% Options for packages loaded elsewhere
\PassOptionsToPackage{unicode}{hyperref}
\PassOptionsToPackage{hyphens}{url}
\PassOptionsToPackage{dvipsnames,svgnames,x11names}{xcolor}
%
\documentclass[
  letterpaper,
  DIV=11,
  numbers=noendperiod]{scrreprt}

\usepackage{amsmath,amssymb}
\usepackage{iftex}
\ifPDFTeX
  \usepackage[T1]{fontenc}
  \usepackage[utf8]{inputenc}
  \usepackage{textcomp} % provide euro and other symbols
\else % if luatex or xetex
  \usepackage{unicode-math}
  \defaultfontfeatures{Scale=MatchLowercase}
  \defaultfontfeatures[\rmfamily]{Ligatures=TeX,Scale=1}
\fi
\usepackage{lmodern}
\ifPDFTeX\else  
    % xetex/luatex font selection
\fi
% Use upquote if available, for straight quotes in verbatim environments
\IfFileExists{upquote.sty}{\usepackage{upquote}}{}
\IfFileExists{microtype.sty}{% use microtype if available
  \usepackage[]{microtype}
  \UseMicrotypeSet[protrusion]{basicmath} % disable protrusion for tt fonts
}{}
\makeatletter
\@ifundefined{KOMAClassName}{% if non-KOMA class
  \IfFileExists{parskip.sty}{%
    \usepackage{parskip}
  }{% else
    \setlength{\parindent}{0pt}
    \setlength{\parskip}{6pt plus 2pt minus 1pt}}
}{% if KOMA class
  \KOMAoptions{parskip=half}}
\makeatother
\usepackage{xcolor}
\setlength{\emergencystretch}{3em} % prevent overfull lines
\setcounter{secnumdepth}{5}
% Make \paragraph and \subparagraph free-standing
\ifx\paragraph\undefined\else
  \let\oldparagraph\paragraph
  \renewcommand{\paragraph}[1]{\oldparagraph{#1}\mbox{}}
\fi
\ifx\subparagraph\undefined\else
  \let\oldsubparagraph\subparagraph
  \renewcommand{\subparagraph}[1]{\oldsubparagraph{#1}\mbox{}}
\fi

\usepackage{color}
\usepackage{fancyvrb}
\newcommand{\VerbBar}{|}
\newcommand{\VERB}{\Verb[commandchars=\\\{\}]}
\DefineVerbatimEnvironment{Highlighting}{Verbatim}{commandchars=\\\{\}}
% Add ',fontsize=\small' for more characters per line
\usepackage{framed}
\definecolor{shadecolor}{RGB}{241,243,245}
\newenvironment{Shaded}{\begin{snugshade}}{\end{snugshade}}
\newcommand{\AlertTok}[1]{\textcolor[rgb]{0.68,0.00,0.00}{#1}}
\newcommand{\AnnotationTok}[1]{\textcolor[rgb]{0.37,0.37,0.37}{#1}}
\newcommand{\AttributeTok}[1]{\textcolor[rgb]{0.40,0.45,0.13}{#1}}
\newcommand{\BaseNTok}[1]{\textcolor[rgb]{0.68,0.00,0.00}{#1}}
\newcommand{\BuiltInTok}[1]{\textcolor[rgb]{0.00,0.23,0.31}{#1}}
\newcommand{\CharTok}[1]{\textcolor[rgb]{0.13,0.47,0.30}{#1}}
\newcommand{\CommentTok}[1]{\textcolor[rgb]{0.37,0.37,0.37}{#1}}
\newcommand{\CommentVarTok}[1]{\textcolor[rgb]{0.37,0.37,0.37}{\textit{#1}}}
\newcommand{\ConstantTok}[1]{\textcolor[rgb]{0.56,0.35,0.01}{#1}}
\newcommand{\ControlFlowTok}[1]{\textcolor[rgb]{0.00,0.23,0.31}{#1}}
\newcommand{\DataTypeTok}[1]{\textcolor[rgb]{0.68,0.00,0.00}{#1}}
\newcommand{\DecValTok}[1]{\textcolor[rgb]{0.68,0.00,0.00}{#1}}
\newcommand{\DocumentationTok}[1]{\textcolor[rgb]{0.37,0.37,0.37}{\textit{#1}}}
\newcommand{\ErrorTok}[1]{\textcolor[rgb]{0.68,0.00,0.00}{#1}}
\newcommand{\ExtensionTok}[1]{\textcolor[rgb]{0.00,0.23,0.31}{#1}}
\newcommand{\FloatTok}[1]{\textcolor[rgb]{0.68,0.00,0.00}{#1}}
\newcommand{\FunctionTok}[1]{\textcolor[rgb]{0.28,0.35,0.67}{#1}}
\newcommand{\ImportTok}[1]{\textcolor[rgb]{0.00,0.46,0.62}{#1}}
\newcommand{\InformationTok}[1]{\textcolor[rgb]{0.37,0.37,0.37}{#1}}
\newcommand{\KeywordTok}[1]{\textcolor[rgb]{0.00,0.23,0.31}{#1}}
\newcommand{\NormalTok}[1]{\textcolor[rgb]{0.00,0.23,0.31}{#1}}
\newcommand{\OperatorTok}[1]{\textcolor[rgb]{0.37,0.37,0.37}{#1}}
\newcommand{\OtherTok}[1]{\textcolor[rgb]{0.00,0.23,0.31}{#1}}
\newcommand{\PreprocessorTok}[1]{\textcolor[rgb]{0.68,0.00,0.00}{#1}}
\newcommand{\RegionMarkerTok}[1]{\textcolor[rgb]{0.00,0.23,0.31}{#1}}
\newcommand{\SpecialCharTok}[1]{\textcolor[rgb]{0.37,0.37,0.37}{#1}}
\newcommand{\SpecialStringTok}[1]{\textcolor[rgb]{0.13,0.47,0.30}{#1}}
\newcommand{\StringTok}[1]{\textcolor[rgb]{0.13,0.47,0.30}{#1}}
\newcommand{\VariableTok}[1]{\textcolor[rgb]{0.07,0.07,0.07}{#1}}
\newcommand{\VerbatimStringTok}[1]{\textcolor[rgb]{0.13,0.47,0.30}{#1}}
\newcommand{\WarningTok}[1]{\textcolor[rgb]{0.37,0.37,0.37}{\textit{#1}}}

\providecommand{\tightlist}{%
  \setlength{\itemsep}{0pt}\setlength{\parskip}{0pt}}\usepackage{longtable,booktabs,array}
\usepackage{calc} % for calculating minipage widths
% Correct order of tables after \paragraph or \subparagraph
\usepackage{etoolbox}
\makeatletter
\patchcmd\longtable{\par}{\if@noskipsec\mbox{}\fi\par}{}{}
\makeatother
% Allow footnotes in longtable head/foot
\IfFileExists{footnotehyper.sty}{\usepackage{footnotehyper}}{\usepackage{footnote}}
\makesavenoteenv{longtable}
\usepackage{graphicx}
\makeatletter
\def\maxwidth{\ifdim\Gin@nat@width>\linewidth\linewidth\else\Gin@nat@width\fi}
\def\maxheight{\ifdim\Gin@nat@height>\textheight\textheight\else\Gin@nat@height\fi}
\makeatother
% Scale images if necessary, so that they will not overflow the page
% margins by default, and it is still possible to overwrite the defaults
% using explicit options in \includegraphics[width, height, ...]{}
\setkeys{Gin}{width=\maxwidth,height=\maxheight,keepaspectratio}
% Set default figure placement to htbp
\makeatletter
\def\fps@figure{htbp}
\makeatother

\KOMAoption{captions}{tableheading}
\makeatletter
\@ifpackageloaded{bookmark}{}{\usepackage{bookmark}}
\makeatother
\makeatletter
\@ifpackageloaded{caption}{}{\usepackage{caption}}
\AtBeginDocument{%
\ifdefined\contentsname
  \renewcommand*\contentsname{Table of contents}
\else
  \newcommand\contentsname{Table of contents}
\fi
\ifdefined\listfigurename
  \renewcommand*\listfigurename{List of Figures}
\else
  \newcommand\listfigurename{List of Figures}
\fi
\ifdefined\listtablename
  \renewcommand*\listtablename{List of Tables}
\else
  \newcommand\listtablename{List of Tables}
\fi
\ifdefined\figurename
  \renewcommand*\figurename{Figure}
\else
  \newcommand\figurename{Figure}
\fi
\ifdefined\tablename
  \renewcommand*\tablename{Table}
\else
  \newcommand\tablename{Table}
\fi
}
\@ifpackageloaded{float}{}{\usepackage{float}}
\floatstyle{ruled}
\@ifundefined{c@chapter}{\newfloat{codelisting}{h}{lop}}{\newfloat{codelisting}{h}{lop}[chapter]}
\floatname{codelisting}{Listing}
\newcommand*\listoflistings{\listof{codelisting}{List of Listings}}
\makeatother
\makeatletter
\makeatother
\makeatletter
\@ifpackageloaded{caption}{}{\usepackage{caption}}
\@ifpackageloaded{subcaption}{}{\usepackage{subcaption}}
\makeatother
\ifLuaTeX
  \usepackage{selnolig}  % disable illegal ligatures
\fi
\usepackage{bookmark}

\IfFileExists{xurl.sty}{\usepackage{xurl}}{} % add URL line breaks if available
\urlstyle{same} % disable monospaced font for URLs
\hypersetup{
  pdftitle={Python Básico},
  pdfauthor={Luis Miguel de la Cruz Salas},
  colorlinks=true,
  linkcolor={blue},
  filecolor={Maroon},
  citecolor={Blue},
  urlcolor={Blue},
  pdfcreator={LaTeX via pandoc}}

\title{Python Básico}
\author{Luis Miguel de la Cruz Salas}
\date{2023-01-12}

\begin{document}
\maketitle

\renewcommand*\contentsname{Table of contents}
{
\hypersetup{linkcolor=}
\setcounter{tocdepth}{2}
\tableofcontents
}
\bookmarksetup{startatroot}

\chapter*{Introducción}\label{introducciuxf3n}
\addcontentsline{toc}{chapter}{Introducción}

\markboth{Introducción}{Introducción}

Este es un primer curso sobre las funcionalidades básicas de Python. Es
un complemento al curso ``Diseño de Cursos Interactivos con la
Plataforma MACTI''.

Los Jupyter Notebooks de este curso se pueden obtener en
\url{https://github.com/repomacti/macti/tutoriales/python_basico}.

\bookmarksetup{startatroot}

\chapter{Definición de variables.}\label{definiciuxf3n-de-variables.}

\textbf{Objetivo.} Explicar el concepto de variable, etiqueta, objetos y
como se usan mediante algunos ejemplos.

\textbf{Funciones de Python}: - \texttt{print()}, \texttt{type()},
\texttt{id()}, \texttt{chr()}, \texttt{ord()}, \texttt{del()}

MACTI-Algebra\_Lineal\_01 by Luis M. de la Cruz is licensed under
Attribution-ShareAlike 4.0 International

\section{Variables.}\label{variables.}

\begin{itemize}
\tightlist
\item
  Son \textbf{símbolos} que permiten identificar la información que se
  almacena en la memoria de la computadora.
\item
  Son \textbf{nombres} o \textbf{etiquetas} para los objetos que se
  crean en Python.
\item
  Se crean con ayuda del operador de asignación \texttt{=}.
\item
  No se tiene que establecer explícitamente el tipo de dato de la
  variable, pues esto se realiza de manera dinámica (tipado dinámico).
\end{itemize}

\subsection{\texorpdfstring{\textbf{Ejemplos de variables
válidas.}}{Ejemplos de variables válidas.}}\label{ejemplos-de-variables-vuxe1lidas.}

Los nombres de las variables: * pueden contener \textbf{letras},
\textbf{números} y \textbf{guiones bajos}, * deben comenzar con una
letra o un guion bajo, * se distingue entre mayúsculas y minúsculas, es
decir, \texttt{variable} y \texttt{Variable} son nombres diferentes.

A continuación se muestran algunos ejemplos.

\begin{Shaded}
\begin{Highlighting}[]
\NormalTok{\_luis }\OperatorTok{=} \StringTok{"Luis Miguel de la Cruz"}    \CommentTok{\# El nombre de la variable es \_luis, el contenido es una cadena.}
\NormalTok{LuisXV }\OperatorTok{=} \StringTok{"Louis Michel de la Croix"} 
\NormalTok{luigi }\OperatorTok{=} \DecValTok{25}
\NormalTok{luis\_b }\OperatorTok{=} \BaseNTok{0b01110} \CommentTok{\# Binario}
\NormalTok{luis\_o }\OperatorTok{=} \BaseNTok{0o12376} \CommentTok{\# Octal}
\NormalTok{luis\_h }\OperatorTok{=} \BaseNTok{0x12323} \CommentTok{\# Hexadecimal}

\CommentTok{\# Sensibilidad a mayúsculas y minúsculas}
\CommentTok{\# los siguientes nombres son diferentes}
\NormalTok{pi }\OperatorTok{=} \FloatTok{3.14}
\NormalTok{PI }\OperatorTok{=} \FloatTok{31416e{-}4}
\NormalTok{Pi }\OperatorTok{=} \FloatTok{3.141592}
\end{Highlighting}
\end{Shaded}

Podemos ver el contenido de la variable usando la función
\texttt{print()}:

\begin{Shaded}
\begin{Highlighting}[]
\CommentTok{\# El contenido de cada variable se imprime en renglones}
\CommentTok{\# diferentes debido a que usamos el argumento sep=\textquotesingle{}\textbackslash{}n\textquotesingle{}}
\BuiltInTok{print}\NormalTok{(\_luis, LuisXV, luigi, luis\_b, luis\_o, luis\_h, pi, PI, Pi, sep}\OperatorTok{=}\StringTok{\textquotesingle{}}\CharTok{\textbackslash{}n}\StringTok{\textquotesingle{}}\NormalTok{)}
\end{Highlighting}
\end{Shaded}

\begin{verbatim}
Luis Miguel de la Cruz
Louis Michel de la Croix
25
14
5374
74531
3.14
3.1416
3.141592
\end{verbatim}

\textbf{NOTA}. Para saber más sobre la función \texttt{print()} revisa
la sección XXX.

Para saber el tipo de objeto que se creo cuando se definieron las
variables anteriores, podemos hacer uso de la función \texttt{type()}:

\begin{Shaded}
\begin{Highlighting}[]
\BuiltInTok{print}\NormalTok{(}\BuiltInTok{type}\NormalTok{(\_luis), }\BuiltInTok{type}\NormalTok{(LuisXV), }\BuiltInTok{type}\NormalTok{(luigi), }
      \BuiltInTok{type}\NormalTok{(luis\_b), }\BuiltInTok{type}\NormalTok{(luis\_o), }\BuiltInTok{type}\NormalTok{(luis\_h),}
      \BuiltInTok{type}\NormalTok{(pi), }\BuiltInTok{type}\NormalTok{(PI), }\BuiltInTok{type}\NormalTok{(Pi), sep }\OperatorTok{=} \StringTok{\textquotesingle{}}\CharTok{\textbackslash{}n}\StringTok{\textquotesingle{}}\NormalTok{)}
\end{Highlighting}
\end{Shaded}

\begin{verbatim}
<class 'str'>
<class 'str'>
<class 'int'>
<class 'int'>
<class 'int'>
<class 'int'>
<class 'float'>
<class 'float'>
<class 'float'>
\end{verbatim}

También es posible usar la función \texttt{id()} para conocer el
identificador en la memoria de cada objeto como sigue:

\begin{Shaded}
\begin{Highlighting}[]
\BuiltInTok{print}\NormalTok{(}\BuiltInTok{id}\NormalTok{(\_luis), }\BuiltInTok{id}\NormalTok{(LuisXV), }\BuiltInTok{id}\NormalTok{(luigi), }
      \BuiltInTok{id}\NormalTok{(luis\_b), }\BuiltInTok{id}\NormalTok{(luis\_o), }\BuiltInTok{id}\NormalTok{(luis\_h),}
      \BuiltInTok{id}\NormalTok{(pi), }\BuiltInTok{id}\NormalTok{(PI), }\BuiltInTok{id}\NormalTok{(Pi), sep }\OperatorTok{=} \StringTok{\textquotesingle{}}\CharTok{\textbackslash{}n}\StringTok{\textquotesingle{}}\NormalTok{)}
\end{Highlighting}
\end{Shaded}

\begin{verbatim}
139672517675408
139672517886160
94157725269672
94157725269320
139672517639248
139672517630576
139672517623920
139672517638992
139672518259056
\end{verbatim}

Observa que cada objeto tiene un identificador diferente. Es posible que
un objeto tenga más de un nombre, por ejemplo

\begin{Shaded}
\begin{Highlighting}[]
\NormalTok{luiggi }\OperatorTok{=}\NormalTok{ \_luis }
\end{Highlighting}
\end{Shaded}

La etiqueta o variable \texttt{luiggi} hace referencia al mismo objeto
que la variable \texttt{\_luis}, y eso lo podemos comprbar usando la
función \texttt{id()}:

\begin{Shaded}
\begin{Highlighting}[]
\BuiltInTok{print}\NormalTok{(}\BuiltInTok{id}\NormalTok{(luiggi))}
\BuiltInTok{print}\NormalTok{(}\BuiltInTok{id}\NormalTok{(\_luis))}
\end{Highlighting}
\end{Shaded}

\begin{verbatim}
139672517675408
139672517675408
\end{verbatim}

\subsection{\texorpdfstring{\textbf{Ejemplos con
Unicode.}}{Ejemplos con Unicode.}}\label{ejemplos-con-unicode.}

\textbf{Unicode}: estándar para la codificación de caracteres, que
permite el tratamiento informático, la transmisión y visualización de
textos de muchos idiomas y disciplinas técnicas. Unicode intenta tener
universalidad, uniformidad y unicidad. Unicode define tres formas de
codificación bajo el nombre UTF (Unicode transformation format): UTF8,
UTF16, UTF32. Véase https://es.wikipedia.org/wiki/Unicode

Python 3 utiliza internamente el tipo de datos \texttt{str} para
representar cadenas de texto Unicode, lo que significa que se puede
escribir y manipular texto en cualquier idioma sin preocuparte por la
codificación. La compatibilidad con UTF-8 en Python significa que se
puede leer, escribir y manipular archivos de texto en cualquier idioma,
y también trabajar con datos provenientes de fuentes diversas, como
bases de datos, API web, etc., que pueden contener texto en diferentes
idiomas y codificaciones.

A continuación se muestran algunos ejemplos.

\begin{Shaded}
\begin{Highlighting}[]
\NormalTok{compañero }\OperatorTok{=} \StringTok{\textquotesingle{}Luismi\textquotesingle{}} \CommentTok{\# puedo usar la ñ como parte del nombre de la variable}
\BuiltInTok{print}\NormalTok{(compañero)}
\end{Highlighting}
\end{Shaded}

\begin{verbatim}
Luismi
\end{verbatim}

Los códigos Unicode de cada caracter se pueden dar en decimal o
hexadecimal, por ejemplo para el símbolo \(\pi\) se tiene el código
decimal \texttt{120587} y hexadecimal \texttt{0x1D70B}. La función
\texttt{chr()} convierte ese código en el caracter correspondiente:

\begin{Shaded}
\begin{Highlighting}[]
\BuiltInTok{chr}\NormalTok{(}\BaseNTok{0x1D70B}\NormalTok{)}
\end{Highlighting}
\end{Shaded}

\begin{verbatim}
'𝜋'
\end{verbatim}

\begin{Shaded}
\begin{Highlighting}[]
\BuiltInTok{chr}\NormalTok{(}\DecValTok{120587}\NormalTok{)}
\end{Highlighting}
\end{Shaded}

\begin{verbatim}
'𝜋'
\end{verbatim}

La función \texttt{ord()} obtiene el código Unicode de un caracter y lo
regresa en decimal:

\begin{Shaded}
\begin{Highlighting}[]
\BuiltInTok{ord}\NormalTok{(}\StringTok{\textquotesingle{}𝜋\textquotesingle{}}\NormalTok{)}
\end{Highlighting}
\end{Shaded}

\begin{verbatim}
120587
\end{verbatim}

Podemos usar la función \texttt{print()} para realizar una impresión con
formato como sigue:

\begin{Shaded}
\begin{Highlighting}[]
\NormalTok{𝜋 }\OperatorTok{=} \FloatTok{3.141592} 
\BuiltInTok{print}\NormalTok{(}\StringTok{\textquotesingle{}}\SpecialCharTok{\{:04d\}}\StringTok{ }\CharTok{\textbackslash{}t}\StringTok{ }\SpecialCharTok{\{\}}\StringTok{ = }\SpecialCharTok{\{\}}\StringTok{\textquotesingle{}}\NormalTok{.}\BuiltInTok{format}\NormalTok{(}\BuiltInTok{ord}\NormalTok{(}\StringTok{\textquotesingle{}𝜋\textquotesingle{}}\NormalTok{), }\StringTok{\textquotesingle{}𝜋\textquotesingle{}}\NormalTok{, 𝜋)) }\CommentTok{\# Impresión en decimal}
\BuiltInTok{print}\NormalTok{(}\StringTok{\textquotesingle{}}\SpecialCharTok{\{:04o\}}\StringTok{ }\CharTok{\textbackslash{}t}\StringTok{ }\SpecialCharTok{\{\}}\StringTok{ = }\SpecialCharTok{\{\}}\StringTok{\textquotesingle{}}\NormalTok{.}\BuiltInTok{format}\NormalTok{(}\BuiltInTok{ord}\NormalTok{(}\StringTok{\textquotesingle{}𝜋\textquotesingle{}}\NormalTok{), }\StringTok{\textquotesingle{}𝜋\textquotesingle{}}\NormalTok{, 𝜋)) }\CommentTok{\# Impresión en octal}
\BuiltInTok{print}\NormalTok{(}\StringTok{\textquotesingle{}}\SpecialCharTok{\{:04x\}}\StringTok{ }\CharTok{\textbackslash{}t}\StringTok{ }\SpecialCharTok{\{\}}\StringTok{ = }\SpecialCharTok{\{\}}\StringTok{\textquotesingle{}}\NormalTok{.}\BuiltInTok{format}\NormalTok{(}\BuiltInTok{ord}\NormalTok{(}\StringTok{\textquotesingle{}𝜋\textquotesingle{}}\NormalTok{), }\StringTok{\textquotesingle{}𝜋\textquotesingle{}}\NormalTok{, 𝜋)) }\CommentTok{\# Impresión en hexadecimal}
\end{Highlighting}
\end{Shaded}

\begin{verbatim}
120587   𝜋 = 3.141592
353413   𝜋 = 3.141592
1d70b    𝜋 = 3.141592
\end{verbatim}

Podemos usar acentos:

\begin{Shaded}
\begin{Highlighting}[]
\NormalTok{México }\OperatorTok{=} \StringTok{\textquotesingle{}El ombligo de la luna\textquotesingle{}}
\BuiltInTok{print}\NormalTok{(México)}
\end{Highlighting}
\end{Shaded}

\begin{verbatim}
El ombligo de la luna
\end{verbatim}

Puedo saber el tipo de codificación que usa Python de la siguiente
manera:

\begin{Shaded}
\begin{Highlighting}[]
\ImportTok{import}\NormalTok{ sys}
\NormalTok{sys.stdout.encoding}
\end{Highlighting}
\end{Shaded}

\begin{verbatim}
'UTF-8'
\end{verbatim}

También es posible obtener más información de los códigos unicode como
sigue:

\begin{Shaded}
\begin{Highlighting}[]
\ImportTok{import}\NormalTok{ unicodedata}

\NormalTok{u }\OperatorTok{=} \BuiltInTok{chr}\NormalTok{(}\DecValTok{233}\NormalTok{) }\OperatorTok{+} \BuiltInTok{chr}\NormalTok{(}\BaseNTok{0x0bf2}\NormalTok{) }\OperatorTok{+} \BuiltInTok{chr}\NormalTok{(}\DecValTok{6000}\NormalTok{) }\OperatorTok{+} \BuiltInTok{chr}\NormalTok{(}\DecValTok{13231}\NormalTok{) }
\BuiltInTok{print}\NormalTok{(}\StringTok{\textquotesingle{}cadena : \textquotesingle{}}\NormalTok{, u)}
\BuiltInTok{print}\NormalTok{()}
\ControlFlowTok{for}\NormalTok{ i, c }\KeywordTok{in} \BuiltInTok{enumerate}\NormalTok{(u):}
    \BuiltInTok{print}\NormalTok{(}\StringTok{\textquotesingle{}}\SpecialCharTok{\{\}}\StringTok{ }\SpecialCharTok{\{:\textgreater{}5x\}}\StringTok{ }\SpecialCharTok{\{:\textgreater{}3\}}\StringTok{\textquotesingle{}}\NormalTok{.}\BuiltInTok{format}\NormalTok{(c, }\BuiltInTok{ord}\NormalTok{(c), unicodedata.category(c)), end}\OperatorTok{=}\StringTok{" "}\NormalTok{)}
    \BuiltInTok{print}\NormalTok{(unicodedata.name(c))}
\end{Highlighting}
\end{Shaded}

\begin{verbatim}
cadena :  é௲ᝰ㎯

é    e9  Ll LATIN SMALL LETTER E WITH ACUTE
௲   bf2  No TAMIL NUMBER ONE THOUSAND
ᝰ  1770  Lo TAGBANWA LETTER SA
㎯  33af  So SQUARE RAD OVER S SQUARED
\end{verbatim}

Véase: https://docs.python.org/3/howto/unicode.html

\section{Asignación múltiple.}\label{asignaciuxf3n-muxfaltiple.}

Es posible definir varias variables en una sola instrucción:

\begin{Shaded}
\begin{Highlighting}[]
\NormalTok{x }\OperatorTok{=}\NormalTok{ y }\OperatorTok{=}\NormalTok{ z }\OperatorTok{=} \DecValTok{25}
\end{Highlighting}
\end{Shaded}

\begin{Shaded}
\begin{Highlighting}[]
\BuiltInTok{print}\NormalTok{(}\BuiltInTok{type}\NormalTok{(x), }\BuiltInTok{type}\NormalTok{(y), }\BuiltInTok{type}\NormalTok{(z))}
\end{Highlighting}
\end{Shaded}

\begin{verbatim}
<class 'int'> <class 'int'> <class 'int'>
\end{verbatim}

\begin{Shaded}
\begin{Highlighting}[]
\BuiltInTok{print}\NormalTok{(}\BuiltInTok{id}\NormalTok{(x), }\BuiltInTok{id}\NormalTok{(y), }\BuiltInTok{id}\NormalTok{(z))}
\end{Highlighting}
\end{Shaded}

\begin{verbatim}
94157725269672 94157725269672 94157725269672
\end{verbatim}

Observa que se creó el objeto \texttt{25} de tipo
\texttt{\textless{}class\ \textquotesingle{}int\textquotesingle{}\textgreater{}}
y los nombres \texttt{x}, \texttt{y} y \texttt{z} son etiquetas al mismo
objeto, como se verifica imprimiendo el identificador de cada variable
usando la función \texttt{id()}.

Podemos eliminar la etiqueta \texttt{x} con la función \texttt{del()}:

\begin{Shaded}
\begin{Highlighting}[]
\KeywordTok{del}\NormalTok{(x)}
\end{Highlighting}
\end{Shaded}

Ahora ya no es posible hacer referencia al objeto \texttt{25} usando
\texttt{x}:

\begin{Shaded}
\begin{Highlighting}[]
\BuiltInTok{print}\NormalTok{(x)}
\end{Highlighting}
\end{Shaded}

\begin{verbatim}
NameError: name 'x' is not defined
\end{verbatim}

Pero si es posible hacer referencia al objeto \texttt{25} con los
nombres \texttt{y} y \texttt{z}:

\begin{Shaded}
\begin{Highlighting}[]
\BuiltInTok{print}\NormalTok{(y,z)}
\end{Highlighting}
\end{Shaded}

\begin{verbatim}
25 25
\end{verbatim}

Podemos hacer una asignación múltiples de objetos diferentes a variables
diferentes:

\begin{Shaded}
\begin{Highlighting}[]
\NormalTok{x, y, z }\OperatorTok{=} \StringTok{\textquotesingle{}eje x\textquotesingle{}}\NormalTok{, }\FloatTok{3.141592}\NormalTok{, }\DecValTok{50}
\end{Highlighting}
\end{Shaded}

\begin{Shaded}
\begin{Highlighting}[]
\BuiltInTok{print}\NormalTok{(}\BuiltInTok{type}\NormalTok{(x), }\BuiltInTok{type}\NormalTok{(y), }\BuiltInTok{type}\NormalTok{(z))}
\end{Highlighting}
\end{Shaded}

\begin{verbatim}
<class 'str'> <class 'float'> <class 'int'>
\end{verbatim}

\begin{Shaded}
\begin{Highlighting}[]
\BuiltInTok{print}\NormalTok{(}\BuiltInTok{id}\NormalTok{(x), }\BuiltInTok{id}\NormalTok{(y), }\BuiltInTok{id}\NormalTok{(z))}
\end{Highlighting}
\end{Shaded}

\begin{verbatim}
139672217233008 139672517638832 94157725270472
\end{verbatim}

Como se observa, ahora las variables \texttt{x}, \texttt{y} y~\texttt{z}
hacen referencia a diferentes objetos, de distinto tipo.

\subsection{\texorpdfstring{\textbf{Ejemplos de nombres NO
válidos.}}{Ejemplos de nombres NO válidos.}}\label{ejemplos-de-nombres-no-vuxe1lidos.}

Los siguientes son ejemplos NO VALIDOS para el nombre de variables. Al
ejecutar las celdas se obtendrá un error en cada una de ellas.

\begin{Shaded}
\begin{Highlighting}[]
\DecValTok{1l}\ErrorTok{uis} \OperatorTok{=} \DecValTok{20}      \CommentTok{\# No se puede iniciar con un número}
\end{Highlighting}
\end{Shaded}

\begin{verbatim}
SyntaxError: invalid decimal literal (953519616.py, line 1)
\end{verbatim}

\begin{Shaded}
\begin{Highlighting}[]
\NormalTok{luis$ }\OperatorTok{=} \FloatTok{8.2323}  \CommentTok{\# No puede contener caractéres especiales}
\end{Highlighting}
\end{Shaded}

\begin{verbatim}
SyntaxError: invalid syntax (2653363214.py, line 1)
\end{verbatim}

\begin{Shaded}
\begin{Highlighting}[]
\ControlFlowTok{for} \OperatorTok{=} \DecValTok{35}        \CommentTok{\# Algunos nombres ya están reservados}
\end{Highlighting}
\end{Shaded}

\begin{verbatim}
SyntaxError: invalid syntax (2521306807.py, line 1)
\end{verbatim}

\section{Palabras reservadas.}\label{palabras-reservadas.}

Tampoco es posible usar las palabras reservadas para nombrar variables.
Podemos conocer las palabras reservadas como sigue:

\begin{Shaded}
\begin{Highlighting}[]
\BuiltInTok{help}\NormalTok{(}\StringTok{\textquotesingle{}keywords\textquotesingle{}}\NormalTok{)}
\end{Highlighting}
\end{Shaded}

\begin{verbatim}

Here is a list of the Python keywords.  Enter any keyword to get more help.

False               class               from                or
None                continue            global              pass
True                def                 if                  raise
and                 del                 import              return
as                  elif                in                  try
assert              else                is                  while
async               except              lambda              with
await               finally             nonlocal            yield
break               for                 not                 
\end{verbatim}

\bookmarksetup{startatroot}

\chapter{Expresiones y
declaraciones.}\label{expresiones-y-declaraciones.}

\textbf{Objetivo.} Explicar el concepto de variable, etiqueta, objetos y
como se usan mediante algunos ejemplos.

\textbf{Funciones de Python}: - \texttt{print()}, \texttt{type()},
\texttt{id()}, \texttt{chr()}, \texttt{ord()}, \texttt{del()}

MACTI-Algebra\_Lineal\_01 by Luis M. de la Cruz is licensed under
Attribution-ShareAlike 4.0 International

\section{Expresiones}\label{expresiones}

En matemáticas se define una expresión como una colección de símbolos
que juntos expresan una cantidad, por ejemplo, el perímetro de una
circunferencia es 2\(\pi r\).

En Python una \textbf{expresión} está compuesta de una combinación
válida de valores, variables, operadores, funciones y métodos, que se
puede evaluar y \textbf{da como resultado al menos un valor}.

En esencia, una \textbf{expresión es cualquier cosa que pueda ser
evaluada y producir un resultado}.

Las expresiones pueden ser simples o complejas, pero en general,
representan un valor único, por ejemplo:

\begin{Shaded}
\begin{Highlighting}[]
\NormalTok{a }\OperatorTok{=} \DecValTok{2}\OperatorTok{**}\DecValTok{32}
\end{Highlighting}
\end{Shaded}

Véase más en The Python language reference: Expressions y Python
expressions .

Veamos algunos ejemplos:

\textbf{Expresiones simples}

\begin{Shaded}
\begin{Highlighting}[]
\DecValTok{23} 
\end{Highlighting}
\end{Shaded}

\begin{verbatim}
23
\end{verbatim}

\begin{Shaded}
\begin{Highlighting}[]
\DecValTok{5} \OperatorTok{+} \DecValTok{3}
\end{Highlighting}
\end{Shaded}

\begin{verbatim}
8
\end{verbatim}

\begin{Shaded}
\begin{Highlighting}[]
\NormalTok{a }\OperatorTok{=} \DecValTok{5}
\NormalTok{a }\OperatorTok{**} \DecValTok{2}
\end{Highlighting}
\end{Shaded}

\begin{verbatim}
25
\end{verbatim}

\textbf{Expresión que ejecuta una función}

\begin{Shaded}
\begin{Highlighting}[]
\BuiltInTok{len}\NormalTok{(}\StringTok{\textquotesingle{}Hola mundo\textquotesingle{}}\NormalTok{) }
\end{Highlighting}
\end{Shaded}

\begin{verbatim}
10
\end{verbatim}

\textbf{Expresiones usando operadores}

\begin{Shaded}
\begin{Highlighting}[]
\CommentTok{\# Otros ejemplos}
\NormalTok{x }\OperatorTok{=} \DecValTok{1}
\NormalTok{y }\OperatorTok{=}\NormalTok{ x }\OperatorTok{+} \DecValTok{2}
\NormalTok{z }\OperatorTok{=}\NormalTok{ y }\OperatorTok{**} \DecValTok{3}

\BuiltInTok{print}\NormalTok{(x)}
\BuiltInTok{print}\NormalTok{(y)}
\BuiltInTok{print}\NormalTok{(z)}

\CommentTok{\# Operación Booleana}
\DecValTok{7} \OperatorTok{==} \DecValTok{2} \OperatorTok{*} \DecValTok{2} \OperatorTok{*} \DecValTok{2}
\end{Highlighting}
\end{Shaded}

\begin{verbatim}
1
3
27
\end{verbatim}

\begin{verbatim}
False
\end{verbatim}

\textbf{Expresiones más complicadas}

\begin{Shaded}
\begin{Highlighting}[]
\CommentTok{\# Se combinan varias operaciones matemáticas}
\NormalTok{b }\OperatorTok{=} \FloatTok{2.14}
\NormalTok{c }\OperatorTok{=} \FloatTok{0.1} \OperatorTok{+} \OtherTok{4j}

\NormalTok{(}\FloatTok{3.141592} \OperatorTok{*}\NormalTok{ c }\OperatorTok{+}\NormalTok{ b) }\OperatorTok{/}\NormalTok{ a}
\end{Highlighting}
\end{Shaded}

\begin{verbatim}
(0.4908318400000001+2.5132736j)
\end{verbatim}

Observa que en todos los ejemplos anteriores se produce al menos un
valor como resultado de la ejecución de cada expresión.

\section{Declaraciones}\label{declaraciones}

Una \textbf{declaración} (\emph{statement}) se puede pensar como el
elemento autónomo más corto de un lenguaje de programación. Un programa
se forma de una secuencia que contiene una o más declaraciones. Una
declaración contiene componentes internos, que pueden ser otras
declaraciones y varias expresiones.

En términos simples, una \textbf{declaración} es una \textbf{instrucción
que realiza una acción}.

Puede ser una asignación de valor a una variable, una llamada a una
función, una estructura de control de flujo (como un ciclo o una
condición), una definición de función, etc.

Véase más en Simple statements, Compound statements y Python statements
(wikipedia).

Veamos algunos ejemplos:

\textbf{Declaración que hace una asignación}

\begin{Shaded}
\begin{Highlighting}[]
\NormalTok{x }\OperatorTok{=} \DecValTok{0}
\end{Highlighting}
\end{Shaded}

\textbf{Declaración usando un condicional}

\begin{Shaded}
\begin{Highlighting}[]
\ControlFlowTok{if}\NormalTok{ x }\OperatorTok{\textless{}} \DecValTok{0}\NormalTok{:}
    \ControlFlowTok{pass}
\end{Highlighting}
\end{Shaded}

\textbf{Declaración que realiza un ciclo}

\begin{Shaded}
\begin{Highlighting}[]
\ControlFlowTok{for}\NormalTok{ i }\KeywordTok{in} \BuiltInTok{range}\NormalTok{(}\DecValTok{0}\NormalTok{,}\DecValTok{5}\NormalTok{):}
    \ControlFlowTok{pass}
\end{Highlighting}
\end{Shaded}

\textbf{Declaración de una función}

\begin{Shaded}
\begin{Highlighting}[]
\KeywordTok{def}\NormalTok{ mult(a, b):}
    \ControlFlowTok{return}\NormalTok{ a }\OperatorTok{*}\NormalTok{ b}
\end{Highlighting}
\end{Shaded}

\begin{Shaded}
\begin{Highlighting}[]
\NormalTok{Here is a note}
\end{Highlighting}
\end{Shaded}

\bookmarksetup{startatroot}

\chapter{Tipos de datos básicos y
operadores.}\label{tipos-de-datos-buxe1sicos-y-operadores.}

\textbf{Objetivo.} Explicar el concepto de variable, etiqueta, objetos y
como se usan mediante algunos ejemplos.

\textbf{Funciones de Python}: - \texttt{print()}, \texttt{type()},
\texttt{id()}, \texttt{chr()}, \texttt{ord()}, \texttt{del()}

MACTI-Algebra\_Lineal\_01 by Luis M. de la Cruz is licensed under
Attribution-ShareAlike 4.0 International

\section{Tipos y operadores}\label{tipos-y-operadores}

En Python se tienen tres tipos de datos básicos principales:

\begin{longtable}[]{@{}ll@{}}
\toprule\noalign{}
Tipo & Ejemplo \\
\midrule\noalign{}
\endhead
\bottomrule\noalign{}
\endlastfoot
Númerico & 13, 3.1416, 1+5j \\
Cadena & ``Frida'', ``Diego'' \\
Lógico & True, False \\
\end{longtable}

\section{Tipos númericos}\label{tipos-nuxfamericos}

En Python se tienen tres tipos de datos númericos: 1. Enteros 2. Reales
3. Complejos

A continuación se realiza una descripción de estos tipos numéricos. Más
información se puede encontrar aquí: Numeric types.

\textbf{1. Enteros}

Son aquellos que carecen de parte decimal. Para definir un entero
hacemos lo siguiente:

\begin{Shaded}
\begin{Highlighting}[]
\NormalTok{entero }\OperatorTok{=} \DecValTok{13}
\end{Highlighting}
\end{Shaded}

Cuando se ejecuta la celda anterior, se crea el objeto \texttt{13} cuyo
nombre es \texttt{entero}. Podemos imprimir el valor de \texttt{entero}
y su tipo como sigue:

\begin{Shaded}
\begin{Highlighting}[]
\BuiltInTok{print}\NormalTok{(entero)}
\BuiltInTok{print}\NormalTok{(}\BuiltInTok{type}\NormalTok{(entero))}
\end{Highlighting}
\end{Shaded}

\begin{verbatim}
13
<class 'int'>
\end{verbatim}

Es posible obtener más información del tipo \texttt{int} usando la
biblioteca \texttt{sys}:

\begin{Shaded}
\begin{Highlighting}[]
\ImportTok{import}\NormalTok{ sys}
\NormalTok{sys.int\_info}
\end{Highlighting}
\end{Shaded}

\begin{verbatim}
sys.int_info(bits_per_digit=30, sizeof_digit=4, default_max_str_digits=4300, str_digits_check_threshold=640)
\end{verbatim}

\textbf{2. Reales}

Son aquellos que tienen una parte decimal. Para definir un número real
(flotante) se hace como sigue:

\begin{Shaded}
\begin{Highlighting}[]
\NormalTok{pi }\OperatorTok{=} \FloatTok{3.141592}
\end{Highlighting}
\end{Shaded}

Cuando se ejecuta la celda anterior, se crea el objeto \texttt{3.141592}
cuyo nombre es \texttt{pi}. Podemos imprimir el valor de \texttt{pi} y
su tipo como sigue:

\begin{Shaded}
\begin{Highlighting}[]
\BuiltInTok{print}\NormalTok{(pi)}
\BuiltInTok{print}\NormalTok{(}\BuiltInTok{type}\NormalTok{(pi))}
\end{Highlighting}
\end{Shaded}

\begin{verbatim}
3.141592
<class 'float'>
\end{verbatim}

\begin{Shaded}
\begin{Highlighting}[]
\CommentTok{\# para obtener más información:}
\NormalTok{sys.float\_info}
\end{Highlighting}
\end{Shaded}

\begin{verbatim}
sys.float_info(max=1.7976931348623157e+308, max_exp=1024, max_10_exp=308, min=2.2250738585072014e-308, min_exp=-1021, min_10_exp=-307, dig=15, mant_dig=53, epsilon=2.220446049250313e-16, radix=2, rounds=1)
\end{verbatim}

\textbf{3. Complejos}

Son aquellos que tienen una parte real y una parte imaginaria, y ambas
partes son números reales. Para definir un número complejo se hace como
sigue:

\begin{Shaded}
\begin{Highlighting}[]
\NormalTok{complejo }\OperatorTok{=} \DecValTok{12} \OperatorTok{+} \OtherTok{5j} \CommentTok{\# La parte imaginaria lleva una j al final}
\end{Highlighting}
\end{Shaded}

Cuando se ejecuta la celda anterior, se crea el objeto
\texttt{12\ +\ 5j} cuyo nombre es \texttt{complejo}. En este caso, el
contenido de \texttt{complejo} tiene dos partes: la real y la
imaginaria. Podemos imprimir el valor de \texttt{complejo} y su tipo
como sigue:

\begin{Shaded}
\begin{Highlighting}[]
\BuiltInTok{print}\NormalTok{(complejo)}
\BuiltInTok{print}\NormalTok{(}\BuiltInTok{type}\NormalTok{(complejo))}
\end{Highlighting}
\end{Shaded}

\begin{verbatim}
(12+5j)
<class 'complex'>
\end{verbatim}

\begin{Shaded}
\begin{Highlighting}[]
\NormalTok{complejo.imag }\CommentTok{\# accedemos a la parte imaginaria}
\end{Highlighting}
\end{Shaded}

\begin{verbatim}
5.0
\end{verbatim}

\begin{Shaded}
\begin{Highlighting}[]
\NormalTok{complejo.real }\CommentTok{\# accedemos a la parte real}
\end{Highlighting}
\end{Shaded}

\begin{verbatim}
12.0
\end{verbatim}

\begin{Shaded}
\begin{Highlighting}[]
\NormalTok{complejo.conjugate() }\CommentTok{\# calculamos el conjugado del número complejo.}
\end{Highlighting}
\end{Shaded}

\begin{verbatim}
(12-5j)
\end{verbatim}

\textbf{Nota}: observa que hemos aplicado el método \texttt{conjugate()}
al objeto \texttt{complejo}, esto es posible debido a que existe la
clase
\texttt{\textless{}class\ \textquotesingle{}complex\textquotesingle{}\textgreater{}}
en Python, y en ella se definen atributos y métodos para los objetos de
esta clase. Más acerca de programación orientada a objetos la puedes ver
en esta sección XXX.

\subsection{Operadores aritméticos}\label{operadores-aritmuxe9ticos}

Para los tipos numéricos descritos antes, existen operaciones
aritméticas que se pueden aplicar sobre ellos. Veamos:

\begin{Shaded}
\begin{Highlighting}[]
\CommentTok{\# Suma}
\DecValTok{1} \OperatorTok{+} \DecValTok{2}
\end{Highlighting}
\end{Shaded}

\begin{Shaded}
\begin{Highlighting}[]
\CommentTok{\# Resta}
\DecValTok{5} \OperatorTok{{-}} \DecValTok{32}
\end{Highlighting}
\end{Shaded}

\begin{Shaded}
\begin{Highlighting}[]
\CommentTok{\# Multiplicación}
\DecValTok{3} \OperatorTok{*} \DecValTok{3}
\end{Highlighting}
\end{Shaded}

\begin{Shaded}
\begin{Highlighting}[]
\CommentTok{\# División}
\DecValTok{3} \OperatorTok{/} \DecValTok{2}
\end{Highlighting}
\end{Shaded}

\begin{Shaded}
\begin{Highlighting}[]
\CommentTok{\# Potencia}
\DecValTok{81} \OperatorTok{**}\NormalTok{ (}\DecValTok{1}\OperatorTok{/}\DecValTok{2}\NormalTok{)}
\end{Highlighting}
\end{Shaded}

\subsection{Precedencia de
operadores.}\label{precedencia-de-operadores.}

La aplicación de los operadores tiene cierta precedencia que está
definida en cada implementación del lenguaje de programación. La
siguiente tabla muestra el orden en que se aplicarán los operadores en
una expresión.

\begin{longtable}[]{@{}lll@{}}
\toprule\noalign{}
Nivel & Categoría & Operadores \\
\midrule\noalign{}
\endhead
\bottomrule\noalign{}
\endlastfoot
7 & exponenciación & \texttt{**} \\
6 & multiplicación & \texttt{*},\texttt{/},\texttt{//},\texttt{\%} \\
5 & adición & \texttt{+},\texttt{-} \\
4 & relacional &
\texttt{==},\texttt{!=},\texttt{\textless{}=},\texttt{\textgreater{}=},\texttt{\textgreater{}},\texttt{\textless{}} \\
3 & logicos & \texttt{not} \\
2 & logicos & \texttt{and} \\
1 & logicos & \texttt{or} \\
\end{longtable}

Como se puede ver, siempre se aplican primero los operadores aritméticos
(niveles 7,6, y 5), luego los relacionales (nivel 4) y finalmente los
lógicos (3, 2 y 1).

Más acerca de este tema se puede ver aquí: Operator precedence.

A continuación se muestran ejemplos de operaciones aritméticas en donde
se resalta est precedencia:

\begin{Shaded}
\begin{Highlighting}[]
\CommentTok{\# Precedencia de operaciones: primero se realiza la multiplicación}
\DecValTok{1} \OperatorTok{+} \DecValTok{2} \OperatorTok{*} \DecValTok{3} \OperatorTok{+} \DecValTok{4}
\end{Highlighting}
\end{Shaded}

\begin{verbatim}
11
\end{verbatim}

\begin{Shaded}
\begin{Highlighting}[]
\CommentTok{\# Es posible usar paréntesis para modificar la precedencia: primero la suma}
\NormalTok{(}\DecValTok{1} \OperatorTok{+} \DecValTok{2}\NormalTok{) }\OperatorTok{*}\NormalTok{ (}\DecValTok{3} \OperatorTok{+} \DecValTok{4}\NormalTok{)}
\end{Highlighting}
\end{Shaded}

\begin{verbatim}
21
\end{verbatim}

\begin{Shaded}
\begin{Highlighting}[]
\CommentTok{\# ¿Puedes explicar el resultado de acuerdo con la precedencia}
\CommentTok{\# descrita en la tabla anterior?}
\DecValTok{6}\OperatorTok{/}\DecValTok{2}\OperatorTok{*}\NormalTok{(}\DecValTok{2}\OperatorTok{+}\DecValTok{1}\NormalTok{) }
\end{Highlighting}
\end{Shaded}

\begin{verbatim}
9.0
\end{verbatim}

\subsection{Operaciones entre tipos
diferentes}\label{operaciones-entre-tipos-diferentes}

Es posible combinar operaciones entre tipos de números diferentes. Lo
que Python hará es promover cada número al tipo más sofisticado, siendo
el orden de sofisticación, de menos a más, como sigue: \texttt{int},
\texttt{float}, \texttt{complex}.

\begin{Shaded}
\begin{Highlighting}[]
\NormalTok{a }\OperatorTok{=} \DecValTok{1} \CommentTok{\# un entero}
\NormalTok{b }\OperatorTok{=} \DecValTok{2} \OperatorTok{*} \OtherTok{3j} \CommentTok{\# un complejo}
\NormalTok{a }\OperatorTok{+}\NormalTok{ b  }\CommentTok{\# resultará en un complejo}
\end{Highlighting}
\end{Shaded}

\subsection{Operadores de asignación}\label{operadores-de-asignaciuxf3n}

Existen varios operadores para realizar asignaciones: \texttt{=},
\texttt{+=}, \texttt{-=}, \texttt{*=}, \texttt{/=}, \texttt{**=},
\texttt{\%=}. La forma de uso de estos operadores se muestra en los
siguientes ejemplos:

\begin{Shaded}
\begin{Highlighting}[]
\NormalTok{etiqueta }\OperatorTok{=} \FloatTok{1.0} 
\NormalTok{suma }\OperatorTok{=} \FloatTok{1.0}
\NormalTok{suma }\OperatorTok{+=}\NormalTok{ etiqueta  }\CommentTok{\# Equivalente a : suma = suma + etiqueta}
\BuiltInTok{print}\NormalTok{(suma)}
\end{Highlighting}
\end{Shaded}

\begin{Shaded}
\begin{Highlighting}[]
\NormalTok{etiqueta }\OperatorTok{=}  \DecValTok{4}
\NormalTok{resta }\OperatorTok{=} \DecValTok{16}
\NormalTok{resta }\OperatorTok{{-}=}\NormalTok{ etiqueta }\CommentTok{\# Equivalente a : resta = resta {-} etiqueta}
\BuiltInTok{print}\NormalTok{(resta)}
\end{Highlighting}
\end{Shaded}

\begin{Shaded}
\begin{Highlighting}[]
\NormalTok{etiqueta }\OperatorTok{=} \DecValTok{2}
\NormalTok{mult }\OperatorTok{=} \DecValTok{12}
\NormalTok{mult }\OperatorTok{*=}\NormalTok{ etiqueta  }\CommentTok{\# Equivalente a : mult = mult * etiqueta}
\BuiltInTok{print}\NormalTok{(mult)}
\end{Highlighting}
\end{Shaded}

\begin{Shaded}
\begin{Highlighting}[]
\NormalTok{etiqueta }\OperatorTok{=} \DecValTok{5}
\NormalTok{div }\OperatorTok{=} \DecValTok{50}
\NormalTok{div }\OperatorTok{/=}\NormalTok{ etiqueta  }\CommentTok{\# Equivalente a : divide = divide / etiqueta}
\BuiltInTok{print}\NormalTok{(div)}
\end{Highlighting}
\end{Shaded}

\begin{Shaded}
\begin{Highlighting}[]
\NormalTok{etiqueta }\OperatorTok{=} \DecValTok{2}
\NormalTok{pot }\OperatorTok{=} \DecValTok{3}
\NormalTok{pot }\OperatorTok{**=}\NormalTok{ etiqueta }\CommentTok{\# Equivalente a : pot = pot ** etiqueta}
\BuiltInTok{print}\NormalTok{(pot)}
\end{Highlighting}
\end{Shaded}

\begin{Shaded}
\begin{Highlighting}[]
\NormalTok{etiqueta }\OperatorTok{=} \DecValTok{5}
\NormalTok{modulo }\OperatorTok{=} \DecValTok{50}
\NormalTok{modulo }\OperatorTok{\%=}\NormalTok{ etiqueta }\CommentTok{\# Equivalente a : modulo = modulo \% etiqueta}
\BuiltInTok{print}\NormalTok{(modulo)}
\end{Highlighting}
\end{Shaded}

\section{Tipos lógicos}\label{tipos-luxf3gicos}

Es un tipo utilizado para realizar operaciones lógicas y puede tomar dos
valores: \texttt{True} o \texttt{False}.

\begin{Shaded}
\begin{Highlighting}[]
\NormalTok{bandera }\OperatorTok{=} \VariableTok{True}
\BuiltInTok{print}\NormalTok{(}\BuiltInTok{type}\NormalTok{(bandera))}
\end{Highlighting}
\end{Shaded}

\subsection{Operadores relacionales}\label{operadores-relacionales}

Cuando se aplica un operador relacional a dos expresiones, se realiza
una comparación entre dichas expresiones y se obtiene como resultado un
tipo lógico. \texttt{True} o \texttt{False}.

Los operadores relacionales que se pueden usar son:
\texttt{==},\texttt{!=},\texttt{\textless{}=},\texttt{\textgreater{}=},\texttt{\textgreater{}},\texttt{\textless{}}.
A continuación se muestran algunos ejemplos:

\begin{Shaded}
\begin{Highlighting}[]
\DecValTok{35} \OperatorTok{\textgreater{}} \DecValTok{562} \CommentTok{\# ¿Es 35 mayor que 562?}
\end{Highlighting}
\end{Shaded}

\begin{verbatim}
False
\end{verbatim}

\begin{Shaded}
\begin{Highlighting}[]
\DecValTok{32} \OperatorTok{\textgreater{}=} \DecValTok{21} \CommentTok{\# ¿Es 32 mayor o igual que 21?}
\end{Highlighting}
\end{Shaded}

\begin{verbatim}
True
\end{verbatim}

\begin{Shaded}
\begin{Highlighting}[]
\DecValTok{12} \OperatorTok{\textless{}} \DecValTok{34} \CommentTok{\# ¿Es 12 menor que 34?}
\end{Highlighting}
\end{Shaded}

\begin{verbatim}
True
\end{verbatim}

\begin{Shaded}
\begin{Highlighting}[]
\DecValTok{12} \OperatorTok{\textless{}=} \DecValTok{25} \CommentTok{\# ¿Es 12 menor o igual que 25?}
\end{Highlighting}
\end{Shaded}

\begin{verbatim}
True
\end{verbatim}

\begin{Shaded}
\begin{Highlighting}[]
\DecValTok{5} \OperatorTok{==} \DecValTok{5} \CommentTok{\# ¿Es 5 igual a 5?}
\end{Highlighting}
\end{Shaded}

\begin{verbatim}
True
\end{verbatim}

\begin{Shaded}
\begin{Highlighting}[]
\DecValTok{23} \OperatorTok{!=} \DecValTok{23} \CommentTok{\# ¿Es 23 diferente de 23?}
\end{Highlighting}
\end{Shaded}

\begin{verbatim}
False
\end{verbatim}

\begin{Shaded}
\begin{Highlighting}[]
\CommentTok{\textquotesingle{}aaa\textquotesingle{}} \OperatorTok{==} \StringTok{\textquotesingle{}aaa\textquotesingle{}} \CommentTok{\# Se pueden comparar otros tipos de datos}
\end{Highlighting}
\end{Shaded}

\begin{verbatim}
True
\end{verbatim}

\begin{Shaded}
\begin{Highlighting}[]
\DecValTok{5} \OperatorTok{\textgreater{}} \BuiltInTok{len}\NormalTok{(}\StringTok{\textquotesingle{}5\textquotesingle{}}\NormalTok{)}
\end{Highlighting}
\end{Shaded}

\begin{verbatim}
True
\end{verbatim}

\subsection{Operaciones lógicas.}\label{operaciones-luxf3gicas.}

Los operadores lógicos que se pueden usar son: \texttt{not},
\texttt{and} y \texttt{or}. Veamos algunos ejemplos

\begin{Shaded}
\begin{Highlighting}[]
\NormalTok{(}\DecValTok{5} \OperatorTok{\textless{}} \DecValTok{32}\NormalTok{) }\KeywordTok{and}\NormalTok{ (}\DecValTok{63} \OperatorTok{\textgreater{}} \DecValTok{32}\NormalTok{) }
\end{Highlighting}
\end{Shaded}

\begin{verbatim}
True
\end{verbatim}

Debido a la precedencia de operadores, no son necesarios los paréntesis
en la operaciones relacionales de la expresión anterior (véase tabla
\ldots):

\begin{Shaded}
\begin{Highlighting}[]
\DecValTok{5} \OperatorTok{\textless{}} \DecValTok{32} \KeywordTok{and} \DecValTok{63} \OperatorTok{\textgreater{}} \DecValTok{32}
\end{Highlighting}
\end{Shaded}

\begin{verbatim}
True
\end{verbatim}

Aunque a veces el uso de paréntesis hace la lectura del código más
clara:

\begin{Shaded}
\begin{Highlighting}[]
\NormalTok{(}\FloatTok{2.32} \OperatorTok{\textless{}} \DecValTok{21}\NormalTok{) }\KeywordTok{and}\NormalTok{ (}\DecValTok{23} \OperatorTok{\textgreater{}} \DecValTok{63}\NormalTok{)}
\end{Highlighting}
\end{Shaded}

\begin{verbatim}
False
\end{verbatim}

\begin{Shaded}
\begin{Highlighting}[]
\NormalTok{(}\DecValTok{32} \OperatorTok{==} \DecValTok{32}\NormalTok{) }\KeywordTok{or}\NormalTok{ (}\DecValTok{5} \OperatorTok{\textless{}} \DecValTok{31}\NormalTok{)}
\end{Highlighting}
\end{Shaded}

\begin{verbatim}
True
\end{verbatim}

\begin{Shaded}
\begin{Highlighting}[]
\NormalTok{(}\DecValTok{32} \OperatorTok{==} \DecValTok{21}\NormalTok{) }\KeywordTok{or}\NormalTok{ (}\DecValTok{31} \OperatorTok{\textless{}} \DecValTok{5}\NormalTok{) }
\end{Highlighting}
\end{Shaded}

\begin{verbatim}
False
\end{verbatim}

\begin{Shaded}
\begin{Highlighting}[]
\KeywordTok{not} \VariableTok{True}
\end{Highlighting}
\end{Shaded}

\begin{verbatim}
False
\end{verbatim}

\begin{Shaded}
\begin{Highlighting}[]
\KeywordTok{not}\NormalTok{ (}\DecValTok{32} \OperatorTok{!=} \DecValTok{32}\NormalTok{)}
\end{Highlighting}
\end{Shaded}

\begin{verbatim}
True
\end{verbatim}

\subsubsection{Comparación entre números
flotantes.}\label{comparaciuxf3n-entre-nuxfameros-flotantes.}

La comparación entre números de tipo flotante debe reaizarse con
cuidado, veamos el siguiente ejemplo:

\begin{Shaded}
\begin{Highlighting}[]
\NormalTok{(}\FloatTok{0.4} \OperatorTok{{-}} \FloatTok{0.3}\NormalTok{) }\OperatorTok{==} \FloatTok{0.1}
\end{Highlighting}
\end{Shaded}

\begin{verbatim}
False
\end{verbatim}

El cálculo a mano de \texttt{(0.4\ -0.3)} da como resultado
\texttt{0.1}; pero en una computadora este cálculo es aproximado y
depende de las características del hardware (exponente, mantisa, base,
véase ). En Python el resultado de la operación \texttt{(0.4\ -0.3)} es
diferente de \texttt{0.1}, veamos:

\begin{Shaded}
\begin{Highlighting}[]
\BuiltInTok{print}\NormalTok{(}\FloatTok{0.4} \OperatorTok{{-}}\FloatTok{0.3}\NormalTok{)}
\end{Highlighting}
\end{Shaded}

\begin{verbatim}
0.10000000000000003
\end{verbatim}

Python ofrece herramientas que permiten realizar una mejor comparación
entre números de tipo flotante. Por ejemplo la biblioteca \texttt{math}
contiene la función \texttt{isclose(a,\ b)} en donde se puede definir
una tolerancia mínima para que las dos expresiones, \texttt{a} y
\texttt{b} se consideren iguales (\emph{cercanas}), por ejemplo:

\begin{Shaded}
\begin{Highlighting}[]
\ImportTok{import}\NormalTok{ math}
\NormalTok{math.isclose((}\FloatTok{0.4} \OperatorTok{{-}} \FloatTok{0.3}\NormalTok{), }\FloatTok{0.1}\NormalTok{)}
\end{Highlighting}
\end{Shaded}

\begin{verbatim}
True
\end{verbatim}

Se recomienda revisar el manual de
\href{https://docs.python.org/3/library/math.html}{\texttt{math.isclose()}}
y el de
\href{https://numpy.org/doc/stable/reference/generated/numpy.isclose.html}{\texttt{numpy.isclose()}}para
comparación de arreglos con elementos de tipo flotante.

\section{Fuertemente Tipado.}\label{fuertemente-tipado.}

Python es fuertemente tipado, lo que significa que el tipo de un objeto
no puede cambiar repentinamente; se debe realizar una conversión
explícita si se desea cambiar el tipo de un objeto.

Esta característica también impide que se realizen operaciones entre
tipos no compatibles.

Veamos unos ejemplos:

\begin{Shaded}
\begin{Highlighting}[]
\NormalTok{lógico }\OperatorTok{=} \VariableTok{True} 
\NormalTok{real   }\OperatorTok{=} \FloatTok{220.0}  
\NormalTok{entero }\OperatorTok{=} \DecValTok{284}
\NormalTok{complejo }\OperatorTok{=} \DecValTok{1}\OperatorTok{+}\OtherTok{1j}
\NormalTok{cadena }\OperatorTok{=} \StringTok{\textquotesingle{}numeros hermanos\textquotesingle{}}
\end{Highlighting}
\end{Shaded}

\begin{Shaded}
\begin{Highlighting}[]
\NormalTok{lógico }\OperatorTok{+}\NormalTok{ real }\CommentTok{\# Los tipos son compatibles}
\end{Highlighting}
\end{Shaded}

\begin{verbatim}
221.0
\end{verbatim}

\begin{Shaded}
\begin{Highlighting}[]
\NormalTok{lógico }\OperatorTok{+}\NormalTok{ complejo }\CommentTok{\# Los tipos son compatibles}
\end{Highlighting}
\end{Shaded}

\begin{verbatim}
(2+1j)
\end{verbatim}

\begin{Shaded}
\begin{Highlighting}[]
\NormalTok{cadena }\OperatorTok{+}\NormalTok{ real  }\CommentTok{\# Los tipos no son compatibles}
\end{Highlighting}
\end{Shaded}

\begin{verbatim}
TypeError: can only concatenate str (not "float") to str
\end{verbatim}

\section{\texorpdfstring{Conversión entre tipos
(\emph{casting})}{Conversión entre tipos (casting)}}\label{conversiuxf3n-entre-tipos-casting}

Es posible transformar un tipo en otro tipo compatible; a esta operación
se lo conoce como \emph{casting}.

\subsection{\texorpdfstring{Función
\texttt{int()}}{Función int()}}\label{funciuxf3n-int}

Transforma objetos en enteros, siempre y cuando haya compatibilidad.

\begin{Shaded}
\begin{Highlighting}[]
\NormalTok{cadena }\OperatorTok{=} \StringTok{\textquotesingle{}1000\textquotesingle{}}
\BuiltInTok{print}\NormalTok{(}\BuiltInTok{type}\NormalTok{(cadena))}
\NormalTok{entero }\OperatorTok{=} \BuiltInTok{int}\NormalTok{(cadena)}
\BuiltInTok{print}\NormalTok{(}\BuiltInTok{type}\NormalTok{(entero))}
\BuiltInTok{print}\NormalTok{(entero)}
\end{Highlighting}
\end{Shaded}

\begin{verbatim}
<class 'str'>
<class 'int'>
1000
\end{verbatim}

\begin{Shaded}
\begin{Highlighting}[]
\NormalTok{flotante }\OperatorTok{=} \FloatTok{3.141592}
\NormalTok{entero  }\OperatorTok{=} \BuiltInTok{int}\NormalTok{(flotante) }\CommentTok{\# Trunca la parte decimal}
\BuiltInTok{print}\NormalTok{(entero)}
\end{Highlighting}
\end{Shaded}

\begin{verbatim}
3
\end{verbatim}

\begin{Shaded}
\begin{Highlighting}[]
\NormalTok{complejo}\OperatorTok{=} \DecValTok{4}\OperatorTok{{-}}\OtherTok{4j}
\NormalTok{entero }\OperatorTok{=} \BuiltInTok{int}\NormalTok{(complejo) }\CommentTok{\# Tipos NO COMPATIBLES}
\end{Highlighting}
\end{Shaded}

\begin{verbatim}
TypeError: int() argument must be a string, a bytes-like object or a real number, not 'complex'
\end{verbatim}

\begin{Shaded}
\begin{Highlighting}[]
\NormalTok{entero }\OperatorTok{=} \BuiltInTok{int}\NormalTok{(}\VariableTok{True}\NormalTok{) }
\BuiltInTok{print}\NormalTok{(entero)}
\end{Highlighting}
\end{Shaded}

\begin{verbatim}
1
\end{verbatim}

\begin{Shaded}
\begin{Highlighting}[]
\BuiltInTok{print}\NormalTok{(}\DecValTok{1} \OperatorTok{==} \VariableTok{True}\NormalTok{)}
\end{Highlighting}
\end{Shaded}

\begin{verbatim}
True
\end{verbatim}

Función \texttt{str()}

Transforma objetos en cadenas, siempre y cuando haya compatibilidad.

\begin{Shaded}
\begin{Highlighting}[]
\NormalTok{entero }\OperatorTok{=} \DecValTok{1000}
\BuiltInTok{print}\NormalTok{(}\BuiltInTok{type}\NormalTok{(entero))}
\NormalTok{cadena }\OperatorTok{=} \BuiltInTok{str}\NormalTok{(entero) }
\BuiltInTok{print}\NormalTok{(}\BuiltInTok{type}\NormalTok{(cadena))}
\BuiltInTok{print}\NormalTok{(cadena)}
\end{Highlighting}
\end{Shaded}

\begin{verbatim}
<class 'int'>
<class 'str'>
1000
\end{verbatim}

\begin{Shaded}
\begin{Highlighting}[]
\NormalTok{complejo }\OperatorTok{=} \DecValTok{5}\OperatorTok{+}\OtherTok{1j}
\BuiltInTok{print}\NormalTok{(complejo)}
\BuiltInTok{print}\NormalTok{(}\BuiltInTok{type}\NormalTok{(complejo))}
\NormalTok{cadena }\OperatorTok{=} \BuiltInTok{str}\NormalTok{(complejo)}
\BuiltInTok{print}\NormalTok{(cadena)}
\BuiltInTok{print}\NormalTok{(}\BuiltInTok{type}\NormalTok{(cadena))}
\end{Highlighting}
\end{Shaded}

\begin{verbatim}
(5+1j)
<class 'complex'>
(5+1j)
<class 'str'>
\end{verbatim}

Función \texttt{float()}

Transforma objetos en flotantes, siempre y cuando haya compatibilidad.

\begin{Shaded}
\begin{Highlighting}[]
\NormalTok{cadena }\OperatorTok{=} \StringTok{\textquotesingle{}3.141592\textquotesingle{}}
\BuiltInTok{print}\NormalTok{(cadena)}
\BuiltInTok{print}\NormalTok{(}\BuiltInTok{type}\NormalTok{(cadena))}
\NormalTok{real }\OperatorTok{=} \BuiltInTok{float}\NormalTok{(cadena)}
\BuiltInTok{print}\NormalTok{(real)}
\BuiltInTok{print}\NormalTok{(}\BuiltInTok{type}\NormalTok{(real))}
\end{Highlighting}
\end{Shaded}

\begin{verbatim}
3.141592
<class 'str'>
3.141592
<class 'float'>
\end{verbatim}

\begin{Shaded}
\begin{Highlighting}[]
\BuiltInTok{float}\NormalTok{(}\DecValTok{33}\NormalTok{)}
\end{Highlighting}
\end{Shaded}

\begin{verbatim}
33.0
\end{verbatim}

\begin{Shaded}
\begin{Highlighting}[]
\BuiltInTok{float}\NormalTok{(}\VariableTok{False}\NormalTok{)}
\end{Highlighting}
\end{Shaded}

\begin{verbatim}
0.0
\end{verbatim}

\begin{Shaded}
\begin{Highlighting}[]
\BuiltInTok{float}\NormalTok{(}\DecValTok{3}\OperatorTok{+}\OtherTok{3j}\NormalTok{) }\CommentTok{\# NO hay compatibilidad}
\end{Highlighting}
\end{Shaded}

\begin{verbatim}
TypeError: float() argument must be a string or a real number, not 'complex'
\end{verbatim}

En general, si existe el tipo
\texttt{\textless{}class\ \textquotesingle{}MiClase\textquotesingle{}\textgreater{}},
donde \texttt{MiClase} puede ser un tipo de dato definido dentro de
Python, alguna biblioteca o creada por el usuario, es posible realizar
el \emph{casting} del objeto \texttt{a} al tipo
\texttt{\textless{}class\ \textquotesingle{}MiClase\textquotesingle{}\textgreater{}}
haciendo : \texttt{MiClase(a)} siempre y cuando haya compatibilidad.

\section{Constantes}\label{constantes}

Python contiene una serie de constantes integradas a las que no se les
puede cambiar su valor.

Más detalles se pueden encontrar en:
\href{https://docs.python.org/3/library/constants.html}{Built-in
Constants}

Las principales constantes son las siguientes:

\begin{itemize}
\item
  \texttt{False}: de tipo Booleano.
\item
  \texttt{True}: de tipo Booleano.
\item
  \texttt{None}: El único valor para el tipo NoneType. Es usado
  frecuentemente para representar la ausencia de un valor, por ejemplo
  cuando no se pasa un argumento a una función.
\item
  \texttt{NotImplemented}: es un valor especial que es regresado por
  métodos binarios especiales (por ejemplo \texttt{\_\_eq\_\_()},
  \texttt{\_\_lt\_\_()}, \texttt{\_\_add\_\_()},
  \texttt{\_\_rsub\_\_()}, etc.) para indicar que la operación no está
  implementada con respecto a otro tipo.
\item
  \texttt{Ellipsis}: equivalente a \texttt{...}, es un valor especial
  usado mayormente en conjunción con la sintáxis de \emph{slicing} de
  arreglos.
\item
  \texttt{\_\_debug\_\_} : Esta constante es verdadera si Python no se
  inició con la opción -O.
\end{itemize}

Las siguiente constantes son usadas dentro del intérprete interactivo
(no se pueden usar dentro de programas ejecutados fuera del intérprete).

\begin{itemize}
\tightlist
\item
  \texttt{quit}(code=None)
\item
  \texttt{exit}(code=None)
\item
  \texttt{copyright}
\item
  \texttt{credits}
\item
  \texttt{license}
\end{itemize}

\begin{Shaded}
\begin{Highlighting}[]
\NormalTok{copyright()}
\end{Highlighting}
\end{Shaded}

\begin{verbatim}
Copyright (c) 2001-2023 Python Software Foundation.
All Rights Reserved.

Copyright (c) 2000 BeOpen.com.
All Rights Reserved.

Copyright (c) 1995-2001 Corporation for National Research Initiatives.
All Rights Reserved.

Copyright (c) 1991-1995 Stichting Mathematisch Centrum, Amsterdam.
All Rights Reserved.
\end{verbatim}

\begin{Shaded}
\begin{Highlighting}[]
\NormalTok{license()}
\end{Highlighting}
\end{Shaded}

\begin{verbatim}
A. HISTORY OF THE SOFTWARE
==========================

Python was created in the early 1990s by Guido van Rossum at Stichting
Mathematisch Centrum (CWI, see https://www.cwi.nl) in the Netherlands
as a successor of a language called ABC.  Guido remains Python's
principal author, although it includes many contributions from others.

In 1995, Guido continued his work on Python at the Corporation for
National Research Initiatives (CNRI, see https://www.cnri.reston.va.us)
in Reston, Virginia where he released several versions of the
software.

In May 2000, Guido and the Python core development team moved to
BeOpen.com to form the BeOpen PythonLabs team.  In October of the same
year, the PythonLabs team moved to Digital Creations, which became
Zope Corporation.  In 2001, the Python Software Foundation (PSF, see
https://www.python.org/psf/) was formed, a non-profit organization
created specifically to own Python-related Intellectual Property.
Zope Corporation was a sponsoring member of the PSF.

All Python releases are Open Source (see https://opensource.org for
the Open Source Definition).  Historically, most, but not all, Python
\end{verbatim}

\begin{verbatim}
Hit Return for more, or q (and Return) to quit:  q
\end{verbatim}

\bookmarksetup{startatroot}

\chapter{Cadenas.}\label{cadenas.}

\textbf{Objetivo.} Explicar el concepto de variable, etiqueta, objetos y
como se usan mediante algunos ejemplos.

\textbf{Funciones de Python}: - \texttt{print()}, \texttt{type()},
\texttt{id()}, \texttt{chr()}, \texttt{ord()}, \texttt{del()}

MACTI-Algebra\_Lineal\_01 by Luis M. de la Cruz is licensed under
Attribution-ShareAlike 4.0 International

\section{Definición de cadenas}\label{definiciuxf3n-de-cadenas}

Para definir una cadena se utilizan comillas simples
\texttt{\textquotesingle{}}, comillas dobles \texttt{"} o comillas
triples \texttt{"""} o
\texttt{\textquotesingle{}\textquotesingle{}\textquotesingle{}}.

\begin{Shaded}
\begin{Highlighting}[]
\NormalTok{simples }\OperatorTok{=} \StringTok{\textquotesingle{}este es un ejemplo usando }\CharTok{\textbackslash{}\textquotesingle{}}\StringTok{ }\CharTok{\textbackslash{}\textquotesingle{}}\StringTok{ \textquotesingle{}}
\BuiltInTok{print}\NormalTok{(simples)}

\NormalTok{dobles }\OperatorTok{=} \StringTok{"este es un ejemplo usando }\CharTok{\textbackslash{}"}\StringTok{ }\CharTok{\textbackslash{}"}\StringTok{ "}
\BuiltInTok{print}\NormalTok{(dobles)}

\NormalTok{triples1 }\OperatorTok{=} \StringTok{\textquotesingle{}\textquotesingle{}\textquotesingle{}este es un ejemplo usando }\CharTok{\textbackslash{}\textquotesingle{}}\StringTok{\textquotesingle{}\textquotesingle{} }\CharTok{\textbackslash{}\textquotesingle{}}\StringTok{\textquotesingle{}\textquotesingle{} \textquotesingle{}\textquotesingle{}\textquotesingle{}}
\BuiltInTok{print}\NormalTok{(triples1)}

\NormalTok{triples2 }\OperatorTok{=} \StringTok{"""este es un ejemplo usando }\CharTok{\textbackslash{}"}\StringTok{"" }\CharTok{\textbackslash{}"}\StringTok{"" """}
\BuiltInTok{print}\NormalTok{(triples2)}
\end{Highlighting}
\end{Shaded}

\begin{verbatim}
este es un ejemplo usando ' ' 
este es un ejemplo usando " " 
este es un ejemplo usando ''' ''' 
este es un ejemplo usando """ """ 
\end{verbatim}

Observa que para poder imprimir \texttt{\textquotesingle{}} dentro de
una cadena definida con \texttt{\textquotesingle{}\ \textquotesingle{}}
es necesario usar el caracter \texttt{\textbackslash{}} antes de
\texttt{\textquotesingle{}} para que se imprima correctamente. Lo mismo
sucede en los otros ejemplos.

Es posible imprimir \texttt{\textquotesingle{}} sin usar el caracter
\texttt{\textbackslash{}} si la cadena se define con \texttt{"} y
viceversa, veamos unos ejemplos:

\begin{Shaded}
\begin{Highlighting}[]
\CommentTok{\# La cadena puede tener \textquotesingle{} dentro de " ... "}
\NormalTok{poema }\OperatorTok{=} \StringTok{"Enjoy the moments now, because they don\textquotesingle{}t last forever"}
\BuiltInTok{print}\NormalTok{(poema)}
\end{Highlighting}
\end{Shaded}

\begin{verbatim}
Enjoy the moments now, because they don't last forever
\end{verbatim}

\begin{Shaded}
\begin{Highlighting}[]
\CommentTok{\# La cadena puede tener " dentro de \textquotesingle{} ... \textquotesingle{}}
\NormalTok{titulo }\OperatorTok{=} \StringTok{\textquotesingle{}Python "pythonico" \textquotesingle{}}
\BuiltInTok{print}\NormalTok{(titulo)}
\end{Highlighting}
\end{Shaded}

\begin{verbatim}
Python "pythonico" 
\end{verbatim}

\begin{Shaded}
\begin{Highlighting}[]
\CommentTok{\# La cadena puede tener " y \textquotesingle{} dentro de \textquotesingle{}\textquotesingle{}\textquotesingle{} ... \textquotesingle{}\textquotesingle{}\textquotesingle{}}
\NormalTok{queja }\OperatorTok{=} \StringTok{"""}
\StringTok{Desde muy niño}
\StringTok{tuve que "interrumpir" \textquotesingle{}mi\textquotesingle{} educación}
\StringTok{para ir a la escuela}
\StringTok{"""}
\BuiltInTok{print}\NormalTok{(queja)}
\end{Highlighting}
\end{Shaded}

\begin{verbatim}

Desde muy niño
tuve que "interrumpir" 'mi' educación
para ir a la escuela
\end{verbatim}

\begin{Shaded}
\begin{Highlighting}[]
\CommentTok{\# La cadena puede tener " y \textquotesingle{} dentro de """ ... """}
\NormalTok{queja }\OperatorTok{=} \StringTok{"""}
\StringTok{Desde muy niño}
\StringTok{tuve que "interrumpir" \textquotesingle{}mi\textquotesingle{} educación}
\StringTok{para ir a la escuela}
\StringTok{"""}
\BuiltInTok{print}\NormalTok{(queja)}
\end{Highlighting}
\end{Shaded}

\begin{verbatim}

Desde muy niño
tuve que "interrumpir" 'mi' educación
para ir a la escuela
\end{verbatim}

\section{Indexación de las
cadenas.}\label{indexaciuxf3n-de-las-cadenas.}

La indexación de las cadenas permite acceder a diferentes elementos, o
rangos de elementos, de una cadena.

\begin{itemize}
\tightlist
\item
  Todos los elementos de una cadena se numeran empezando en \texttt{0} y
  terminando en \texttt{N}, el cual representa el último elemento de la
  cadena.
\item
  También se pueden usar índices negativos donde \texttt{-1} representa
  el último elemento y \texttt{-(N+1)} el primer elemento.
\end{itemize}

Veamos la siguiente tabla:

\begin{longtable}[]{@{}lllllllllll@{}}
\toprule\noalign{}
\endhead
\bottomrule\noalign{}
\endlastfoot
cadena : & M & u & r & c & i & é & l & a & g & o \\
índice +: & 0 & 1 & 2 & 3 & 4 & 5 & 6 & 7 & 8 & 9 \\
índice -: & -10 & -9 & -8 & -7 & -6 & -5 & -4 & -3 & -2 & -1 \\
\end{longtable}

\begin{Shaded}
\begin{Highlighting}[]
\NormalTok{ejemplo }\OperatorTok{=} \StringTok{\textquotesingle{}Murciélago\textquotesingle{}}
\end{Highlighting}
\end{Shaded}

\begin{Shaded}
\begin{Highlighting}[]
\NormalTok{ejemplo[}\DecValTok{0}\NormalTok{]}
\end{Highlighting}
\end{Shaded}

\begin{verbatim}
'M'
\end{verbatim}

\begin{Shaded}
\begin{Highlighting}[]
\NormalTok{ejemplo[}\DecValTok{5}\NormalTok{]}
\end{Highlighting}
\end{Shaded}

\begin{verbatim}
'é'
\end{verbatim}

\begin{Shaded}
\begin{Highlighting}[]
\NormalTok{ejemplo[}\DecValTok{9}\NormalTok{]}
\end{Highlighting}
\end{Shaded}

\begin{verbatim}
'o'
\end{verbatim}

\begin{Shaded}
\begin{Highlighting}[]
\BuiltInTok{len}\NormalTok{(ejemplo) }\CommentTok{\# Longitud total de la cadena}
\end{Highlighting}
\end{Shaded}

\begin{verbatim}
10
\end{verbatim}

\begin{Shaded}
\begin{Highlighting}[]
\NormalTok{ejemplo[}\OperatorTok{{-}}\DecValTok{1}\NormalTok{]}
\end{Highlighting}
\end{Shaded}

\begin{verbatim}
'o'
\end{verbatim}

\begin{Shaded}
\begin{Highlighting}[]
\NormalTok{ejemplo[}\OperatorTok{{-}}\DecValTok{5}\NormalTok{]}
\end{Highlighting}
\end{Shaded}

\begin{verbatim}
'é'
\end{verbatim}

\begin{Shaded}
\begin{Highlighting}[]
\NormalTok{ejemplo[}\OperatorTok{{-}}\DecValTok{10}\NormalTok{]}
\end{Highlighting}
\end{Shaded}

\begin{verbatim}
'M'
\end{verbatim}

\section{Inmutabilidad de las
cadenas}\label{inmutabilidad-de-las-cadenas}

Los elementos de las cadenas no se pueden modificar:

\begin{Shaded}
\begin{Highlighting}[]
\NormalTok{ejemplo[}\DecValTok{5}\NormalTok{] }\OperatorTok{=} \StringTok{"e"}
\end{Highlighting}
\end{Shaded}

\begin{verbatim}
TypeError: 'str' object does not support item assignment
\end{verbatim}

\begin{Shaded}
\begin{Highlighting}[]
\NormalTok{cadena}\OperatorTok{=}\StringTok{\textquotesingle{}\textquotesingle{}\textquotesingle{}}
\StringTok{esta es una}
\StringTok{oración}
\StringTok{larga}
\StringTok{\textquotesingle{}\textquotesingle{}\textquotesingle{}}
\end{Highlighting}
\end{Shaded}

\begin{Shaded}
\begin{Highlighting}[]
\BuiltInTok{print}\NormalTok{(}\BuiltInTok{type}\NormalTok{(cadena))}
\end{Highlighting}
\end{Shaded}

\begin{verbatim}
<class 'str'>
\end{verbatim}

\begin{Shaded}
\begin{Highlighting}[]
\BuiltInTok{len}\NormalTok{(cadena)}
\end{Highlighting}
\end{Shaded}

\begin{verbatim}
27
\end{verbatim}

\begin{Shaded}
\begin{Highlighting}[]
\NormalTok{cadena[}\DecValTok{0}\NormalTok{]}
\end{Highlighting}
\end{Shaded}

\begin{verbatim}
'\n'
\end{verbatim}

\begin{Shaded}
\begin{Highlighting}[]
\NormalTok{cadena[}\OperatorTok{{-}}\DecValTok{1}\NormalTok{]}
\end{Highlighting}
\end{Shaded}

\begin{verbatim}
'\n'
\end{verbatim}

\begin{Shaded}
\begin{Highlighting}[]
\NormalTok{cadena[}\DecValTok{5}\NormalTok{] }\OperatorTok{=} \StringTok{\textquotesingle{}h\textquotesingle{}}
\end{Highlighting}
\end{Shaded}

\begin{verbatim}
TypeError: 'str' object does not support item assignment
\end{verbatim}

\section{\texorpdfstring{Acceso a porciones de las cadenas
(\emph{slicing})}{Acceso a porciones de las cadenas (slicing)}}\label{acceso-a-porciones-de-las-cadenas-slicing}

Se puede obtener una subcadena a partir de la cadena original. La
sintaxis es la siguiente:

\texttt{cadena{[}Start:End:Stride{]}}

\textbf{Start} :Índice del primer caracter para formar la subcadena.

\textbf{End} : Índice (menos uno) que indica el caracter final de la
subcadena.

\textbf{Stride}: Salto entre elementos.

\begin{Shaded}
\begin{Highlighting}[]
\NormalTok{ejemplo[:] }\CommentTok{\# Cadena completa}
\end{Highlighting}
\end{Shaded}

\begin{verbatim}
'Murciélago'
\end{verbatim}

\begin{Shaded}
\begin{Highlighting}[]
\NormalTok{ejemplo[}\DecValTok{0}\NormalTok{:}\DecValTok{5}\NormalTok{] }\CommentTok{\# Elementos del 0 al 4 }
\end{Highlighting}
\end{Shaded}

\begin{verbatim}
'Murci'
\end{verbatim}

\begin{Shaded}
\begin{Highlighting}[]
\NormalTok{ejemplo[::}\DecValTok{2}\NormalTok{] }\CommentTok{\# Todos los elementos, con saltos de 2}
\end{Highlighting}
\end{Shaded}

\begin{verbatim}
'Mrilg'
\end{verbatim}

\begin{Shaded}
\begin{Highlighting}[]
\NormalTok{ejemplo[}\DecValTok{1}\NormalTok{:}\DecValTok{8}\NormalTok{:}\DecValTok{2}\NormalTok{] }\CommentTok{\# Los elementos de 1 a 7, con saltos de 2}
\end{Highlighting}
\end{Shaded}

\begin{verbatim}
'ucéa'
\end{verbatim}

\begin{Shaded}
\begin{Highlighting}[]
\NormalTok{ejemplo[::}\OperatorTok{{-}}\DecValTok{1}\NormalTok{] }\CommentTok{\# La cadena en reversa}
\end{Highlighting}
\end{Shaded}

\begin{verbatim}
'ogaléicruM'
\end{verbatim}

\section{Operaciones básicas con
cadenas}\label{operaciones-buxe1sicas-con-cadenas}

Los operadores: \texttt{+} y \texttt{*} están definidos para las
cadenas.

\begin{Shaded}
\begin{Highlighting}[]
\CommentTok{\textquotesingle{}Luis\textquotesingle{}} \OperatorTok{+} \StringTok{\textquotesingle{} \textquotesingle{}} \OperatorTok{+} \StringTok{\textquotesingle{}Miguel\textquotesingle{}} \CommentTok{\# Concatenación}
\end{Highlighting}
\end{Shaded}

\begin{verbatim}
'Luis Miguel'
\end{verbatim}

\begin{Shaded}
\begin{Highlighting}[]
\CommentTok{\textquotesingle{}ABC\textquotesingle{}} \OperatorTok{*} \DecValTok{3} \CommentTok{\# Repetición}
\end{Highlighting}
\end{Shaded}

\begin{verbatim}
'ABCABCABC'
\end{verbatim}

\section{Funciones aplicables sobre las
cadenas}\label{funciones-aplicables-sobre-las-cadenas}

Existen métodos definidos que se pueden aplicar a las cadenas. Véase
\href{https://docs.python.org/3/library/string.html}{Common string
operations} para más información.

\begin{Shaded}
\begin{Highlighting}[]
\NormalTok{ejemplo }\OperatorTok{=} \StringTok{\textquotesingle{}murcielago\textquotesingle{}}
\end{Highlighting}
\end{Shaded}

\begin{Shaded}
\begin{Highlighting}[]
\NormalTok{ejemplo.capitalize()}
\end{Highlighting}
\end{Shaded}

\begin{verbatim}
'Murcielago'
\end{verbatim}

\begin{Shaded}
\begin{Highlighting}[]
\BuiltInTok{print}\NormalTok{(ejemplo)}
\BuiltInTok{print}\NormalTok{(ejemplo.center(}\DecValTok{20}\NormalTok{,}\StringTok{\textquotesingle{}{-}\textquotesingle{}}\NormalTok{))}
\BuiltInTok{print}\NormalTok{(ejemplo.upper())}
\BuiltInTok{print}\NormalTok{(ejemplo.find(}\StringTok{\textquotesingle{}e\textquotesingle{}}\NormalTok{))}
\BuiltInTok{print}\NormalTok{(ejemplo.count(}\StringTok{\textquotesingle{}g\textquotesingle{}}\NormalTok{))}
\BuiltInTok{print}\NormalTok{(ejemplo.isprintable())}
\end{Highlighting}
\end{Shaded}

\begin{verbatim}
murcielago
-----murcielago-----
MURCIELAGO
5
1
True
\end{verbatim}

\section{Construcción de cadenas con
variables}\label{construcciuxf3n-de-cadenas-con-variables}

\begin{Shaded}
\begin{Highlighting}[]
\NormalTok{edad }\OperatorTok{=} \DecValTok{15}
\NormalTok{nombre }\OperatorTok{=} \StringTok{\textquotesingle{}Pedro\textquotesingle{}}
\NormalTok{apellido }\OperatorTok{=} \StringTok{\textquotesingle{}Páramo\textquotesingle{}}
\NormalTok{peso }\OperatorTok{=} \FloatTok{70.5}
\end{Highlighting}
\end{Shaded}

\textbf{Concatenación y casting}.

\begin{Shaded}
\begin{Highlighting}[]
\NormalTok{datos }\OperatorTok{=}\NormalTok{ nombre }\OperatorTok{+}\NormalTok{ apellido }\OperatorTok{+} \StringTok{\textquotesingle{}tiene\textquotesingle{}} \OperatorTok{+} \BuiltInTok{str}\NormalTok{(}\DecValTok{15}\NormalTok{) }\OperatorTok{+} \StringTok{\textquotesingle{}años y pesa \textquotesingle{}} \OperatorTok{+} \BuiltInTok{str}\NormalTok{(}\FloatTok{70.5}\NormalTok{)}
\NormalTok{datos}
\end{Highlighting}
\end{Shaded}

\begin{verbatim}
'PedroPáramotiene15años y pesa 70.5'
\end{verbatim}

\textbf{Método \texttt{format()}}

\begin{Shaded}
\begin{Highlighting}[]
\NormalTok{datos }\OperatorTok{=} \StringTok{\textquotesingle{}}\SpecialCharTok{\{\}}\StringTok{ }\SpecialCharTok{\{\}}\StringTok{ tiene }\SpecialCharTok{\{\}}\StringTok{ años y pesa }\SpecialCharTok{\{\}}\StringTok{\textquotesingle{}}\NormalTok{.}\BuiltInTok{format}\NormalTok{(nombre, apellido, edad, peso)}
\NormalTok{datos}
\end{Highlighting}
\end{Shaded}

\begin{verbatim}
'Pedro Páramo tiene 15 años y pesa 70.5'
\end{verbatim}

\textbf{Cadenas formateadas (\emph{f-string}, \emph{formatted string
literals})}

\begin{Shaded}
\begin{Highlighting}[]
\NormalTok{datos }\OperatorTok{=} \SpecialStringTok{f\textquotesingle{}}\SpecialCharTok{\{}\NormalTok{nombre}\SpecialCharTok{\}}\SpecialStringTok{ }\SpecialCharTok{\{}\NormalTok{apellido}\SpecialCharTok{\}}\SpecialStringTok{ tiene }\SpecialCharTok{\{}\NormalTok{edad}\SpecialCharTok{\}}\SpecialStringTok{ años y pesa }\SpecialCharTok{\{}\NormalTok{peso}\SpecialCharTok{\}}\SpecialStringTok{\textquotesingle{}}
\NormalTok{datos}
\end{Highlighting}
\end{Shaded}

\begin{verbatim}
'Pedro Páramo tiene 15 años y pesa 70.5'
\end{verbatim}

\bookmarksetup{startatroot}

\chapter{Estructura de datos.}\label{estructura-de-datos.}

\textbf{Objetivo.} \ldots{}

\textbf{Funciones de Python}: \ldots{}

MACTI-Algebra\_Lineal\_01 by Luis M. de la Cruz is licensed under
Attribution-ShareAlike 4.0 International

\bookmarksetup{startatroot}

\chapter{Introducción}\label{introducciuxf3n-1}

Hay cuatro tipos de estructuras de datos, también conocidas como
\emph{colecciones}. La siguiente tabla resume estos cuatro tipos:

\begin{longtable}[]{@{}rcccc@{}}
\toprule\noalign{}
Tipo & Ordenada & Inmutable & Indexable & Duplicidad \\
\midrule\noalign{}
\endhead
\bottomrule\noalign{}
\endlastfoot
List & SI & NO & SI & SI \\
Tuple & SI & SI & SI & SI \\
Sets & NO & NO & NO & NO \\
Dict & NO & NO & SI & NO \\
\end{longtable}

Cuando se selecciona un tipo de colección, es importante conocer sus
propiedades para incrementar la eficiencia y/o la seguridad de los
datos.

\bookmarksetup{startatroot}

\chapter{Listas}\label{listas}

\begin{itemize}
\tightlist
\item
  Consisten en una secuencia \textbf{ordenada} y \textbf{mutable} de
  elementos.

  \begin{itemize}
  \tightlist
  \item
    Ordenadas significa que cada elemento dentro de la lista está
    indexado y mantiene su orden definido en su creación.
  \item
    Mutable significa que los elementos de la lista se pueden modificar,
    y además que se pueden agregar o eliminar elementos.
  \end{itemize}
\item
  Las listas pueden tener elementos \textbf{duplicados}, es decir,
  \textbf{elementos del mismo tipo y con el mismo contenido}.
\end{itemize}

\section{\texorpdfstring{\textbf{Ejemplo
1.}}{Ejemplo 1.}}\label{ejemplo-1.}

Creamos 4 listas:

\begin{itemize}
\tightlist
\item
  \texttt{gatos} : Razas de gatos.
\item
  \texttt{origen} : Origen de cada raza de gatos.
\item
  \texttt{pelo\_largo}: Si tienen pelo largo o no.
\item
  \texttt{pelo\_corto}: Si tienen pelo corto o no.
\item
  \texttt{peso\_minimo}: El peso mínimo que pueden tener.
\item
  \texttt{peso\_maximo}: El peso máximo que pueden tener.
\end{itemize}

\begin{Shaded}
\begin{Highlighting}[]
\CommentTok{\# Las lista se definen usando corchetes []}
\NormalTok{gatos }\OperatorTok{=}\NormalTok{ [}\StringTok{\textquotesingle{}Persa\textquotesingle{}}\NormalTok{, }\StringTok{\textquotesingle{}Sphynx\textquotesingle{}}\NormalTok{, }\StringTok{\textquotesingle{}Ragdoll\textquotesingle{}}\NormalTok{,}\StringTok{\textquotesingle{}Siamés\textquotesingle{}}\NormalTok{]}
\NormalTok{origen }\OperatorTok{=}\NormalTok{ [}\StringTok{\textquotesingle{}Irán\textquotesingle{}}\NormalTok{, }\StringTok{\textquotesingle{}Toronto\textquotesingle{}}\NormalTok{, }\StringTok{\textquotesingle{}California\textquotesingle{}}\NormalTok{, }\StringTok{\textquotesingle{}Tailandia\textquotesingle{}}\NormalTok{]}
\NormalTok{pelo\_largo }\OperatorTok{=}\NormalTok{ [}\VariableTok{True}\NormalTok{, }\VariableTok{False}\NormalTok{, }\VariableTok{True}\NormalTok{, }\VariableTok{True}\NormalTok{]}
\NormalTok{pelo\_corto }\OperatorTok{=}\NormalTok{ [}\VariableTok{False}\NormalTok{, }\VariableTok{False}\NormalTok{, }\VariableTok{False}\NormalTok{, }\VariableTok{True}\NormalTok{]}
\NormalTok{peso\_minimo }\OperatorTok{=}\NormalTok{ [}\FloatTok{2.3}\NormalTok{, }\FloatTok{3.5}\NormalTok{, }\FloatTok{5.4}\NormalTok{, }\FloatTok{2.5}\NormalTok{]}
\NormalTok{peso\_maximo }\OperatorTok{=}\NormalTok{ [}\FloatTok{6.8}\NormalTok{, }\FloatTok{7.0}\NormalTok{, }\FloatTok{9.1}\NormalTok{, }\FloatTok{4.5}\NormalTok{]}
\end{Highlighting}
\end{Shaded}

\textbf{Observaciones}: * Cada lista contiene 4 elementos. * Los
elementos de cada lista son del mismo tipo. * Los elementos son cadenas,
tipos lógicos y flotantes.

Se puede obtaner el tipo de las listas como sigue:

\begin{Shaded}
\begin{Highlighting}[]
\BuiltInTok{print}\NormalTok{(}\BuiltInTok{type}\NormalTok{(gatos))}
\end{Highlighting}
\end{Shaded}

\begin{Shaded}
\begin{Highlighting}[]
\BuiltInTok{print}\NormalTok{(gatos)}
\end{Highlighting}
\end{Shaded}

\section{Indexado}\label{indexado}

Se puede acceder a cada elemento de las listas de manera similar a como
se hace con las cadenas, veáse la notebook \ldots{}

Por ejemplo:

\begin{Shaded}
\begin{Highlighting}[]
\NormalTok{gatos[}\DecValTok{0}\NormalTok{] }\CommentTok{\# Primer elemento}
\end{Highlighting}
\end{Shaded}

\begin{Shaded}
\begin{Highlighting}[]
\NormalTok{gatos[}\DecValTok{1}\NormalTok{:}\DecValTok{4}\NormalTok{] }\CommentTok{\# Todos los elementos, desde el 1 hasta el 3}
\end{Highlighting}
\end{Shaded}

\begin{Shaded}
\begin{Highlighting}[]
\NormalTok{gatos[}\OperatorTok{{-}}\DecValTok{1}\NormalTok{] }\CommentTok{\# Último elemento}
\end{Highlighting}
\end{Shaded}

\begin{Shaded}
\begin{Highlighting}[]
\NormalTok{gatos[::}\OperatorTok{{-}}\DecValTok{1}\NormalTok{] }\CommentTok{\# Todos los elementos en reversa}
\end{Highlighting}
\end{Shaded}

Para conocer el tipo de objeto de uno de los elementos podemos hacer lo
siguiente:

\begin{Shaded}
\begin{Highlighting}[]
\BuiltInTok{print}\NormalTok{(}\BuiltInTok{type}\NormalTok{(gatos[}\DecValTok{0}\NormalTok{]))}
\end{Highlighting}
\end{Shaded}

\begin{Shaded}
\begin{Highlighting}[]
\BuiltInTok{print}\NormalTok{(}\BuiltInTok{type}\NormalTok{(peso\_maximo[}\DecValTok{2}\NormalTok{]))}
\end{Highlighting}
\end{Shaded}

\section{Operaciones sobre las
listas}\label{operaciones-sobre-las-listas}

Existen muchas operaciones que se pueden realizar sobre las listas. A
continuación se muestran unos ejemplos

\begin{Shaded}
\begin{Highlighting}[]
\BuiltInTok{len}\NormalTok{(gatos) }\CommentTok{\# Determinar la longitud de la lista}
\end{Highlighting}
\end{Shaded}

\begin{Shaded}
\begin{Highlighting}[]
\BuiltInTok{max}\NormalTok{(gatos) }\CommentTok{\# Determinar el máximo elemento de la lista}
\end{Highlighting}
\end{Shaded}

\begin{Shaded}
\begin{Highlighting}[]
\BuiltInTok{min}\NormalTok{(gatos) }\CommentTok{\# Determinar el mínimo elemento de la lista}
\end{Highlighting}
\end{Shaded}

\begin{Shaded}
\begin{Highlighting}[]
\CommentTok{\# Operación lógica elemento a elemento.}
\CommentTok{\# Produce una lista con elementos lógicos.}
\NormalTok{sin\_pelo }\OperatorTok{=}\NormalTok{ pelo\_largo }\KeywordTok{or}\NormalTok{ pelo\_corto}
\BuiltInTok{print}\NormalTok{(sin\_pelo)}
\end{Highlighting}
\end{Shaded}

\begin{Shaded}
\begin{Highlighting}[]
\NormalTok{gatos }\OperatorTok{+}\NormalTok{ peso\_maximo }\CommentTok{\# Concatenación de dos listas}
\end{Highlighting}
\end{Shaded}

\begin{Shaded}
\begin{Highlighting}[]
\NormalTok{origen }\OperatorTok{*} \DecValTok{2} \CommentTok{\# Duplicación de la lista, intenta multiplicar por 3}
\end{Highlighting}
\end{Shaded}

\begin{Shaded}
\begin{Highlighting}[]
\CommentTok{\textquotesingle{}Siamés\textquotesingle{}} \KeywordTok{in}\NormalTok{ gatos }\CommentTok{\# ¿Está el elemento \textasciigrave{}Siamés\textasciigrave{} en la lista gatos?}
\end{Highlighting}
\end{Shaded}

\section{Métodos de las listas
(comportamiento)}\label{muxe9todos-de-las-listas-comportamiento}

En términos de Programación Orientada a Objetos, la clase
\texttt{\textless{}class\ \textquotesingle{}list\textquotesingle{}\textgreater{}}
define una serie de métodos que se pueden aplicar sobre los objetos del
tipo \texttt{list}. Veamos algunos ejemplos:

\begin{Shaded}
\begin{Highlighting}[]
\BuiltInTok{print}\NormalTok{(gatos) }\CommentTok{\# Imprimimos la lista original}
\end{Highlighting}
\end{Shaded}

\begin{Shaded}
\begin{Highlighting}[]
\NormalTok{gatos.append(}\StringTok{\textquotesingle{}Siberiano\textquotesingle{}}\NormalTok{) }\CommentTok{\# Se agrega un elemento al final de la lista}
\end{Highlighting}
\end{Shaded}

\begin{Shaded}
\begin{Highlighting}[]
\BuiltInTok{print}\NormalTok{(gatos)}
\end{Highlighting}
\end{Shaded}

\begin{Shaded}
\begin{Highlighting}[]
\NormalTok{gatos.append(}\StringTok{\textquotesingle{}Persa\textquotesingle{}}\NormalTok{) }\CommentTok{\# Se agrega otro elemento al final de la lista, repetido}
\end{Highlighting}
\end{Shaded}

\begin{Shaded}
\begin{Highlighting}[]
\BuiltInTok{print}\NormalTok{(gatos)}
\end{Highlighting}
\end{Shaded}

\begin{Shaded}
\begin{Highlighting}[]
\NormalTok{gatos.remove(}\StringTok{\textquotesingle{}Persa\textquotesingle{}}\NormalTok{) }\CommentTok{\# Eliminamos el elemento \textquotesingle{}Persa\textquotesingle{} de la lista}
\end{Highlighting}
\end{Shaded}

\begin{Shaded}
\begin{Highlighting}[]
\BuiltInTok{print}\NormalTok{(gatos) }
\end{Highlighting}
\end{Shaded}

Observa que solo se elimina el primer elemento `Persa' que encuentra.

\begin{Shaded}
\begin{Highlighting}[]
\NormalTok{gatos.insert(}\DecValTok{0}\NormalTok{,}\StringTok{\textquotesingle{}Persa\textquotesingle{}}\NormalTok{) }\CommentTok{\# Podemos insertar un elemento en un lugar específico de la lista}
\end{Highlighting}
\end{Shaded}

\begin{Shaded}
\begin{Highlighting}[]
\BuiltInTok{print}\NormalTok{(gatos)}
\end{Highlighting}
\end{Shaded}

\begin{Shaded}
\begin{Highlighting}[]
\NormalTok{gatos.pop() }\CommentTok{\# Extrae el último elemento de la lista}
\end{Highlighting}
\end{Shaded}

\begin{Shaded}
\begin{Highlighting}[]
\BuiltInTok{print}\NormalTok{(gatos)}
\end{Highlighting}
\end{Shaded}

\begin{Shaded}
\begin{Highlighting}[]
\NormalTok{gatos.sort() }\CommentTok{\# Ordena la lista}
\end{Highlighting}
\end{Shaded}

\begin{Shaded}
\begin{Highlighting}[]
\BuiltInTok{print}\NormalTok{(gatos)}
\end{Highlighting}
\end{Shaded}

\begin{Shaded}
\begin{Highlighting}[]
\NormalTok{gatos.reverse() }\CommentTok{\# Modifca la lista con los elementos en reversa}
\end{Highlighting}
\end{Shaded}

\begin{Shaded}
\begin{Highlighting}[]
\BuiltInTok{print}\NormalTok{(gatos)}
\end{Highlighting}
\end{Shaded}

Una descripción detallada de los métodos de la listas se puede ver en
\href{https://docs.python.org/3/tutorial/datastructures.html\#more-on-lists}{More
on Lists}.

\section{Copiando listas}\label{copiando-listas}

Una lista es un objeto que contiene varios elementos. Para crear una
copia de una lista, se debe generar un espacio de memoria en donde se
copien todos los elementos de la lista original y asignar un nuevo
nombre para esta nueva lista. Lo anterior no se puede hacer simplemente
con el operador de asignación. Veamos un ejemplo:

\begin{Shaded}
\begin{Highlighting}[]
\NormalTok{gatitos }\OperatorTok{=}\NormalTok{ gatos}
\end{Highlighting}
\end{Shaded}

\begin{Shaded}
\begin{Highlighting}[]
\BuiltInTok{print}\NormalTok{(gatos)}
\BuiltInTok{print}\NormalTok{(gatitos)}
\end{Highlighting}
\end{Shaded}

Podemos observar que al imprimir la lista mediante los nombres
\texttt{gatos} y \texttt{gatitos} obtenemos el mismo resultado. Ahora,
modifiquemos el primer elemento usando el nombre \texttt{gatitos}:

\begin{Shaded}
\begin{Highlighting}[]
\NormalTok{gatitos[}\DecValTok{0}\NormalTok{] }\OperatorTok{=} \StringTok{\textquotesingle{}Singapur\textquotesingle{}}
\end{Highlighting}
\end{Shaded}

\begin{Shaded}
\begin{Highlighting}[]
\BuiltInTok{print}\NormalTok{(gatos)}
\BuiltInTok{print}\NormalTok{(gatitos)}
\end{Highlighting}
\end{Shaded}

Observamos que al imprimir la lista usando \texttt{gatos} y
\texttt{gatitos} volvemos a obtener el mismo resultado. Lo anterior
significa que el operador de asignación solamente creó un nuevo nombre
para el mismo objeto en memoria, por lo que en realidad no hizo una
copia de la lista. Lo anterior lo podemos verificar usando la función
\texttt{id()} para ver la dirección en memoria del objeto:

\begin{Shaded}
\begin{Highlighting}[]
\BuiltInTok{print}\NormalTok{(}\BuiltInTok{id}\NormalTok{(gatitos))}
\BuiltInTok{print}\NormalTok{(}\BuiltInTok{id}\NormalTok{(gatos))}
\end{Highlighting}
\end{Shaded}

\subsection{\texorpdfstring{Copiando con
\texttt{{[}:{]}}}{Copiando con {[}:{]}}}\label{copiando-con}

Crear una nueva lista copiando todos los elementos podemos hacer lo
siguiente:

\begin{Shaded}
\begin{Highlighting}[]
\NormalTok{gatitos }\OperatorTok{=}\NormalTok{ gatos[:]}
\end{Highlighting}
\end{Shaded}

\begin{Shaded}
\begin{Highlighting}[]
\BuiltInTok{print}\NormalTok{(}\BuiltInTok{type}\NormalTok{(gatitos))}
\BuiltInTok{print}\NormalTok{(}\BuiltInTok{id}\NormalTok{(gatitos))}
\BuiltInTok{print}\NormalTok{(gatitos)}

\BuiltInTok{print}\NormalTok{(}\BuiltInTok{type}\NormalTok{(gatos))}
\BuiltInTok{print}\NormalTok{(}\BuiltInTok{id}\NormalTok{(gatos))}
\BuiltInTok{print}\NormalTok{(gatos)}
\end{Highlighting}
\end{Shaded}

Observa que el identificador en memoria de cada lista es diferente.

\subsection{\texorpdfstring{Copiando con el método
\texttt{copy()}}{Copiando con el método copy()}}\label{copiando-con-el-muxe9todo-copy}

La clase
\texttt{\textless{}class\ \textquotesingle{}list\textquotesingle{}\textgreater{}}
contiene un método llamado \texttt{copy()} que efectivamente realiza una
copia de la lista:

\begin{Shaded}
\begin{Highlighting}[]
\NormalTok{gatitos }\OperatorTok{=}\NormalTok{ gatos.copy()}
\end{Highlighting}
\end{Shaded}

\begin{Shaded}
\begin{Highlighting}[]
\BuiltInTok{print}\NormalTok{(}\BuiltInTok{type}\NormalTok{(gatitos))}
\BuiltInTok{print}\NormalTok{(}\BuiltInTok{id}\NormalTok{(gatitos))}
\BuiltInTok{print}\NormalTok{(gatitos)}

\BuiltInTok{print}\NormalTok{(}\BuiltInTok{type}\NormalTok{(gatos))}
\BuiltInTok{print}\NormalTok{(}\BuiltInTok{id}\NormalTok{(gatos))}
\BuiltInTok{print}\NormalTok{(gatos)}
\end{Highlighting}
\end{Shaded}

Observa que el identificador en memoria de cada lista es diferente.

\subsection{Copiando con el
constructor}\label{copiando-con-el-constructor}

La función \texttt{list()} transforma un objeto \emph{iterable} en una
lista. La podemos usar para copiar una lista como sigue:

\begin{Shaded}
\begin{Highlighting}[]
\NormalTok{gatitos }\OperatorTok{=} \BuiltInTok{list}\NormalTok{(gatos)}
\end{Highlighting}
\end{Shaded}

\begin{Shaded}
\begin{Highlighting}[]
\BuiltInTok{print}\NormalTok{(}\BuiltInTok{type}\NormalTok{(gatitos))}
\BuiltInTok{print}\NormalTok{(}\BuiltInTok{id}\NormalTok{(gatitos))}
\BuiltInTok{print}\NormalTok{(gatitos)}

\BuiltInTok{print}\NormalTok{(}\BuiltInTok{type}\NormalTok{(gatos))}
\BuiltInTok{print}\NormalTok{(}\BuiltInTok{id}\NormalTok{(gatos))}
\BuiltInTok{print}\NormalTok{(gatos)}
\end{Highlighting}
\end{Shaded}

Observa que el identificador en memoria de cada lista es diferente.

\textbf{NOTA}. Lo que sucede en este último caso, es que se ejecuta el
constructor de la clase
\texttt{\textless{}class\ \textquotesingle{}list\textquotesingle{}\textgreater{}},
el cual recibe un objeto iterable (lista, tupla, diccionario, entre
otros), copia todos los elementos de ese iterable y los pone en una
lista que se almacena en un espacio en memoria diferente al iterable
original.

\subsection{\texorpdfstring{Copiando con la biblioteca
\texttt{copy}}{Copiando con la biblioteca copy}}\label{copiando-con-la-biblioteca-copy}

\begin{Shaded}
\begin{Highlighting}[]
\ImportTok{import}\NormalTok{ copy}
\NormalTok{gatitos }\OperatorTok{=}\NormalTok{ copy.copy(gatos)}
\end{Highlighting}
\end{Shaded}

\begin{Shaded}
\begin{Highlighting}[]
\BuiltInTok{print}\NormalTok{(}\BuiltInTok{type}\NormalTok{(gatitos))}
\BuiltInTok{print}\NormalTok{(}\BuiltInTok{id}\NormalTok{(gatitos))}
\BuiltInTok{print}\NormalTok{(gatitos)}

\BuiltInTok{print}\NormalTok{(}\BuiltInTok{type}\NormalTok{(gatos))}
\BuiltInTok{print}\NormalTok{(}\BuiltInTok{id}\NormalTok{(gatos))}
\BuiltInTok{print}\NormalTok{(gatos)}
\end{Highlighting}
\end{Shaded}

Observa que el identificador en memoria de cada lista es diferente.

Más información sobre el uso de esta biblioteca se puede ver en
\href{https://docs.python.org/3/library/copy.html}{Shallow and deep copy
operations}.

\section{Listas mas complejas.}\label{listas-mas-complejas.}

Las listas pueden tener elementos de distintos tipos. Por ejemplo:

\begin{Shaded}
\begin{Highlighting}[]
\NormalTok{superlista }\OperatorTok{=}\NormalTok{ [}\StringTok{\textquotesingle{}México\textquotesingle{}}\NormalTok{, }\FloatTok{3.141592}\NormalTok{, }\DecValTok{20}\NormalTok{, }\OtherTok{1j}\NormalTok{, [}\DecValTok{1}\NormalTok{,}\DecValTok{2}\NormalTok{,}\DecValTok{3}\NormalTok{,}\StringTok{\textquotesingle{}lista\textquotesingle{}}\NormalTok{]]}
\end{Highlighting}
\end{Shaded}

\begin{Shaded}
\begin{Highlighting}[]
\NormalTok{superlista}
\end{Highlighting}
\end{Shaded}

\begin{Shaded}
\begin{Highlighting}[]
\NormalTok{superlista[}\DecValTok{0}\NormalTok{] }\CommentTok{\# El elemento 0 de la lista}
\end{Highlighting}
\end{Shaded}

\begin{Shaded}
\begin{Highlighting}[]
\NormalTok{superlista[}\DecValTok{4}\NormalTok{] }\CommentTok{\# El elemento 4 de la lista (este elemento es otra lista)}
\end{Highlighting}
\end{Shaded}

\begin{Shaded}
\begin{Highlighting}[]
\NormalTok{superlista[}\DecValTok{4}\NormalTok{][}\DecValTok{2}\NormalTok{] }\CommentTok{\# El elemento 2 del elemento 4 de la lista original}
\end{Highlighting}
\end{Shaded}

\bookmarksetup{startatroot}

\chapter{Tuplas}\label{tuplas}

\begin{itemize}
\tightlist
\item
  Consisten en una secuencia \textbf{ordenada} e \textbf{inmutable} de
  elementos.

  \begin{itemize}
  \tightlist
  \item
    Ordenadas significa que cada elemento dentro de la tupla está
    indexado y mantiene su orden definido en su creación.
  \item
    Inmutable significa que los elementos de la tupla \textbf{NO se
    pueden modificar}, tampoco que se pueden agregar o eliminar
    elementos.
  \end{itemize}
\item
  Las tuplas pueden tener elementos \textbf{duplicados}, es decir,
  \textbf{elementos del mismo tipo y con el mismo contenido}.
\end{itemize}

Veamos algunos ejemplos:

\begin{Shaded}
\begin{Highlighting}[]
\CommentTok{\# Las tuplas se definen usando paréntesis ()}
\NormalTok{tupla1 }\OperatorTok{=}\NormalTok{ () }\CommentTok{\# tupla vacía}

\BuiltInTok{print}\NormalTok{(}\BuiltInTok{type}\NormalTok{(tupla1))}
\BuiltInTok{print}\NormalTok{(}\BuiltInTok{id}\NormalTok{(tupla1))}
\BuiltInTok{print}\NormalTok{(tupla1)}
\end{Highlighting}
\end{Shaded}

La clase
\texttt{\textless{}class\ \textquotesingle{}tuple\textquotesingle{}\textgreater{}}
solo contiene dos funciones: * \texttt{index(o)}, determina el índice
dentro de la tupla del objeto \texttt{o}. * \texttt{count(o)}, determina
el número de objetos iguales a \texttt{o} existen dentro de la tupla.

\begin{Shaded}
\begin{Highlighting}[]
\NormalTok{tupla }\OperatorTok{=}\NormalTok{ (}\StringTok{\textquotesingle{}a\textquotesingle{}}\NormalTok{, }\StringTok{\textquotesingle{}b\textquotesingle{}}\NormalTok{, }\StringTok{\textquotesingle{}c\textquotesingle{}}\NormalTok{, }\StringTok{\textquotesingle{}b\textquotesingle{}}\NormalTok{, }\StringTok{\textquotesingle{}d\textquotesingle{}}\NormalTok{, }\StringTok{\textquotesingle{}e\textquotesingle{}}\NormalTok{, }\StringTok{\textquotesingle{}f\textquotesingle{}}\NormalTok{, }\StringTok{\textquotesingle{}b\textquotesingle{}}\NormalTok{)}
\BuiltInTok{print}\NormalTok{(tupla)}
\end{Highlighting}
\end{Shaded}

\begin{Shaded}
\begin{Highlighting}[]
\NormalTok{tupla.index(}\StringTok{\textquotesingle{}a\textquotesingle{}}\NormalTok{)}
\end{Highlighting}
\end{Shaded}

\begin{Shaded}
\begin{Highlighting}[]
\NormalTok{tupla.count(}\StringTok{\textquotesingle{}b\textquotesingle{}}\NormalTok{)}
\end{Highlighting}
\end{Shaded}

Si deseamos una tupla de un solo elemento debemos realizar lo siguiente:

\begin{Shaded}
\begin{Highlighting}[]
\NormalTok{tupla\_1 }\OperatorTok{=}\NormalTok{ (}\DecValTok{1}\NormalTok{,)}
\BuiltInTok{print}\NormalTok{(}\BuiltInTok{type}\NormalTok{(tupla\_1))}
\BuiltInTok{print}\NormalTok{(tupla\_1)}
\end{Highlighting}
\end{Shaded}

La siguiente expresión no construye una tupla, si no un entero:

\begin{Shaded}
\begin{Highlighting}[]
\NormalTok{tupla\_1 }\OperatorTok{=}\NormalTok{ (}\DecValTok{1}\NormalTok{)}
\BuiltInTok{print}\NormalTok{(}\BuiltInTok{type}\NormalTok{(tupla\_1))}
\BuiltInTok{print}\NormalTok{(tupla\_1)}
\end{Highlighting}
\end{Shaded}

\section{Indexado.}\label{indexado.}

El indexado de las tuplas es similar al de las listas.

\begin{Shaded}
\begin{Highlighting}[]
\BuiltInTok{print}\NormalTok{(tupla)}
\end{Highlighting}
\end{Shaded}

\begin{Shaded}
\begin{Highlighting}[]
\NormalTok{tupla[}\DecValTok{0}\NormalTok{]}
\end{Highlighting}
\end{Shaded}

\begin{Shaded}
\begin{Highlighting}[]
\NormalTok{tupla[}\OperatorTok{{-}}\DecValTok{1}\NormalTok{]}
\end{Highlighting}
\end{Shaded}

\begin{Shaded}
\begin{Highlighting}[]
\NormalTok{tupla[}\DecValTok{2}\NormalTok{:}\DecValTok{5}\NormalTok{]}
\end{Highlighting}
\end{Shaded}

\begin{Shaded}
\begin{Highlighting}[]
\NormalTok{tupla[::}\OperatorTok{{-}}\DecValTok{1}\NormalTok{]}
\end{Highlighting}
\end{Shaded}

\section{Inmutabilidad}\label{inmutabilidad}

Los elementos de las tuplas no se pueden modificar.

\begin{Shaded}
\begin{Highlighting}[]
\NormalTok{tupla[}\DecValTok{2}\NormalTok{]}
\end{Highlighting}
\end{Shaded}

\begin{Shaded}
\begin{Highlighting}[]
\NormalTok{tupla[}\DecValTok{2}\NormalTok{] }\OperatorTok{=} \StringTok{\textquotesingle{}h\textquotesingle{}}
\end{Highlighting}
\end{Shaded}

\section{¿Copiando tuplas?}\label{copiando-tuplas}

No es posible crear una copia de una tupla en otra. Lo que se recomienda
es transformar la tupla en otra estructura de datos compatible (por
ejemplo \texttt{list} o \texttt{set}).

\bookmarksetup{startatroot}

\chapter{Conjuntos}\label{conjuntos}

\begin{itemize}
\tightlist
\item
  Consisten en una secuencia \textbf{NO ordenada}, \textbf{modificable},
  \textbf{NO indexable} y \textbf{NO} permite miembros duplicados.
\end{itemize}

Veamos algunos ejemplos:

\begin{Shaded}
\begin{Highlighting}[]
\CommentTok{\# Los conjuntos se definen con \{\}}
\NormalTok{conjunto }\OperatorTok{=}\NormalTok{ \{}\DecValTok{4}\NormalTok{,}\DecValTok{1}\NormalTok{,}\DecValTok{8}\NormalTok{,}\DecValTok{0}\NormalTok{,}\DecValTok{4}\NormalTok{,}\DecValTok{20}\NormalTok{\}}
\end{Highlighting}
\end{Shaded}

\begin{Shaded}
\begin{Highlighting}[]
\BuiltInTok{print}\NormalTok{(}\BuiltInTok{type}\NormalTok{(conjunto))}
\BuiltInTok{print}\NormalTok{(}\BuiltInTok{id}\NormalTok{(conjunto))}
\BuiltInTok{print}\NormalTok{(conjunto)}
\end{Highlighting}
\end{Shaded}

\section{Funciones y operaciones sobre
conjuntos}\label{funciones-y-operaciones-sobre-conjuntos}

\begin{Shaded}
\begin{Highlighting}[]
\CommentTok{\# Adiccionar un elemento}
\NormalTok{conjunto.add(}\OperatorTok{{-}}\DecValTok{8}\NormalTok{)}
\BuiltInTok{print}\NormalTok{(conjunto)}
\end{Highlighting}
\end{Shaded}

\begin{Shaded}
\begin{Highlighting}[]
\CommentTok{\# Eliminar un elemento del conjunto (el elemento debe existir dentro del conjunto)}
\NormalTok{conjunto.remove(}\DecValTok{4}\NormalTok{)}
\BuiltInTok{print}\NormalTok{(conjunto)}
\end{Highlighting}
\end{Shaded}

\begin{Shaded}
\begin{Highlighting}[]
\CommentTok{\# ¿El elemento 0 está en el conjunto?}
\DecValTok{0} \KeywordTok{in}\NormalTok{ conjunto}
\end{Highlighting}
\end{Shaded}

\begin{Shaded}
\begin{Highlighting}[]
\CommentTok{\# ¿El elemento 10 está en el conjunto?}
\DecValTok{10} \KeywordTok{in}\NormalTok{ conjunto}
\end{Highlighting}
\end{Shaded}

\begin{Shaded}
\begin{Highlighting}[]
\CommentTok{\# Limpiar todos los elementos del conjunto}
\NormalTok{conjunto.clear()}
\BuiltInTok{print}\NormalTok{(}\BuiltInTok{type}\NormalTok{(conjunto))}
\BuiltInTok{print}\NormalTok{(}\BuiltInTok{id}\NormalTok{(conjunto))}
\BuiltInTok{print}\NormalTok{(conjunto)}
\end{Highlighting}
\end{Shaded}

Observa que el identificador no ha cambiado, solo se eliminaron todos
los elementos.

\begin{Shaded}
\begin{Highlighting}[]
\CommentTok{\# Definimos dos conjuntos}
\NormalTok{A }\OperatorTok{=}\NormalTok{ \{}\StringTok{\textquotesingle{}Taza\textquotesingle{}}\NormalTok{, }\StringTok{\textquotesingle{}Vaso\textquotesingle{}}\NormalTok{, }\StringTok{\textquotesingle{}Mesa\textquotesingle{}}\NormalTok{\}}
\NormalTok{B }\OperatorTok{=}\NormalTok{ \{}\StringTok{\textquotesingle{}Casa\textquotesingle{}}\NormalTok{, }\StringTok{\textquotesingle{}Mesa\textquotesingle{}}\NormalTok{, }\StringTok{\textquotesingle{}Silla\textquotesingle{}}\NormalTok{\}}
\end{Highlighting}
\end{Shaded}

\begin{Shaded}
\begin{Highlighting}[]
\BuiltInTok{print}\NormalTok{(A)}
\BuiltInTok{print}\NormalTok{(B)}
\end{Highlighting}
\end{Shaded}

\begin{Shaded}
\begin{Highlighting}[]
\NormalTok{A }\OperatorTok{{-}}\NormalTok{ B }\CommentTok{\# elementos en A, pero no en B}
\end{Highlighting}
\end{Shaded}

\begin{Shaded}
\begin{Highlighting}[]
\NormalTok{A }\OperatorTok{|}\NormalTok{ B }\CommentTok{\# elementos en A o en B o en ambos}
\end{Highlighting}
\end{Shaded}

\begin{Shaded}
\begin{Highlighting}[]
\NormalTok{A }\OperatorTok{\&}\NormalTok{ B }\CommentTok{\# elementos en ambos conjuntos}
\end{Highlighting}
\end{Shaded}

\begin{Shaded}
\begin{Highlighting}[]
\NormalTok{A }\OperatorTok{\^{}}\NormalTok{ B }\CommentTok{\# elementos en A o en B, pero no en ambos}
\end{Highlighting}
\end{Shaded}

\section{Copiando conjuntos}\label{copiando-conjuntos}

\begin{Shaded}
\begin{Highlighting}[]
\NormalTok{conjunto }\OperatorTok{=}\NormalTok{ \{}\DecValTok{4}\NormalTok{,}\DecValTok{1}\NormalTok{,}\DecValTok{8}\NormalTok{,}\DecValTok{0}\NormalTok{,}\DecValTok{4}\NormalTok{,}\DecValTok{20}\NormalTok{\}}
\end{Highlighting}
\end{Shaded}

\subsection{\texorpdfstring{Copiando con el método
\texttt{copy()}}{Copiando con el método copy()}}\label{copiando-con-el-muxe9todo-copy-1}

\begin{Shaded}
\begin{Highlighting}[]
\CommentTok{\# Crear otro conjunto haciendo una copia}
\NormalTok{conjunto2 }\OperatorTok{=}\NormalTok{ conjunto.copy()}

\BuiltInTok{print}\NormalTok{(}\BuiltInTok{type}\NormalTok{(conjunto2))}
\BuiltInTok{print}\NormalTok{(}\BuiltInTok{id}\NormalTok{(conjunto2))}
\BuiltInTok{print}\NormalTok{(conjunto2)}

\BuiltInTok{print}\NormalTok{(}\BuiltInTok{type}\NormalTok{(conjunto))}
\BuiltInTok{print}\NormalTok{(}\BuiltInTok{id}\NormalTok{(conjunto))}
\BuiltInTok{print}\NormalTok{(conjunto)}
\end{Highlighting}
\end{Shaded}

\subsection{\texorpdfstring{Copiando con la biblioteca
\texttt{copy()}}{Copiando con la biblioteca copy()}}\label{copiando-con-la-biblioteca-copy-1}

\begin{Shaded}
\begin{Highlighting}[]
\ImportTok{import}\NormalTok{ copy}
\NormalTok{conjunto2 }\OperatorTok{=}\NormalTok{ conjunto.copy()}

\BuiltInTok{print}\NormalTok{(}\BuiltInTok{type}\NormalTok{(conjunto2))}
\BuiltInTok{print}\NormalTok{(}\BuiltInTok{id}\NormalTok{(conjunto2))}
\BuiltInTok{print}\NormalTok{(conjunto2)}

\BuiltInTok{print}\NormalTok{(}\BuiltInTok{type}\NormalTok{(conjunto))}
\BuiltInTok{print}\NormalTok{(}\BuiltInTok{id}\NormalTok{(conjunto))}
\BuiltInTok{print}\NormalTok{(conjunto)}
\end{Highlighting}
\end{Shaded}

\bookmarksetup{startatroot}

\chapter{Diccionarios}\label{diccionarios}

\begin{itemize}
\tightlist
\item
  Diccionarios son colecciones que \textbf{NO} son ordenadas, son
  \textbf{modificables}, \textbf{indexables} y \textbf{NO} permiten
  miembros duplicados.
\item
  Las colecciones están compuesta por pares \texttt{clave:valor}.
\item
  Se accede a los valores mediante las claves en lugar de índices.
\end{itemize}

Veamos algunos ejemplos:

\begin{Shaded}
\begin{Highlighting}[]
\NormalTok{dicc }\OperatorTok{=}\NormalTok{ \{}\StringTok{\textquotesingle{}Luis\textquotesingle{}}\NormalTok{: }\DecValTok{20}\NormalTok{, }\StringTok{\textquotesingle{}Miguel\textquotesingle{}}\NormalTok{: }\DecValTok{25}\NormalTok{\}}
\end{Highlighting}
\end{Shaded}

\begin{Shaded}
\begin{Highlighting}[]
\BuiltInTok{print}\NormalTok{(}\BuiltInTok{type}\NormalTok{(dicc))}
\BuiltInTok{print}\NormalTok{(dicc)}
\end{Highlighting}
\end{Shaded}

\section{Operaciones sobre
diccionarios}\label{operaciones-sobre-diccionarios}

\begin{Shaded}
\begin{Highlighting}[]
\NormalTok{dicc[}\StringTok{\textquotesingle{}Luis\textquotesingle{}}\NormalTok{] }\CommentTok{\# acceder a un elemento del diccionario}
\end{Highlighting}
\end{Shaded}

Se puede acceder a las claves y a los valores de manera independiente
como sigue:

\begin{Shaded}
\begin{Highlighting}[]
\NormalTok{dicc.keys() }\CommentTok{\# Obtener todas las claves}
\end{Highlighting}
\end{Shaded}

\begin{Shaded}
\begin{Highlighting}[]
\NormalTok{dicc.values() }\CommentTok{\# Obtener todos los valores}
\end{Highlighting}
\end{Shaded}

\begin{Shaded}
\begin{Highlighting}[]
\CommentTok{\# ¿Existe el elemento \textquotesingle{}Miguel\textquotesingle{} en el diccionario?}
\CommentTok{\textquotesingle{}Miguel\textquotesingle{}} \KeywordTok{in}\NormalTok{ dicc}
\end{Highlighting}
\end{Shaded}

\begin{Shaded}
\begin{Highlighting}[]
\CommentTok{\# ¿Existe el valor 25 en los valores del diccionario?}
\DecValTok{25} \KeywordTok{in}\NormalTok{ dicc.values()}
\end{Highlighting}
\end{Shaded}

\begin{Shaded}
\begin{Highlighting}[]
\CommentTok{\# ¿Existe la clave \textquotesingle{}Luis\textquotesingle{} en las claves del diccionario?}
\CommentTok{\textquotesingle{}Luis\textquotesingle{}} \KeywordTok{in}\NormalTok{ dicc.keys()}
\end{Highlighting}
\end{Shaded}

\begin{Shaded}
\begin{Highlighting}[]
\BuiltInTok{len}\NormalTok{(dicc) }\CommentTok{\# Calcular la longitud (el número de parejas)}
\end{Highlighting}
\end{Shaded}

\begin{Shaded}
\begin{Highlighting}[]
\NormalTok{dicc[}\StringTok{\textquotesingle{}fulano\textquotesingle{}}\NormalTok{] }\OperatorTok{=} \DecValTok{100} \CommentTok{\# Agregar el par \textasciigrave{}\textquotesingle{}fulano\textquotesingle{}:100\textasciigrave{}}
\end{Highlighting}
\end{Shaded}

\begin{Shaded}
\begin{Highlighting}[]
\BuiltInTok{print}\NormalTok{(dicc)}
\end{Highlighting}
\end{Shaded}

Observa que cuando el elemento no existe, la expresión
\texttt{dicc{[}\textquotesingle{}fulano\textquotesingle{}{]}\ =\ 100}
agrega el par \texttt{\textquotesingle{}fulano\textquotesingle{}:\ 100}
al diccionario.

\begin{Shaded}
\begin{Highlighting}[]
\KeywordTok{del}\NormalTok{ dicc[}\StringTok{\textquotesingle{}Miguel\textquotesingle{}}\NormalTok{] }\CommentTok{\# Eliminar el par \textasciigrave{}\textquotesingle{}Miguel\textquotesingle{}:25\textasciigrave{}}
\end{Highlighting}
\end{Shaded}

\begin{Shaded}
\begin{Highlighting}[]
\BuiltInTok{print}\NormalTok{(dicc)}
\end{Highlighting}
\end{Shaded}

Podemos agregar un diccionario en otro:

\begin{Shaded}
\begin{Highlighting}[]
\NormalTok{dicc\_otro }\OperatorTok{=}\NormalTok{ \{}\StringTok{\textquotesingle{}nuevo\textquotesingle{}}\NormalTok{:}\StringTok{\textquotesingle{}estrellas\textquotesingle{}}\NormalTok{, }\StringTok{\textquotesingle{}viejo\textquotesingle{}}\NormalTok{:}\StringTok{\textquotesingle{}cosmos\textquotesingle{}}\NormalTok{, }\StringTok{\textquotesingle{}edad\textquotesingle{}}\NormalTok{:}\DecValTok{15000000}\NormalTok{\}}
\end{Highlighting}
\end{Shaded}

\begin{Shaded}
\begin{Highlighting}[]
\BuiltInTok{print}\NormalTok{(dicc\_otro)}
\end{Highlighting}
\end{Shaded}

\begin{Shaded}
\begin{Highlighting}[]
\NormalTok{dicc.update(dicc\_otro) }\CommentTok{\# Agregamos el diccionario dicc\_otro al diccionario dicc}
\end{Highlighting}
\end{Shaded}

\begin{Shaded}
\begin{Highlighting}[]
\BuiltInTok{print}\NormalTok{(dicc)}
\end{Highlighting}
\end{Shaded}

Si los elementos ya existen, solo se actualizan los valores:

\begin{Shaded}
\begin{Highlighting}[]
\NormalTok{nuevo }\OperatorTok{=}\NormalTok{ \{}\StringTok{\textquotesingle{}Luis\textquotesingle{}}\NormalTok{:}\DecValTok{512}\NormalTok{, }\StringTok{\textquotesingle{}viejo\textquotesingle{}}\NormalTok{:}\FloatTok{2.1}\NormalTok{\}}
\end{Highlighting}
\end{Shaded}

\begin{Shaded}
\begin{Highlighting}[]
\NormalTok{dicc.update(nuevo)}
\end{Highlighting}
\end{Shaded}

\begin{Shaded}
\begin{Highlighting}[]
\BuiltInTok{print}\NormalTok{(dicc)}
\end{Highlighting}
\end{Shaded}

\section{Copiando diccionarios}\label{copiando-diccionarios}

\begin{Shaded}
\begin{Highlighting}[]
\BuiltInTok{print}\NormalTok{(dicc)}
\end{Highlighting}
\end{Shaded}

\subsection{\texorpdfstring{Copiando con el método
\texttt{copy()}}{Copiando con el método copy()}}\label{copiando-con-el-muxe9todo-copy-2}

\begin{Shaded}
\begin{Highlighting}[]
\NormalTok{dicc\_cp }\OperatorTok{=}\NormalTok{ dicc.copy()}

\BuiltInTok{print}\NormalTok{(}\BuiltInTok{type}\NormalTok{(dicc\_cp))}
\BuiltInTok{print}\NormalTok{(}\BuiltInTok{id}\NormalTok{(dicc\_cp))}
\BuiltInTok{print}\NormalTok{(dicc\_cp)}

\BuiltInTok{print}\NormalTok{(}\BuiltInTok{type}\NormalTok{(dicc))}
\BuiltInTok{print}\NormalTok{(}\BuiltInTok{id}\NormalTok{(dicc))}
\BuiltInTok{print}\NormalTok{(dicc)}
\end{Highlighting}
\end{Shaded}

\subsection{\texorpdfstring{Copiando con la biblioteca
\texttt{copy()}}{Copiando con la biblioteca copy()}}\label{copiando-con-la-biblioteca-copy-2}

\begin{Shaded}
\begin{Highlighting}[]
\ImportTok{import}\NormalTok{ copy}
\NormalTok{dicc\_cp }\OperatorTok{=}\NormalTok{ dicc.copy()}

\BuiltInTok{print}\NormalTok{(}\BuiltInTok{type}\NormalTok{(dicc\_cp))}
\BuiltInTok{print}\NormalTok{(}\BuiltInTok{id}\NormalTok{(dicc\_cp))}
\BuiltInTok{print}\NormalTok{(dicc\_cp)}

\BuiltInTok{print}\NormalTok{(}\BuiltInTok{type}\NormalTok{(dicc))}
\BuiltInTok{print}\NormalTok{(}\BuiltInTok{id}\NormalTok{(dicc))}
\BuiltInTok{print}\NormalTok{(dicc)}
\end{Highlighting}
\end{Shaded}

\bookmarksetup{startatroot}

\chapter{Transformación entre
colecciones}\label{transformaciuxf3n-entre-colecciones}

\subsection{\texorpdfstring{listas \(\to\)
tuplas}{listas \textbackslash to tuplas}}\label{listas-to-tuplas}

\begin{Shaded}
\begin{Highlighting}[]
\NormalTok{lista }\OperatorTok{=}\NormalTok{ [}\StringTok{\textquotesingle{}a\textquotesingle{}}\NormalTok{, }\StringTok{\textquotesingle{}b\textquotesingle{}}\NormalTok{, }\StringTok{\textquotesingle{}c\textquotesingle{}}\NormalTok{]}
\end{Highlighting}
\end{Shaded}

\begin{Shaded}
\begin{Highlighting}[]
\NormalTok{tupla\_l }\OperatorTok{=} \BuiltInTok{tuple}\NormalTok{(lista)}

\BuiltInTok{print}\NormalTok{(}\BuiltInTok{type}\NormalTok{(tupla\_l))}
\BuiltInTok{print}\NormalTok{(}\BuiltInTok{id}\NormalTok{(tupla\_l))}
\BuiltInTok{print}\NormalTok{(tupla\_l)}

\BuiltInTok{print}\NormalTok{(}\BuiltInTok{type}\NormalTok{(lista))}
\BuiltInTok{print}\NormalTok{(}\BuiltInTok{id}\NormalTok{(lista))}
\BuiltInTok{print}\NormalTok{(lista)}
\end{Highlighting}
\end{Shaded}

\begin{verbatim}
<class 'tuple'>
2215532231808
('a', 'b', 'c')
<class 'list'>
2215532334976
['a', 'b', 'c']
\end{verbatim}

\subsection{\texorpdfstring{listas \(\to\)
conjuntos}{listas \textbackslash to conjuntos}}\label{listas-to-conjuntos}

\begin{Shaded}
\begin{Highlighting}[]
\NormalTok{lista }\OperatorTok{=}\NormalTok{ [}\StringTok{\textquotesingle{}a\textquotesingle{}}\NormalTok{, }\StringTok{\textquotesingle{}b\textquotesingle{}}\NormalTok{, }\StringTok{\textquotesingle{}c\textquotesingle{}}\NormalTok{, }\StringTok{\textquotesingle{}a\textquotesingle{}}\NormalTok{]}
\end{Highlighting}
\end{Shaded}

\begin{Shaded}
\begin{Highlighting}[]
\NormalTok{conj\_l }\OperatorTok{=} \BuiltInTok{set}\NormalTok{(lista)}

\BuiltInTok{print}\NormalTok{(}\BuiltInTok{type}\NormalTok{(conj\_l))}
\BuiltInTok{print}\NormalTok{(}\BuiltInTok{id}\NormalTok{(conj\_l))}
\BuiltInTok{print}\NormalTok{(conj\_l)}

\BuiltInTok{print}\NormalTok{(}\BuiltInTok{type}\NormalTok{(lista))}
\BuiltInTok{print}\NormalTok{(}\BuiltInTok{id}\NormalTok{(lista))}
\BuiltInTok{print}\NormalTok{(lista)}
\end{Highlighting}
\end{Shaded}

\begin{verbatim}
<class 'set'>
2215531882848
{'c', 'a', 'b'}
<class 'list'>
2215532336832
['a', 'b', 'c', 'a']
\end{verbatim}

Observa que en esta transformación el conjunto elimina los elementos
repetidos.

\subsection{\texorpdfstring{listas \(\to\)
diccionarios}{listas \textbackslash to diccionarios}}\label{listas-to-diccionarios}

\begin{Shaded}
\begin{Highlighting}[]
\NormalTok{key\_l }\OperatorTok{=}\NormalTok{ [}\StringTok{\textquotesingle{}a\textquotesingle{}}\NormalTok{,}\StringTok{\textquotesingle{}b\textquotesingle{}}\NormalTok{,}\StringTok{\textquotesingle{}c\textquotesingle{}}\NormalTok{,}\StringTok{\textquotesingle{}d\textquotesingle{}}\NormalTok{, }\StringTok{\textquotesingle{}e\textquotesingle{}}\NormalTok{]}
\NormalTok{val\_l }\OperatorTok{=}\NormalTok{ [}\DecValTok{1}\NormalTok{, }\DecValTok{2}\NormalTok{, }\DecValTok{3}\NormalTok{, }\DecValTok{4}\NormalTok{, }\DecValTok{5}\NormalTok{]}
\end{Highlighting}
\end{Shaded}

\begin{Shaded}
\begin{Highlighting}[]
\NormalTok{dicc }\OperatorTok{=} \BuiltInTok{dict}\NormalTok{(}\BuiltInTok{zip}\NormalTok{(key\_l, val\_l))}

\BuiltInTok{print}\NormalTok{(key\_l)}
\BuiltInTok{print}\NormalTok{(val\_l)}

\BuiltInTok{print}\NormalTok{(}\BuiltInTok{type}\NormalTok{(dicc))}
\BuiltInTok{print}\NormalTok{(}\BuiltInTok{id}\NormalTok{(dicc))}
\BuiltInTok{print}\NormalTok{(dicc)}
\end{Highlighting}
\end{Shaded}

\begin{verbatim}
['a', 'b', 'c', 'd', 'e']
[1, 2, 3, 4, 5]
<class 'dict'>
2215509691520
{'a': 1, 'b': 2, 'c': 3, 'd': 4, 'e': 5}
\end{verbatim}

\subsection{\texorpdfstring{tuplas \(\to\)
listas}{tuplas \textbackslash to listas}}\label{tuplas-to-listas}

\begin{Shaded}
\begin{Highlighting}[]
\NormalTok{tupla }\OperatorTok{=}\NormalTok{ (}\DecValTok{1}\NormalTok{,}\DecValTok{2}\NormalTok{,}\DecValTok{3}\NormalTok{,}\DecValTok{4}\NormalTok{)}
\end{Highlighting}
\end{Shaded}

\begin{Shaded}
\begin{Highlighting}[]
\NormalTok{lista\_t }\OperatorTok{=} \BuiltInTok{list}\NormalTok{(tupla)}

\BuiltInTok{print}\NormalTok{(}\BuiltInTok{type}\NormalTok{(lista\_t))}
\BuiltInTok{print}\NormalTok{(}\BuiltInTok{id}\NormalTok{(lista\_t))}
\BuiltInTok{print}\NormalTok{(lista\_t)}

\BuiltInTok{print}\NormalTok{(}\BuiltInTok{type}\NormalTok{(tupla))}
\BuiltInTok{print}\NormalTok{(}\BuiltInTok{id}\NormalTok{(tupla))}
\BuiltInTok{print}\NormalTok{(tupla)}
\end{Highlighting}
\end{Shaded}

\begin{verbatim}
<class 'list'>
2215537704832
[1, 2, 3, 4]
<class 'tuple'>
2215537784000
(1, 2, 3, 4)
\end{verbatim}

\subsection{\texorpdfstring{tuplas \(\to\)
conjuntos}{tuplas \textbackslash to conjuntos}}\label{tuplas-to-conjuntos}

\begin{Shaded}
\begin{Highlighting}[]
\NormalTok{tupla }\OperatorTok{=}\NormalTok{ (}\DecValTok{1}\NormalTok{,}\DecValTok{2}\NormalTok{,}\DecValTok{3}\NormalTok{,}\DecValTok{1}\NormalTok{,}\DecValTok{2}\NormalTok{)}
\end{Highlighting}
\end{Shaded}

\begin{Shaded}
\begin{Highlighting}[]
\NormalTok{conj\_t }\OperatorTok{=} \BuiltInTok{set}\NormalTok{(tupla)}

\BuiltInTok{print}\NormalTok{(}\BuiltInTok{type}\NormalTok{(conj\_t))}
\BuiltInTok{print}\NormalTok{(}\BuiltInTok{id}\NormalTok{(conj\_t))}
\BuiltInTok{print}\NormalTok{(conj\_t)}

\BuiltInTok{print}\NormalTok{(}\BuiltInTok{type}\NormalTok{(tupla))}
\BuiltInTok{print}\NormalTok{(}\BuiltInTok{id}\NormalTok{(tupla))}
\BuiltInTok{print}\NormalTok{(tupla)}
\end{Highlighting}
\end{Shaded}

\begin{verbatim}
<class 'set'>
2215537722528
{1, 2, 3}
<class 'tuple'>
2215537786320
(1, 2, 3, 1, 2)
\end{verbatim}

\subsection{\texorpdfstring{tuplas \(\to\)
diccionarios}{tuplas \textbackslash to diccionarios}}\label{tuplas-to-diccionarios}

\begin{Shaded}
\begin{Highlighting}[]
\NormalTok{key\_t }\OperatorTok{=}\NormalTok{ (}\StringTok{\textquotesingle{}a\textquotesingle{}}\NormalTok{,}\StringTok{\textquotesingle{}b\textquotesingle{}}\NormalTok{,}\StringTok{\textquotesingle{}c\textquotesingle{}}\NormalTok{,}\StringTok{\textquotesingle{}d\textquotesingle{}}\NormalTok{, }\StringTok{\textquotesingle{}e\textquotesingle{}}\NormalTok{)}
\NormalTok{val\_t }\OperatorTok{=}\NormalTok{ (}\DecValTok{1}\NormalTok{, }\DecValTok{2}\NormalTok{, }\DecValTok{3}\NormalTok{, }\DecValTok{4}\NormalTok{, }\DecValTok{5}\NormalTok{)}
\end{Highlighting}
\end{Shaded}

\begin{Shaded}
\begin{Highlighting}[]
\NormalTok{dicc }\OperatorTok{=} \BuiltInTok{dict}\NormalTok{(}\BuiltInTok{zip}\NormalTok{(key\_t, val\_t))}

\BuiltInTok{print}\NormalTok{(key\_t)}
\BuiltInTok{print}\NormalTok{(val\_t)}

\BuiltInTok{print}\NormalTok{(}\BuiltInTok{type}\NormalTok{(dicc))}
\BuiltInTok{print}\NormalTok{(}\BuiltInTok{id}\NormalTok{(dicc))}
\BuiltInTok{print}\NormalTok{(dicc)}
\end{Highlighting}
\end{Shaded}

\begin{verbatim}
('a', 'b', 'c', 'd', 'e')
(1, 2, 3, 4, 5)
<class 'dict'>
2215537751808
{'a': 1, 'b': 2, 'c': 3, 'd': 4, 'e': 5}
\end{verbatim}

\subsection{\texorpdfstring{conjunto \(\to\) lista y
tupla}{conjunto \textbackslash to lista y tupla}}\label{conjunto-to-lista-y-tupla}

\begin{Shaded}
\begin{Highlighting}[]
\NormalTok{conj }\OperatorTok{=}\NormalTok{ \{}\DecValTok{1}\NormalTok{,}\DecValTok{2}\NormalTok{,}\DecValTok{3}\NormalTok{,}\StringTok{\textquotesingle{}a\textquotesingle{}}\NormalTok{,}\StringTok{\textquotesingle{}b\textquotesingle{}}\NormalTok{,}\StringTok{\textquotesingle{}c\textquotesingle{}}\NormalTok{\}}
\end{Highlighting}
\end{Shaded}

\begin{Shaded}
\begin{Highlighting}[]
\NormalTok{lista\_s }\OperatorTok{=} \BuiltInTok{list}\NormalTok{(conj)}
\NormalTok{tupla\_s }\OperatorTok{=} \BuiltInTok{tuple}\NormalTok{(conj)}
\end{Highlighting}
\end{Shaded}

\begin{Shaded}
\begin{Highlighting}[]
\BuiltInTok{print}\NormalTok{(}\BuiltInTok{type}\NormalTok{(conj))}
\BuiltInTok{print}\NormalTok{(}\BuiltInTok{id}\NormalTok{(conj))}
\BuiltInTok{print}\NormalTok{(conj)}

\BuiltInTok{print}\NormalTok{(}\BuiltInTok{type}\NormalTok{(lista\_s))}
\BuiltInTok{print}\NormalTok{(}\BuiltInTok{id}\NormalTok{(lista\_s))}
\BuiltInTok{print}\NormalTok{(lista\_s)}

\BuiltInTok{print}\NormalTok{(}\BuiltInTok{type}\NormalTok{(tupla\_s))}
\BuiltInTok{print}\NormalTok{(}\BuiltInTok{id}\NormalTok{(tupla\_s))}
\BuiltInTok{print}\NormalTok{(tupla\_s)}
\end{Highlighting}
\end{Shaded}

\begin{verbatim}
<class 'set'>
2215537723424
{1, 2, 3, 'b', 'c', 'a'}
<class 'list'>
2215537788864
[1, 2, 3, 'b', 'c', 'a']
<class 'tuple'>
2215534718752
(1, 2, 3, 'b', 'c', 'a')
\end{verbatim}

\subsection{\texorpdfstring{conjuntos \(\to\)
diccionarios}{conjuntos \textbackslash to diccionarios}}\label{conjuntos-to-diccionarios}

\begin{Shaded}
\begin{Highlighting}[]
\NormalTok{conj1 }\OperatorTok{=}\NormalTok{ \{}\StringTok{\textquotesingle{}a\textquotesingle{}}\NormalTok{,}\StringTok{\textquotesingle{}b\textquotesingle{}}\NormalTok{,}\StringTok{\textquotesingle{}c\textquotesingle{}}\NormalTok{\}}
\NormalTok{conj2 }\OperatorTok{=}\NormalTok{ \{}\DecValTok{1}\NormalTok{,}\DecValTok{2}\NormalTok{,}\DecValTok{3}\NormalTok{\}}

\NormalTok{dicc }\OperatorTok{=} \BuiltInTok{dict}\NormalTok{(}\BuiltInTok{zip}\NormalTok{(conj1, conj2))}

\BuiltInTok{print}\NormalTok{(conj1)}
\BuiltInTok{print}\NormalTok{(conj2)}

\BuiltInTok{print}\NormalTok{(}\BuiltInTok{type}\NormalTok{(dicc))}
\BuiltInTok{print}\NormalTok{(}\BuiltInTok{id}\NormalTok{(dicc))}
\BuiltInTok{print}\NormalTok{(dicc)}
\end{Highlighting}
\end{Shaded}

\begin{verbatim}
{'c', 'a', 'b'}
{1, 2, 3}
<class 'dict'>
2215537993856
{'c': 1, 'a': 2, 'b': 3}
\end{verbatim}

\subsection{\texorpdfstring{diccionarios \(\to\) listas, tuplas y
conjuntos}{diccionarios \textbackslash to listas, tuplas y conjuntos}}\label{diccionarios-to-listas-tuplas-y-conjuntos}

\begin{Shaded}
\begin{Highlighting}[]
\NormalTok{dicc }\OperatorTok{=}\NormalTok{ \{}\StringTok{\textquotesingle{}x\textquotesingle{}}\NormalTok{:}\DecValTok{3}\NormalTok{, }\StringTok{\textquotesingle{}y\textquotesingle{}}\NormalTok{:}\DecValTok{4}\NormalTok{, }\StringTok{\textquotesingle{}z\textquotesingle{}}\NormalTok{:}\DecValTok{5}\NormalTok{\}}
\end{Highlighting}
\end{Shaded}

Conversión directa:

\begin{Shaded}
\begin{Highlighting}[]
\NormalTok{lista }\OperatorTok{=} \BuiltInTok{list}\NormalTok{(dicc)}
\NormalTok{tupla }\OperatorTok{=} \BuiltInTok{tuple}\NormalTok{(dicc)}
\NormalTok{conj }\OperatorTok{=} \BuiltInTok{set}\NormalTok{(dicc)}
\end{Highlighting}
\end{Shaded}

\begin{Shaded}
\begin{Highlighting}[]
\BuiltInTok{print}\NormalTok{(dicc)}

\BuiltInTok{print}\NormalTok{(}\BuiltInTok{type}\NormalTok{(lista))}
\BuiltInTok{print}\NormalTok{(lista)}
\BuiltInTok{print}\NormalTok{(}\BuiltInTok{type}\NormalTok{(tupla))}
\BuiltInTok{print}\NormalTok{(tupla)}
\BuiltInTok{print}\NormalTok{(}\BuiltInTok{type}\NormalTok{(conj))}
\BuiltInTok{print}\NormalTok{(conj)}
\end{Highlighting}
\end{Shaded}

\begin{verbatim}
{'x': 3, 'y': 4, 'z': 5}
<class 'list'>
['x', 'y', 'z']
<class 'tuple'>
('x', 'y', 'z')
<class 'set'>
{'y', 'x', 'z'}
\end{verbatim}

Conversión desde las claves:

\begin{Shaded}
\begin{Highlighting}[]
\NormalTok{lista }\OperatorTok{=} \BuiltInTok{list}\NormalTok{(dicc.keys())}
\NormalTok{tupla }\OperatorTok{=} \BuiltInTok{tuple}\NormalTok{(dicc.keys())}
\NormalTok{conj }\OperatorTok{=} \BuiltInTok{set}\NormalTok{(dicc.keys())}
\end{Highlighting}
\end{Shaded}

\begin{Shaded}
\begin{Highlighting}[]
\BuiltInTok{print}\NormalTok{(dicc)}

\BuiltInTok{print}\NormalTok{(}\BuiltInTok{type}\NormalTok{(lista))}
\BuiltInTok{print}\NormalTok{(lista)}
\BuiltInTok{print}\NormalTok{(}\BuiltInTok{type}\NormalTok{(tupla))}
\BuiltInTok{print}\NormalTok{(tupla)}
\BuiltInTok{print}\NormalTok{(}\BuiltInTok{type}\NormalTok{(conj))}
\BuiltInTok{print}\NormalTok{(conj)}
\end{Highlighting}
\end{Shaded}

\begin{verbatim}
{'x': 3, 'y': 4, 'z': 5}
<class 'list'>
['x', 'y', 'z']
<class 'tuple'>
('x', 'y', 'z')
<class 'set'>
{'y', 'x', 'z'}
\end{verbatim}

Conversión desde las valores:

\begin{Shaded}
\begin{Highlighting}[]
\NormalTok{lista }\OperatorTok{=} \BuiltInTok{list}\NormalTok{(dicc.values())}
\NormalTok{tupla }\OperatorTok{=} \BuiltInTok{tuple}\NormalTok{(dicc.values())}
\NormalTok{conj }\OperatorTok{=} \BuiltInTok{set}\NormalTok{(dicc.values())}
\end{Highlighting}
\end{Shaded}

\begin{Shaded}
\begin{Highlighting}[]
\BuiltInTok{print}\NormalTok{(dicc)}

\BuiltInTok{print}\NormalTok{(}\BuiltInTok{type}\NormalTok{(lista))}
\BuiltInTok{print}\NormalTok{(lista)}
\BuiltInTok{print}\NormalTok{(}\BuiltInTok{type}\NormalTok{(tupla))}
\BuiltInTok{print}\NormalTok{(tupla)}
\BuiltInTok{print}\NormalTok{(}\BuiltInTok{type}\NormalTok{(conj))}
\BuiltInTok{print}\NormalTok{(conj)}
\end{Highlighting}
\end{Shaded}

\begin{verbatim}
{'x': 3, 'y': 4, 'z': 5}
<class 'list'>
[3, 4, 5]
<class 'tuple'>
(3, 4, 5)
<class 'set'>
{3, 4, 5}
\end{verbatim}

\bookmarksetup{startatroot}

\chapter{Control de flujo.}\label{control-de-flujo.}

\textbf{Objetivo.} \ldots{}

\textbf{Funciones de Python}: \ldots{}

MACTI-Algebra\_Lineal\_01 by Luis M. de la Cruz is licensed under
Attribution-ShareAlike 4.0 International

En Python existen declaraciones que permiten controlar el flujo de un
programa para realizar acciones complejas. Entre estas declaraciones
tenemos las siguientes:

\begin{itemize}
\tightlist
\item
  \texttt{while}
\item
  \texttt{for}
\item
  \texttt{if}
\item
  \texttt{match}
\end{itemize}

Junto con estas declaraciones generalmente se utilizan las siguientes
operaciones lógicas cuyo resultado puede ser \texttt{True} o
\texttt{False}:

\begin{longtable}[]{@{}
  >{\raggedright\arraybackslash}p{(\columnwidth - 2\tabcolsep) * \real{0.5000}}
  >{\raggedright\arraybackslash}p{(\columnwidth - 2\tabcolsep) * \real{0.5000}}@{}}
\toprule\noalign{}
\begin{minipage}[b]{\linewidth}\raggedright
Python
\end{minipage} & \begin{minipage}[b]{\linewidth}\raggedright
Significado
\end{minipage} \\
\midrule\noalign{}
\endhead
\bottomrule\noalign{}
\endlastfoot
\texttt{a\ ==\ b} & ¿son iguales \texttt{a} y \texttt{b}? \\
\texttt{a\ !=\ b} & ¿son diferentes \texttt{a} y \texttt{b}? \\
\texttt{a\ \textless{}\ b} & ¿\texttt{a} es menor que \texttt{b}?: \\
\texttt{a\ \textless{}=\ b} & ¿\texttt{a} es menor o igual que
\texttt{b}? \\
\texttt{a\ \textgreater{}\ b} & ¿\texttt{a} es mayor que \texttt{b}? \\
\texttt{a\ \textgreater{}=\ b} & ¿\texttt{a} es mayor o igual que
\texttt{b}? \\
\textbf{\texttt{not}} \texttt{A} & El inverso de la expresión
\texttt{A} \\
\texttt{A} \textbf{\texttt{and}} \texttt{B} & ¿La expresión \texttt{A} y
la expresión \texttt{B} son verdaderas? \\
\texttt{A} \textbf{\texttt{or}} \texttt{B} & ¿La expresión \texttt{A} o
la expresión \texttt{B} es verdadera?: \\
\end{longtable}

\bookmarksetup{startatroot}

\chapter{\texorpdfstring{\texttt{while}}{while}}\label{while}

Se utiliza para repetir un conjunto de instrucciones mientras una
expresión sea verdadera:

\begin{Shaded}
\begin{Highlighting}[]
\ControlFlowTok{while}\NormalTok{ expresión:}
\NormalTok{    código ...}
\end{Highlighting}
\end{Shaded}

Por ejemplo:

\begin{Shaded}
\begin{Highlighting}[]
\NormalTok{a }\OperatorTok{=} \DecValTok{0} \CommentTok{\# Inicializamos a en 0}

\BuiltInTok{print}\NormalTok{(}\StringTok{\textquotesingle{}Inicia while\textquotesingle{}}\NormalTok{) }\CommentTok{\# Instrucción fuera del bloque while}

\ControlFlowTok{while}\NormalTok{ a }\OperatorTok{\textless{}} \DecValTok{5}\NormalTok{: }\CommentTok{\# Mientras a sea menor que 5 realiza lo siguiente:}
    \BuiltInTok{print}\NormalTok{(a) }\CommentTok{\# Imprime el valor de a}
\NormalTok{    a }\OperatorTok{+=} \DecValTok{1}   \CommentTok{\# Incrementa el valor de a en 1}
    
\BuiltInTok{print}\NormalTok{(}\StringTok{\textquotesingle{}Finaliza while\textquotesingle{}}\NormalTok{)  }\CommentTok{\# Instrucción fuera del bloque while}
\end{Highlighting}
\end{Shaded}

\begin{itemize}
\item
  Como se observa, el código después de \texttt{while} tiene una sangría
  (\emph{indentation}): las líneas de código están recorridas hacia la
  derecha.
\item
  Este espacio en blanco debe ser al menos de uno, pero pueden ser más.
\item
  Por omisión, en JupyterLab (y algunos otros editores, se usan 4
  espacios en blanco para cada línea de código dentro del bloque.
\item
  El número de espacios en blanco se debe mantener durante todo el
  bloque de código.
\item
  Cuando termina el sangrado, es decir las líneas de código ya no tienen
  ningún espacio en blanco al inicio, se cierra el bloque de código, en
  este caso el \texttt{while}.
\item
  El uso de una sangría para organizar los bloques de código lo hace
  Python para que el código sea más entendible.
\item
  \textbf{Ejemplos válidos}:
\end{itemize}

\begin{Shaded}
\begin{Highlighting}[]
\ControlFlowTok{while}\NormalTok{ a }\OperatorTok{\textless{}} \DecValTok{5}\NormalTok{: }
 \BuiltInTok{print}\NormalTok{(a)}
\NormalTok{ a }\OperatorTok{+=} \DecValTok{1}
\end{Highlighting}
\end{Shaded}

\begin{Shaded}
\begin{Highlighting}[]
\ControlFlowTok{while}\NormalTok{ a }\OperatorTok{\textless{}} \DecValTok{5}\NormalTok{: }
        \BuiltInTok{print}\NormalTok{(a)}
\NormalTok{        a }\OperatorTok{+=} \DecValTok{1}
\end{Highlighting}
\end{Shaded}

\begin{itemize}
\tightlist
\item
  \textbf{Ejemplos NO válidos}
\end{itemize}

\begin{Shaded}
\begin{Highlighting}[]
\ControlFlowTok{while}\NormalTok{ a }\OperatorTok{\textless{}} \DecValTok{5}\NormalTok{: }
 \BuiltInTok{print}\NormalTok{(a)}
\NormalTok{   a }\OperatorTok{+=} \DecValTok{1}
\end{Highlighting}
\end{Shaded}

\begin{Shaded}
\begin{Highlighting}[]
\ControlFlowTok{while}\NormalTok{ a }\OperatorTok{\textless{}} \DecValTok{5}\NormalTok{: }
\BuiltInTok{print}\NormalTok{(a)}
\NormalTok{a }\OperatorTok{+=} \DecValTok{1}
\end{Highlighting}
\end{Shaded}

\section{\texorpdfstring{\textbf{Ejemplo
1.}}{Ejemplo 1.}}\label{ejemplo-1.-1}

Los número de Fibonacci, denotados con \(F_n\) forman una secuencia tal
que cada número es la suma de dos números precedentes e inicia con el 0
y el 1. Matemáticamente se escribe como:

\[
\begin{eqnarray}
F_0 & = & 0 \\
F_1 & = & 1 \\
F_n & = & F_{n − 1} + F_{n − 2} \;\;\; \text{para} \; n > 1.
\end{eqnarray}
\] La secuencia es entonces: 0 , 1 , 1 , 2 , 3 , 5 , 8 , 13 , 21 , 34 ,
55 , 89 , 144 , \(\ldots\)

Vamos a calcular esta secuencia usando la instrucción \texttt{while}:

\begin{Shaded}
\begin{Highlighting}[]
\NormalTok{a, b }\OperatorTok{=} \DecValTok{0}\NormalTok{, }\DecValTok{1} \CommentTok{\# Definimos los primeros dos elementos:}

\ControlFlowTok{while}\NormalTok{ a }\OperatorTok{\textless{}} \DecValTok{1000}\NormalTok{:       }\CommentTok{\# Mientras a sea menor que 1000 realiza lo siguiente:}
    \BuiltInTok{print}\NormalTok{(a, end}\OperatorTok{=}\StringTok{\textquotesingle{}, \textquotesingle{}}\NormalTok{)   }\CommentTok{\# Imprime a y b (separados por una coma)}
\NormalTok{    a, b }\OperatorTok{=}\NormalTok{ b, a}\OperatorTok{+}\NormalTok{b       }\CommentTok{\# Calcula los siguientes dos elementos}
\end{Highlighting}
\end{Shaded}

\bookmarksetup{startatroot}

\chapter{\texorpdfstring{\texttt{if}, \texttt{elif},
\texttt{else}}{if, elif, else}}\label{if-elif-else}

Esta declaración permite ejecutar un código dependiendo del resultado de
una o varias expresiones lógicas. La estructura es como sigue:

\begin{Shaded}
\begin{Highlighting}[]
\ControlFlowTok{if}\NormalTok{ expresion1:}
\NormalTok{    codigo1 ...}
\ControlFlowTok{elif}\NormalTok{ expresion2:}
\NormalTok{    codigo2 ...}
\ControlFlowTok{elif}\NormalTok{ expresion3:}
\NormalTok{    codigo3 ...}
\ControlFlowTok{else}\NormalTok{:}
\NormalTok{    codigo4}
\end{Highlighting}
\end{Shaded}

Si la \texttt{expresion1} es verdadera, entonces se ejecuta el
\texttt{codigo1}, en otro caso se evalúan las siguientes expresiones y
dependiendo de cuál es verdadera se ejecuta el código correspondiente.
Cuando ninguna de las expresiones es verdadera, entonces se ejecuta el
código de la sección \texttt{else}, es decir el \texttt{codigo4}.

Observa que se siguen las mismas reglas de sangrado que en el
\texttt{while}.

Veamos un ejemplo:

\begin{Shaded}
\begin{Highlighting}[]
\CommentTok{\# Modifica los valores de a y b, y observa el resultado}
\NormalTok{a }\OperatorTok{=} \DecValTok{10}
\NormalTok{b }\OperatorTok{=} \DecValTok{20}
\ControlFlowTok{if}\NormalTok{ a }\OperatorTok{\textless{}}\NormalTok{ b:}
    \BuiltInTok{print}\NormalTok{(}\StringTok{\textquotesingle{}a es menor que b\textquotesingle{}}\NormalTok{)}
\ControlFlowTok{elif}\NormalTok{ a }\OperatorTok{\textgreater{}}\NormalTok{ b:}
    \BuiltInTok{print}\NormalTok{(}\StringTok{\textquotesingle{}a es mayor que b\textquotesingle{}}\NormalTok{)}
\ControlFlowTok{elif}\NormalTok{ a }\OperatorTok{==}\NormalTok{ b:}
    \BuiltInTok{print}\NormalTok{(}\StringTok{\textquotesingle{}a es igual a b\textquotesingle{}}\NormalTok{)}
\ControlFlowTok{else}\NormalTok{:}
    \BuiltInTok{print}\NormalTok{(}\StringTok{\textquotesingle{}Esto nunca pasa\textquotesingle{}}\NormalTok{)}
\end{Highlighting}
\end{Shaded}

Las expresiones pueden ser más complejas:

\begin{Shaded}
\begin{Highlighting}[]
\CommentTok{\# Modifica los valores de a y b, y observa el resultado}
\NormalTok{a }\OperatorTok{=} \DecValTok{10}
\NormalTok{b }\OperatorTok{=} \DecValTok{20}
\ControlFlowTok{if}\NormalTok{ (a }\OperatorTok{\textless{}}\NormalTok{ b) }\KeywordTok{or}\NormalTok{ (a }\OperatorTok{\textgreater{}}\NormalTok{ b):}
    \BuiltInTok{print}\NormalTok{(}\SpecialStringTok{f\textquotesingle{}a = }\SpecialCharTok{\{}\NormalTok{a}\SpecialCharTok{\}}\SpecialStringTok{, b = }\SpecialCharTok{\{}\NormalTok{b}\SpecialCharTok{\}}\SpecialStringTok{\textquotesingle{}}\NormalTok{)}
\end{Highlighting}
\end{Shaded}

\bookmarksetup{startatroot}

\chapter{Operador ternario}\label{operador-ternario}

Este operador permite evaluar una expresión lógica y generar un valor
para un resultado \texttt{True} y otro diferente para un resultado
\texttt{False}; todo esto se logra en una sola línea de código como
sigue:

\begin{Shaded}
\begin{Highlighting}[]
\NormalTok{resultado }\OperatorTok{=}\NormalTok{ valor1 }\ControlFlowTok{if}\NormalTok{ expresion }\ControlFlowTok{else}\NormalTok{ valor2}
\end{Highlighting}
\end{Shaded}

Por ejemplo:

\begin{Shaded}
\begin{Highlighting}[]
\CommentTok{\# Usa valores para c = 1, 2, 4, 4, 5, 6, 20 y observa el resultado}
\NormalTok{c }\OperatorTok{=} \DecValTok{1}
\NormalTok{r }\OperatorTok{=}\NormalTok{ c }\ControlFlowTok{if}\NormalTok{ c }\OperatorTok{\textgreater{}} \DecValTok{5} \ControlFlowTok{else} \DecValTok{0}
\BuiltInTok{print}\NormalTok{(r)}
\end{Highlighting}
\end{Shaded}

\bookmarksetup{startatroot}

\chapter{\texorpdfstring{\texttt{for}}{for}}\label{for}

Permite iterar sobre el contenido de cualquier secuencia (cadena, lista,
tupla, conjunto, diccionario, archivo, \ldots). La forma de esta
declaración es como sigue:

\begin{Shaded}
\begin{Highlighting}[]
\ControlFlowTok{for}\NormalTok{ i }\KeywordTok{in}\NormalTok{ secuencia:}
\NormalTok{    codigo}
\end{Highlighting}
\end{Shaded}

Las reglas de sangrado se siguen en esta declaración.

Por ejemplo:

\begin{Shaded}
\begin{Highlighting}[]
\NormalTok{gatos }\OperatorTok{=}\NormalTok{ [}\StringTok{\textquotesingle{}Persa\textquotesingle{}}\NormalTok{, }\StringTok{\textquotesingle{}Sphynx\textquotesingle{}}\NormalTok{, }\StringTok{\textquotesingle{}Ragdoll\textquotesingle{}}\NormalTok{,}\StringTok{\textquotesingle{}Siamés\textquotesingle{}}\NormalTok{]}

\ControlFlowTok{for}\NormalTok{ i }\KeywordTok{in}\NormalTok{ gatos:}
    \BuiltInTok{print}\NormalTok{(i)}
\end{Highlighting}
\end{Shaded}

\section{\texorpdfstring{Función
\texttt{zip}}{Función zip}}\label{funciuxf3n-zip}

La función \texttt{zip(s1,\ s2,\ ...)} permite combinar dos o más
secuencias en una sola; genera tuplas con los elementos de: las
secuencias y va iterando sobre ellas.

Por ejemplo

\begin{Shaded}
\begin{Highlighting}[]
\CommentTok{\# Dos listas de la misma longitud}
\NormalTok{gatos }\OperatorTok{=}\NormalTok{ [}\StringTok{\textquotesingle{}Persa\textquotesingle{}}\NormalTok{, }\StringTok{\textquotesingle{}Sphynx\textquotesingle{}}\NormalTok{, }\StringTok{\textquotesingle{}Ragdoll\textquotesingle{}}\NormalTok{,}\StringTok{\textquotesingle{}Siamés\textquotesingle{}}\NormalTok{]}
\NormalTok{origen }\OperatorTok{=}\NormalTok{ [}\StringTok{\textquotesingle{}Irán\textquotesingle{}}\NormalTok{, }\StringTok{\textquotesingle{}Toronto\textquotesingle{}}\NormalTok{, }\StringTok{\textquotesingle{}California\textquotesingle{}}\NormalTok{, }\StringTok{\textquotesingle{}Tailandia\textquotesingle{}}\NormalTok{]}
\BuiltInTok{print}\NormalTok{(gatos)}
\BuiltInTok{print}\NormalTok{(origen)}

\CommentTok{\# Combinamos las listas en una sola secuencia}
\BuiltInTok{print}\NormalTok{(}\StringTok{\textquotesingle{}}\CharTok{\textbackslash{}n}\StringTok{(Raza, Origen)\textquotesingle{}}\NormalTok{)}
\BuiltInTok{print}\NormalTok{(}\StringTok{\textquotesingle{}{-}\textquotesingle{}}\OperatorTok{*}\DecValTok{20}\NormalTok{)}
\ControlFlowTok{for}\NormalTok{ t }\KeywordTok{in} \BuiltInTok{zip}\NormalTok{(gatos, origen):}
    \BuiltInTok{print}\NormalTok{(t)}
\end{Highlighting}
\end{Shaded}

\begin{verbatim}
['Persa', 'Sphynx', 'Ragdoll', 'Siamés']
['Irán', 'Toronto', 'California', 'Tailandia']

(Raza, Origen)
--------------------
('Persa', 'Irán')
('Sphynx', 'Toronto')
('Ragdoll', 'California')
('Siamés', 'Tailandia')
\end{verbatim}

\begin{Shaded}
\begin{Highlighting}[]
\CommentTok{\# Se puede extraer la información de cada secuencia:}
\ControlFlowTok{for}\NormalTok{ g, o }\KeywordTok{in} \BuiltInTok{zip}\NormalTok{(gatos, origen):}
    \BuiltInTok{print}\NormalTok{(}\StringTok{\textquotesingle{}La raza }\SpecialCharTok{\{\}}\StringTok{ proviene de }\SpecialCharTok{\{\}}\StringTok{\textquotesingle{}}\NormalTok{.}\BuiltInTok{format}\NormalTok{(g, o))}
\end{Highlighting}
\end{Shaded}

\begin{verbatim}
La raza Persa proviene de Irán
La raza Sphynx proviene de Toronto
La raza Ragdoll proviene de California
La raza Siamés proviene de Tailandia
\end{verbatim}

\section{\texorpdfstring{Conversión de \texttt{zip} a \texttt{list},
\texttt{tuple}, \texttt{set},
\texttt{dict}}{Conversión de zip a list, tuple, set, dict}}\label{conversiuxf3n-de-zip-a-list-tuple-set-dict}

Estrictamente \texttt{zip} es una clase que define un tipo dentro de
Python, por lo que es posible convertir del tipo \texttt{zip} a alguna
otra secuencia básica de datos de Python.

\begin{Shaded}
\begin{Highlighting}[]
\NormalTok{z }\OperatorTok{=} \BuiltInTok{zip}\NormalTok{(gatos, origen)}

\CommentTok{\# Verificar el tipo de zip}
\BuiltInTok{print}\NormalTok{(}\BuiltInTok{type}\NormalTok{(z))}
\BuiltInTok{print}\NormalTok{(z)}
\end{Highlighting}
\end{Shaded}

\begin{verbatim}
<class 'zip'>
<zip object at 0x7f0598954a40>
\end{verbatim}

\begin{Shaded}
\begin{Highlighting}[]
\NormalTok{lista }\OperatorTok{=} \BuiltInTok{list}\NormalTok{(}\BuiltInTok{zip}\NormalTok{(gatos, origen))}
\NormalTok{tupla }\OperatorTok{=} \BuiltInTok{tuple}\NormalTok{(}\BuiltInTok{zip}\NormalTok{(gatos, origen))}
\NormalTok{conj }\OperatorTok{=} \BuiltInTok{set}\NormalTok{(}\BuiltInTok{zip}\NormalTok{(gatos, origen))}
\NormalTok{dicc }\OperatorTok{=} \BuiltInTok{dict}\NormalTok{(}\BuiltInTok{zip}\NormalTok{(gatos, origen)) }\CommentTok{\# Solo funciona para dos secuencias}

\BuiltInTok{print}\NormalTok{(lista)}
\BuiltInTok{print}\NormalTok{(tupla)}
\BuiltInTok{print}\NormalTok{(conj)}
\BuiltInTok{print}\NormalTok{(dicc)}
\end{Highlighting}
\end{Shaded}

\begin{verbatim}
[('Persa', 'Irán'), ('Sphynx', 'Toronto'), ('Ragdoll', 'California'), ('Siamés', 'Tailandia')]
(('Persa', 'Irán'), ('Sphynx', 'Toronto'), ('Ragdoll', 'California'), ('Siamés', 'Tailandia'))
{('Persa', 'Irán'), ('Sphynx', 'Toronto'), ('Siamés', 'Tailandia'), ('Ragdoll', 'California')}
{'Persa': 'Irán', 'Sphynx': 'Toronto', 'Ragdoll': 'California', 'Siamés': 'Tailandia'}
\end{verbatim}

\section{\texorpdfstring{Función
\texttt{enumerate}}{Función enumerate}}\label{funciuxf3n-enumerate}

Permite enumerar los elementos de una secuencia. Genera tuplas con el
número del elemento y el elemento de la secuencia.

Por ejemplo:

\begin{Shaded}
\begin{Highlighting}[]
\BuiltInTok{print}\NormalTok{(gatos)}

\CommentTok{\# Enumeramos la secuencia}
\BuiltInTok{print}\NormalTok{(}\StringTok{\textquotesingle{}}\CharTok{\textbackslash{}n}\StringTok{(Numero, Raza)\textquotesingle{}}\NormalTok{)}
\BuiltInTok{print}\NormalTok{(}\StringTok{\textquotesingle{}{-}\textquotesingle{}}\OperatorTok{*}\DecValTok{20}\NormalTok{)}

\ControlFlowTok{for}\NormalTok{ t }\KeywordTok{in} \BuiltInTok{enumerate}\NormalTok{(gatos):}
    \BuiltInTok{print}\NormalTok{(t)}
\end{Highlighting}
\end{Shaded}

\begin{verbatim}
['Persa', 'Sphynx', 'Ragdoll', 'Siamés']

(Numero, Raza)
--------------------
(0, 'Persa')
(1, 'Sphynx')
(2, 'Ragdoll')
(3, 'Siamés')
\end{verbatim}

\begin{Shaded}
\begin{Highlighting}[]
\ControlFlowTok{for}\NormalTok{ i, g }\KeywordTok{in} \BuiltInTok{enumerate}\NormalTok{(gatos):}
    \BuiltInTok{print}\NormalTok{(i, g)}
\end{Highlighting}
\end{Shaded}

\begin{verbatim}
0 Persa
1 Sphynx
2 Ragdoll
3 Siamés
\end{verbatim}

Lo anterior permite usar el indexado para acceder a los elementos de una
secuencia:

\begin{Shaded}
\begin{Highlighting}[]
\ControlFlowTok{for}\NormalTok{ i, g }\KeywordTok{in} \BuiltInTok{enumerate}\NormalTok{(gatos):}
    \BuiltInTok{print}\NormalTok{(i, gatos[i])}
\end{Highlighting}
\end{Shaded}

\begin{verbatim}
0 Persa
1 Sphynx
2 Ragdoll
3 Siamés
\end{verbatim}

\section{\texorpdfstring{Conversión de \texttt{enumerate} a
\texttt{list}, \texttt{tuple}, \texttt{set},
\texttt{dict}}{Conversión de enumerate a list, tuple, set, dict}}\label{conversiuxf3n-de-enumerate-a-list-tuple-set-dict}

Estrictamente \texttt{enumerate} es una clase que define un tipo dentro
de Python, por lo que es posible convertir del tipo \texttt{enumerate} a
alguna otra secuencia básica de datos de Python:

\begin{Shaded}
\begin{Highlighting}[]
\NormalTok{e }\OperatorTok{=} \BuiltInTok{enumerate}\NormalTok{(gatos)}

\CommentTok{\# Verificar el tipo de enumerate}
\BuiltInTok{print}\NormalTok{(}\BuiltInTok{type}\NormalTok{(e))}
\BuiltInTok{print}\NormalTok{(e)}
\end{Highlighting}
\end{Shaded}

\begin{verbatim}
<class 'enumerate'>
<enumerate object at 0x7f0598472d90>
\end{verbatim}

\begin{Shaded}
\begin{Highlighting}[]
\NormalTok{lista }\OperatorTok{=} \BuiltInTok{list}\NormalTok{(}\BuiltInTok{enumerate}\NormalTok{(gatos))}
\NormalTok{tupla }\OperatorTok{=} \BuiltInTok{tuple}\NormalTok{(}\BuiltInTok{enumerate}\NormalTok{(gatos))}
\NormalTok{conj }\OperatorTok{=} \BuiltInTok{set}\NormalTok{(}\BuiltInTok{enumerate}\NormalTok{(gatos))}
\NormalTok{dicc }\OperatorTok{=} \BuiltInTok{dict}\NormalTok{(}\BuiltInTok{enumerate}\NormalTok{(gatos))}

\BuiltInTok{print}\NormalTok{(lista)}
\BuiltInTok{print}\NormalTok{(tupla)}
\BuiltInTok{print}\NormalTok{(conj)}
\BuiltInTok{print}\NormalTok{(dicc)}
\end{Highlighting}
\end{Shaded}

\begin{verbatim}
[(0, 'Persa'), (1, 'Sphynx'), (2, 'Ragdoll'), (3, 'Siamés')]
((0, 'Persa'), (1, 'Sphynx'), (2, 'Ragdoll'), (3, 'Siamés'))
{(1, 'Sphynx'), (0, 'Persa'), (3, 'Siamés'), (2, 'Ragdoll')}
{0: 'Persa', 1: 'Sphynx', 2: 'Ragdoll', 3: 'Siamés'}
\end{verbatim}

\section{Funcion range}\label{funcion-range}

Esta función genera una secuencia iterable con un inicio, un final y un
salta:

\begin{Shaded}
\begin{Highlighting}[]
\BuiltInTok{range}\NormalTok{(start, stop, step)}
\end{Highlighting}
\end{Shaded}

La secuencia irá desde \texttt{start} hasta \texttt{stop-1} en pasos de
\texttt{step}. Por ejemplo:

\begin{Shaded}
\begin{Highlighting}[]
\ControlFlowTok{for}\NormalTok{ i }\KeywordTok{in} \BuiltInTok{range}\NormalTok{(}\DecValTok{1}\NormalTok{,}\DecValTok{20}\NormalTok{): }\CommentTok{\# Por omisión step = 1}
    \BuiltInTok{print}\NormalTok{(i, end}\OperatorTok{=} \StringTok{\textquotesingle{}, \textquotesingle{}}\NormalTok{)}
\end{Highlighting}
\end{Shaded}

\begin{verbatim}
1, 2, 3, 4, 5, 6, 7, 8, 9, 10, 11, 12, 13, 14, 15, 16, 17, 18, 19, 
\end{verbatim}

\begin{Shaded}
\begin{Highlighting}[]
\ControlFlowTok{for}\NormalTok{ i }\KeywordTok{in} \BuiltInTok{range}\NormalTok{(}\DecValTok{1}\NormalTok{,}\DecValTok{20}\NormalTok{, }\DecValTok{2}\NormalTok{):}
    \BuiltInTok{print}\NormalTok{(i, end}\OperatorTok{=} \StringTok{\textquotesingle{}, \textquotesingle{}}\NormalTok{)}
\end{Highlighting}
\end{Shaded}

\begin{verbatim}
1, 3, 5, 7, 9, 11, 13, 15, 17, 19, 
\end{verbatim}

\begin{Shaded}
\begin{Highlighting}[]
\ControlFlowTok{for}\NormalTok{ i }\KeywordTok{in} \BuiltInTok{range}\NormalTok{(}\DecValTok{20}\NormalTok{, }\DecValTok{1}\NormalTok{, }\OperatorTok{{-}}\DecValTok{2}\NormalTok{): }\CommentTok{\# El paso puede ser negativo}
    \BuiltInTok{print}\NormalTok{(i, end}\OperatorTok{=} \StringTok{\textquotesingle{}, \textquotesingle{}}\NormalTok{)}
\end{Highlighting}
\end{Shaded}

\begin{verbatim}
20, 18, 16, 14, 12, 10, 8, 6, 4, 2, 
\end{verbatim}

Usando \texttt{range()} se puede acceder a una seccuencia mediante el
indexado:

\begin{Shaded}
\begin{Highlighting}[]
\NormalTok{N }\OperatorTok{=} \BuiltInTok{len}\NormalTok{(gatos) }\CommentTok{\# Longitud de la lista gatos}

\ControlFlowTok{for}\NormalTok{ i }\KeywordTok{in} \BuiltInTok{range}\NormalTok{(}\DecValTok{0}\NormalTok{, N):}
    \BuiltInTok{print}\NormalTok{(i, gatos[i])}
\end{Highlighting}
\end{Shaded}

\begin{verbatim}
0 Persa
1 Sphynx
2 Ragdoll
3 Siamés
\end{verbatim}

\section{\texorpdfstring{Conversión de \texttt{range} a \texttt{list},
\texttt{tuple},
\texttt{set}}{Conversión de range a list, tuple, set}}\label{conversiuxf3n-de-range-a-list-tuple-set}

Estrictamente \texttt{range} es una clase que define un tipo dentro de
Python, por lo que es posible convertir del tipo \texttt{range} a alguna
otra secuencia básica de datos de Python:

\begin{Shaded}
\begin{Highlighting}[]
\NormalTok{N }\OperatorTok{=} \BuiltInTok{len}\NormalTok{(gatos) }\CommentTok{\# Longitud de la lista gatos}
\NormalTok{r }\OperatorTok{=} \BuiltInTok{range}\NormalTok{(}\DecValTok{0}\NormalTok{,N)}

\CommentTok{\# Verificar el tipo de range}
\BuiltInTok{print}\NormalTok{(}\BuiltInTok{type}\NormalTok{(r))}
\BuiltInTok{print}\NormalTok{(r)}
\end{Highlighting}
\end{Shaded}

\begin{verbatim}
<class 'range'>
range(0, 4)
\end{verbatim}

\begin{Shaded}
\begin{Highlighting}[]
\NormalTok{lista }\OperatorTok{=} \BuiltInTok{list}\NormalTok{(}\BuiltInTok{range}\NormalTok{(}\DecValTok{0}\NormalTok{,N))}
\NormalTok{tupla }\OperatorTok{=} \BuiltInTok{tuple}\NormalTok{(}\BuiltInTok{range}\NormalTok{(}\DecValTok{0}\NormalTok{,N))}
\NormalTok{conj }\OperatorTok{=} \BuiltInTok{set}\NormalTok{(}\BuiltInTok{range}\NormalTok{(}\DecValTok{0}\NormalTok{,N))}

\BuiltInTok{print}\NormalTok{(lista)}
\BuiltInTok{print}\NormalTok{(tupla)}
\BuiltInTok{print}\NormalTok{(conj)}
\end{Highlighting}
\end{Shaded}

\begin{verbatim}
[0, 1, 2, 3]
(0, 1, 2, 3)
{0, 1, 2, 3}
\end{verbatim}

\section{break, continue, else, pass}\label{break-continue-else-pass}

Estas son palabras clave que se pueden usar en ciclos \texttt{while} o
\texttt{for}: * \texttt{break}: terminar el ciclo más interno. *
\texttt{continue}: saltarse a la siguiente iteración sin terminar de
ejecutar el código que sigue. * \texttt{else}: \textbf{NO} se ejecuta el
código de esta cláusula si el ciclo es finalizado por el \texttt{break}.
* \texttt{pass}: no hacer nada y continuar.

Veamos algunos ejemplos:

\begin{Shaded}
\begin{Highlighting}[]
\CommentTok{\# Se itera por una lista de palabras, cuando se encuentra}
\CommentTok{\# la letra \textquotesingle{}h\textquotesingle{} se termina el ciclo interno y se continua con}
\CommentTok{\# la siguiente palabra.}
\ControlFlowTok{for}\NormalTok{ palabra }\KeywordTok{in}\NormalTok{ [}\StringTok{"Hola"}\NormalTok{, }\StringTok{"mundo"}\NormalTok{, }\StringTok{"Pythonico"}\NormalTok{]:}
    \BuiltInTok{print}\NormalTok{(}\StringTok{\textquotesingle{}Palabra: \textquotesingle{}}\NormalTok{, palabra)}
    \ControlFlowTok{for}\NormalTok{ letra }\KeywordTok{in}\NormalTok{ palabra:}
        \BuiltInTok{print}\NormalTok{(}\StringTok{\textquotesingle{}}\CharTok{\textbackslash{}t}\StringTok{ letra: \textquotesingle{}}\NormalTok{, letra)}
        \ControlFlowTok{if}\NormalTok{ letra }\OperatorTok{==} \StringTok{"h"}\NormalTok{:}
            \ControlFlowTok{break}
\end{Highlighting}
\end{Shaded}

\begin{verbatim}
Palabra:  Hola
     letra:  H
     letra:  o
     letra:  l
     letra:  a
Palabra:  mundo
     letra:  m
     letra:  u
     letra:  n
     letra:  d
     letra:  o
Palabra:  Pythonico
     letra:  P
     letra:  y
     letra:  t
     letra:  h
\end{verbatim}

\begin{Shaded}
\begin{Highlighting}[]
\CommentTok{\# Se itera por una lista de palabras, cuando se encuentra}
\CommentTok{\# la letra \textquotesingle{}h\textquotesingle{} se termina el ciclo interno y se continua con}
\CommentTok{\# la siguiente palabra. La cláusula \textquotesingle{}else\textquotesingle{} se ejecuta si no}
\CommentTok{\# se encuentra la letra.}
\ControlFlowTok{for}\NormalTok{ palabra }\KeywordTok{in}\NormalTok{ [}\StringTok{"Hola"}\NormalTok{, }\StringTok{"mundo"}\NormalTok{, }\StringTok{"Pythonico"}\NormalTok{]:}
    \BuiltInTok{print}\NormalTok{(}\StringTok{\textquotesingle{}Palabra: \textquotesingle{}}\NormalTok{, palabra)}
    \ControlFlowTok{for}\NormalTok{ letra }\KeywordTok{in}\NormalTok{ palabra:}
        \BuiltInTok{print}\NormalTok{(}\StringTok{\textquotesingle{}}\CharTok{\textbackslash{}t}\StringTok{ letra: \textquotesingle{}}\NormalTok{, letra)}
        \ControlFlowTok{if}\NormalTok{ letra }\OperatorTok{==} \StringTok{"h"}\NormalTok{:}
            \ControlFlowTok{break}
    \ControlFlowTok{else}\NormalTok{:}
        \BuiltInTok{print}\NormalTok{(}\StringTok{\textquotesingle{}No encontré la letra "h"\textquotesingle{}}\NormalTok{)}
\end{Highlighting}
\end{Shaded}

\begin{verbatim}
Palabra:  Hola
     letra:  H
     letra:  o
     letra:  l
     letra:  a
No encontré la letra "h"
Palabra:  mundo
     letra:  m
     letra:  u
     letra:  n
     letra:  d
     letra:  o
No encontré la letra "h"
Palabra:  Pythonico
     letra:  P
     letra:  y
     letra:  t
     letra:  h
\end{verbatim}

\begin{Shaded}
\begin{Highlighting}[]
\CommentTok{\# Esta declaración pass no hace nada. Se usa principalmente }
\CommentTok{\# para cuestiones de desarrollo de código a un nivel abstracto.}
\NormalTok{i }\OperatorTok{=} \DecValTok{0}
\ControlFlowTok{while}\NormalTok{ i }\OperatorTok{\textgreater{}} \DecValTok{10}\NormalTok{:}
    \ControlFlowTok{pass}
\end{Highlighting}
\end{Shaded}

\begin{Shaded}
\begin{Highlighting}[]
\CommentTok{\# La siguiente función calcula la secuencia de Fibonacci}
\KeywordTok{def}\NormalTok{ fib(n):}
\CommentTok{\#    print(i, end=" ")}
    \ControlFlowTok{pass}

\CommentTok{\# En este punto del programa requiero el uso de la función fib(n):}

\NormalTok{fib(}\DecValTok{100000}\NormalTok{) }\CommentTok{\# }
\end{Highlighting}
\end{Shaded}

\subsection{Más ejemplos.}\label{muxe1s-ejemplos.}

\begin{Shaded}
\begin{Highlighting}[]
\CommentTok{\# Calcula números primos usando la}
\CommentTok{\# criba de Eratóstenes}
\ControlFlowTok{for}\NormalTok{ n }\KeywordTok{in} \BuiltInTok{range}\NormalTok{(}\DecValTok{2}\NormalTok{, }\DecValTok{10}\NormalTok{):}
    \ControlFlowTok{for}\NormalTok{ x }\KeywordTok{in} \BuiltInTok{range}\NormalTok{(}\DecValTok{2}\NormalTok{, n):}
        \ControlFlowTok{if}\NormalTok{ n }\OperatorTok{\%}\NormalTok{ x }\OperatorTok{==} \DecValTok{0}\NormalTok{:}
            \BuiltInTok{print}\NormalTok{(n, }\StringTok{\textquotesingle{}igual a \textquotesingle{}}\NormalTok{, x, }\StringTok{\textquotesingle{}*\textquotesingle{}}\NormalTok{, n}\OperatorTok{//}\NormalTok{x)}
            \ControlFlowTok{break}
    \ControlFlowTok{else}\NormalTok{:}
        \BuiltInTok{print}\NormalTok{(n, }\StringTok{\textquotesingle{}es un número primo\textquotesingle{}}\NormalTok{)}
\end{Highlighting}
\end{Shaded}

\begin{verbatim}
2 es un número primo
3 es un número primo
4 igual a  2 * 2
5 es un número primo
6 igual a  2 * 3
7 es un número primo
8 igual a  2 * 4
9 igual a  3 * 3
\end{verbatim}

\begin{Shaded}
\begin{Highlighting}[]
\CommentTok{\# Determina números pares e impares}
\ControlFlowTok{for}\NormalTok{ num }\KeywordTok{in} \BuiltInTok{range}\NormalTok{(}\DecValTok{2}\NormalTok{, }\DecValTok{10}\NormalTok{):}
    \ControlFlowTok{if}\NormalTok{ num }\OperatorTok{\%} \DecValTok{2} \OperatorTok{==} \DecValTok{0}\NormalTok{:}
        \BuiltInTok{print}\NormalTok{(}\StringTok{"Número par "}\NormalTok{, num)}
        \ControlFlowTok{continue}
    \BuiltInTok{print}\NormalTok{(}\StringTok{"Número impar"}\NormalTok{, num)}
\end{Highlighting}
\end{Shaded}

\begin{verbatim}
Número par  2
Número impar 3
Número par  4
Número impar 5
Número par  6
Número impar 7
Número par  8
Número impar 9
\end{verbatim}

\begin{Shaded}
\begin{Highlighting}[]
\CommentTok{\# Checa la clave de un usuario. Después de tres}
\CommentTok{\# intentos fallidos termina. Si se da la clave}
\CommentTok{\# correcta (despedida) se termina.}
\NormalTok{suma }\OperatorTok{=} \DecValTok{0}
\ControlFlowTok{while}\NormalTok{ suma }\OperatorTok{\textless{}} \DecValTok{3}\NormalTok{:}
\NormalTok{    entrada }\OperatorTok{=} \BuiltInTok{input}\NormalTok{(}\StringTok{"Clave:"}\NormalTok{)}
    \ControlFlowTok{if}\NormalTok{ entrada }\OperatorTok{==} \StringTok{"despedida"}\NormalTok{:}
        \ControlFlowTok{break}
\NormalTok{    suma }\OperatorTok{=}\NormalTok{ suma }\OperatorTok{+} \DecValTok{1}
    \BuiltInTok{print}\NormalTok{(}\StringTok{"Intento }\SpecialCharTok{\%d}\StringTok{. }\CharTok{\textbackslash{}n}\StringTok{ "} \OperatorTok{\%}\NormalTok{ suma)}
\BuiltInTok{print}\NormalTok{(}\StringTok{"Tuviste }\SpecialCharTok{\{\}}\StringTok{ intentos fallidos."}\NormalTok{.}\BuiltInTok{format}\NormalTok{(suma))}
\end{Highlighting}
\end{Shaded}

\begin{verbatim}
Clave: despedida
\end{verbatim}

\begin{verbatim}
Tuviste 0 intentos fallidos.
\end{verbatim}

\bookmarksetup{startatroot}

\chapter{\texorpdfstring{\texttt{match} (desde la versión
3.10)}{match (desde la versión 3.10)}}\label{match-desde-la-versiuxf3n-3.10}

Similar al switch de lenguajes como C, C++, Java.

\begin{Shaded}
\begin{Highlighting}[]
\KeywordTok{def}\NormalTok{ http\_error(status):}
    \ControlFlowTok{match}\NormalTok{ status:}
        \ControlFlowTok{case} \DecValTok{400}\NormalTok{:}
            \ControlFlowTok{return} \StringTok{"Bad request"}
        \ControlFlowTok{case} \DecValTok{404}\NormalTok{:}
            \ControlFlowTok{return} \StringTok{"Not found"}
        \ControlFlowTok{case} \DecValTok{418}\NormalTok{:}
            \ControlFlowTok{return} \StringTok{"I\textquotesingle{}m a teapot"}
        \ControlFlowTok{case}\NormalTok{ \_:}
            \ControlFlowTok{return} \StringTok{"Something\textquotesingle{}s wrong with the internet"}
\end{Highlighting}
\end{Shaded}

\begin{Shaded}
\begin{Highlighting}[]
\CommentTok{\# Modifica el valor del argumento y observa lo que sucede}
\NormalTok{http\_error(}\DecValTok{500}\NormalTok{)}
\end{Highlighting}
\end{Shaded}

\begin{verbatim}
"Something's wrong with the internet"
\end{verbatim}

\begin{Shaded}
\begin{Highlighting}[]
\CommentTok{\# Modifica los valores de la siguiente tupla y observa el resultado}
\NormalTok{point }\OperatorTok{=}\NormalTok{ (}\DecValTok{0}\NormalTok{,}\DecValTok{0}\NormalTok{)}

\ControlFlowTok{match}\NormalTok{ point:}
\NormalTok{    case (}\DecValTok{0}\NormalTok{, }\DecValTok{0}\NormalTok{):}
        \BuiltInTok{print}\NormalTok{(}\StringTok{"Origin"}\NormalTok{)}
\NormalTok{    case (}\DecValTok{0}\NormalTok{, y):}
        \BuiltInTok{print}\NormalTok{(}\SpecialStringTok{f"Y=}\SpecialCharTok{\{}\NormalTok{y}\SpecialCharTok{\}}\SpecialStringTok{"}\NormalTok{)}
\NormalTok{    case (x, }\DecValTok{0}\NormalTok{):}
        \BuiltInTok{print}\NormalTok{(}\SpecialStringTok{f"X=}\SpecialCharTok{\{}\NormalTok{x}\SpecialCharTok{\}}\SpecialStringTok{"}\NormalTok{)}
\NormalTok{    case (x, y):}
        \BuiltInTok{print}\NormalTok{(}\SpecialStringTok{f"X=}\SpecialCharTok{\{}\NormalTok{x}\SpecialCharTok{\}}\SpecialStringTok{, Y=}\SpecialCharTok{\{}\NormalTok{y}\SpecialCharTok{\}}\SpecialStringTok{"}\NormalTok{)}
    \ControlFlowTok{case}\NormalTok{ \_:}
        \ControlFlowTok{raise} \PreprocessorTok{ValueError}\NormalTok{(}\StringTok{"Not a point"}\NormalTok{)}
\end{Highlighting}
\end{Shaded}

\begin{verbatim}
Origin
\end{verbatim}

Para más detalles véase
\href{https://docs.python.org/3/tutorial/controlflow.html\#match-statements}{match
Statements}.

\bookmarksetup{startatroot}

\chapter{Entrada y salida
estándar.}\label{entrada-y-salida-estuxe1ndar.}

\textbf{Objetivo.} \ldots{}

\textbf{Funciones de Python}: \ldots{}

MACTI-Algebra\_Lineal\_01 by Luis M. de la Cruz is licensed under
Attribution-ShareAlike 4.0 International

\bookmarksetup{startatroot}

\chapter{\texorpdfstring{Entrada estándar:
\texttt{input()}}{Entrada estándar: input()}}\label{entrada-estuxe1ndar-input}

La entrada estándar para proporcionar información a un programa se
realiza mediante la función \texttt{input()}.

Veamos algunos ejemplos:

\begin{Shaded}
\begin{Highlighting}[]
\CommentTok{\# Cuando se ejecuta, se espera a que el usuario teclee algo y luego de \textless{}enter\textgreater{}}
\BuiltInTok{input}\NormalTok{() }
\end{Highlighting}
\end{Shaded}

\begin{verbatim}
 34
\end{verbatim}

\begin{verbatim}
'34'
\end{verbatim}

\begin{Shaded}
\begin{Highlighting}[]
\CommentTok{\# Se puede asignar el valor que se teclea a una variable}
\NormalTok{entrada }\OperatorTok{=} \BuiltInTok{input}\NormalTok{()}
\end{Highlighting}
\end{Shaded}

\begin{verbatim}
 34
\end{verbatim}

\begin{Shaded}
\begin{Highlighting}[]
\CommentTok{\# Ahora imprimimos el valor guardado}
\BuiltInTok{print}\NormalTok{(entrada) }
\end{Highlighting}
\end{Shaded}

\begin{verbatim}
34
\end{verbatim}

\begin{Shaded}
\begin{Highlighting}[]
\CommentTok{\# Se puede poner un mensaje para que el usuario sepa}
\CommentTok{\# lo que se espera:}
\NormalTok{entrada }\OperatorTok{=} \BuiltInTok{input}\NormalTok{(}\StringTok{\textquotesingle{}Teclea un valor entero :\textquotesingle{}}\NormalTok{)}
\end{Highlighting}
\end{Shaded}

\begin{verbatim}
Teclea un valor entero : 5
\end{verbatim}

\begin{Shaded}
\begin{Highlighting}[]
\CommentTok{\# Lo que se lee siempre se transforma en una cadena de texto:}
\BuiltInTok{print}\NormalTok{(}\BuiltInTok{type}\NormalTok{(entrada))}
\BuiltInTok{print}\NormalTok{(entrada)}
\end{Highlighting}
\end{Shaded}

\begin{verbatim}
<class 'str'>
5
\end{verbatim}

\begin{Shaded}
\begin{Highlighting}[]
\CommentTok{\# Podemos hacer el \textquotesingle{}casting\textquotesingle{} para transformar lo que teclea el usuario}
\CommentTok{\# en el tipo de dato requerido:}
\NormalTok{entrada }\OperatorTok{=} \BuiltInTok{int}\NormalTok{(}\BuiltInTok{input}\NormalTok{(}\StringTok{\textquotesingle{}Teclea un valor entero :\textquotesingle{}}\NormalTok{))}
\end{Highlighting}
\end{Shaded}

\begin{verbatim}
Teclea un valor entero : 5
\end{verbatim}

\begin{Shaded}
\begin{Highlighting}[]
\BuiltInTok{print}\NormalTok{(}\BuiltInTok{type}\NormalTok{(entrada))}
\BuiltInTok{print}\NormalTok{(entrada)}
\end{Highlighting}
\end{Shaded}

\begin{verbatim}
<class 'int'>
5
\end{verbatim}

Lo anterior se debe realizar con más cuidado e incluso usando
declaraciones para capturar posibles errores del usuario al teclear un
valor.

\bookmarksetup{startatroot}

\chapter{\texorpdfstring{Salida estándar:
\texttt{print()}}{Salida estándar: print()}}\label{salida-estuxe1ndar-print}

Existen varias formas de presentar la salida de un programa al usuario,
la más común es en la pantalla (la salida estándar).

En todos los casos se desea un control adecuado sobre el formato de la
salida. Para ello se tienen varias maneras de controlar esta salida:

\section{\texorpdfstring{Cadenas con formato \texttt{f} o
\texttt{F}.}{Cadenas con formato f o F.}}\label{cadenas-con-formato-f-o-f.}

Este tipo de cadenas se forman anteponiendo una \texttt{f} o \texttt{F}
al principio de la misma. De esta forma, es posible poner variables
entre llaves \texttt{\{\}} dentro de la definición de la cadena.

\begin{Shaded}
\begin{Highlighting}[]
\NormalTok{nombre }\OperatorTok{=} \StringTok{\textquotesingle{}LUIS MIGUEL\textquotesingle{}}
\NormalTok{edad }\OperatorTok{=} \DecValTok{25}
\SpecialStringTok{f\textquotesingle{}Hola mi nombre es }\SpecialCharTok{\{}\NormalTok{nombre}\SpecialCharTok{\}}\SpecialStringTok{ y tengo }\SpecialCharTok{\{}\NormalTok{edad}\SpecialCharTok{\}}\SpecialStringTok{ años\textquotesingle{}}
\end{Highlighting}
\end{Shaded}

\begin{verbatim}
'Hola mi nombre es LUIS MIGUEL y tengo 25 años'
\end{verbatim}

En el caso de números es posible agregar un formato:

\begin{Shaded}
\begin{Highlighting}[]
\ImportTok{import}\NormalTok{ math}
\BuiltInTok{print}\NormalTok{(}\SpecialStringTok{f\textquotesingle{}El valor de PI es aproximadamente }\SpecialCharTok{\{}\NormalTok{math}\SpecialCharTok{.}\NormalTok{pi}\SpecialCharTok{:.10f\}}\SpecialStringTok{.\textquotesingle{}}\NormalTok{) }\CommentTok{\# \{valor:formato\}}
\end{Highlighting}
\end{Shaded}

\begin{verbatim}
El valor de PI es aproximadamente 3.1415926536.
\end{verbatim}

También es posible alinear el texto de la salida.

\begin{Shaded}
\begin{Highlighting}[]
\NormalTok{na1 }\OperatorTok{=} \StringTok{\textquotesingle{}Fulano\textquotesingle{}}\OperatorTok{;}\NormalTok{ n1 }\OperatorTok{=} \DecValTok{5\_521\_345\_678}
\NormalTok{na2 }\OperatorTok{=} \StringTok{\textquotesingle{}Sutano\textquotesingle{}}\OperatorTok{;}\NormalTok{ n2 }\OperatorTok{=} \DecValTok{7\_712\_932\_143}
\BuiltInTok{print}\NormalTok{(}\SpecialStringTok{f\textquotesingle{}}\SpecialCharTok{\{}\NormalTok{na1}\SpecialCharTok{:10\}}\SpecialStringTok{ ==\textgreater{} }\SpecialCharTok{\{}\NormalTok{n1}\SpecialCharTok{:15d\}}\SpecialStringTok{\textquotesingle{}}\NormalTok{) }\CommentTok{\# alineación del texto}
\BuiltInTok{print}\NormalTok{(}\SpecialStringTok{f\textquotesingle{}}\SpecialCharTok{\{}\NormalTok{na2}\SpecialCharTok{:10\}}\SpecialStringTok{ ==\textgreater{} }\SpecialCharTok{\{}\NormalTok{n2}\SpecialCharTok{:15d\}}\SpecialStringTok{\textquotesingle{}}\NormalTok{) }\CommentTok{\# alineación del texto}
\end{Highlighting}
\end{Shaded}

\begin{verbatim}
Fulano     ==>      5521345678
Sutano     ==>      7712932143
\end{verbatim}

Para más información véase Format Specification Mini-Language

\section{\texorpdfstring{Método
\texttt{format()}}{Método format()}}\label{muxe9todo-format}

Las cadenas tienen un método llamado \texttt{format()} que permite darle
formato a la misma.

Veamos unos ejemplos¡:

\begin{Shaded}
\begin{Highlighting}[]
\BuiltInTok{print}\NormalTok{(}\StringTok{\textquotesingle{}El curso se llama "}\SpecialCharTok{\{\}}\StringTok{" y tenemos }\SpecialCharTok{\{\}}\StringTok{ alumnos\textquotesingle{}}\NormalTok{.}\BuiltInTok{format}\NormalTok{(}\StringTok{\textquotesingle{}Python de cero a experto\textquotesingle{}}\NormalTok{, }\DecValTok{100}\NormalTok{))}
\end{Highlighting}
\end{Shaded}

\begin{verbatim}
El curso se llama "Python de cero a experto" y tenemos 100 alumnos
\end{verbatim}

\begin{Shaded}
\begin{Highlighting}[]
\NormalTok{votos\_a\_favor }\OperatorTok{=} \DecValTok{42\_572\_654}   \CommentTok{\# Este es un formato de número entero}
\NormalTok{votos\_en\_contra }\OperatorTok{=} \DecValTok{43\_132\_495} \CommentTok{\# que usa \_ para separar los miles}
\NormalTok{total\_de\_votos }\OperatorTok{=}\NormalTok{ votos\_a\_favor }\OperatorTok{+}\NormalTok{ votos\_en\_contra}
\NormalTok{porcentaje }\OperatorTok{=}\NormalTok{ votos\_a\_favor }\OperatorTok{/}\NormalTok{ total\_de\_votos}

\CommentTok{\# El primer dato se alinea al centro usando \^{}}
\CommentTok{\# El segundo dato tendrá dos valores antes del punto y dos valores después.}
\BuiltInTok{print}\NormalTok{(}\StringTok{\textquotesingle{}}\SpecialCharTok{\{:\^{}20\}}\StringTok{ votos a favor (}\SpecialCharTok{\{:2.2\%\}}\StringTok{)\textquotesingle{}}\NormalTok{.}\BuiltInTok{format}\NormalTok{(votos\_a\_favor, porcentaje))}
\end{Highlighting}
\end{Shaded}

\begin{verbatim}
      42572654       votos a favor (49.67%)
\end{verbatim}

\begin{Shaded}
\begin{Highlighting}[]
\CommentTok{\# Se pueden usar números para identificar los argumentos de format()}
\BuiltInTok{print}\NormalTok{(}\StringTok{\textquotesingle{}}\SpecialCharTok{\{0\}}\StringTok{ y }\SpecialCharTok{\{1\}}\StringTok{\textquotesingle{}}\NormalTok{.}\BuiltInTok{format}\NormalTok{(}\StringTok{\textquotesingle{}el huevo\textquotesingle{}}\NormalTok{, }\StringTok{\textquotesingle{}la gallina\textquotesingle{}}\NormalTok{))}
\BuiltInTok{print}\NormalTok{(}\StringTok{\textquotesingle{}}\SpecialCharTok{\{1\}}\StringTok{ y }\SpecialCharTok{\{0:\^{}20\}}\StringTok{\textquotesingle{}}\NormalTok{.}\BuiltInTok{format}\NormalTok{(}\StringTok{\textquotesingle{}el huevo\textquotesingle{}}\NormalTok{, }\StringTok{\textquotesingle{}la gallina\textquotesingle{}}\NormalTok{))}
\end{Highlighting}
\end{Shaded}

\begin{verbatim}
el huevo y la gallina
la gallina y       el huevo      
\end{verbatim}

\begin{Shaded}
\begin{Highlighting}[]
\CommentTok{\# Se le puede dar nombre a los argumentos para que }
\CommentTok{\# sea más fácil entender la salida}
\BuiltInTok{print}\NormalTok{(}\StringTok{\textquotesingle{}Esta }\SpecialCharTok{\{sustantivo\}}\StringTok{ es }\SpecialCharTok{\{adjetivo\}}\StringTok{.\textquotesingle{}}\NormalTok{.}\BuiltInTok{format}\NormalTok{(adjetivo}\OperatorTok{=}\StringTok{\textquotesingle{}exquisita\textquotesingle{}}\NormalTok{, sustantivo}\OperatorTok{=}\StringTok{\textquotesingle{}comida\textquotesingle{}}\NormalTok{))}
\end{Highlighting}
\end{Shaded}

\begin{verbatim}
Esta comida es exquisita.
\end{verbatim}

\begin{Shaded}
\begin{Highlighting}[]
\CommentTok{\# Se pueden combinar números con nombres de argumentos}
\BuiltInTok{print}\NormalTok{(}\StringTok{\textquotesingle{}El }\SpecialCharTok{\{0\}}\StringTok{, el }\SpecialCharTok{\{1\}}\StringTok{, y el }\SpecialCharTok{\{otro\}}\StringTok{.\textquotesingle{}}\NormalTok{.}\BuiltInTok{format}\NormalTok{(}\StringTok{\textquotesingle{}Bueno\textquotesingle{}}\NormalTok{, }\StringTok{\textquotesingle{}Malo\textquotesingle{}}\NormalTok{, otro}\OperatorTok{=}\StringTok{\textquotesingle{}Feo\textquotesingle{}}\NormalTok{))}
\end{Highlighting}
\end{Shaded}

\begin{verbatim}
El Bueno, el Malo, y el Feo.
\end{verbatim}

\begin{Shaded}
\begin{Highlighting}[]
\NormalTok{gatos }\OperatorTok{=}\NormalTok{ \{}\StringTok{\textquotesingle{}Siamés\textquotesingle{}}\NormalTok{: }\DecValTok{5}\NormalTok{, }\StringTok{\textquotesingle{}Siberiano\textquotesingle{}}\NormalTok{: }\DecValTok{4}\NormalTok{, }\StringTok{\textquotesingle{}Sphynx\textquotesingle{}}\NormalTok{: }\DecValTok{0}\NormalTok{\}}

\CommentTok{\# Podemos usar las características de los diccionarios}
\CommentTok{\# para imprimir la salida.}
\BuiltInTok{print}\NormalTok{(}\StringTok{\textquotesingle{}Sphynx: }\SpecialCharTok{\{g[Sphynx]:d\}}\StringTok{; Siamés: }\SpecialCharTok{\{g[Siamés]:d\}}\StringTok{; Siberiano: }\SpecialCharTok{\{g[Siberiano]:d\}}\StringTok{\textquotesingle{}}\NormalTok{.}\BuiltInTok{format}\NormalTok{(g }\OperatorTok{=}\NormalTok{ gatos))}
\end{Highlighting}
\end{Shaded}

\begin{verbatim}
Sphynx: 0; Siamés: 5; Siberiano: 4
\end{verbatim}

\begin{Shaded}
\begin{Highlighting}[]
\CommentTok{\# Una manera más entendible:}
\BuiltInTok{print}\NormalTok{(}\StringTok{\textquotesingle{}Sphynx: }\SpecialCharTok{\{Sphynx:d\}}\StringTok{; Siamés: \{Siamés:d\}; Siberiano: }\SpecialCharTok{\{Siberiano:d\}}\StringTok{\textquotesingle{}}\NormalTok{.}\BuiltInTok{format}\NormalTok{(}\OperatorTok{**}\NormalTok{gatos))}
\end{Highlighting}
\end{Shaded}

\begin{verbatim}
Sphynx: 0; Siamés: 5; Siberiano: 4
\end{verbatim}

\begin{Shaded}
\begin{Highlighting}[]
\ControlFlowTok{for}\NormalTok{ x }\KeywordTok{in} \BuiltInTok{range}\NormalTok{(}\DecValTok{1}\NormalTok{, }\DecValTok{11}\NormalTok{):}
    \BuiltInTok{print}\NormalTok{(}\StringTok{\textquotesingle{}}\SpecialCharTok{\{0:2d\}}\StringTok{ }\SpecialCharTok{\{1:3d\}}\StringTok{ }\SpecialCharTok{\{2:4d\}}\StringTok{\textquotesingle{}}\NormalTok{.}\BuiltInTok{format}\NormalTok{(x, x}\OperatorTok{*}\NormalTok{x, x}\OperatorTok{*}\NormalTok{x}\OperatorTok{*}\NormalTok{x))}
\end{Highlighting}
\end{Shaded}

\begin{verbatim}
 1   1    1
 2   4    8
 3   9   27
 4  16   64
 5  25  125
 6  36  216
 7  49  343
 8  64  512
 9  81  729
10 100 1000
\end{verbatim}

Para más información véase Format String Syntax

\section{Forma antigua de formatear la
salida}\label{forma-antigua-de-formatear-la-salida}

Se puede seguir usando la forma antigua de la salida.

\begin{Shaded}
\begin{Highlighting}[]
\ImportTok{import}\NormalTok{ math}
\BuiltInTok{print}\NormalTok{(}\StringTok{\textquotesingle{}El valor aproximado de pi es }\SpecialCharTok{\%5.6f}\StringTok{.\textquotesingle{}} \OperatorTok{\%}\NormalTok{ math.pi)}
\end{Highlighting}
\end{Shaded}

\begin{verbatim}
El valor aproximado de pi es 3.141593.
\end{verbatim}

\bookmarksetup{startatroot}

\chapter{Gestión de archivos y gestores de
contexto.}\label{gestiuxf3n-de-archivos-y-gestores-de-contexto.}

\textbf{Objetivo.} \ldots{}

\textbf{Funciones de Python}: \ldots{}

MACTI-Algebra\_Lineal\_01 by Luis M. de la Cruz is licensed under
Attribution-ShareAlike 4.0 International

\bookmarksetup{startatroot}

\chapter{Gestión de archivos}\label{gestiuxf3n-de-archivos}

La función para gestionar archivos es \texttt{open()}. Toma dos
parámetros: el nombre del archivo y el modo. Existen cuatro diferentes
modos:

\begin{itemize}
\tightlist
\item
  \texttt{"r"} - Read - Default value. Abre el archivo para lectura. Se
  produce un error si el archvio no existe.
\item
  \texttt{"a"} - Append - Abre el archivo para agregar información. Si
  el archivo no existe, lo crea.
\item
  \texttt{"w"} - Write - Abre el archivo para escritura. Si el archivo
  no existe, lo crea. Si el archivo existe, lo sobreescribe.
\item
  \texttt{"x"} - Create - Crea el archivo, regresa un error si el
  archivo existe.
\end{itemize}

Adicionalmente se puede especificar si el archivo se abre en modo texto
o binario:

\begin{itemize}
\tightlist
\item
  \texttt{"t"} - Text - Valor por omisión.
\item
  \texttt{"b"} - Binary
\end{itemize}

\begin{Shaded}
\begin{Highlighting}[]
\CommentTok{\# Abrimos el archivo gatos.txt en modo escritura}
\CommentTok{\# Ojo: si el archivo existe, lo sobreescribe.}
\NormalTok{f }\OperatorTok{=} \BuiltInTok{open}\NormalTok{(}\StringTok{\textquotesingle{}gatos.txt\textquotesingle{}}\NormalTok{, }\StringTok{\textquotesingle{}w\textquotesingle{}}\NormalTok{)}
\end{Highlighting}
\end{Shaded}

\begin{Shaded}
\begin{Highlighting}[]
\CommentTok{\# Construimos un diccionario con información}
\NormalTok{gatos }\OperatorTok{=}\NormalTok{ [}\StringTok{\textquotesingle{}Persa\textquotesingle{}}\NormalTok{, }\StringTok{\textquotesingle{}Sphynx\textquotesingle{}}\NormalTok{, }\StringTok{\textquotesingle{}Ragdoll\textquotesingle{}}\NormalTok{,}\StringTok{\textquotesingle{}Siamés\textquotesingle{}}\NormalTok{]}
\NormalTok{peso\_minimo }\OperatorTok{=}\NormalTok{ [}\FloatTok{2.3}\NormalTok{, }\FloatTok{3.5}\NormalTok{, }\FloatTok{5.4}\NormalTok{, }\FloatTok{2.5}\NormalTok{]}
\NormalTok{dicc }\OperatorTok{=} \BuiltInTok{dict}\NormalTok{(}\BuiltInTok{zip}\NormalTok{(gatos,peso\_minimo))}
\BuiltInTok{print}\NormalTok{(dicc)}
\end{Highlighting}
\end{Shaded}

\begin{verbatim}
{'Persa': 2.3, 'Sphynx': 3.5, 'Ragdoll': 5.4, 'Siamés': 2.5}
\end{verbatim}

\begin{Shaded}
\begin{Highlighting}[]
\CommentTok{\# Guardamos cada elemento del diccionario en el archivo}
\ControlFlowTok{for}\NormalTok{ i }\KeywordTok{in}\NormalTok{ dicc:}
\NormalTok{    f.write(}\StringTok{\textquotesingle{}}\SpecialCharTok{\{:\textgreater{}10\}}\StringTok{ }\SpecialCharTok{\{:\textgreater{}10\}}\StringTok{ }\CharTok{\textbackslash{}n}\StringTok{\textquotesingle{}}\NormalTok{.}\BuiltInTok{format}\NormalTok{(i,dicc[i]))}

\CommentTok{\# Es importante siempre cerrar el archivo cuando ya no se use.}
\NormalTok{f.close()}
\end{Highlighting}
\end{Shaded}

\begin{Shaded}
\begin{Highlighting}[]
\CommentTok{\# Ahora abrimos el archivo en modo lectura}
\CommentTok{\# El archivo debe existir, de otro modo se genera un error.}
\NormalTok{f }\OperatorTok{=} \BuiltInTok{open}\NormalTok{(}\StringTok{\textquotesingle{}gatos.txt\textquotesingle{}}\NormalTok{,}\StringTok{\textquotesingle{}r\textquotesingle{}}\NormalTok{)}
\ControlFlowTok{for}\NormalTok{ i }\KeywordTok{in}\NormalTok{ f:}
    \BuiltInTok{print}\NormalTok{(i)}
    
\NormalTok{f.close()}
\end{Highlighting}
\end{Shaded}

\begin{verbatim}
     Persa        2.3 

    Sphynx        3.5 

   Ragdoll        5.4 

    Siamés        2.5 
\end{verbatim}

\bookmarksetup{startatroot}

\chapter{Gestores de contexto}\label{gestores-de-contexto}

Permiten asignar y liberar recursos justo en el momento requerido. Se
usa mayormente con \texttt{with}, veamos un ejemplo:

\begin{Shaded}
\begin{Highlighting}[]
\ControlFlowTok{with} \BuiltInTok{open}\NormalTok{(}\StringTok{\textquotesingle{}contexto\_ejemplo\textquotesingle{}}\NormalTok{, }\StringTok{\textquotesingle{}w\textquotesingle{}}\NormalTok{) }\ImportTok{as}\NormalTok{ archivo\_abierto:}
\NormalTok{    archivo\_abierto.write(}\StringTok{\textquotesingle{}En este ejemplo, todo se realiza con un gestor de contexto.\textquotesingle{}}\NormalTok{)}
\end{Highlighting}
\end{Shaded}

Lo que realiza la operación anterior es: 1. Abre el archivo
\texttt{contexto\_ejemplo}. 2. Escribe algo en el archivo, 3. Cierra el
archivo. Si ocurre algún error mientras se escribe en el archivo,
entonces se intenta cerrar el archivo.

El código es equivalente a:

\begin{Shaded}
\begin{Highlighting}[]
\NormalTok{archivo\_abierto }\OperatorTok{=} \BuiltInTok{open}\NormalTok{(}\StringTok{\textquotesingle{}contexto\_ejemplo\textquotesingle{}}\NormalTok{, }\StringTok{\textquotesingle{}w\textquotesingle{}}\NormalTok{)}
\ControlFlowTok{try}\NormalTok{:}
\NormalTok{    archivo\_abierto.write(}\StringTok{\textquotesingle{}En este ... contexto!\textquotesingle{}}\NormalTok{)}
\ControlFlowTok{finally}\NormalTok{:}
\NormalTok{    archivo\_abierto.close()}
\end{Highlighting}
\end{Shaded}

Es posible implemetar nuestros propios gestores de contexto. Para ello
se requiere un conocimiento más profundo de Programación Orientada a
Objetos.

\bookmarksetup{startatroot}

\chapter{Funciones y docstring.}\label{funciones-y-docstring.}

\textbf{Objetivo.} \ldots{}

\textbf{Funciones de Python}: \ldots{}

MACTI-Algebra\_Lineal\_01 by Luis M. de la Cruz is licensed under
Attribution-ShareAlike 4.0 International

\bookmarksetup{startatroot}

\chapter{Definición de funciones}\label{definiciuxf3n-de-funciones}

Las funciones son la primera forma de estructurar un programa. Esto nos
lleva al paradigma de programación estructurada, junto con las
construcciones de control de flujo. Las funciones nos permiten agrupar y
reutilizar líneas de código.

La sintáxis es:

\begin{Shaded}
\begin{Highlighting}[]
\KeywordTok{def}\NormalTok{ nombre\_de\_la\_función(parm1, parm2, ...):}
\NormalTok{    bloque\_de\_código}
    \ControlFlowTok{return}\NormalTok{ resultado}
\end{Highlighting}
\end{Shaded}

Una vez definida la función, es posible ejecutarla (hacer una llamada a
la función) como sigue:

\begin{Shaded}
\begin{Highlighting}[]
\NormalTok{nombre\_de\_la\_función(arg1, arg2, ...)}
\end{Highlighting}
\end{Shaded}

También es posible hacer lo siguiente:

En ambos casos, la función regresa un resultado debido a que existe la
declaración \texttt{return} dentro de la función. Este resultado puede
ser referenciado por una variable haciendo lo siguiente:

\begin{Shaded}
\begin{Highlighting}[]
\NormalTok{variable }\OperatorTok{=}\NormalTok{ nombre\_de\_la\_función(arg1, arg2, ...)}
\end{Highlighting}
\end{Shaded}

La \texttt{variable} puede ser utilizada posteriormente para otros
cálculos.

Observa que: * Cuando se define la función, se definen los
\textbf{parámetros} que recibirá, en este caso \texttt{param1},
\texttt{param2}, \texttt{...} * Cuando se ejecuta la función, se pasan
los valores los \texttt{arg1}, \texttt{arg2}, \texttt{...}, los cuales
son los \textbf{argumentos} de la ejecución y serán sustituidos en los
parámetros de la función.

Veamos un ejemplo simple:

\begin{Shaded}
\begin{Highlighting}[]
\CommentTok{\# Función que calcula el cuadrado de su argumento.}
\KeywordTok{def}\NormalTok{ squared(f):}
    \ControlFlowTok{return}\NormalTok{ f }\OperatorTok{**} \DecValTok{2}
\end{Highlighting}
\end{Shaded}

\begin{Shaded}
\begin{Highlighting}[]
\CommentTok{\# Se ejecuta la función con el argumento 2}
\NormalTok{squared(}\DecValTok{2}\NormalTok{) }
\end{Highlighting}
\end{Shaded}

\begin{verbatim}
4
\end{verbatim}

\begin{Shaded}
\begin{Highlighting}[]
\CommentTok{\# Se ejecuta la función con el argumento 3}
\CommentTok{\# el resultado se almacena en f2}
\NormalTok{f2 }\OperatorTok{=}\NormalTok{ squared(}\DecValTok{3}\NormalTok{)}
\BuiltInTok{print}\NormalTok{(f2)}
\end{Highlighting}
\end{Shaded}

\begin{verbatim}
9
\end{verbatim}

Veamos ahora un ejemplo más interesante

\begin{Shaded}
\begin{Highlighting}[]
\CommentTok{\# La siguiente función calcula la secuencia de Fibonacci}
\KeywordTok{def}\NormalTok{ fib(n):  }\CommentTok{\# La función se llama fib y tiene el parámetro n}
\NormalTok{    a, b }\OperatorTok{=} \DecValTok{0}\NormalTok{, }\DecValTok{1}
    \ControlFlowTok{while}\NormalTok{ a }\OperatorTok{\textless{}}\NormalTok{ n:}
        \BuiltInTok{print}\NormalTok{(a, end}\OperatorTok{=}\StringTok{\textquotesingle{},\textquotesingle{}}\NormalTok{)}
\NormalTok{        a, b }\OperatorTok{=}\NormalTok{ b, a}\OperatorTok{+}\NormalTok{b}
\end{Highlighting}
\end{Shaded}

Observa que esta función no regresa ningún valor, solo imprime en
pantalla un valor conforme lo calcula.

\begin{Shaded}
\begin{Highlighting}[]
\NormalTok{fib(}\DecValTok{50}\NormalTok{) }\CommentTok{\# ejecutamos la función fib con el argumento 10}
\end{Highlighting}
\end{Shaded}

\begin{verbatim}
0,1,1,2,3,5,8,13,21,34,
\end{verbatim}

Le podemos poner otro nombre a la función

\begin{Shaded}
\begin{Highlighting}[]
\NormalTok{Fibonacci }\OperatorTok{=}\NormalTok{ fib}
\end{Highlighting}
\end{Shaded}

\begin{Shaded}
\begin{Highlighting}[]
\NormalTok{Fibonacci(}\DecValTok{200}\NormalTok{)}
\end{Highlighting}
\end{Shaded}

\begin{verbatim}
0,1,1,2,3,5,8,13,21,34,55,89,144,
\end{verbatim}

\begin{Shaded}
\begin{Highlighting}[]
\BuiltInTok{print}\NormalTok{(}\BuiltInTok{type}\NormalTok{(fib))}
\BuiltInTok{print}\NormalTok{(}\BuiltInTok{type}\NormalTok{(Fibonacci))}
\end{Highlighting}
\end{Shaded}

\begin{verbatim}
<class 'function'>
<class 'function'>
\end{verbatim}

\begin{Shaded}
\begin{Highlighting}[]
\BuiltInTok{print}\NormalTok{(}\BuiltInTok{id}\NormalTok{(fib))}
\BuiltInTok{print}\NormalTok{(}\BuiltInTok{id}\NormalTok{(Fibonacci))}
\end{Highlighting}
\end{Shaded}

\begin{verbatim}
140670158000416
140670158000416
\end{verbatim}

Observamos que se puede ejecutar la función \texttt{fib()} a través de
\texttt{Fibonacci()} y que ambos nombres hacen referencia a la misma
función.

\bookmarksetup{startatroot}

\chapter{Ámbitos}\label{uxe1mbitos}

Las funciones (y otros operadores también), crean su propio ámbito, de
tal manera que las etiquetas declaradas dentro de funciones son locales.

\begin{Shaded}
\begin{Highlighting}[]
\NormalTok{a }\OperatorTok{=} \DecValTok{20} \CommentTok{\# Objeto global etiquetado con a}

\KeywordTok{def}\NormalTok{ f():}
\NormalTok{    a }\OperatorTok{=} \DecValTok{21} \CommentTok{\# Objeto local etiquetado con a}
    \ControlFlowTok{return}\NormalTok{ a}
\end{Highlighting}
\end{Shaded}

\begin{Shaded}
\begin{Highlighting}[]
\CommentTok{\# ¿Que valor tiene \textquotesingle{}a\textquotesingle{} fuera de la función?}
\BuiltInTok{print}\NormalTok{(a)}
\end{Highlighting}
\end{Shaded}

\begin{verbatim}
20
\end{verbatim}

\begin{Shaded}
\begin{Highlighting}[]
\CommentTok{\# ¿Qué valor tiene la \textquotesingle{}a\textquotesingle{} dentro de la función?}
\BuiltInTok{print}\NormalTok{(f())}
\end{Highlighting}
\end{Shaded}

\begin{verbatim}
20
\end{verbatim}

Para usar el objeto global dentro de la función debemos usar
\texttt{global}

\begin{Shaded}
\begin{Highlighting}[]
\NormalTok{a }\OperatorTok{=} \DecValTok{20}

\KeywordTok{def}\NormalTok{ f():}
    \KeywordTok{global}\NormalTok{ a }
    \ControlFlowTok{return}\NormalTok{ a}
\end{Highlighting}
\end{Shaded}

\begin{Shaded}
\begin{Highlighting}[]
\CommentTok{\# ¿Que valor tiene \textquotesingle{}a\textquotesingle{} fuera de la función?}
\BuiltInTok{print}\NormalTok{(a)}
\end{Highlighting}
\end{Shaded}

\begin{verbatim}
20
\end{verbatim}

\begin{Shaded}
\begin{Highlighting}[]
\CommentTok{\# La \textquotesingle{}a\textquotesingle{} dentro de la función hace referencia a la \textquotesingle{}a\textquotesingle{} global}
\BuiltInTok{print}\NormalTok{(f())}
\end{Highlighting}
\end{Shaded}

\begin{verbatim}
20
\end{verbatim}

\bookmarksetup{startatroot}

\chapter{Retorno de una función}\label{retorno-de-una-funciuxf3n}

Como se mencionó antes, la declaración \texttt{return}, dentro de una
función, regresa un objeto que en principio contiene el resultado de las
operaciones realizadas por la función.

Veamos un ejemplo.

\begin{Shaded}
\begin{Highlighting}[]
\NormalTok{g }\OperatorTok{=} \FloatTok{9.81}
\CommentTok{\# Función que calcula la posición y velocidad en el tiro vertical de un objeto.}
\KeywordTok{def}\NormalTok{ verticalThrow(t, v0): }
\NormalTok{    g }\OperatorTok{=} \FloatTok{3.1416} \CommentTok{\# [m / s**2]}
\NormalTok{    y }\OperatorTok{=}\NormalTok{ v0 }\OperatorTok{*}\NormalTok{ t }\OperatorTok{{-}} \FloatTok{0.5} \OperatorTok{*}\NormalTok{ g }\OperatorTok{*}\NormalTok{ t}\OperatorTok{**}\DecValTok{2} 
\NormalTok{    v }\OperatorTok{=}\NormalTok{ v0 }\OperatorTok{{-}}\NormalTok{ g }\OperatorTok{*}\NormalTok{ t }
    \ControlFlowTok{return}\NormalTok{ (y, v)  }\CommentTok{\# regresa la posición [m] y la velocidad [m/s] en un objeto de tipo tupla}
\end{Highlighting}
\end{Shaded}

\begin{Shaded}
\begin{Highlighting}[]
\NormalTok{t }\OperatorTok{=} \FloatTok{2.0}   \CommentTok{\# [s]}
\NormalTok{v0 }\OperatorTok{=} \DecValTok{20}   \CommentTok{\# [m/s]}
\NormalTok{verticalThrow(t, v0)}
\end{Highlighting}
\end{Shaded}

\begin{verbatim}
(33.7168, 13.7168)
\end{verbatim}

\begin{Shaded}
\begin{Highlighting}[]
\NormalTok{resultado }\OperatorTok{=}\NormalTok{ verticalThrow(t, v0)}
\end{Highlighting}
\end{Shaded}

\begin{Shaded}
\begin{Highlighting}[]
\BuiltInTok{print}\NormalTok{(resultado)}
\end{Highlighting}
\end{Shaded}

\begin{verbatim}
(33.7168, 13.7168)
\end{verbatim}

\bookmarksetup{startatroot}

\chapter{Argumentos por omisión}\label{argumentos-por-omisiuxf3n}

Los parámetros de una función pueden tener valores (argumentos) por
omisión, es decir, si no se da un valor para uno de los parámetros,
entonces se toma el valor definido por omisión. Esto crea una función
que se puede llamar con menos argumentos de los que está definida
inicialmente.

Por ejemplo:

\begin{Shaded}
\begin{Highlighting}[]
\CommentTok{\# Función que calcula la posición y velocidad en el tiro vertical de un objeto.}
\KeywordTok{def}\NormalTok{ verticalThrow(t, v0 }\OperatorTok{=} \DecValTok{20}\NormalTok{):  }\CommentTok{\# El valor 20 es un argumento por omisión}
\NormalTok{    g }\OperatorTok{=} \FloatTok{9.81} \CommentTok{\# [m / s**2]}
\NormalTok{    y }\OperatorTok{=}\NormalTok{ v0 }\OperatorTok{*}\NormalTok{ t }\OperatorTok{{-}} \FloatTok{0.5} \OperatorTok{*}\NormalTok{ g }\OperatorTok{*}\NormalTok{ t}\OperatorTok{**}\DecValTok{2} 
\NormalTok{    v }\OperatorTok{=}\NormalTok{ v0 }\OperatorTok{{-}}\NormalTok{ g }\OperatorTok{*}\NormalTok{ t }
    \ControlFlowTok{return}\NormalTok{ (y, v)}
\end{Highlighting}
\end{Shaded}

\begin{Shaded}
\begin{Highlighting}[]
\NormalTok{pos, vel }\OperatorTok{=}\NormalTok{ verticalThrow(}\FloatTok{2.0}\NormalTok{) }\CommentTok{\# El valor 2.0 corresponde al primer parámetro de la función, \textquotesingle{}t\textquotesingle{}}
                              \CommentTok{\# En este caso v0 será igual a 20.}
\BuiltInTok{print}\NormalTok{(pos, vel)}
\end{Highlighting}
\end{Shaded}

\begin{verbatim}
20.38 0.379999999999999
\end{verbatim}

\begin{Shaded}
\begin{Highlighting}[]
\NormalTok{pos, vel }\OperatorTok{=}\NormalTok{ verticalThrow(}\FloatTok{2.0}\NormalTok{, }\DecValTok{30}\NormalTok{) }\CommentTok{\# En este caso v0 = 30}
\BuiltInTok{print}\NormalTok{(pos, vel)}
\end{Highlighting}
\end{Shaded}

\begin{verbatim}
40.379999999999995 10.379999999999999
\end{verbatim}

Una función puede tener más de un argumento por omisión. Todos los
parámetros que tienen argumentos por omisión deben estar al final de la
lista en la declaración de la función.

Por ejemplo:

\begin{Shaded}
\begin{Highlighting}[]
\KeywordTok{def}\NormalTok{ f(a,b,c,d}\OperatorTok{=}\DecValTok{10}\NormalTok{,e}\OperatorTok{=}\DecValTok{20}\NormalTok{):}
    \ControlFlowTok{return}\NormalTok{ a}\OperatorTok{+}\NormalTok{b}\OperatorTok{+}\NormalTok{c}\OperatorTok{+}\NormalTok{d}\OperatorTok{+}\NormalTok{e}
\end{Highlighting}
\end{Shaded}

\begin{Shaded}
\begin{Highlighting}[]
\BuiltInTok{print}\NormalTok{(f(}\DecValTok{1}\NormalTok{,}\DecValTok{2}\NormalTok{,}\DecValTok{3}\NormalTok{))  }\CommentTok{\# Se los dos argumentos por omisión 10 y 20}
\end{Highlighting}
\end{Shaded}

\begin{verbatim}
36
\end{verbatim}

\begin{Shaded}
\begin{Highlighting}[]
\BuiltInTok{print}\NormalTok{(f(}\DecValTok{1}\NormalTok{,}\DecValTok{2}\NormalTok{,}\DecValTok{3}\NormalTok{,}\DecValTok{4}\NormalTok{)) }\CommentTok{\# Se usa el último argumento por omisión 20}
\end{Highlighting}
\end{Shaded}

\begin{verbatim}
30
\end{verbatim}

\begin{Shaded}
\begin{Highlighting}[]
\BuiltInTok{print}\NormalTok{(f(}\DecValTok{1}\NormalTok{,}\DecValTok{2}\NormalTok{,}\DecValTok{3}\NormalTok{,}\DecValTok{4}\NormalTok{,}\DecValTok{5}\NormalTok{)) }\CommentTok{\# Se dan todos los argumentos.}
\end{Highlighting}
\end{Shaded}

\begin{verbatim}
15
\end{verbatim}

\bookmarksetup{startatroot}

\chapter{\texorpdfstring{Argumentos posicionales y
\texttt{keyword}}{Argumentos posicionales y keyword}}\label{argumentos-posicionales-y-keyword}

Un \textbf{argumento} es el valor que se le pasa a una función cuando se
llama. Hay dos tipos de argumentos:

\textbf{\emph{Positional argument}} :

\begin{enumerate}
\def\labelenumi{\arabic{enumi}.}
\item
  Un argumento que no es precedido por un identificador:
  \texttt{verticalThrow(3,\ 50)}
\item
  Un argumento que es pasado en una tupla precedido por \texttt{*}:
  \texttt{verticalThrow(*(3,\ 50))}.
\end{enumerate}

En este caso, el \texttt{*} indica a Python que la tupla
\texttt{(3,\ 50)} debe desempacarse cuando se reciba en la función, de
tal manera que \texttt{3} será el primer argumento y \texttt{5} el
segundo.

La llamada a la función del punto 2 es equivalente a la del punto 1.

\textbf{\emph{Keyword argument}} :

\begin{enumerate}
\def\labelenumi{\arabic{enumi}.}
\setcounter{enumi}{2}
\item
  Un argumento precedido por un identificador.
  \texttt{verticalThrow(t=3,\ v0=50)}
\item
  Un argumento que se pasa en un diccionario precedido por
  \texttt{**}:\texttt{verticalThrow(**\{\textquotesingle{}t\textquotesingle{}:\ 3,\ \textquotesingle{}v0\textquotesingle{}:\ 50\})}.
\end{enumerate}

En este caso, el \texttt{**} indica a Python que el diccionario
\texttt{\{\textquotesingle{}t\textquotesingle{}:\ 3,\ \textquotesingle{}v0\textquotesingle{}:\ 50\}}
debe desempacarse cuando se reciba en la función. Observa que el
diccionario contiene dos pares clave-valor:
\texttt{\textquotesingle{}t\textquotesingle{}:\ 3} y
\texttt{\textquotesingle{}v0\textquotesingle{}:\ 50}.

La llamada a la función del punto 4 es equivalente a la del punto 3.

Veamos los ejemplos en código:

Primer recordemos que la firma de la función es
\texttt{def\ verticalThrow(t,\ v0\ =\ 20):} es decir, el primer
parámetro es \texttt{t} y el segundo \texttt{v0}.

\begin{Shaded}
\begin{Highlighting}[]
\CommentTok{\# Los argumentos se sustituyen en los parámetros en el orden de acuerdo a su posición.}
\NormalTok{verticalThrow(}\DecValTok{3}\NormalTok{,}\DecValTok{50}\NormalTok{)}
\end{Highlighting}
\end{Shaded}

\begin{verbatim}
(105.85499999999999, 20.57)
\end{verbatim}

\begin{Shaded}
\begin{Highlighting}[]
\CommentTok{\# Lo anterior NO es equivalente a:}
\NormalTok{verticalThrow(}\DecValTok{50}\NormalTok{,}\DecValTok{3}\NormalTok{)}
\end{Highlighting}
\end{Shaded}

\begin{verbatim}
(-12112.5, -487.5)
\end{verbatim}

\begin{Shaded}
\begin{Highlighting}[]
\CommentTok{\# Se pueden pasar los argumentos empacados en una tupla}
\NormalTok{verticalThrow(}\OperatorTok{*}\NormalTok{(}\DecValTok{3}\NormalTok{,}\DecValTok{50}\NormalTok{))}
\end{Highlighting}
\end{Shaded}

\begin{verbatim}
(105.85499999999999, 20.57)
\end{verbatim}

\begin{Shaded}
\begin{Highlighting}[]
\CommentTok{\# Se puede usar el nombre del parámetro para determinar}
\CommentTok{\# como se sustituyen los argumentos:}
\NormalTok{verticalThrow(t}\OperatorTok{=}\DecValTok{3}\NormalTok{,v0}\OperatorTok{=}\DecValTok{50}\NormalTok{)}
\end{Highlighting}
\end{Shaded}

\begin{verbatim}
(105.85499999999999, 20.57)
\end{verbatim}

\begin{Shaded}
\begin{Highlighting}[]
\CommentTok{\# Lo anterior SI es equivalente a:}
\NormalTok{verticalThrow(v0}\OperatorTok{=}\DecValTok{50}\NormalTok{, t}\OperatorTok{=}\DecValTok{3}\NormalTok{)}
\end{Highlighting}
\end{Shaded}

\begin{verbatim}
(105.85499999999999, 20.57)
\end{verbatim}

\begin{Shaded}
\begin{Highlighting}[]
\CommentTok{\# Se pueden pasar los argumentos empacados en un diccionario}
\NormalTok{verticalThrow(}\OperatorTok{**}\NormalTok{\{}\StringTok{\textquotesingle{}t\textquotesingle{}}\NormalTok{:}\DecValTok{3}\NormalTok{,}\StringTok{\textquotesingle{}v0\textquotesingle{}}\NormalTok{:}\DecValTok{50}\NormalTok{\})}
\end{Highlighting}
\end{Shaded}

\begin{verbatim}
(105.85499999999999, 20.57)
\end{verbatim}

\begin{Shaded}
\begin{Highlighting}[]
\CommentTok{\# También se acepta esta forma:}
\NormalTok{verticalThrow(}\OperatorTok{**}\BuiltInTok{dict}\NormalTok{(t }\OperatorTok{=} \DecValTok{3}\NormalTok{,v0 }\OperatorTok{=} \DecValTok{50}\NormalTok{))}
\end{Highlighting}
\end{Shaded}

\begin{verbatim}
(105.85499999999999, 20.57)
\end{verbatim}

\bookmarksetup{startatroot}

\chapter{Número variable de
parámetros}\label{nuxfamero-variable-de-paruxe1metros}

Dada la funcionalidad descrita en la sección anterior, es posible que
una función reciba un número variable de argumentos.

\begin{Shaded}
\begin{Highlighting}[]
\CommentTok{\# *args: número variable de Positional arguments empacados en una tupla}
\CommentTok{\# *kwargs: número variable de Keyword arguments empacados en un diccionario}
\KeywordTok{def}\NormalTok{ parametrosVariables(}\OperatorTok{*}\NormalTok{args, }\OperatorTok{**}\NormalTok{kwargs):}
    \BuiltInTok{print}\NormalTok{(}\StringTok{\textquotesingle{}args es una tupla : \textquotesingle{}}\NormalTok{, args)}
    \BuiltInTok{print}\NormalTok{(}\StringTok{\textquotesingle{}kwargs es un diccionario: \textquotesingle{}}\NormalTok{, kwargs)}
    \BuiltInTok{print}\NormalTok{(}\BuiltInTok{set}\NormalTok{(kwargs))}
\end{Highlighting}
\end{Shaded}

\begin{Shaded}
\begin{Highlighting}[]
\NormalTok{parametrosVariables(}\StringTok{\textquotesingle{}one\textquotesingle{}}\NormalTok{, }\StringTok{\textquotesingle{}two\textquotesingle{}}\NormalTok{,}\StringTok{\textquotesingle{}three\textquotesingle{}}\NormalTok{, }\StringTok{\textquotesingle{}four\textquotesingle{}}\NormalTok{, a }\OperatorTok{=} \DecValTok{4}\NormalTok{,  x}\OperatorTok{=}\DecValTok{1}\NormalTok{, y}\OperatorTok{=}\DecValTok{2}\NormalTok{, z}\OperatorTok{=}\DecValTok{3}\NormalTok{, w}\OperatorTok{=}\NormalTok{[}\DecValTok{1}\NormalTok{,}\DecValTok{2}\NormalTok{,}\DecValTok{2}\NormalTok{])}
\end{Highlighting}
\end{Shaded}

\begin{verbatim}
args es una tupla :  ('one', 'two', 'three', 'four')
kwargs es un diccionario:  {'a': 4, 'x': 1, 'y': 2, 'z': 3, 'w': [1, 2, 2]}
{'a', 'y', 'w', 'z', 'x'}
\end{verbatim}

\begin{Shaded}
\begin{Highlighting}[]
\NormalTok{parametrosVariables(}\DecValTok{1}\NormalTok{,}\DecValTok{2}\NormalTok{,}\DecValTok{3}\NormalTok{, w}\OperatorTok{=}\DecValTok{8}\NormalTok{, y}\OperatorTok{=}\StringTok{\textquotesingle{}cadena\textquotesingle{}}\NormalTok{)}
\end{Highlighting}
\end{Shaded}

\begin{verbatim}
args es una tupla :  (1, 2, 3)
kwargs es un diccionario:  {'w': 8, 'y': 'cadena'}
{'w', 'y'}
\end{verbatim}

\begin{Shaded}
\begin{Highlighting}[]
\CommentTok{\# Los argumentos que vienen en un diccionario se desempacan}
\CommentTok{\#  y se pueden usar dentro de la función:}
\KeywordTok{def}\NormalTok{ funcion\_kargs(}\OperatorTok{**}\NormalTok{argumentos):}
    \ControlFlowTok{for}\NormalTok{ key, val }\KeywordTok{in}\NormalTok{ argumentos.items():}
        \BuiltInTok{print}\NormalTok{(}\SpecialStringTok{f" }\SpecialCharTok{\{}\NormalTok{key}\SpecialCharTok{\}}\SpecialStringTok{ : }\SpecialCharTok{\{}\NormalTok{val}\SpecialCharTok{\}}\SpecialStringTok{"}\NormalTok{)}
\end{Highlighting}
\end{Shaded}

\begin{Shaded}
\begin{Highlighting}[]
\NormalTok{funcion\_kargs(nombre }\OperatorTok{=} \StringTok{\textquotesingle{}Luis\textquotesingle{}}\NormalTok{, apellido}\OperatorTok{=}\StringTok{\textquotesingle{}de la Cruz\textquotesingle{}}\NormalTok{, edad}\OperatorTok{=}\DecValTok{15}\NormalTok{, peso}\OperatorTok{=}\FloatTok{80.5}\NormalTok{ )}
\end{Highlighting}
\end{Shaded}

\begin{verbatim}
 nombre : Luis
 apellido : de la Cruz
 edad : 15
 peso : 80.5
\end{verbatim}

\begin{Shaded}
\begin{Highlighting}[]
\NormalTok{funcion\_kargs(nombre }\OperatorTok{=} \StringTok{\textquotesingle{}Luis\textquotesingle{}}\NormalTok{, apellido}\OperatorTok{=}\StringTok{\textquotesingle{}de la Cruz\textquotesingle{}}\NormalTok{, estudios}\OperatorTok{=}\StringTok{\textquotesingle{}primaria\textquotesingle{}}\NormalTok{, edad}\OperatorTok{=}\DecValTok{15}\NormalTok{, peso}\OperatorTok{=}\FloatTok{80.5}\NormalTok{, num\_cuenta }\OperatorTok{=} \DecValTok{12334457}\NormalTok{ )}
\end{Highlighting}
\end{Shaded}

\begin{verbatim}
 nombre : Luis
 apellido : de la Cruz
 estudios : primaria
 edad : 15
 peso : 80.5
 num_cuenta : 12334457
\end{verbatim}

\begin{Shaded}
\begin{Highlighting}[]
\CommentTok{\# Se puede definir un diccionario}
\NormalTok{mi\_dicc }\OperatorTok{=}\NormalTok{ \{}\StringTok{\textquotesingle{}nombre\textquotesingle{}}\NormalTok{:}\StringTok{\textquotesingle{}Luis\textquotesingle{}}\NormalTok{, }\StringTok{\textquotesingle{}apellido\textquotesingle{}}\NormalTok{:}\StringTok{\textquotesingle{}de la Cruz\textquotesingle{}}\NormalTok{, }\StringTok{\textquotesingle{}edad\textquotesingle{}}\NormalTok{:}\DecValTok{15}\NormalTok{, }\StringTok{\textquotesingle{}peso\textquotesingle{}}\NormalTok{:}\FloatTok{80.5}\NormalTok{\}}
\end{Highlighting}
\end{Shaded}

\begin{Shaded}
\begin{Highlighting}[]
\CommentTok{\# Se usa el diccionario para llamar a la función}
\NormalTok{funcion\_kargs(}\OperatorTok{**}\NormalTok{mi\_dicc)}
\end{Highlighting}
\end{Shaded}

\begin{verbatim}
 nombre : Luis
 apellido : de la Cruz
 edad : 15
 peso : 80.5
\end{verbatim}

\bookmarksetup{startatroot}

\chapter{Funciones como parámetros de otras
funciones.}\label{funciones-como-paruxe1metros-de-otras-funciones.}

Las funciones pueden recibir como argumentos objetos muy complejos,
incluso otras funciones. Veamos un ejemplo simple:

\begin{Shaded}
\begin{Highlighting}[]
\CommentTok{\# Un función simple}
\KeywordTok{def}\NormalTok{ g():}
    \BuiltInTok{print}\NormalTok{(}\StringTok{"Iniciando la función \textquotesingle{}g()\textquotesingle{}"}\NormalTok{)}

\CommentTok{\# Una función que reibirá otra función:}
\KeywordTok{def}\NormalTok{ func(f):}
    \BuiltInTok{print}\NormalTok{(}\StringTok{"Iniciando la función \textquotesingle{}func()\textquotesingle{}"}\NormalTok{)}
    \BuiltInTok{print}\NormalTok{(}\StringTok{"Ejecución de la función \textquotesingle{}f()\textquotesingle{}, nombre real \textquotesingle{}"} \OperatorTok{+}\NormalTok{ f.}\VariableTok{\_\_name\_\_} \OperatorTok{+} \StringTok{"()\textquotesingle{}"}\NormalTok{)}
\NormalTok{    f() }\CommentTok{\# Se ejecuta la función que se recibió en el parámetro f}
\end{Highlighting}
\end{Shaded}

\begin{Shaded}
\begin{Highlighting}[]
\NormalTok{func(g)}
\end{Highlighting}
\end{Shaded}

\begin{verbatim}
Iniciando la función 'func()'
Ejecución de la función 'f()', nombre real 'g()'
Iniciando la función 'g()'
\end{verbatim}

\subsection{\texorpdfstring{\textbf{Ejemplo 1. Integración
numérica.}}{Ejemplo 1. Integración numérica.}}\label{ejemplo-1.-integraciuxf3n-numuxe9rica.}

En este ejemplo el objetivo es crear un función que recibirá como
argumentos la funcióm matemática a integrar, los límites de integración
y el número de puntos para realizar la integración. Regresará como
resultado un número que es la aproximación de la integral.

\begin{Shaded}
\begin{Highlighting}[]
\ImportTok{import}\NormalTok{ math}

\KeywordTok{def}\NormalTok{ integra(func,a,b,N):}
    \CommentTok{\# Se utiliza el método de Simpson para la integración.}
    \CommentTok{\# El parámetro \textquotesingle{}func\textquotesingle{} es la función a integrar}
    \BuiltInTok{print}\NormalTok{(}\SpecialStringTok{f"Integral de la función }\SpecialCharTok{\{}\NormalTok{func}\SpecialCharTok{.}\VariableTok{\_\_name\_\_}\SpecialCharTok{\}}\SpecialStringTok{() en el intervalo (}\SpecialCharTok{\{}\NormalTok{a}\SpecialCharTok{\}}\SpecialStringTok{,}\SpecialCharTok{\{}\NormalTok{b}\SpecialCharTok{\}}\SpecialStringTok{) usando }\SpecialCharTok{\{}\NormalTok{N}\SpecialCharTok{\}}\SpecialStringTok{ puntos"}\NormalTok{)}
\NormalTok{    h }\OperatorTok{=}\NormalTok{ (b }\OperatorTok{{-}}\NormalTok{ a) }\OperatorTok{/}\NormalTok{ N}
\NormalTok{    resultado }\OperatorTok{=} \DecValTok{0}
\NormalTok{    x }\OperatorTok{=}\NormalTok{ [a }\OperatorTok{+}\NormalTok{ h}\OperatorTok{*}\NormalTok{i }\ControlFlowTok{for}\NormalTok{ i }\KeywordTok{in} \BuiltInTok{range}\NormalTok{(N}\OperatorTok{+}\DecValTok{1}\NormalTok{)]}
    \ControlFlowTok{for}\NormalTok{ xi }\KeywordTok{in}\NormalTok{ x:}
\NormalTok{        resultado }\OperatorTok{+=}\NormalTok{ func(xi) }\OperatorTok{*}\NormalTok{ h}
    \ControlFlowTok{return}\NormalTok{ resultado}
\end{Highlighting}
\end{Shaded}

\begin{Shaded}
\begin{Highlighting}[]
\CommentTok{\# Integral de la función sin() de la biblioteca math.}
\BuiltInTok{print}\NormalTok{(integra(math.sin, }\DecValTok{0}\NormalTok{, math.pi, }\DecValTok{100}\NormalTok{))}

\CommentTok{\# Integral de la función cos() de la biblioteca math.}
\BuiltInTok{print}\NormalTok{(integra(math.cos, }\OperatorTok{{-}}\FloatTok{0.5} \OperatorTok{*}\NormalTok{ math.pi, }\FloatTok{0.5} \OperatorTok{*}\NormalTok{ math.pi, }\DecValTok{50}\NormalTok{))}
\end{Highlighting}
\end{Shaded}

\begin{verbatim}
Integral de la función sin() en el intervalo (0,3.141592653589793) usando 100 puntos
1.9998355038874436
Integral de la función cos() en el intervalo (-1.5707963267948966,1.5707963267948966) usando 50 puntos
1.9993419830762613
\end{verbatim}

\bookmarksetup{startatroot}

\chapter{Funciones que regresan otra
función.}\label{funciones-que-regresan-otra-funciuxf3n.}

Como vimos antes, una función puede regresar un objeto de cualquier
tipo, incluyendo una función. Veamos un ejemplo:

\begin{Shaded}
\begin{Highlighting}[]
\CommentTok{\# La funcionPadre() regresará como resultado una de dos funciones}
\CommentTok{\# definidas dentro de ella.}
\KeywordTok{def}\NormalTok{ funcionPadre(n):}

    \CommentTok{\# Se define la función 1}
    \KeywordTok{def}\NormalTok{ funcionHijo1():}
        \ControlFlowTok{return} \StringTok{"funcionHijo1(): n = }\SpecialCharTok{\{\}}\StringTok{"}\NormalTok{.}\BuiltInTok{format}\NormalTok{(n)}

    \CommentTok{\# Se define la función 2}
    \KeywordTok{def}\NormalTok{ funcionHijo2():}
        \ControlFlowTok{return} \StringTok{"funcionHijo2(): n = }\SpecialCharTok{\{\}}\StringTok{"}\NormalTok{.}\BuiltInTok{format}\NormalTok{(n)}

    \CommentTok{\# Se determina la función que se va a regresar}
    \ControlFlowTok{if}\NormalTok{ n }\OperatorTok{==} \DecValTok{10}\NormalTok{:}
        \ControlFlowTok{return}\NormalTok{ funcionHijo1}
    \ControlFlowTok{else}\NormalTok{:}
        \ControlFlowTok{return}\NormalTok{ funcionHijo2}
\end{Highlighting}
\end{Shaded}

\begin{Shaded}
\begin{Highlighting}[]
\CommentTok{\# La funcionPadre() regresa una función}
\NormalTok{funcionPadre(}\DecValTok{36}\NormalTok{)}
\end{Highlighting}
\end{Shaded}

\begin{verbatim}
<function __main__.funcionPadre.<locals>.funcionHijo2()>
\end{verbatim}

\begin{Shaded}
\begin{Highlighting}[]
\CommentTok{\# Asignamos el resultado de la funcionPadre() a un nombre}
\NormalTok{f1 }\OperatorTok{=}\NormalTok{ funcionPadre(}\DecValTok{10}\NormalTok{)}
\NormalTok{f2 }\OperatorTok{=}\NormalTok{ funcionPadre(}\DecValTok{20}\NormalTok{)}
\end{Highlighting}
\end{Shaded}

\begin{Shaded}
\begin{Highlighting}[]
\BuiltInTok{print}\NormalTok{(f1()) }\CommentTok{\# Resultado de la funcionf1(), generada con la funcionPadre()}
\BuiltInTok{print}\NormalTok{(f2()) }\CommentTok{\# Resultado de la funcionf2(), generada con la funcionPadre()}
\end{Highlighting}
\end{Shaded}

\begin{verbatim}
funcionHijo2(): n = 20
funcionHijo1(): n = 10
\end{verbatim}

\subsection{\texorpdfstring{\textbf{Ejemplo 2. Polinomios de segundo
grado.}}{Ejemplo 2. Polinomios de segundo grado.}}\label{ejemplo-2.-polinomios-de-segundo-grado.}

Implementar una fábrica de polinomios de segundo grado:

\[
p(x) = a x^2 + b x + c
\]

\begin{Shaded}
\begin{Highlighting}[]
\CommentTok{\# Esta función recibe los coeficientes del polinomio}
\CommentTok{\# y regresa una función que calcula el polinomio de}
\CommentTok{\# segundo grado.}
\KeywordTok{def}\NormalTok{ polinomio(a, b, c):}
    
    \KeywordTok{def}\NormalTok{ polSegundoGrado(x):}
        \ControlFlowTok{return}\NormalTok{ a }\OperatorTok{*}\NormalTok{ x}\OperatorTok{**}\DecValTok{2} \OperatorTok{+}\NormalTok{ b }\OperatorTok{*}\NormalTok{ x }\OperatorTok{+}\NormalTok{ c}
    
    \ControlFlowTok{return}\NormalTok{ polSegundoGrado}
\end{Highlighting}
\end{Shaded}

\begin{Shaded}
\begin{Highlighting}[]
\CommentTok{\# Dos polinomios de segundo grado}
\NormalTok{p1 }\OperatorTok{=}\NormalTok{ polinomio(}\DecValTok{2}\NormalTok{, }\DecValTok{3}\NormalTok{, }\OperatorTok{{-}}\DecValTok{1}\NormalTok{) }\CommentTok{\# 2x\^{}2 + 3x {-} 1}
\NormalTok{p2 }\OperatorTok{=}\NormalTok{ polinomio(}\OperatorTok{{-}}\DecValTok{1}\NormalTok{, }\DecValTok{2}\NormalTok{, }\DecValTok{1}\NormalTok{) }\CommentTok{\# {-}x\^{}2 + 2x + 1}

\CommentTok{\# Evaluación de los polinomios en el intervalo}
\CommentTok{\# ({-}2,2) con pasos de 1}
\ControlFlowTok{for}\NormalTok{ x }\KeywordTok{in} \BuiltInTok{range}\NormalTok{(}\OperatorTok{{-}}\DecValTok{2}\NormalTok{, }\DecValTok{2}\NormalTok{, }\DecValTok{1}\NormalTok{):}
    \BuiltInTok{print}\NormalTok{(}\SpecialStringTok{f\textquotesingle{}x = }\SpecialCharTok{\{}\NormalTok{x}\SpecialCharTok{:3d\}}\SpecialStringTok{ }\CharTok{\textbackslash{}t}\SpecialStringTok{ p1(x) = }\SpecialCharTok{\{}\NormalTok{p1(x)}\SpecialCharTok{:3d\}}\SpecialStringTok{ }\CharTok{\textbackslash{}t}\SpecialStringTok{ p2(x) = }\SpecialCharTok{\{}\NormalTok{p2(x)}\SpecialCharTok{:3d\}}\SpecialStringTok{\textquotesingle{}}\NormalTok{)}
\end{Highlighting}
\end{Shaded}

\begin{verbatim}
x =  -2      p1(x) =   1     p2(x) =  -7
x =  -1      p1(x) =  -2     p2(x) =  -2
x =   0      p1(x) =  -1     p2(x) =   1
x =   1      p1(x) =   4     p2(x) =   2
\end{verbatim}

\subsection{\texorpdfstring{\textbf{Ejemplo 2. Polinomios de cualquier
grado.}}{Ejemplo 2. Polinomios de cualquier grado.}}\label{ejemplo-2.-polinomios-de-cualquier-grado.}

Implementar una fábrica de polinomios de cualquier grado:

\[
\sum\limits_{k=0}^{n} a_k x^k = a_n x^n + a_{n-1} x^{n-1} + \dots + a_1 x + a_0 
\]

\begin{Shaded}
\begin{Highlighting}[]
\CommentTok{\# Esta función recibe un conjunto de argumentos variable}
\CommentTok{\# para construir un polinomio de cualquier grado.}
\CommentTok{\# Regresa la función que implementa el polinomio.}
\KeywordTok{def}\NormalTok{ polinomioFactory(}\OperatorTok{*}\NormalTok{coeficientes):}

    \KeywordTok{def}\NormalTok{ polinomio(x):}
\NormalTok{        res }\OperatorTok{=} \DecValTok{0}
        \ControlFlowTok{for}\NormalTok{ i, coef }\KeywordTok{in} \BuiltInTok{enumerate}\NormalTok{(coeficientes):}
\NormalTok{            res }\OperatorTok{+=}\NormalTok{ coef }\OperatorTok{*}\NormalTok{ x }\OperatorTok{**}\NormalTok{ i}
        \ControlFlowTok{return}\NormalTok{ res}
    
    \ControlFlowTok{return}\NormalTok{ polinomio}
\end{Highlighting}
\end{Shaded}

\begin{Shaded}
\begin{Highlighting}[]
\CommentTok{\# Se generan 4 polinomios de diferente grado}
\NormalTok{p1 }\OperatorTok{=}\NormalTok{ polinomioFactory(}\DecValTok{5}\NormalTok{)           }\CommentTok{\# a\_0 = 5}
\NormalTok{p2 }\OperatorTok{=}\NormalTok{ polinomioFactory(}\DecValTok{2}\NormalTok{, }\DecValTok{4}\NormalTok{)        }\CommentTok{\# 4 x + 2}
\NormalTok{p3 }\OperatorTok{=}\NormalTok{ polinomioFactory(}\OperatorTok{{-}}\DecValTok{1}\NormalTok{, }\DecValTok{2}\NormalTok{, }\DecValTok{1}\NormalTok{)    }\CommentTok{\# x\^{}2 + 2x {-} 1}
\NormalTok{p4 }\OperatorTok{=}\NormalTok{ polinomioFactory(}\DecValTok{0}\NormalTok{, }\DecValTok{3}\NormalTok{, }\OperatorTok{{-}}\DecValTok{1}\NormalTok{, }\DecValTok{1}\NormalTok{) }\CommentTok{\# x\^{}3 {-} x\^{}2 + 3x + 0}

\CommentTok{\# Evaluación de los polinomios en el intervalo}
\CommentTok{\# ({-}2,2) con pasos de 1}
\ControlFlowTok{for}\NormalTok{ x }\KeywordTok{in} \BuiltInTok{range}\NormalTok{(}\OperatorTok{{-}}\DecValTok{2}\NormalTok{, }\DecValTok{2}\NormalTok{, }\DecValTok{1}\NormalTok{):}
    \BuiltInTok{print}\NormalTok{(}\SpecialStringTok{f\textquotesingle{}x = }\SpecialCharTok{\{}\NormalTok{x}\SpecialCharTok{:3d\}}\SpecialStringTok{ }\CharTok{\textbackslash{}t}\SpecialStringTok{ p1(x) = }\SpecialCharTok{\{}\NormalTok{p1(x)}\SpecialCharTok{:3d\}}\SpecialStringTok{ }\CharTok{\textbackslash{}t}\SpecialStringTok{ p2(x) = }\SpecialCharTok{\{}\NormalTok{p2(x)}\SpecialCharTok{:3d\}}\SpecialStringTok{ }\CharTok{\textbackslash{}t}\SpecialStringTok{ p3(x) = }\SpecialCharTok{\{}\NormalTok{p3(x)}\SpecialCharTok{:3d\}}\SpecialStringTok{ }\CharTok{\textbackslash{}t}\SpecialStringTok{ p4(x) = }\SpecialCharTok{\{}\NormalTok{p4(x)}\SpecialCharTok{:3d\}}\SpecialStringTok{\textquotesingle{}}\NormalTok{)}
\end{Highlighting}
\end{Shaded}

\begin{verbatim}
x =  -2      p1(x) =   5     p2(x) =  -6     p3(x) =  -1     p4(x) = -18
x =  -1      p1(x) =   5     p2(x) =  -2     p3(x) =  -2     p4(x) =  -5
x =   0      p1(x) =   5     p2(x) =   2     p3(x) =  -1     p4(x) =   0
x =   1      p1(x) =   5     p2(x) =   6     p3(x) =   2     p4(x) =   3
\end{verbatim}

\bookmarksetup{startatroot}

\chapter{\texorpdfstring{Documentación con
\emph{docstring}}{Documentación con docstring}}\label{documentaciuxf3n-con-docstring}

Python ofrece dos tipos básicos de comentarios para documentar el
código:

\begin{enumerate}
\def\labelenumi{\arabic{enumi}.}
\tightlist
\item
  Lineal. Este tipo de comentarios se llevan a cabo utilizando el
  símbolo especial \texttt{\#}. El intérprete de Python sabrá que todo
  lo que sigue delante de este símbolo es un comentario y por lo tanto
  no se toma en cuenta en la ejecución:
\end{enumerate}

\begin{Shaded}
\begin{Highlighting}[]
\NormalTok{a }\OperatorTok{=} \DecValTok{10} \CommentTok{\# Este es un comentario}
\end{Highlighting}
\end{Shaded}

\begin{enumerate}
\def\labelenumi{\arabic{enumi}.}
\setcounter{enumi}{1}
\tightlist
\item
  Docstrings En programación, un \emph{docstring} es una cadena de
  caracteres embedidas en el código fuente, similares a un comentario,
  para documentar un segmento de código específico. A diferencia de los
  comentarios tradicionales, las docstrings no se quitan del código
  cuando es analizado, sino que son retenidas a través de la ejecución
  del programa. Esto permite al programador inspeccionar esos
  comentarios en tiempo de ejecución, por ejemplo como un sistema de
  ayuda interactivo o como metadatos. En Python se utilizan las triples
  comillas para definir un \emph{docstring}.
\end{enumerate}

\begin{Shaded}
\begin{Highlighting}[]
\KeywordTok{def}\NormalTok{ funcion(x):}
    \CommentTok{\textquotesingle{}\textquotesingle{}\textquotesingle{}}
\CommentTok{    Esta es una descripción de la función ...}
\CommentTok{    \textquotesingle{}\textquotesingle{}\textquotesingle{}}
    
\KeywordTok{def}\NormalTok{ foo(y):}
    \CommentTok{"""}
\CommentTok{    También de esta manera se puede definir una docstring}
\CommentTok{    """}
   
\end{Highlighting}
\end{Shaded}

\begin{Shaded}
\begin{Highlighting}[]
\KeywordTok{def}\NormalTok{ suma(a,b):}
    \CommentTok{\textquotesingle{}\textquotesingle{}\textquotesingle{}}
\CommentTok{    Esta función calcula  la suma de los parámetros a y b. }
\CommentTok{    Regresa el resultado de la suma}
\CommentTok{    \textquotesingle{}\textquotesingle{}\textquotesingle{}}
    \ControlFlowTok{return}\NormalTok{ a }\OperatorTok{+}\NormalTok{ b}
\end{Highlighting}
\end{Shaded}

\begin{Shaded}
\begin{Highlighting}[]
\NormalTok{suma}
\end{Highlighting}
\end{Shaded}

\begin{verbatim}
<function __main__.suma(a, b)>
\end{verbatim}

\begin{Shaded}
\begin{Highlighting}[]
\CommentTok{\# En numpy se usa la siguiente definición de docstrings}
\KeywordTok{def}\NormalTok{ suma(a,b):}
    \CommentTok{\textquotesingle{}\textquotesingle{}\textquotesingle{}}
\CommentTok{    Calcula la suma de los dos parámetros a y b.}
\CommentTok{    }
\CommentTok{    Args: }
\CommentTok{        a: int Numero a sumar}
\CommentTok{        b: int Numero a sumar}
\CommentTok{    Return:}
\CommentTok{        c: int Suma del numero a y b}
\CommentTok{    \textquotesingle{}\textquotesingle{}\textquotesingle{}}
\NormalTok{    c }\OperatorTok{=}\NormalTok{ a }\OperatorTok{+}\NormalTok{ b}
    \ControlFlowTok{return}\NormalTok{ c}
\end{Highlighting}
\end{Shaded}

\begin{Shaded}
\begin{Highlighting}[]
\NormalTok{suma}
\end{Highlighting}
\end{Shaded}

\begin{verbatim}
<function __main__.suma(a, b)>
\end{verbatim}

Existen diferentes estilos de documentación tipo \emph{docstring} vease
por ejemplo:
\href{https://numpydoc.readthedocs.io/en/latest/format.html}{Numpy},
\href{https://matplotlib.org/devdocs/devel/document.html\#example-docstring}{Matplotlib}.

Para más información véase \href{https://peps.python.org/pep-0257/}{PEP
257 -- Docstring Conventions} y
\href{https://peps.python.org/pep-0008/}{PEP 8 -- Style Guide for Python
Code}.

\bookmarksetup{startatroot}

\chapter{Funciones lambda.}\label{funciones-lambda.}

\textbf{Objetivo.} \ldots{}

\textbf{Funciones de Python}: \ldots{}

MACTI-Algebra\_Lineal\_01 by Luis M. de la Cruz is licensed under
Attribution-ShareAlike 4.0 International

\bookmarksetup{startatroot}

\chapter{Programación funcional.}\label{programaciuxf3n-funcional.}

\begin{itemize}
\item
  Paradigma de programación basado en el uso de funciones, entendiendo
  el concepto de función según su definición matemática, y no como los
  subprogramas de los lenguajes imperativos.
\item
  Tiene sus raíces en el cálculo lambda (un sistema formal desarrollado
  en los años 1930 para investigar la definición de función, la
  aplicación de las funciones y la recursión).
\item
  Muchos lenguajes de programación funcionales pueden ser vistos como
  elaboraciones del cálculo lambda.
\item
  Las funciones que se usan en este paradigma son \emph{funciones
  puras}, es decir, que no tienen efectos secundarios, que no manejan
  datos mutables o de estado.
\item
  Lo anterior está en contraposición con la programación imperativa.
\item
  Uno de sus principales representantes es el lenguaje Haskell, que
  compite en belleza, elegancia y expresividad con Python.
\item
  Los programas escritos en un estilo funcional son más fáciles de
  probar y depurar.
\item
  Por su característica modular facilita el cómputo concurrente y
  paralelo.
\item
  El estilo funcional se lleva muy bien con los datos, permitiendo crear
  algoritmos y programas más expresivos para trabajar en \emph{Big
  Data}.
\end{itemize}

\bookmarksetup{startatroot}

\chapter{Lambda expressions}\label{lambda-expressions}

\begin{itemize}
\item
  Una expresión Lambda (\emph{Lambda expressions}) nos permite crear una
  función ``anónima'', es decir podemos crear funciones \emph{ad-hoc},
  \textbf{sin} la necesidad de definir una función propiamente con el
  comando \textbf{def}.
\item
  Una expresión Lambda o función anónima, es una expresión simple, no un
  bloque de declaraciones.
\item
  Solo hay que escribir el resultado de una expresión en vez de regresar
  un valor explícitamente.
\item
  Dado que se limita a una expresión, una función anónima es menos
  general que una función normal \textbf{def}.
\end{itemize}

Por ejemplo, para calcular el cuadrado de un número podemos escribir la
siguiente función:

\begin{Shaded}
\begin{Highlighting}[]
\KeywordTok{def}\NormalTok{ square\_v1(n):}
    \CommentTok{"""}
\CommentTok{    Calcula el cuadrado de n y lo regresa. Versión 1.0}
\CommentTok{    """}
\NormalTok{    result }\OperatorTok{=}\NormalTok{ n}\OperatorTok{**}\DecValTok{2}
    \ControlFlowTok{return}\NormalTok{ result}
\end{Highlighting}
\end{Shaded}

\begin{Shaded}
\begin{Highlighting}[]
\BuiltInTok{print}\NormalTok{(square\_v1(}\DecValTok{5}\NormalTok{))}
\end{Highlighting}
\end{Shaded}

Se puede reducir el código anterior como sigue:

\begin{Shaded}
\begin{Highlighting}[]
\KeywordTok{def}\NormalTok{ square\_v2(n):}
    \CommentTok{"""}
\CommentTok{    Calcula el cuadrado de n y lo regresa. Versión 2.0}
\CommentTok{    """}
    \ControlFlowTok{return}\NormalTok{ n}\OperatorTok{**}\DecValTok{2}
\end{Highlighting}
\end{Shaded}

\begin{Shaded}
\begin{Highlighting}[]
\BuiltInTok{print}\NormalTok{(square\_v2(}\DecValTok{5}\NormalTok{))}
\end{Highlighting}
\end{Shaded}

Se puede reducir aún más, pero puede llevarnos a un mal estilo de
programación. Por ejemplo:

\begin{Shaded}
\begin{Highlighting}[]
\KeywordTok{def}\NormalTok{ square\_v3(n): }\ControlFlowTok{return}\NormalTok{ n}\OperatorTok{**}\DecValTok{2}
\end{Highlighting}
\end{Shaded}

\begin{Shaded}
\begin{Highlighting}[]
\BuiltInTok{print}\NormalTok{(square\_v3(}\DecValTok{5}\NormalTok{))}
\end{Highlighting}
\end{Shaded}

\textbf{Definición}. La sintáxis de una expresión lambda en Python
(función lambda o función anónima) es muy simple:

\begin{Shaded}
\begin{Highlighting}[]
\KeywordTok{lambda}\NormalTok{ argument\_list: expression}
\end{Highlighting}
\end{Shaded}

\begin{enumerate}
\def\labelenumi{\arabic{enumi}.}
\tightlist
\item
  La lista de argumentos consiste de objetos separados por coma.
\item
  La expresión es cualquiera que sea válida en Python.
\end{enumerate}

Se puede asignar la función a una etiqueta para darle un nombre.

\section{\texorpdfstring{\textbf{Ejemplo
1.}}{Ejemplo 1.}}\label{ejemplo-1.-2}

Función anónima para el cálculo del cuadrado de un número.

\begin{Shaded}
\begin{Highlighting}[]
\CommentTok{\# Se crea una función anónima}
\KeywordTok{lambda}\NormalTok{ n: n}\OperatorTok{**}\DecValTok{2}
\end{Highlighting}
\end{Shaded}

Para poder usar la función anterior debe estar en un contexto donde
pueda ser ejecutada o podemos darle un nombre como sigue:

\begin{Shaded}
\begin{Highlighting}[]
\CommentTok{\# La función anónima se llama ahora cuadrado()}
\NormalTok{cuadrado }\OperatorTok{=} \KeywordTok{lambda}\NormalTok{ num: num}\OperatorTok{**}\DecValTok{2}
\end{Highlighting}
\end{Shaded}

\begin{Shaded}
\begin{Highlighting}[]
\CommentTok{\# Usamos la función cuadrado()}
\BuiltInTok{print}\NormalTok{(cuadrado(}\DecValTok{7}\NormalTok{))}
\end{Highlighting}
\end{Shaded}

\section{\texorpdfstring{\textbf{Ejemplo
2.}}{Ejemplo 2.}}\label{ejemplo-2.}

Escribir una función lambda para calcular el cubo de un número usando la
función lambda que calcula el cuadrado.

\begin{Shaded}
\begin{Highlighting}[]
\CommentTok{\# Construimos la función cubo() usando la función cuadrado()}
\NormalTok{cubo }\OperatorTok{=} \KeywordTok{lambda}\NormalTok{ n: cuadrado(n) }\OperatorTok{*}\NormalTok{ n}
\end{Highlighting}
\end{Shaded}

\begin{Shaded}
\begin{Highlighting}[]
\NormalTok{cubo(}\DecValTok{5}\NormalTok{)}
\end{Highlighting}
\end{Shaded}

\section{\texorpdfstring{\textbf{Ejemplo
3.}}{Ejemplo 3.}}\label{ejemplo-3.}

Construir una función que genere funciones para elevar un número
\texttt{a} a una potencia \texttt{n}.

Este ejemplo nos permite mostrar que es posible combinar la definición
de funciones normales de Python con las funciones lambda.

\begin{Shaded}
\begin{Highlighting}[]
\KeywordTok{def}\NormalTok{ potencia(n):}
    \ControlFlowTok{return} \KeywordTok{lambda}\NormalTok{ a: a }\OperatorTok{**}\NormalTok{ n }\CommentTok{\# regresa una función lambda}
\end{Highlighting}
\end{Shaded}

\begin{Shaded}
\begin{Highlighting}[]
\CommentTok{\# Creamos dos funciones.}
\NormalTok{cuadrado }\OperatorTok{=}\NormalTok{ potencia(}\DecValTok{2}\NormalTok{) }\CommentTok{\# función para elevar al cuadrado}
\NormalTok{cubo }\OperatorTok{=}\NormalTok{ potencia(}\DecValTok{3}\NormalTok{) }\CommentTok{\# función para elevar al cubo}
\end{Highlighting}
\end{Shaded}

\begin{Shaded}
\begin{Highlighting}[]
\BuiltInTok{print}\NormalTok{(cuadrado(}\DecValTok{5}\NormalTok{))}
\BuiltInTok{print}\NormalTok{(cubo(}\DecValTok{2}\NormalTok{))}
\end{Highlighting}
\end{Shaded}

\section{\texorpdfstring{\textbf{Ejemplo
4.}}{Ejemplo 4.}}\label{ejemplo-4.}

Escribir una función lambda para multiplicar dos números.

En este ejemplo vemos como una función lambda puede recibir dos
argumentos.

\begin{Shaded}
\begin{Highlighting}[]
\NormalTok{mult }\OperatorTok{=} \KeywordTok{lambda}\NormalTok{ a, b: a }\OperatorTok{*}\NormalTok{ b}
\end{Highlighting}
\end{Shaded}

\begin{Shaded}
\begin{Highlighting}[]
\BuiltInTok{print}\NormalTok{(mult(}\DecValTok{5}\NormalTok{,}\DecValTok{3}\NormalTok{))}
\end{Highlighting}
\end{Shaded}

\section{\texorpdfstring{\textbf{Ejemplo
5.}}{Ejemplo 5.}}\label{ejemplo-5.}

Checar si un número es par.

En este ejemplo usamos el operador ternario para probar una condición.

\begin{Shaded}
\begin{Highlighting}[]
\NormalTok{esPar }\OperatorTok{=} \KeywordTok{lambda}\NormalTok{ n: }\VariableTok{False} \ControlFlowTok{if}\NormalTok{ n }\OperatorTok{\%} \DecValTok{2} \ControlFlowTok{else} \VariableTok{True}
\end{Highlighting}
\end{Shaded}

\begin{Shaded}
\begin{Highlighting}[]
\BuiltInTok{print}\NormalTok{(esPar(}\DecValTok{2}\NormalTok{))}
\BuiltInTok{print}\NormalTok{(esPar(}\DecValTok{3}\NormalTok{))}
\end{Highlighting}
\end{Shaded}

\section{\texorpdfstring{\textbf{Ejemplo
6.}}{Ejemplo 6.}}\label{ejemplo-6.}

Obtener el primer y último elemento de una secuencia, la cual puede ser
una cadena, una lista y una tupla.

\begin{Shaded}
\begin{Highlighting}[]
\NormalTok{primer\_ultimo}\OperatorTok{=} \KeywordTok{lambda}\NormalTok{ s: (s[}\DecValTok{0}\NormalTok{], s[}\OperatorTok{{-}}\DecValTok{1}\NormalTok{])}
\end{Highlighting}
\end{Shaded}

\begin{Shaded}
\begin{Highlighting}[]
\CommentTok{\# Cadena}
\NormalTok{primer\_ultimo(}\StringTok{\textquotesingle{}Pythonico\textquotesingle{}}\NormalTok{)}
\end{Highlighting}
\end{Shaded}

\begin{Shaded}
\begin{Highlighting}[]
\CommentTok{\# Lista}
\NormalTok{primer\_ultimo([}\DecValTok{1}\NormalTok{,}\DecValTok{2}\NormalTok{,}\DecValTok{3}\NormalTok{,}\DecValTok{4}\NormalTok{,}\DecValTok{5}\NormalTok{,}\DecValTok{6}\NormalTok{,}\DecValTok{7}\NormalTok{,}\DecValTok{8}\NormalTok{,}\DecValTok{9}\NormalTok{])}
\end{Highlighting}
\end{Shaded}

\begin{Shaded}
\begin{Highlighting}[]
\CommentTok{\# Tupla}
\NormalTok{primer\_ultimo( (}\FloatTok{1.2}\NormalTok{, }\FloatTok{3.4}\NormalTok{, }\FloatTok{5.6}\NormalTok{, }\FloatTok{8.4}\NormalTok{) )}
\end{Highlighting}
\end{Shaded}

\section{\texorpdfstring{\textbf{Ejemplo
7.}}{Ejemplo 7.}}\label{ejemplo-7.}

Escribir en reversa una secuencia qu puede ser una cadena, una lista y
una tupla.

\begin{Shaded}
\begin{Highlighting}[]
\NormalTok{c }\OperatorTok{=} \StringTok{\textquotesingle{}Pythonico\textquotesingle{}}

\NormalTok{reversa }\OperatorTok{=} \KeywordTok{lambda}\NormalTok{ l: l[::}\OperatorTok{{-}}\DecValTok{1}\NormalTok{]}

\BuiltInTok{print}\NormalTok{(c)}
\BuiltInTok{print}\NormalTok{(reversa(c))}
\end{Highlighting}
\end{Shaded}

\bookmarksetup{startatroot}

\chapter{Funciones puras e impuras}\label{funciones-puras-e-impuras}

\begin{itemize}
\item
  La programación funcional busca usar funciones \emph{puras}, es decir,
  que no tienen efectos secundarios, no manejan datos mutables o de
  estado.
\item
  Estas funciones puras devuelven un valor que depende solo de sus
  argumentos.
\end{itemize}

Por ejemplo, podemos construir funciones que hagan un cálculo aritmético
el cuál solo depende de sus entradas y no modifica otra cosa:

\begin{Shaded}
\begin{Highlighting}[]
\CommentTok{\# La siguiente es una función pura}
\KeywordTok{def}\NormalTok{ pura(x, y):}
    \ControlFlowTok{return}\NormalTok{ (x }\OperatorTok{+} \DecValTok{2} \OperatorTok{*}\NormalTok{ y) }\OperatorTok{/}\NormalTok{ (}\DecValTok{2} \OperatorTok{*}\NormalTok{ x }\OperatorTok{+}\NormalTok{ y)}

\NormalTok{pura(}\DecValTok{1}\NormalTok{,}\DecValTok{2}\NormalTok{)}
\end{Highlighting}
\end{Shaded}

\begin{verbatim}
1.25
\end{verbatim}

\begin{Shaded}
\begin{Highlighting}[]
\CommentTok{\# La siguiente es una función lambda pura}
\NormalTok{lambda\_pura }\OperatorTok{=} \KeywordTok{lambda}\NormalTok{ x,y: (x }\OperatorTok{+} \DecValTok{2} \OperatorTok{*}\NormalTok{ y) }\OperatorTok{/}\NormalTok{ (}\DecValTok{2} \OperatorTok{*}\NormalTok{ x }\OperatorTok{+}\NormalTok{ y)}

\NormalTok{lambda\_pura(}\DecValTok{1}\NormalTok{,}\DecValTok{2}\NormalTok{)}
\end{Highlighting}
\end{Shaded}

\begin{verbatim}
1.25
\end{verbatim}

El que sigue es un ejemplo de una función impura que tiene efectos
colaterales en la \texttt{lista}:

\begin{Shaded}
\begin{Highlighting}[]
\CommentTok{\# Esta es una función impura}
\NormalTok{lista }\OperatorTok{=}\NormalTok{ []}

\KeywordTok{def}\NormalTok{ impura(arg):}
\NormalTok{    potencia }\OperatorTok{=} \DecValTok{2}
\NormalTok{    lista.append(arg) }\CommentTok{\# Se modifica la lista}
    \ControlFlowTok{return}\NormalTok{ arg }\OperatorTok{**}\NormalTok{ potencia}

\NormalTok{impura(}\DecValTok{5}\NormalTok{)}

\BuiltInTok{print}\NormalTok{(lista)}
\end{Highlighting}
\end{Shaded}

\begin{verbatim}
[5]
\end{verbatim}

Lo anterior también puede suceder usando funciones lambda:

\begin{Shaded}
\begin{Highlighting}[]
\CommentTok{\# podemos crear funciones lambda impuras :o}
\NormalTok{lambda\_impura }\OperatorTok{=} \KeywordTok{lambda}\NormalTok{ l, arg : (l.append(arg), arg}\OperatorTok{**}\DecValTok{2}\NormalTok{)}
\end{Highlighting}
\end{Shaded}

\begin{Shaded}
\begin{Highlighting}[]
\BuiltInTok{print}\NormalTok{(lambda\_impura(lista,}\DecValTok{5}\NormalTok{))}
\NormalTok{lista}
\end{Highlighting}
\end{Shaded}

\begin{verbatim}
(None, 25)
\end{verbatim}

\begin{verbatim}
[5, 5]
\end{verbatim}

Una buena práctica del estilo funcional es evitar los efectos
secundarios, es decir, \textbf{que nuestras funciones NO modifiquen los
valores de sus argumentos}.

\bookmarksetup{startatroot}

\chapter{Manejo de excepciones.}\label{manejo-de-excepciones.}

\textbf{Objetivo.} \ldots{}

\textbf{Funciones de Python}: \ldots{}
\href{https://docs.python.org/3/tutorial/errors.html}{Errors and
Exceptions}

MACTI-Algebra\_Lineal\_01 by Luis M. de la Cruz is licensed under
Attribution-ShareAlike 4.0 International

\bookmarksetup{startatroot}

\chapter{Tipos de errores:}\label{tipos-de-errores}

En Python existen dos tipos de errores por los cuales un programa se
detiene y no continua con su ejecución normal.

\section{Errores de sintaxis.}\label{errores-de-sintaxis.}

Ocurren cuando no se escriben correctamente las expresiones y
declaraciones, siguiendo la especificación de la interfaz de Python.

Por ejemplo:

\begin{Shaded}
\begin{Highlighting}[]
\CommentTok{\# Escribimos a propósito \textquotesingle{}printf\textquotesingle{} que es un nombre incorrecto.}
\NormalTok{printf(}\StringTok{\textquotesingle{}Hola mundo!\textquotesingle{}}\NormalTok{)}
\end{Highlighting}
\end{Shaded}

\begin{verbatim}
NameError: name 'printf' is not defined
\end{verbatim}

\begin{itemize}
\tightlist
\item
  Observa que el tipo de error se imprime cuando éste ocurre.
\item
  En el caso anterior el error fue de tipo \texttt{NameError}, por lo
  que hay que revisar que todo esté correctamente escrito.
\end{itemize}

\section{Errores provocados por
excepciones.}\label{errores-provocados-por-excepciones.}

Son errores lógicos que detienen la ejecución de un programa aún cuando
la sintaxis sea la correcta.

Por ejemplo:

\begin{Shaded}
\begin{Highlighting}[]
\KeywordTok{def}\NormalTok{ raizCuadrada(numero):}
\NormalTok{    numero }\OperatorTok{=} \BuiltInTok{float}\NormalTok{(numero)}
    \BuiltInTok{print}\NormalTok{(}\StringTok{"La raíz cuadrada del número }\SpecialCharTok{\{\}}\StringTok{ es }\SpecialCharTok{\{\}}\StringTok{"}\NormalTok{.}\BuiltInTok{format}\NormalTok{(numero, numero }\OperatorTok{**} \FloatTok{0.5}\NormalTok{))}
\end{Highlighting}
\end{Shaded}

\begin{Shaded}
\begin{Highlighting}[]
\CommentTok{\# Ejemplo correcto que se ejecuta sin problemas.}
\NormalTok{raizCuadrada(}\DecValTok{1}\NormalTok{)}
\end{Highlighting}
\end{Shaded}

\begin{verbatim}
La raíz cuadrada del número 1.0 es 1.0
\end{verbatim}

\begin{Shaded}
\begin{Highlighting}[]
\CommentTok{\# Ejemplo correcto, se calcula la raíz cuadrada de {-}1 en}
\CommentTok{\# El resultado es un número complejo. En este caso Python}
\CommentTok{\# se encarga de realizar las conversiones necesarias.}
\NormalTok{raizCuadrada(}\OperatorTok{{-}}\DecValTok{1}\NormalTok{)}
\end{Highlighting}
\end{Shaded}

\begin{verbatim}
La raíz cuadrada del número -1.0 es (6.123233995736766e-17+1j)
\end{verbatim}

\begin{Shaded}
\begin{Highlighting}[]
\CommentTok{\# Ejemplo incorrecto. No es posible calcular la raíz cuadrada}
\CommentTok{\# de un número complejo, es una operación no definida.}
\NormalTok{raizCuadrada(}\DecValTok{1}\OperatorTok{+}\OtherTok{1j}\NormalTok{)}
\end{Highlighting}
\end{Shaded}

\begin{verbatim}
TypeError: float() argument must be a string or a real number, not 'complex'
\end{verbatim}

En el ejemplo anterior se produce un error de tipo \texttt{TypeError},
es decir hay incompatibilidad con los tipos de datos que se están
manipulando.

\begin{Shaded}
\begin{Highlighting}[]
\CommentTok{\# Ejemplo incorrecto. No se puede calcular la raíz cuadrada}
\CommentTok{\# de una cadena.}
\NormalTok{raizCuadrada(}\StringTok{"hola"}\NormalTok{)}
\end{Highlighting}
\end{Shaded}

\begin{verbatim}
ValueError: could not convert string to float: 'hola'
\end{verbatim}

En el ejemplo anterior se produce un error de tipo \texttt{ValueError},
es decir es decir hay un problema con el contenido del objeto.

\bookmarksetup{startatroot}

\chapter{\texorpdfstring{Manjeo de excepciones con: \texttt{try},
\texttt{except},
\texttt{finally}}{Manjeo de excepciones con: try, except, finally}}\label{manjeo-de-excepciones-con-try-except-finally}

Los errores que se pueden manejar, son aquellos errores lógicos como los
presentados anteriormente en donde es posible ``predecir'' el tipo de
error que puede ocurrir de acuerdo con la implementación que estamos
realizando.

Todas las excepciones en Python son ejemplos concretos de una clase
(\emph{instance}) que se derivan de la clase principal BaseExcepcion.
Más detalles se pueden consultar aquí.

Las excepciones se pueden capturar y manejar adecuadamente. Para ello se
tienen las siguientes herramientas:

\begin{itemize}
\tightlist
\item
  \texttt{try}
\item
  \texttt{except}
\item
  \texttt{else}
\item
  \texttt{finally}
\end{itemize}

Cuando se identifica una sección de código susceptible de errores, ésta
puede ser delimitada con la expresión \texttt{try}. Cualquier excepción
que ocurra dentro de esta sección de código podrá ser capturada y
gestionada.

La expresión \texttt{except} es la encargada de gestionar las
excepciones que se capturan. Si se utiliza sin mayor información, ésta
ejecutará el código que contiene para todas las excepciones que ocurran.

En el ejemplo de la función \texttt{raizCuadrada()} podemos manejar las
excepciones como sigue:

\begin{Shaded}
\begin{Highlighting}[]
\KeywordTok{def}\NormalTok{ raizCuadrada(numero):}
    \CommentTok{"""}
\CommentTok{    Función que calcula la raíz cuadrada de un número.}

\CommentTok{    Parameters}
\CommentTok{    {-}{-}{-}{-}{-}{-}{-}{-}{-}{-}}
\CommentTok{    numero: int o float}
\CommentTok{    Valor al que se le desea calcular la raíz cuadrada.}
\CommentTok{    }
\CommentTok{    """}
    \CommentTok{\# Intenta realizar el cálculo que está dentro de try}
    \ControlFlowTok{try}\NormalTok{:}
\NormalTok{        numero }\OperatorTok{=} \BuiltInTok{float}\NormalTok{(numero)}
        \BuiltInTok{print}\NormalTok{(}\SpecialStringTok{f"La raíz cuadrada del número }\SpecialCharTok{\{}\NormalTok{numero}\SpecialCharTok{\}}\SpecialStringTok{ es }\SpecialCharTok{\{}\NormalTok{numero}\OperatorTok{**}\FloatTok{0.5}\SpecialCharTok{\}}\SpecialStringTok{"}\NormalTok{)}

    \CommentTok{\# Si ocurre una excepción se captura en el except}
    \ControlFlowTok{except}\NormalTok{:}
        \CommentTok{\# No se hace nada con la excepción (por el momento)}
        \ControlFlowTok{pass}

    \BuiltInTok{print}\NormalTok{(}\StringTok{\textquotesingle{}Gracias por usar Python!.\textquotesingle{}}\NormalTok{)}
\end{Highlighting}
\end{Shaded}

Usando la nueva versión de la función \texttt{raizCuadrada()} intentemos
ejecutarla con los ejemplos anteriores:

\begin{Shaded}
\begin{Highlighting}[]
\NormalTok{raizCuadrada(}\DecValTok{1}\NormalTok{)}
\end{Highlighting}
\end{Shaded}

\begin{verbatim}
La raíz cuadrada del número 1.0 es 1.0
Gracias por usar Python!.
\end{verbatim}

\begin{Shaded}
\begin{Highlighting}[]
\NormalTok{raizCuadrada(}\OperatorTok{{-}}\DecValTok{1}\NormalTok{)}
\end{Highlighting}
\end{Shaded}

\begin{verbatim}
La raíz cuadrada del número -1.0 es (6.123233995736766e-17+1j)
Gracias por usar Python!.
\end{verbatim}

\begin{Shaded}
\begin{Highlighting}[]
\NormalTok{raizCuadrada(}\DecValTok{1}\OperatorTok{+}\OtherTok{1j}\NormalTok{)}
\end{Highlighting}
\end{Shaded}

\begin{verbatim}
Gracias por usar Python!.
\end{verbatim}

\begin{Shaded}
\begin{Highlighting}[]
\NormalTok{raizCuadrada(}\StringTok{"hola"}\NormalTok{)}
\end{Highlighting}
\end{Shaded}

\begin{verbatim}
Gracias por usar Python!.
\end{verbatim}

Observa que ya no hay errores. Esto se debe a que la cláusula
\texttt{except} captura todas las posible excepciones, pero no hace
nada, y aún con argumentos erróneos, no se sabe que hubo un error.

\subsection{Gestión general de las
excepciones}\label{gestiuxf3n-general-de-las-excepciones}

Ya que sabemos como capturar las excepciones, veamos cómo pueden ser
tratadas para dar retroalimentación al usuario.

\begin{Shaded}
\begin{Highlighting}[]
\KeywordTok{def}\NormalTok{ raizCuadrada(numero):}
    \CommentTok{"""}
\CommentTok{    Función que calcula la raíz cuadrada de un número.}

\CommentTok{    Parameters}
\CommentTok{    {-}{-}{-}{-}{-}{-}{-}{-}{-}{-}}
\CommentTok{    numero: int o float}
\CommentTok{    Valor al que se le desea calcular la raíz cuadrada.}
\CommentTok{    }
\CommentTok{    """}
    \CommentTok{\# Variable Booleana para manejar las excepciones.}
\NormalTok{    ocurre\_error }\OperatorTok{=} \VariableTok{False}
    
    \CommentTok{\# Intenta realizar el cálculo que está dentro de try    }
    \ControlFlowTok{try}\NormalTok{:}
\NormalTok{        numero }\OperatorTok{=} \BuiltInTok{float}\NormalTok{(numero)}
        \BuiltInTok{print}\NormalTok{(}\StringTok{"La raíz cuadrada del número }\SpecialCharTok{\{\}}\StringTok{ es }\SpecialCharTok{\{\}}\StringTok{"}\NormalTok{.}\BuiltInTok{format}\NormalTok{(numero, numero }\OperatorTok{**} \FloatTok{0.5}\NormalTok{))}

    \CommentTok{\# Si ocurre una excepción se captura en el except}
    \ControlFlowTok{except}\NormalTok{:}
\NormalTok{        ocurre\_error }\OperatorTok{=} \VariableTok{True}

    \CommentTok{\# Cuando ocurre un error se hace lo siguiente:}
    \ControlFlowTok{if}\NormalTok{ ocurre\_error:}
        \BuiltInTok{print}\NormalTok{(}\StringTok{"Cuidado, hubo una falla en el programa, no se pudo realizar el cálculo"}\NormalTok{)}
    \ControlFlowTok{else}\NormalTok{:}
        \BuiltInTok{print}\NormalTok{(}\StringTok{\textquotesingle{}Gracias por usar Python!.\textquotesingle{}}\NormalTok{)}
\end{Highlighting}
\end{Shaded}

\begin{Shaded}
\begin{Highlighting}[]
\NormalTok{raizCuadrada(}\DecValTok{1}\NormalTok{)}
\end{Highlighting}
\end{Shaded}

\begin{verbatim}
La raíz cuadrada del número 1.0 es 1.0
Gracias por usar Python!.
\end{verbatim}

\begin{Shaded}
\begin{Highlighting}[]
\NormalTok{raizCuadrada(}\OperatorTok{{-}}\DecValTok{1}\NormalTok{)}
\end{Highlighting}
\end{Shaded}

\begin{verbatim}
La raíz cuadrada del número -1.0 es (6.123233995736766e-17+1j)
Gracias por usar Python!.
\end{verbatim}

\begin{Shaded}
\begin{Highlighting}[]
\NormalTok{raizCuadrada(}\DecValTok{1}\OperatorTok{+}\OtherTok{1j}\NormalTok{)}
\end{Highlighting}
\end{Shaded}

\begin{verbatim}
Cuidado, hubo una falla en el programa, no se pudo realizar el cálculo
\end{verbatim}

\begin{Shaded}
\begin{Highlighting}[]
\NormalTok{raizCuadrada(}\StringTok{"hola"}\NormalTok{)}
\end{Highlighting}
\end{Shaded}

\begin{verbatim}
Cuidado, hubo una falla en el programa, no se pudo realizar el cálculo
\end{verbatim}

Observa que ahora se avisa al usuario que hubo un error al ejecutar la
función por lo que el cálculo no se realizó. Sin embargo hace falta más
información.

\subsection{Gestión de las excepciones por su
tipo.}\label{gestiuxf3n-de-las-excepciones-por-su-tipo.}

La expresión \texttt{except} puede ser utilizada de forma tal que
ejecute código dependiendo del tipo de error que ocurra. En este caso
sabemos que pueden ocurrir dos tipos de errores: \texttt{TypeError} y
\texttt{ValueError}. Entonces la nueva versión de la función
\texttt{raizCuadrada()} es como sigue:

\begin{Shaded}
\begin{Highlighting}[]
\KeywordTok{def}\NormalTok{ raizCuadrada(numero):}
    \CommentTok{"""}
\CommentTok{    Función que calcula la raíz cuadrada de un número.}

\CommentTok{    Parameters}
\CommentTok{    {-}{-}{-}{-}{-}{-}{-}{-}{-}{-}}
\CommentTok{    numero: int o float}
\CommentTok{    Valor al que se le desea calcular la raíz cuadrada.}
\CommentTok{    }
\CommentTok{    """}
    \CommentTok{\# Variable Booleana para manejar las excepciones.}
\NormalTok{    ocurre\_error }\OperatorTok{=} \VariableTok{False}
    
    \CommentTok{\# Intenta realizar el cálculo que está dentro de try}
    \ControlFlowTok{try}\NormalTok{:}
\NormalTok{        numero }\OperatorTok{=} \BuiltInTok{float}\NormalTok{(numero)}
        \BuiltInTok{print}\NormalTok{(}\StringTok{"La raíz cuadrada del número }\SpecialCharTok{\{\}}\StringTok{ es }\SpecialCharTok{\{\}}\StringTok{"}\NormalTok{.}\BuiltInTok{format}\NormalTok{(numero, numero }\OperatorTok{**} \FloatTok{0.5}\NormalTok{))}

    \CommentTok{\# En esta sección se trata la excepción de tipo TypeError}
    \ControlFlowTok{except} \PreprocessorTok{TypeError}\NormalTok{:}
\NormalTok{        ocurre\_error }\OperatorTok{=} \VariableTok{True}
        \BuiltInTok{print}\NormalTok{(}\StringTok{"Ocurrió un error de tipo: TypeError, verifique que los tipos sean compatibles."}\NormalTok{)}

    \CommentTok{\# En esta sección se trata la excepción de tipo ValueError}
    \ControlFlowTok{except} \PreprocessorTok{ValueError} \ImportTok{as}\NormalTok{ detalles:}
\NormalTok{        ocurre\_error }\OperatorTok{=} \VariableTok{True}
        \BuiltInTok{print}\NormalTok{(}\StringTok{"Ocurrió un error de tipo ValueError, verifique el contenido de los argumentos."}\NormalTok{)}

    \CommentTok{\# En esta sección se tratan todas las otras posible excepciones}
    \ControlFlowTok{except}\NormalTok{:}
\NormalTok{        ocurre\_error }\OperatorTok{=} \VariableTok{True}
        \BuiltInTok{print}\NormalTok{(}\StringTok{"Ocurrió algo misterioso"}\NormalTok{)}
        
    \CommentTok{\# Cuando ocurre un error se hace lo siguiente:}
    \ControlFlowTok{if}\NormalTok{ ocurre\_error:}
        \BuiltInTok{print}\NormalTok{(}\StringTok{"Hubo una falla en el programa, no se pudo realizar el cálculo"}\NormalTok{)}
    \ControlFlowTok{else}\NormalTok{:}
        \BuiltInTok{print}\NormalTok{(}\StringTok{\textquotesingle{}Gracias por usar Python!.\textquotesingle{}}\NormalTok{)}
\end{Highlighting}
\end{Shaded}

\begin{Shaded}
\begin{Highlighting}[]
\NormalTok{raizCuadrada(}\DecValTok{1}\NormalTok{)}
\end{Highlighting}
\end{Shaded}

\begin{verbatim}
La raíz cuadrada del número 1.0 es 1.0
Gracias por usar Python!.
\end{verbatim}

\begin{Shaded}
\begin{Highlighting}[]
\NormalTok{raizCuadrada(}\OperatorTok{{-}}\DecValTok{1}\NormalTok{)}
\end{Highlighting}
\end{Shaded}

\begin{verbatim}
La raíz cuadrada del número -1.0 es (6.123233995736766e-17+1j)
Gracias por usar Python!.
\end{verbatim}

\begin{Shaded}
\begin{Highlighting}[]
\NormalTok{raizCuadrada(}\DecValTok{1}\OperatorTok{+}\OtherTok{4j}\NormalTok{)}
\end{Highlighting}
\end{Shaded}

\begin{verbatim}
Ocurrió un error de tipo: TypeError, verifique que los tipos sean compatibles.
Hubo una falla en el programa, no se pudo realizar el cálculo
\end{verbatim}

\begin{Shaded}
\begin{Highlighting}[]
\NormalTok{raizCuadrada(}\StringTok{"hola"}\NormalTok{)}
\end{Highlighting}
\end{Shaded}

\begin{verbatim}
Ocurrió un error de tipo ValueError, verifique el contenido de los argumentos.
Hubo una falla en el programa, no se pudo realizar el cálculo
\end{verbatim}

Observa que ya se da mayor información sobre el tipo de error que
ocurrió y el usuario puede saber que hacer como corregir los errores.

Todos los tipos de errores que existen en Python se pueden consulta en
Concrete exceptions .

\subsection{Información del error}\label{informaciuxf3n-del-error}

Se puede capturar toda la información del error para pasarla al usuario.
Esto se hace como sigue:

\begin{Shaded}
\begin{Highlighting}[]
\KeywordTok{def}\NormalTok{ raizCuadrada(numero): }
    \CommentTok{"""}
\CommentTok{    Función que calcula la raíz cuadrada de un número.}

\CommentTok{    Parameters}
\CommentTok{    {-}{-}{-}{-}{-}{-}{-}{-}{-}{-}}
\CommentTok{    numero: int o float}
\CommentTok{    Valor al que se le desea calcular la raíz cuadrada.}
\CommentTok{    }
\CommentTok{    """}
    \CommentTok{\# Variable Booleana para manejar las excepciones.}
\NormalTok{    ocurre\_error }\OperatorTok{=} \VariableTok{False}
    
    \CommentTok{\# Intenta realizar el cálculo que está dentro de try}
    \ControlFlowTok{try}\NormalTok{:}
\NormalTok{        numero }\OperatorTok{=} \BuiltInTok{float}\NormalTok{(numero)}
        \BuiltInTok{print}\NormalTok{(}\StringTok{"La raíz cuadrada del número }\SpecialCharTok{\{\}}\StringTok{ es }\SpecialCharTok{\{\}}\StringTok{"}\NormalTok{.}\BuiltInTok{format}\NormalTok{(numero, numero }\OperatorTok{**} \FloatTok{0.5}\NormalTok{))}

    \CommentTok{\# En esta sección se trata la excepción de tipo TypeError y se obtienen los detalles  }
    \ControlFlowTok{except} \PreprocessorTok{TypeError} \ImportTok{as}\NormalTok{ info:}
\NormalTok{        ocurre\_error }\OperatorTok{=} \VariableTok{True}
        \BuiltInTok{print}\NormalTok{(}\StringTok{"Ocurrió un error (TypeError):"}\NormalTok{, info)}

    \CommentTok{\# En esta sección se trata la excepción de tipo ValueError y se obtienen los detalles  }
    \ControlFlowTok{except} \PreprocessorTok{ValueError} \ImportTok{as}\NormalTok{ info:}
\NormalTok{        ocurre\_error }\OperatorTok{=} \VariableTok{True}
        \BuiltInTok{print}\NormalTok{(}\StringTok{"Ocurrió un error (ValueError):"}\NormalTok{, info)}

    \CommentTok{\# En esta sección se tratan todas las otras posible excepciones}
    \ControlFlowTok{except}\NormalTok{:}
\NormalTok{        ocurre\_error }\OperatorTok{=} \VariableTok{True}
        \BuiltInTok{print}\NormalTok{(}\StringTok{"Ocurrió algo misterioso"}\NormalTok{)}

    \CommentTok{\# Cuando ocurre un error se hace lo siguiente:}
    \ControlFlowTok{if}\NormalTok{ ocurre\_error:}
        \BuiltInTok{print}\NormalTok{(}\StringTok{"Hubo una falla en el programa, no se pudo realizar el cálculo"}\NormalTok{)}
    \ControlFlowTok{else}\NormalTok{:}
        \BuiltInTok{print}\NormalTok{(}\StringTok{\textquotesingle{}Gracias por usar Python!.\textquotesingle{}}\NormalTok{)}
\end{Highlighting}
\end{Shaded}

\begin{Shaded}
\begin{Highlighting}[]
\NormalTok{raizCuadrada(}\DecValTok{1}\NormalTok{)}
\end{Highlighting}
\end{Shaded}

\begin{verbatim}
La raíz cuadrada del número 1.0 es 1.0
Gracias por usar Python!.
\end{verbatim}

\begin{Shaded}
\begin{Highlighting}[]
\NormalTok{raizCuadrada(}\OperatorTok{{-}}\DecValTok{1}\NormalTok{)}
\end{Highlighting}
\end{Shaded}

\begin{verbatim}
La raíz cuadrada del número -1.0 es (6.123233995736766e-17+1j)
Gracias por usar Python!.
\end{verbatim}

\begin{Shaded}
\begin{Highlighting}[]
\NormalTok{raizCuadrada(}\DecValTok{1}\OperatorTok{+}\OtherTok{4j}\NormalTok{)}
\end{Highlighting}
\end{Shaded}

\begin{verbatim}
Ocurrió un error (TypeError): float() argument must be a string or a real number, not 'complex'
Hubo una falla en el programa, no se pudo realizar el cálculo
\end{verbatim}

\begin{Shaded}
\begin{Highlighting}[]
\NormalTok{raizCuadrada(}\StringTok{"hola"}\NormalTok{)}
\end{Highlighting}
\end{Shaded}

\begin{verbatim}
Ocurrió un error (ValueError): could not convert string to float: 'hola'
Hubo una falla en el programa, no se pudo realizar el cálculo
\end{verbatim}

Observa que ahora además de conocer el tipo de error, también se muestra
toda la información del error para que el usuario tome las acciones
pertinentes.

\subsection{\texorpdfstring{\texttt{finally}}{finally}}\label{finally}

Esta sección se ejecuta siempre, sin importar si hubo una excepción o
no.

\begin{Shaded}
\begin{Highlighting}[]
\KeywordTok{def}\NormalTok{ raizCuadrada(numero):}
    \CommentTok{"""}
\CommentTok{    Función que calcula la raíz cuadrada de un número.}

\CommentTok{    Parameters}
\CommentTok{    {-}{-}{-}{-}{-}{-}{-}{-}{-}{-}}
\CommentTok{    numero: int o float}
\CommentTok{    Valor al que se le desea calcular la raíz cuadrada.}
\CommentTok{    }
\CommentTok{    """}
    \CommentTok{\# Variable Booleana para manejar las excepciones.}
\NormalTok{    ocurre\_error }\OperatorTok{=} \VariableTok{False}
    
    \CommentTok{\# Intenta realizar el cálculo que está dentro de try}
    \ControlFlowTok{try}\NormalTok{:}
\NormalTok{        numero }\OperatorTok{=} \BuiltInTok{float}\NormalTok{(numero)}
        \BuiltInTok{print}\NormalTok{(}\StringTok{"La raíz cuadrada del número }\SpecialCharTok{\{\}}\StringTok{ es }\SpecialCharTok{\{\}}\StringTok{"}\NormalTok{.}\BuiltInTok{format}\NormalTok{(numero, numero }\OperatorTok{**} \FloatTok{0.5}\NormalTok{))}

    \CommentTok{\# En esta sección se trata la excepción de tipo TypeError y se obtienen los detalles  }
    \ControlFlowTok{except} \PreprocessorTok{TypeError} \ImportTok{as}\NormalTok{ info:}
\NormalTok{        ocurre\_error }\OperatorTok{=} \VariableTok{True}
        \BuiltInTok{print}\NormalTok{(}\StringTok{"Ocurrió un error (TypeError):"}\NormalTok{, info)}

    \CommentTok{\# En esta sección se trata la excepción de tipo ValueError y se obtienen los detalles  }
    \ControlFlowTok{except} \PreprocessorTok{ValueError} \ImportTok{as}\NormalTok{ info:}
\NormalTok{        ocurre\_error }\OperatorTok{=} \VariableTok{True}
        \BuiltInTok{print}\NormalTok{(}\StringTok{"Ocurrió un error (ValueError):"}\NormalTok{, info)}

    \CommentTok{\# En esta sección se tratan todas las otras posible excepciones}
    \ControlFlowTok{except}\NormalTok{:}
\NormalTok{        ocurre\_error }\OperatorTok{=} \VariableTok{True}
        \BuiltInTok{print}\NormalTok{(}\StringTok{"Ocurrió algo misterioso"}\NormalTok{)}

    \CommentTok{\# Cuando ocurre un error se hace lo siguiente:}
    \ControlFlowTok{finally}\NormalTok{:}
        \ControlFlowTok{if}\NormalTok{ ocurre\_error:}
            \BuiltInTok{print}\NormalTok{(}\StringTok{"Hubo una falla en el programa, no se pudo realizar el cálculo"}\NormalTok{)}
        \ControlFlowTok{else}\NormalTok{:}
            \BuiltInTok{print}\NormalTok{(}\StringTok{\textquotesingle{}Gracias por usar Python!.\textquotesingle{}}\NormalTok{)}
\end{Highlighting}
\end{Shaded}

\begin{Shaded}
\begin{Highlighting}[]
\NormalTok{raizCuadrada(}\DecValTok{1}\NormalTok{)}
\end{Highlighting}
\end{Shaded}

\begin{verbatim}
La raíz cuadrada del número 1.0 es 1.0
Gracias por usar Python!.
\end{verbatim}

\begin{Shaded}
\begin{Highlighting}[]
\NormalTok{raizCuadrada(}\OperatorTok{{-}}\DecValTok{1}\NormalTok{)}
\end{Highlighting}
\end{Shaded}

\begin{verbatim}
La raíz cuadrada del número -1.0 es (6.123233995736766e-17+1j)
Gracias por usar Python!.
\end{verbatim}

\begin{Shaded}
\begin{Highlighting}[]
\NormalTok{raizCuadrada(}\DecValTok{1}\OperatorTok{+}\OtherTok{4j}\NormalTok{)}
\end{Highlighting}
\end{Shaded}

\begin{verbatim}
Ocurrió un error (TypeError): float() argument must be a string or a real number, not 'complex'
Hubo una falla en el programa, no se pudo realizar el cálculo
\end{verbatim}

\begin{Shaded}
\begin{Highlighting}[]
\NormalTok{raizCuadrada(}\StringTok{"hola"}\NormalTok{)}
\end{Highlighting}
\end{Shaded}

\begin{verbatim}
Ocurrió un error (ValueError): could not convert string to float: 'hola'
Hubo una falla en el programa, no se pudo realizar el cálculo
\end{verbatim}

\subsection{Lanzar excepciones
controladas.}\label{lanzar-excepciones-controladas.}

Es posible presentar toda la información que genera la excepción y
agregarle notas para el usuario. Para agregar notas usamos le método
\texttt{add\_note()} y para lanzar la excepción una vez controlada
usamos \texttt{raise}. La siguiente versión de la función
\texttt{raizCuadrada()} tiene al final una cláusula \texttt{else}, la
cual se ejecuta cuando no ocurre ninguna excepción. En este caso, dentro
del \texttt{try} realizamo el cálculo de la raíz cuadrada y en el
\texttt{else} hacemos la impresión del resultado.

\begin{Shaded}
\begin{Highlighting}[]
\KeywordTok{def}\NormalTok{ raizCuadrada(numero):}
    \CommentTok{"""}
\CommentTok{    Función que calcula la raíz cuadrada de un número.}

\CommentTok{    Parameters}
\CommentTok{    {-}{-}{-}{-}{-}{-}{-}{-}{-}{-}}
\CommentTok{    numero: int o float}
\CommentTok{    Valor al que se le desea calcular la raíz cuadrada.}
\CommentTok{    }
\CommentTok{    """}
    
    \CommentTok{\# Intenta realizar el cálculo que está dentro de try}
    \ControlFlowTok{try}\NormalTok{:}
\NormalTok{        numero\_cuadrado }\OperatorTok{=} \BuiltInTok{float}\NormalTok{(numero) }\OperatorTok{**} \FloatTok{0.5}

    \CommentTok{\# En esta sección se trata la excepción de tipo TypeError y se obtienen los detalles  }
    \ControlFlowTok{except} \PreprocessorTok{TypeError} \ImportTok{as}\NormalTok{ info:}
\NormalTok{        info.add\_note(}\StringTok{"}\CharTok{\textbackslash{}n}\StringTok{"} \OperatorTok{+} \StringTok{"{-}"}\OperatorTok{*}\DecValTok{20}\NormalTok{)}
\NormalTok{        info.add\_note(}\SpecialStringTok{f"raizCuadrada}\SpecialCharTok{\{}\NormalTok{numero}\SpecialCharTok{\}}\SpecialStringTok{: Para calcular una raíz cuadrada, el argumento \textquotesingle{}numero\textquotesingle{} debe ser compatible con un int o un float"}\NormalTok{)}
\NormalTok{        info.add\_note(}\StringTok{"{-}"}\OperatorTok{*}\DecValTok{20}\NormalTok{)}
        \ControlFlowTok{raise} \CommentTok{\# Lanzamos la excepción con toda la información}

    \CommentTok{\# En esta sección se trata la excepción de tipo ValueError y se obtienen los detalles  }
    \ControlFlowTok{except} \PreprocessorTok{ValueError} \ImportTok{as}\NormalTok{ info:}
\NormalTok{        info.add\_note(}\StringTok{"}\CharTok{\textbackslash{}n}\StringTok{"} \OperatorTok{+} \StringTok{"{-}"}\OperatorTok{*}\DecValTok{20}\NormalTok{)}
\NormalTok{        info.add\_note(}\SpecialStringTok{f"raizCuadrada(\textquotesingle{}}\SpecialCharTok{\{}\NormalTok{numero}\SpecialCharTok{\}}\SpecialStringTok{\textquotesingle{}): Para calcular una raíz cuadrada, el valor del argumento \textquotesingle{}numero\textquotesingle{} debe ser compatible con un int o un float"}\NormalTok{)}
\NormalTok{        info.add\_note(}\StringTok{"{-}"}\OperatorTok{*}\DecValTok{20}\NormalTok{)}
        \ControlFlowTok{raise} \CommentTok{\# Lanzamos la excepción con toda la información}
        
    \CommentTok{\# En esta sección se tratan todas las otras posible excepciones}
    \ControlFlowTok{except}\NormalTok{:}
        \BuiltInTok{print}\NormalTok{(}\StringTok{"Ocurrió algo misterioso"}\NormalTok{)}

    \ControlFlowTok{else}\NormalTok{:}
        \BuiltInTok{print}\NormalTok{(}\StringTok{"La raíz cuadrada del número }\SpecialCharTok{\{\}}\StringTok{ es }\SpecialCharTok{\{\}}\StringTok{"}\NormalTok{.}\BuiltInTok{format}\NormalTok{(numero, numero\_cuadrado))}
\end{Highlighting}
\end{Shaded}

\begin{Shaded}
\begin{Highlighting}[]
\NormalTok{raizCuadrada(}\DecValTok{1}\NormalTok{)}
\end{Highlighting}
\end{Shaded}

\begin{verbatim}
La raíz cuadrada del número 1 es 1.0
\end{verbatim}

\begin{Shaded}
\begin{Highlighting}[]
\NormalTok{raizCuadrada(}\OperatorTok{{-}}\DecValTok{1}\NormalTok{)}
\end{Highlighting}
\end{Shaded}

\begin{verbatim}
La raíz cuadrada del número -1 es (6.123233995736766e-17+1j)
\end{verbatim}

\begin{Shaded}
\begin{Highlighting}[]
\NormalTok{raizCuadrada(}\DecValTok{1}\OperatorTok{+}\OtherTok{4j}\NormalTok{)}
\end{Highlighting}
\end{Shaded}

\begin{verbatim}
TypeError: float() argument must be a string or a real number, not 'complex'
\end{verbatim}

\begin{Shaded}
\begin{Highlighting}[]
\NormalTok{raizCuadrada(}\StringTok{"hola"}\NormalTok{)}
\end{Highlighting}
\end{Shaded}

\begin{verbatim}
ValueError: could not convert string to float: 'hola'
\end{verbatim}

\bookmarksetup{startatroot}

\chapter{Iterables, Mapeo, Filtros y
Reducciones.}\label{iterables-mapeo-filtros-y-reducciones.}

\textbf{Objetivo.} \ldots{}

\textbf{Funciones de Python}: \ldots{}

MACTI-Algebra\_Lineal\_01 by Luis M. de la Cruz is licensed under
Attribution-ShareAlike 4.0 International

\bookmarksetup{startatroot}

\chapter{Iterables}\label{iterables}

\begin{itemize}
\tightlist
\item
  En Python existen objetos que contienen secuencias de otros objetos
  (listas, tuplas, diccionarios, etc).
\item
  Estos objetos se pueden recorrer usando ciclos \textbf{for \ldots{} in
  \ldots{}} .
\item
  A estos objetos se les conoce también como \textbf{iterables} (objetos
  iterables, secuencias iterables, contenedores iterables, conjunto
  iterable, entre otros).
\end{itemize}

\section{\texorpdfstring{\textbf{Ejemplo
1.}}{Ejemplo 1.}}\label{ejemplo-1.-3}

Crear una cadena, una lista, una tupla, un diccionario, un conjunto y
leer un archivo; posteriormente recorrer cada uno de estos iterables
usando un ciclo \texttt{for}:

\begin{Shaded}
\begin{Highlighting}[]
\NormalTok{mi\_cadena }\OperatorTok{=} \StringTok{"pythonico"}
\NormalTok{mi\_lista }\OperatorTok{=}\NormalTok{ [}\StringTok{\textquotesingle{}p\textquotesingle{}}\NormalTok{,}\StringTok{\textquotesingle{}y\textquotesingle{}}\NormalTok{,}\StringTok{\textquotesingle{}t\textquotesingle{}}\NormalTok{,}\StringTok{\textquotesingle{}h\textquotesingle{}}\NormalTok{,}\StringTok{\textquotesingle{}o\textquotesingle{}}\NormalTok{,}\StringTok{\textquotesingle{}n\textquotesingle{}}\NormalTok{,}\StringTok{\textquotesingle{}i\textquotesingle{}}\NormalTok{,}\StringTok{\textquotesingle{}c\textquotesingle{}}\NormalTok{,}\StringTok{\textquotesingle{}o\textquotesingle{}}\NormalTok{]}
\NormalTok{mi\_tupla }\OperatorTok{=}\NormalTok{ (}\StringTok{\textquotesingle{}p\textquotesingle{}}\NormalTok{,}\StringTok{\textquotesingle{}y\textquotesingle{}}\NormalTok{,}\StringTok{\textquotesingle{}t\textquotesingle{}}\NormalTok{,}\StringTok{\textquotesingle{}h\textquotesingle{}}\NormalTok{,}\StringTok{\textquotesingle{}o\textquotesingle{}}\NormalTok{,}\StringTok{\textquotesingle{}n\textquotesingle{}}\NormalTok{,}\StringTok{\textquotesingle{}i\textquotesingle{}}\NormalTok{,}\StringTok{\textquotesingle{}c\textquotesingle{}}\NormalTok{,}\StringTok{\textquotesingle{}o\textquotesingle{}}\NormalTok{)}
\NormalTok{mi\_dict }\OperatorTok{=}\NormalTok{ \{}\StringTok{\textquotesingle{}p\textquotesingle{}}\NormalTok{:}\DecValTok{1}\NormalTok{,}\StringTok{\textquotesingle{}y\textquotesingle{}}\NormalTok{:}\DecValTok{2}\NormalTok{,}\StringTok{\textquotesingle{}t\textquotesingle{}}\NormalTok{:}\DecValTok{3}\NormalTok{,}\StringTok{\textquotesingle{}h\textquotesingle{}}\NormalTok{:}\DecValTok{4}\NormalTok{,}\StringTok{\textquotesingle{}o\textquotesingle{}}\NormalTok{:}\DecValTok{5}\NormalTok{,}\StringTok{\textquotesingle{}n\textquotesingle{}}\NormalTok{:}\DecValTok{6}\NormalTok{,}\StringTok{\textquotesingle{}i\textquotesingle{}}\NormalTok{:}\DecValTok{7}\NormalTok{,}\StringTok{\textquotesingle{}c\textquotesingle{}}\NormalTok{:}\DecValTok{8}\NormalTok{,}\StringTok{\textquotesingle{}o\textquotesingle{}}\NormalTok{:}\DecValTok{9}\NormalTok{\}}
\NormalTok{mi\_conj }\OperatorTok{=}\NormalTok{ \{}\StringTok{\textquotesingle{}p\textquotesingle{}}\NormalTok{,}\StringTok{\textquotesingle{}y\textquotesingle{}}\NormalTok{,}\StringTok{\textquotesingle{}t\textquotesingle{}}\NormalTok{,}\StringTok{\textquotesingle{}h\textquotesingle{}}\NormalTok{,}\StringTok{\textquotesingle{}o\textquotesingle{}}\NormalTok{,}\StringTok{\textquotesingle{}n\textquotesingle{}}\NormalTok{,}\StringTok{\textquotesingle{}i\textquotesingle{}}\NormalTok{,}\StringTok{\textquotesingle{}c\textquotesingle{}}\NormalTok{,}\StringTok{\textquotesingle{}o\textquotesingle{}}\NormalTok{\}}
\NormalTok{mi\_archivo }\OperatorTok{=} \BuiltInTok{open}\NormalTok{(}\StringTok{"mi\_archivo.txt"}\NormalTok{)}

\BuiltInTok{print}\NormalTok{(}\StringTok{\textquotesingle{}}\CharTok{\textbackslash{}n}\StringTok{Cadena:\textquotesingle{}}\NormalTok{, end}\OperatorTok{=}\StringTok{\textquotesingle{} \textquotesingle{}}\NormalTok{)}
\CommentTok{\# Recorremos la cadena e imprimimos cada elemento }
\ControlFlowTok{for}\NormalTok{ char }\KeywordTok{in}\NormalTok{ mi\_cadena:}
    \BuiltInTok{print}\NormalTok{(char, end}\OperatorTok{=}\StringTok{\textquotesingle{} \textquotesingle{}}\NormalTok{)}

\BuiltInTok{print}\NormalTok{(}\StringTok{\textquotesingle{}}\CharTok{\textbackslash{}n}\StringTok{Lista:\textquotesingle{}}\NormalTok{, end}\OperatorTok{=}\StringTok{\textquotesingle{} \textquotesingle{}}\NormalTok{)}
\CommentTok{\# Recorremos la lista e imprimimos cada elemento }
\ControlFlowTok{for}\NormalTok{ element }\KeywordTok{in}\NormalTok{ mi\_lista:}
    \BuiltInTok{print}\NormalTok{(element, end}\OperatorTok{=}\StringTok{\textquotesingle{} \textquotesingle{}}\NormalTok{)}

\BuiltInTok{print}\NormalTok{(}\StringTok{"}\CharTok{\textbackslash{}n}\StringTok{Tupla: "}\NormalTok{, end}\OperatorTok{=}\StringTok{\textquotesingle{}\textquotesingle{}}\NormalTok{)}
\CommentTok{\# Recorremos la tupla e imprimimos cada elemento }
\ControlFlowTok{for}\NormalTok{ element }\KeywordTok{in}\NormalTok{ mi\_tupla:}
    \BuiltInTok{print}\NormalTok{(element, end}\OperatorTok{=}\StringTok{\textquotesingle{} \textquotesingle{}}\NormalTok{)}

\BuiltInTok{print}\NormalTok{(}\StringTok{"}\CharTok{\textbackslash{}n}\StringTok{Diccionario  (claves): "}\NormalTok{, end}\OperatorTok{=}\StringTok{\textquotesingle{}\textquotesingle{}}\NormalTok{) }
\CommentTok{\# Recorremos el diccionario e imprimimos cada clave }
\ControlFlowTok{for}\NormalTok{ key }\KeywordTok{in}\NormalTok{ mi\_dict.keys():}
    \BuiltInTok{print}\NormalTok{(key, end}\OperatorTok{=}\StringTok{\textquotesingle{} \textquotesingle{}}\NormalTok{)}

\BuiltInTok{print}\NormalTok{(}\StringTok{"}\CharTok{\textbackslash{}n}\StringTok{Diccionario (valores): "}\NormalTok{, end}\OperatorTok{=}\StringTok{\textquotesingle{}\textquotesingle{}}\NormalTok{) }
\CommentTok{\# Recorremos el diccionario e imprimimos cada valor }
\ControlFlowTok{for}\NormalTok{ key }\KeywordTok{in}\NormalTok{ mi\_dict.values():}
    \BuiltInTok{print}\NormalTok{(key, end}\OperatorTok{=}\StringTok{\textquotesingle{} \textquotesingle{}}\NormalTok{)}

\BuiltInTok{print}\NormalTok{(}\StringTok{"}\CharTok{\textbackslash{}n}\StringTok{Conjunto: "}\NormalTok{, end}\OperatorTok{=}\StringTok{\textquotesingle{}\textquotesingle{}}\NormalTok{) }
\CommentTok{\# Recorremos el conjunt e imprimimos cada elemento }
\ControlFlowTok{for}\NormalTok{ s }\KeywordTok{in}\NormalTok{ mi\_conj:}
    \BuiltInTok{print}\NormalTok{(s, end }\OperatorTok{=} \StringTok{\textquotesingle{} \textquotesingle{}}\NormalTok{)}
    
\BuiltInTok{print}\NormalTok{(}\StringTok{"}\CharTok{\textbackslash{}n}\StringTok{Archivo: "}\NormalTok{) }
\CommentTok{\# Recorremos el archivo e imprimimos cada elemento }
\ControlFlowTok{for}\NormalTok{ line }\KeywordTok{in}\NormalTok{ mi\_archivo:}
    \BuiltInTok{print}\NormalTok{(line, end }\OperatorTok{=} \StringTok{\textquotesingle{}\textquotesingle{}}\NormalTok{)}
\end{Highlighting}
\end{Shaded}

\begin{verbatim}

Cadena: p y t h o n i c o 
Lista: p y t h o n i c o 
Tupla: p y t h o n i c o 
Diccionario  (claves): p y t h o n i c 
Diccionario (valores): 1 2 3 4 9 6 7 8 
Conjunto: y h t c i n o p 
Archivo: 
p
y
t
h
o
n
i
c
o
\end{verbatim}

Observa el caso del diccionario y del conjunto: * Diccionario: cuando
hay claves repetidas, se sustituye el último valor que toma la clave
(\texttt{\textquotesingle{}0\textquotesingle{}:9}). * Conjunto: los
elementos se ordenan, y no se admiten elementos repetidos.

\bookmarksetup{startatroot}

\chapter{Mapeo.}\label{mapeo.}

En análisis matemático, un \emph{Mapeo} es una regla que asigna a cada
elemento de un primer conjunto, un único elemento de un segundo
conjunto:

\[
\texttt{map} 
\] \[
\left[
\begin{matrix}
s_1 \\
s_2 \\
\vdots \\
s_{n-1}
\end{matrix}
\right]
\begin{matrix}
\longrightarrow \\
\longrightarrow \\
\vdots \\
\longrightarrow
\end{matrix}
\left[
\begin{matrix}
t_1 \\
t_2 \\
\vdots \\
t_{n-1}
\end{matrix}
\right]
\]

\section{\texorpdfstring{\texttt{map}}{map}}\label{map}

En Python existe la función \texttt{map(function,\ sequence)} cuyo
primer parámetro es una función la cual se va a aplicar a una secuencia,
la cual es el segundo parámetro. El resultado será una nueva secuencia
con los elementos obtenidos de aplicar la función a cada elemento de la
secuencia de entrada.

\section{\texorpdfstring{\textbf{Ejemplo
2.}}{Ejemplo 2.}}\label{ejemplo-2.-1}

Crear el siguiente mapeo con una lista, una tupla, un conjunto \[
f(x) = x^2 
\] \[
\left[
\begin{matrix}
0 \\
1 \\
2 \\
3 \\
4
\end{matrix}
\right]
\begin{matrix}
\longrightarrow \\
\longrightarrow \\
\longrightarrow \\
\longrightarrow \\
\longrightarrow
\end{matrix}
\left[
\begin{matrix}
0 \\
1 \\
4 \\
9 \\
16
\end{matrix}
\right]
\]

\begin{Shaded}
\begin{Highlighting}[]
\CommentTok{\# Primero definimos la función}
\KeywordTok{def}\NormalTok{ square(x):}
    \CommentTok{"""}
\CommentTok{    Calcula el cuadrado de x.}
\CommentTok{    """}
    \ControlFlowTok{return}\NormalTok{ x}\OperatorTok{**}\DecValTok{2}

\CommentTok{\# Luego definimos las secuencias}
\NormalTok{l }\OperatorTok{=}\NormalTok{ [}\DecValTok{0}\NormalTok{,}\DecValTok{1}\NormalTok{,}\DecValTok{2}\NormalTok{,}\DecValTok{3}\NormalTok{,}\DecValTok{4}\NormalTok{]}
\NormalTok{t }\OperatorTok{=}\NormalTok{ (}\DecValTok{0}\NormalTok{,}\DecValTok{1}\NormalTok{,}\DecValTok{2}\NormalTok{,}\DecValTok{3}\NormalTok{,}\DecValTok{4}\NormalTok{)}
\NormalTok{s }\OperatorTok{=}\NormalTok{ \{}\DecValTok{0}\NormalTok{,}\DecValTok{1}\NormalTok{,}\DecValTok{2}\NormalTok{,}\DecValTok{3}\NormalTok{,}\DecValTok{4}\NormalTok{\}}

\CommentTok{\# Ahora creamos los mapeos}
\NormalTok{lmap }\OperatorTok{=} \BuiltInTok{map}\NormalTok{(square, l)}
\NormalTok{tmap }\OperatorTok{=} \BuiltInTok{map}\NormalTok{(square, t)}
\NormalTok{smap }\OperatorTok{=} \BuiltInTok{map}\NormalTok{(square, s)}

\CommentTok{\# Checamos el tipo de cada mapeo}
\BuiltInTok{print}\NormalTok{(}\BuiltInTok{type}\NormalTok{(lmap), }\BuiltInTok{type}\NormalTok{(tmap), }\BuiltInTok{type}\NormalTok{(smap))}

\BuiltInTok{print}\NormalTok{(}\StringTok{\textquotesingle{}Lista }\SpecialCharTok{\{\}}\StringTok{\textquotesingle{}}\NormalTok{.}\BuiltInTok{format}\NormalTok{(l))}
\BuiltInTok{print}\NormalTok{(}\StringTok{\textquotesingle{}Mapeo }\SpecialCharTok{\{\}}\CharTok{\textbackslash{}n}\StringTok{\textquotesingle{}}\NormalTok{.}\BuiltInTok{format}\NormalTok{(}\BuiltInTok{list}\NormalTok{(lmap)))}

\BuiltInTok{print}\NormalTok{(}\StringTok{\textquotesingle{}Tupla }\SpecialCharTok{\{\}}\StringTok{\textquotesingle{}}\NormalTok{.}\BuiltInTok{format}\NormalTok{(t))}
\BuiltInTok{print}\NormalTok{(}\StringTok{\textquotesingle{}Mapeo }\SpecialCharTok{\{\}}\CharTok{\textbackslash{}n}\StringTok{\textquotesingle{}}\NormalTok{.}\BuiltInTok{format}\NormalTok{(}\BuiltInTok{tuple}\NormalTok{(tmap)))}

\BuiltInTok{print}\NormalTok{(}\StringTok{\textquotesingle{}Conj }\SpecialCharTok{\{\}}\StringTok{\textquotesingle{}}\NormalTok{.}\BuiltInTok{format}\NormalTok{(s))}
\BuiltInTok{print}\NormalTok{(}\StringTok{\textquotesingle{}Mapeo }\SpecialCharTok{\{\}}\CharTok{\textbackslash{}n}\StringTok{\textquotesingle{}}\NormalTok{.}\BuiltInTok{format}\NormalTok{(}\BuiltInTok{set}\NormalTok{(smap)))}
\end{Highlighting}
\end{Shaded}

\begin{verbatim}
<class 'map'> <class 'map'> <class 'map'>
Lista [0, 1, 2, 3, 4]
Mapeo [0, 1, 4, 9, 16]

Tupla (0, 1, 2, 3, 4)
Mapeo (0, 1, 4, 9, 16)

Conj {0, 1, 2, 3, 4}
Mapeo {0, 1, 4, 9, 16}
\end{verbatim}

Observa que el resultado del mapeo es un objeto de tipo
\texttt{\textless{}class\ \textquotesingle{}map\textquotesingle{}\textgreater{}}
por lo que debemos convertirlo en un tipo que pueda ser desplegado para
imprimir.

\section{\texorpdfstring{\textbf{Ejemplo
3.}}{Ejemplo 3.}}\label{ejemplo-3.-1}

Crear un mapeo para convertir grados Fahrenheit a Celsius y viceversa:

\begin{Shaded}
\begin{Highlighting}[]
\KeywordTok{def}\NormalTok{ toFahrenheit(T):}
    \CommentTok{"""}
\CommentTok{    Transforma los elementos de T en grados Farenheit.}
\CommentTok{    """}
    \ControlFlowTok{return}\NormalTok{ (}\DecValTok{9}\OperatorTok{/}\DecValTok{5}\NormalTok{)}\OperatorTok{*}\NormalTok{T }\OperatorTok{+} \DecValTok{32}

\KeywordTok{def}\NormalTok{ toCelsius(T):}
    \CommentTok{"""}
\CommentTok{    Transforma los elementos de T en grados Celsius.}
\CommentTok{    """}
    \ControlFlowTok{return}\NormalTok{ (}\DecValTok{5}\OperatorTok{/}\DecValTok{9}\NormalTok{)}\OperatorTok{*}\NormalTok{(T}\OperatorTok{{-}}\DecValTok{32}\NormalTok{)}
\end{Highlighting}
\end{Shaded}

\textbf{Celsius \(\to\) Fahrenheit}

\begin{Shaded}
\begin{Highlighting}[]
\CommentTok{\# Lista original con los datos}
\NormalTok{c }\OperatorTok{=}\NormalTok{ [}\DecValTok{0}\NormalTok{, }\FloatTok{22.5}\NormalTok{, }\DecValTok{40}\NormalTok{, }\DecValTok{100}\NormalTok{]}

\CommentTok{\# Construimos el mapeo y lo nombramos en \textasciigrave{}fmap\textasciigrave{}.}
\NormalTok{fmap }\OperatorTok{=} \BuiltInTok{map}\NormalTok{(toFahrenheit, c)}
\end{Highlighting}
\end{Shaded}

\begin{Shaded}
\begin{Highlighting}[]
\CommentTok{\# Imprimos a lista original y el mapeo}
\BuiltInTok{print}\NormalTok{(c)}
\BuiltInTok{print}\NormalTok{(}\BuiltInTok{list}\NormalTok{(fmap))}
\end{Highlighting}
\end{Shaded}

\begin{verbatim}
[0, 22.5, 40, 100]
[32.0, 72.5, 104.0, 212.0]
\end{verbatim}

\textbf{NOTA}. Solo se puede usar el mapeo una vez, si vuelves a
ejecutar la celda anterior el resultado del mapeo estará vacío. Para
volverlo a generar debes ejecutar la celda donde se construye el mapeo.

Lo anterior se puede realiza en una sola línea: crear el mapeo,
convertir a lista e imprimir

\begin{Shaded}
\begin{Highlighting}[]
\BuiltInTok{print}\NormalTok{(}\BuiltInTok{list}\NormalTok{(}\BuiltInTok{map}\NormalTok{(toFahrenheit,c)))}
\end{Highlighting}
\end{Shaded}

\begin{verbatim}
[32.0, 72.5, 104.0, 212.0]
\end{verbatim}

\textbf{Fahrenheit \(\to\) Celsius}

\begin{Shaded}
\begin{Highlighting}[]
\CommentTok{\# Lista original con los datos}
\NormalTok{f }\OperatorTok{=}\NormalTok{ [}\FloatTok{32.0}\NormalTok{, }\FloatTok{72.5}\NormalTok{, }\FloatTok{104.0}\NormalTok{, }\FloatTok{212.0}\NormalTok{]}

\CommentTok{\# Conversión en una sola línea}
\BuiltInTok{print}\NormalTok{(}\BuiltInTok{list}\NormalTok{(}\BuiltInTok{map}\NormalTok{(toCelsius, f)))}
\end{Highlighting}
\end{Shaded}

\begin{verbatim}
[0.0, 22.5, 40.0, 100.0]
\end{verbatim}

\section{\texorpdfstring{\textbf{Ejemplo
4.}}{Ejemplo 4.}}\label{ejemplo-4.-1}

Crear un mapeo para sumar los elementos de tres listas que contienen
números enteros.

\textbf{NOTA}. La función \texttt{map()} se puede aplicar a más de un
conjunto iterable, siempre y cuando los iterables tengan la misma
longitud y la función que se aplique tenga los parámetros
correspondientes.

\begin{Shaded}
\begin{Highlighting}[]
\KeywordTok{def}\NormalTok{ suma(x,y,z):}
    \CommentTok{"""}
\CommentTok{    Suma los números x, y, z.}
\CommentTok{    """}
    \ControlFlowTok{return}\NormalTok{ x}\OperatorTok{+}\NormalTok{y}\OperatorTok{+}\NormalTok{z}

\CommentTok{\# Tres listas con enteros}
\NormalTok{a }\OperatorTok{=}\NormalTok{ [}\DecValTok{1}\NormalTok{,}\DecValTok{2}\NormalTok{,}\DecValTok{3}\NormalTok{,}\DecValTok{4}\NormalTok{]}
\NormalTok{b }\OperatorTok{=}\NormalTok{ [}\DecValTok{5}\NormalTok{,}\DecValTok{6}\NormalTok{,}\DecValTok{7}\NormalTok{,}\DecValTok{8}\NormalTok{]}
\NormalTok{c }\OperatorTok{=}\NormalTok{ [}\DecValTok{9}\NormalTok{,}\DecValTok{10}\NormalTok{,}\DecValTok{11}\NormalTok{,}\DecValTok{12}\NormalTok{]}

\CommentTok{\# Aplicación del mapeo}
\BuiltInTok{print}\NormalTok{(}\BuiltInTok{list}\NormalTok{(}\BuiltInTok{map}\NormalTok{(suma, a,b,c)))}
\end{Highlighting}
\end{Shaded}

\begin{verbatim}
[15, 18, 21, 24]
\end{verbatim}

\bookmarksetup{startatroot}

\chapter{Filtrado.}\label{filtrado.}

\begin{itemize}
\item
  Filtrar es un procedimiento para seleccionar cosas de un conjunto o
  para impedir su paso libremente.
\item
  En matemáticas, un filtro es un subconjunto especial de un conjunto
  parcialmente ordenado.
\end{itemize}

\[
\texttt{filter} 
\] \[
\left[
\begin{matrix}
s_1 \\ s_2 \\ s_3 \\ s_4 \\ s_{n-1} 
\end{matrix}
\right]
\begin{matrix}
\\ \xrightarrow{\texttt{True}} \\  \\ \xrightarrow{\texttt{True}}  \\ \xrightarrow{\texttt{True}}   
\end{matrix}
\left[
\begin{matrix}
- \\ f_1 \\ - \\ f_2 \\ f_{m-1} 
\end{matrix}
\right]
\]

\section{\texorpdfstring{\texttt{filter}.}{filter.}}\label{filter.}

En Python existe la función \texttt{filter(function,\ sequence)} cuyo
primer parámetro es una función la cual se va a aplicar a una secuencia,
la cual es el segundo parámetro. La función debe regresar un objeto de
tipo Booleano: \texttt{True} o \texttt{False}. El resultado será una
nueva secuencia con los elementos obtenidos de aplicar la función a cada
elemento de la secuencia de entrada.

\section{\texorpdfstring{\textbf{Ejemplo
5.}}{Ejemplo 5.}}\label{ejemplo-5.-1}

Usando la función \texttt{filter()}, encontrar los números pares en una
lista.

\begin{Shaded}
\begin{Highlighting}[]
\KeywordTok{def}\NormalTok{ esPar(n):}
    \CommentTok{"""}
\CommentTok{    Función que determina si un número es par o impar.}
\CommentTok{    """}
    \ControlFlowTok{if}\NormalTok{ n}\OperatorTok{\%}\DecValTok{2} \OperatorTok{==} \DecValTok{0}\NormalTok{:}
        \ControlFlowTok{return} \VariableTok{True}
    \ControlFlowTok{else}\NormalTok{:}
        \ControlFlowTok{return} \VariableTok{False}
\end{Highlighting}
\end{Shaded}

\begin{Shaded}
\begin{Highlighting}[]
\CommentTok{\# Probamos la función}
\BuiltInTok{print}\NormalTok{(esPar(}\DecValTok{10}\NormalTok{))}
\BuiltInTok{print}\NormalTok{(esPar(}\DecValTok{9}\NormalTok{))}
\end{Highlighting}
\end{Shaded}

\begin{verbatim}
True
False
\end{verbatim}

\begin{Shaded}
\begin{Highlighting}[]
\CommentTok{\# Creamos una lista de números, del 0 al 19}
\NormalTok{numeros }\OperatorTok{=} \BuiltInTok{list}\NormalTok{(}\BuiltInTok{range}\NormalTok{(}\DecValTok{20}\NormalTok{))}
\BuiltInTok{print}\NormalTok{(numeros)}
\end{Highlighting}
\end{Shaded}

\begin{verbatim}
[0, 1, 2, 3, 4, 5, 6, 7, 8, 9, 10, 11, 12, 13, 14, 15, 16, 17, 18, 19]
\end{verbatim}

\begin{Shaded}
\begin{Highlighting}[]
\CommentTok{\# Aplicamos el filtro}
\BuiltInTok{print}\NormalTok{(}\BuiltInTok{list}\NormalTok{(}\BuiltInTok{filter}\NormalTok{(esPar, numeros)))}
\end{Highlighting}
\end{Shaded}

\begin{verbatim}
[0, 2, 4, 6, 8, 10, 12, 14, 16, 18]
\end{verbatim}

\section{\texorpdfstring{\textbf{Ejemplo
6.}}{Ejemplo 6.}}\label{ejemplo-6.-1}

Encontrar los números pares en una lista que contiene elementos de
muchos tipos.

\textbf{Paso 1.} Creamos la lista.

\begin{Shaded}
\begin{Highlighting}[]
\CommentTok{\# Creamos la lista}
\NormalTok{lista }\OperatorTok{=}\NormalTok{ [}\StringTok{\textquotesingle{}Hola\textquotesingle{}}\NormalTok{, }\DecValTok{4}\NormalTok{, }\FloatTok{3.1416}\NormalTok{, }\DecValTok{3}\NormalTok{, }\DecValTok{8}\NormalTok{, (}\StringTok{\textquotesingle{}a\textquotesingle{}}\NormalTok{,}\DecValTok{2}\NormalTok{), }\DecValTok{10}\NormalTok{, \{}\StringTok{\textquotesingle{}x\textquotesingle{}}\NormalTok{:}\FloatTok{1.5}\NormalTok{, }\StringTok{\textquotesingle{}y\textquotesingle{}}\NormalTok{:}\DecValTok{12}\NormalTok{\} ]}
\BuiltInTok{print}\NormalTok{(lista)}
\end{Highlighting}
\end{Shaded}

\begin{verbatim}
['Hola', 4, 3.1416, 3, 8, ('a', 2), 10, {'x': 1.5, 'y': 12}]
\end{verbatim}

\textbf{Paso 2.} Escribimos una función que verifique si una entrada es
de tipo \texttt{int}.

\begin{Shaded}
\begin{Highlighting}[]
\KeywordTok{def}\NormalTok{ esEntero(i):}
    \CommentTok{"""}
\CommentTok{    Función que determina si un número es entero.}
\CommentTok{    """}
    \ControlFlowTok{if} \BuiltInTok{isinstance}\NormalTok{(i, }\BuiltInTok{int}\NormalTok{): }\CommentTok{\# Checamos si la entrada es de tipo int}
        \ControlFlowTok{return} \VariableTok{True}
    \ControlFlowTok{else}\NormalTok{:}
        \ControlFlowTok{return} \VariableTok{False}
\end{Highlighting}
\end{Shaded}

\begin{Shaded}
\begin{Highlighting}[]
\BuiltInTok{print}\NormalTok{(esEntero(}\StringTok{"Hola"}\NormalTok{))}
\BuiltInTok{print}\NormalTok{(esEntero(}\DecValTok{1}\NormalTok{))}
\BuiltInTok{print}\NormalTok{(esEntero(}\FloatTok{1.}\NormalTok{))}
\end{Highlighting}
\end{Shaded}

\begin{verbatim}
False
True
False
\end{verbatim}

Una forma alternativa, más Pythonica, de construir la función
\texttt{esEntero()} es la siguiente:

\begin{Shaded}
\begin{Highlighting}[]
\KeywordTok{def}\NormalTok{ esEntero(i):}
    \ControlFlowTok{return} \VariableTok{True} \ControlFlowTok{if} \BuiltInTok{isinstance}\NormalTok{(i, }\BuiltInTok{int}\NormalTok{) }\ControlFlowTok{else} \VariableTok{False}
\end{Highlighting}
\end{Shaded}

\textbf{Paso 3.} Usamos la función \texttt{esPar()} para encontrar los
pares de la lista.

\begin{Shaded}
\begin{Highlighting}[]
\BuiltInTok{print}\NormalTok{(}\BuiltInTok{list}\NormalTok{(}\BuiltInTok{filter}\NormalTok{(esPar, }\BuiltInTok{list}\NormalTok{(}\BuiltInTok{filter}\NormalTok{(esEntero,lista)))))}
\end{Highlighting}
\end{Shaded}

\begin{verbatim}
[4, 8, 10]
\end{verbatim}

Observa que se aplicó dos veces la función \texttt{filter()}, la primera
para determinar si el elemento de la lista es entero usando la función
\texttt{esEntero()}, la segunda para determinar si el número es par.

\section{\texorpdfstring{\textbf{Ejemplo
7.}}{Ejemplo 7.}}\label{ejemplo-7.-1}

Encontrar los números primos en el conjunto \(\{2, \dots, 50\}\).

\begin{Shaded}
\begin{Highlighting}[]
\CommentTok{\# Función que crear una }
\KeywordTok{def}\NormalTok{ noPrimo():}
    \CommentTok{"""}
\CommentTok{    Determina la lista de números que no son primos en el }
\CommentTok{    rango [2, 50]}
\CommentTok{    """}
\NormalTok{    np\_list }\OperatorTok{=}\NormalTok{ []}
    \ControlFlowTok{for}\NormalTok{ i }\KeywordTok{in} \BuiltInTok{range}\NormalTok{(}\DecValTok{2}\NormalTok{,}\DecValTok{8}\NormalTok{):}
        \ControlFlowTok{for}\NormalTok{ j }\KeywordTok{in} \BuiltInTok{range}\NormalTok{(i}\OperatorTok{*}\DecValTok{2}\NormalTok{, }\DecValTok{50}\NormalTok{, i):}
\NormalTok{            np\_list.append(j)}
    \ControlFlowTok{return}\NormalTok{ np\_list}

\NormalTok{no\_primo }\OperatorTok{=}\NormalTok{ noPrimo()}

\BuiltInTok{print}\NormalTok{(}\StringTok{"Lista de números NO primos en el rango [2, 50] }\CharTok{\textbackslash{}n}\SpecialCharTok{\{\}}\StringTok{"}\NormalTok{.}\BuiltInTok{format}\NormalTok{(no\_primo))}

\KeywordTok{def}\NormalTok{ esPrimo(number):}
    \CommentTok{"""}
\CommentTok{    Determina si un número es primo o no.}
\CommentTok{    """}
\NormalTok{    np\_list }\OperatorTok{=}\NormalTok{ noPrimo()}
    \ControlFlowTok{if}\NormalTok{(number }\KeywordTok{not} \KeywordTok{in}\NormalTok{ np\_list):}
        \ControlFlowTok{return} \VariableTok{True}

\CommentTok{\# Creación de la lista de números enteros de 2 a 50}
\NormalTok{numeros }\OperatorTok{=} \BuiltInTok{list}\NormalTok{(}\BuiltInTok{range}\NormalTok{(}\DecValTok{2}\NormalTok{,}\DecValTok{50}\NormalTok{))}

\CommentTok{\# Calculamos los primos usando filter(), con }
\CommentTok{\# la función esPrimo() y la lista números.}
\NormalTok{primos }\OperatorTok{=} \BuiltInTok{list}\NormalTok{(}\BuiltInTok{filter}\NormalTok{(esPrimo, numeros))}

\BuiltInTok{print}\NormalTok{(}\StringTok{"}\CharTok{\textbackslash{}n}\StringTok{Números primos en el rango [2, 50] }\CharTok{\textbackslash{}n}\StringTok{ }\SpecialCharTok{\{\}}\StringTok{"}\NormalTok{.}\BuiltInTok{format}\NormalTok{(primos))}
\end{Highlighting}
\end{Shaded}

\begin{verbatim}
Lista de números NO primos en el rango [2, 50] 
[4, 6, 8, 10, 12, 14, 16, 18, 20, 22, 24, 26, 28, 30, 32, 34, 36, 38, 40, 42, 44, 46, 48, 6, 9, 12, 15, 18, 21, 24, 27, 30, 33, 36, 39, 42, 45, 48, 8, 12, 16, 20, 24, 28, 32, 36, 40, 44, 48, 10, 15, 20, 25, 30, 35, 40, 45, 12, 18, 24, 30, 36, 42, 48, 14, 21, 28, 35, 42, 49]

Números primos en el rango [2, 50] 
 [2, 3, 5, 7, 11, 13, 17, 19, 23, 29, 31, 37, 41, 43, 47]
\end{verbatim}

\bookmarksetup{startatroot}

\chapter{Reducción.}\label{reducciuxf3n.}

\begin{itemize}
\item
  \textbf{Reducción} : Disminuir \emph{algo} en tamaño, cantidad, grado,
  importancia, ..
\item
  La operación de reducción es útil en muchos ámbitos, el objetivo es
  reducir un conjunto de objetos en un objeto más simple.
\end{itemize}

Una reducción se hace como sigue:

Dada la función \(f()\) y la secuencia \([s_1, s_2, s_3, s_4]\) se tiene
que

\[
[\underbrace{\underbrace{\underbrace{s_1, s_2}_{a = f(s_1,s_2)}, s_3}_{b = f(a,s_3)}, s_4}_{c = f(b,s_4)}] \qquad \Longrightarrow \qquad \underbrace{f(\underbrace{f(\underbrace{f(s_1,s_2)}_{a}, s_3)}_{b}, s_4)}_{c}
\]

\section{\texorpdfstring{\texttt{filter}.}{filter.}}\label{filter.-1}

En Python existe la función \texttt{reduce(function,\ sequence)} cuyo
primer parámetro es una función la cual se va a aplicar a una secuencia,
la cual es el segundo parámetro. La función debe regresar un objeto que
es el resultado de la reducción.

\section{\texorpdfstring{\textbf{Ejemplo
8.}}{Ejemplo 8.}}\label{ejemplo-8.}

Calcular la siguiente serie:

\(1 + 2 + \dots + n = \sum\limits_{i=1}^{n} i = \dfrac{n(n+1)}{2}\)

Si \(n = 4\) entonces 1+2+3+4 = 10

\begin{Shaded}
\begin{Highlighting}[]
\CommentTok{\# La función reduce() debe importarse del módulo functools}
\ImportTok{from}\NormalTok{ functools }\ImportTok{import} \BuiltInTok{reduce} 
\end{Highlighting}
\end{Shaded}

\begin{Shaded}
\begin{Highlighting}[]
\CommentTok{\# Creamos la lista}
\NormalTok{nums }\OperatorTok{=}\NormalTok{ [}\DecValTok{1}\NormalTok{,}\DecValTok{2}\NormalTok{,}\DecValTok{3}\NormalTok{,}\DecValTok{4}\NormalTok{]}
\BuiltInTok{print}\NormalTok{(nums)}

\CommentTok{\# Calculamos la serie usando reduce y una función lambda}
\NormalTok{suma }\OperatorTok{=} \BuiltInTok{reduce}\NormalTok{(}\KeywordTok{lambda}\NormalTok{ x, y: x }\OperatorTok{+}\NormalTok{ y, nums)}
\BuiltInTok{print}\NormalTok{(suma)}
\end{Highlighting}
\end{Shaded}

\begin{verbatim}
[1, 2, 3, 4]
10
\end{verbatim}

\begin{Shaded}
\begin{Highlighting}[]
\CommentTok{\# Se pueden usar arreglos de numpy}
\ImportTok{import}\NormalTok{ numpy }\ImportTok{as}\NormalTok{ np}

\CommentTok{\# Construimos un arreglo de 1\textquotesingle{}s}
\NormalTok{a }\OperatorTok{=}\NormalTok{ np.ones(}\DecValTok{20}\NormalTok{)}
\BuiltInTok{print}\NormalTok{(a)}

\NormalTok{suma }\OperatorTok{=} \BuiltInTok{reduce}\NormalTok{(}\KeywordTok{lambda}\NormalTok{ x, y: x }\OperatorTok{+}\NormalTok{ y, a)}
\BuiltInTok{print}\NormalTok{(suma)}
\end{Highlighting}
\end{Shaded}

\begin{verbatim}
[1. 1. 1. 1. 1. 1. 1. 1. 1. 1. 1. 1. 1. 1. 1. 1. 1. 1. 1. 1.]
20.0
\end{verbatim}

\section{\texorpdfstring{\textbf{Ejemplo
9.}}{Ejemplo 9.}}\label{ejemplo-9.}

Calcular la siguiente serie:

\$1 + \dfrac{1}{2} + \dots + \dfrac{1}{n} =
\sum\limits\_\{i=1\}\^{}\{n\} \dfrac{1}{i} = \$

\begin{Shaded}
\begin{Highlighting}[]
\NormalTok{numeros }\OperatorTok{=}\NormalTok{ [}\DecValTok{3}\NormalTok{,}\DecValTok{4}\NormalTok{,}\DecValTok{5}\NormalTok{]}
\NormalTok{result }\OperatorTok{=} \BuiltInTok{reduce}\NormalTok{(}\KeywordTok{lambda}\NormalTok{ x, y: }\DecValTok{1}\OperatorTok{/}\NormalTok{x }\OperatorTok{+} \DecValTok{1}\OperatorTok{/}\NormalTok{y, numeros)}
\BuiltInTok{print}\NormalTok{(result)}
\end{Highlighting}
\end{Shaded}

\begin{verbatim}
1.9142857142857144
\end{verbatim}

\section{\texorpdfstring{\textbf{Ejemplo
10.}}{Ejemplo 10.}}\label{ejemplo-10.}

Calcular el máximo de una lista de números.

\begin{Shaded}
\begin{Highlighting}[]
\NormalTok{numeros }\OperatorTok{=}\NormalTok{ [}\DecValTok{23}\NormalTok{,}\DecValTok{5}\NormalTok{,}\DecValTok{23}\NormalTok{,}\DecValTok{56}\NormalTok{,}\DecValTok{87}\NormalTok{,}\DecValTok{98}\NormalTok{,}\DecValTok{23}\NormalTok{]}
\NormalTok{maximo }\OperatorTok{=} \BuiltInTok{reduce}\NormalTok{(}\KeywordTok{lambda}\NormalTok{ a,b: a }\ControlFlowTok{if}\NormalTok{ (a }\OperatorTok{\textgreater{}}\NormalTok{ b) }\ControlFlowTok{else}\NormalTok{ b, numeros)}
\BuiltInTok{print}\NormalTok{(maximo)}
\end{Highlighting}
\end{Shaded}

\begin{verbatim}
98
\end{verbatim}

\section{\texorpdfstring{\textbf{Ejemplo
11.}}{Ejemplo 11.}}\label{ejemplo-11.}

Calcular el factorial de un número.

\bookmarksetup{startatroot}

\chapter{Más ejemplos Pythonicos.}\label{muxe1s-ejemplos-pythonicos.}

\section{\texorpdfstring{\textbf{Ejemplo
12.}}{Ejemplo 12.}}\label{ejemplo-12.}

Convertir grados Fahrenheit a Celsius y viceversa combinando
\texttt{map()} con \texttt{lambda}.

\begin{Shaded}
\begin{Highlighting}[]
\NormalTok{c }\OperatorTok{=}\NormalTok{ [}\DecValTok{0}\NormalTok{, }\FloatTok{22.5}\NormalTok{, }\DecValTok{40}\NormalTok{,}\DecValTok{100}\NormalTok{]}

\CommentTok{\# Conversión a Fahrenheit}
\NormalTok{f }\OperatorTok{=} \BuiltInTok{list}\NormalTok{(}\BuiltInTok{map}\NormalTok{(}\KeywordTok{lambda}\NormalTok{ T: (}\DecValTok{9}\OperatorTok{/}\DecValTok{5}\NormalTok{) }\OperatorTok{*}\NormalTok{ T }\OperatorTok{+} \DecValTok{32}\NormalTok{, c))}
\BuiltInTok{print}\NormalTok{(f)}

\CommentTok{\# Conversión a Celsius}
\BuiltInTok{print}\NormalTok{(}\BuiltInTok{list}\NormalTok{(}\BuiltInTok{map}\NormalTok{(}\KeywordTok{lambda}\NormalTok{ T: (}\DecValTok{5}\OperatorTok{/}\DecValTok{9}\NormalTok{)}\OperatorTok{*}\NormalTok{(T }\OperatorTok{{-}} \DecValTok{32}\NormalTok{), f)))}
\end{Highlighting}
\end{Shaded}

\begin{verbatim}
[32.0, 72.5, 104.0, 212.0]
[0.0, 22.5, 40.0, 100.0]
\end{verbatim}

\section{\texorpdfstring{\textbf{Ejemplo
13.}}{Ejemplo 13.}}\label{ejemplo-13.}

Sumar tres arreglos combinando \texttt{map()} con \texttt{lambda}.

\begin{Shaded}
\begin{Highlighting}[]
\NormalTok{a }\OperatorTok{=}\NormalTok{ [}\DecValTok{1}\NormalTok{,}\DecValTok{2}\NormalTok{,}\DecValTok{3}\NormalTok{,}\DecValTok{4}\NormalTok{]}
\NormalTok{b }\OperatorTok{=}\NormalTok{ [}\DecValTok{5}\NormalTok{,}\DecValTok{6}\NormalTok{,}\DecValTok{7}\NormalTok{,}\DecValTok{8}\NormalTok{]}
\NormalTok{c }\OperatorTok{=}\NormalTok{ [}\DecValTok{9}\NormalTok{,}\DecValTok{10}\NormalTok{,}\DecValTok{11}\NormalTok{,}\DecValTok{12}\NormalTok{]}

\BuiltInTok{print}\NormalTok{(}\BuiltInTok{list}\NormalTok{(}\BuiltInTok{map}\NormalTok{(}\KeywordTok{lambda}\NormalTok{ x,y,z : x}\OperatorTok{+}\NormalTok{y}\OperatorTok{+}\NormalTok{z, a,b,c)))}
\end{Highlighting}
\end{Shaded}

\begin{verbatim}
[15, 18, 21, 24]
\end{verbatim}

\section{\texorpdfstring{\textbf{Ejemplo
14.}}{Ejemplo 14.}}\label{ejemplo-14.}

Encontrar todos los números pares de una lista combinando
\texttt{filter()} con \texttt{lambda}.

\begin{Shaded}
\begin{Highlighting}[]
\CommentTok{\# Lista de números}
\NormalTok{nums }\OperatorTok{=}\NormalTok{ [}\DecValTok{0}\NormalTok{, }\DecValTok{2}\NormalTok{, }\DecValTok{5}\NormalTok{, }\DecValTok{8}\NormalTok{, }\DecValTok{10}\NormalTok{, }\DecValTok{23}\NormalTok{, }\DecValTok{31}\NormalTok{, }\DecValTok{35}\NormalTok{, }\DecValTok{36}\NormalTok{, }\DecValTok{47}\NormalTok{, }\DecValTok{50}\NormalTok{, }\DecValTok{77}\NormalTok{, }\DecValTok{93}\NormalTok{]}

\CommentTok{\# Aplicación de filter y lambda}
\NormalTok{result }\OperatorTok{=} \BuiltInTok{filter}\NormalTok{(}\KeywordTok{lambda}\NormalTok{ x : x }\OperatorTok{\%} \DecValTok{2} \OperatorTok{==} \DecValTok{0}\NormalTok{, nums)}

\BuiltInTok{print}\NormalTok{(}\BuiltInTok{list}\NormalTok{(result))}
\end{Highlighting}
\end{Shaded}

\begin{verbatim}
[0, 2, 8, 10, 36, 50]
\end{verbatim}

\section{\texorpdfstring{\textbf{Ejemplo
15.}}{Ejemplo 15.}}\label{ejemplo-15.}

Encontrar todos los números primos en el conjunto \(\{2, \dots, 50\}\)
combinando combinando \texttt{filter()} con \texttt{lambda}.

\begin{Shaded}
\begin{Highlighting}[]
\CommentTok{\# Lista de números de 2 a 50}
\NormalTok{nums }\OperatorTok{=} \BuiltInTok{list}\NormalTok{(}\BuiltInTok{range}\NormalTok{(}\DecValTok{2}\NormalTok{, }\DecValTok{51}\NormalTok{)) }

\CommentTok{\# Cálculo de los números primos usando}
\CommentTok{\# filter y lambda}
\ControlFlowTok{for}\NormalTok{ i }\KeywordTok{in} \BuiltInTok{range}\NormalTok{(}\DecValTok{2}\NormalTok{, }\DecValTok{8}\NormalTok{):}
\NormalTok{    nums }\OperatorTok{=} \BuiltInTok{list}\NormalTok{(}\BuiltInTok{filter}\NormalTok{(}\KeywordTok{lambda}\NormalTok{ x: x }\OperatorTok{==}\NormalTok{ i }\KeywordTok{or}\NormalTok{ x }\OperatorTok{\%}\NormalTok{ i, nums))}

\BuiltInTok{print}\NormalTok{(nums)}
\end{Highlighting}
\end{Shaded}

\begin{verbatim}
[2, 3, 5, 7, 11, 13, 17, 19, 23, 29, 31, 37, 41, 43, 47]
\end{verbatim}

\section{\texorpdfstring{\textbf{Ejemplo
16.}}{Ejemplo 16.}}\label{ejemplo-16.}

Contar el número de caractéres de un texto, combinando
\texttt{reduce()}, \texttt{map(\ )} y \texttt{lambda}.

\begin{Shaded}
\begin{Highlighting}[]
\CommentTok{\# Contar los caracteres de una cadena}
\NormalTok{texto }\OperatorTok{=} \StringTok{\textquotesingle{}Hola Mundo\textquotesingle{}}

\NormalTok{palabras }\OperatorTok{=}\NormalTok{ texto.split()}
\BuiltInTok{print}\NormalTok{(palabras)}

\CommentTok{\# Conteo de caracteres}
\BuiltInTok{print}\NormalTok{(}\BuiltInTok{reduce}\NormalTok{(}\KeywordTok{lambda}\NormalTok{ x,y: x}\OperatorTok{+}\NormalTok{y, }\BuiltInTok{list}\NormalTok{(}\BuiltInTok{map}\NormalTok{(}\KeywordTok{lambda}\NormalTok{ palabras: }\BuiltInTok{len}\NormalTok{(palabras), palabras))))}
\end{Highlighting}
\end{Shaded}

\begin{verbatim}
['Hola', 'Mundo']
9
\end{verbatim}

\begin{Shaded}
\begin{Highlighting}[]
\CommentTok{\# Contar los caracteres de un texto en un archivo}
\NormalTok{archivo }\OperatorTok{=} \BuiltInTok{open}\NormalTok{(}\StringTok{\textquotesingle{}QueLesQuedaALosJovenes.txt\textquotesingle{}}\NormalTok{,}\StringTok{\textquotesingle{}r\textquotesingle{}}\NormalTok{)}

\NormalTok{suma }\OperatorTok{=} \DecValTok{0}
\ControlFlowTok{for}\NormalTok{ linea }\KeywordTok{in}\NormalTok{ archivo:}
\NormalTok{    palabras }\OperatorTok{=}\NormalTok{ linea.split()}
\NormalTok{    suma }\OperatorTok{+=} \BuiltInTok{reduce}\NormalTok{(}\KeywordTok{lambda}\NormalTok{ x,y: x}\OperatorTok{+}\NormalTok{y, }\BuiltInTok{list}\NormalTok{(}\BuiltInTok{map}\NormalTok{(}\KeywordTok{lambda}\NormalTok{ palabras: }\BuiltInTok{len}\NormalTok{(palabras), palabras)))}
\BuiltInTok{print}\NormalTok{(suma)}
\NormalTok{archivo.close()}
\end{Highlighting}
\end{Shaded}

\begin{verbatim}
824
\end{verbatim}

Lo anterior se puede realizar si construir una función que cuenta los
caracteres de una lista de cadenas:

\begin{Shaded}
\begin{Highlighting}[]
\KeywordTok{def}\NormalTok{ cuentaCaracteres(palabras):}
    \ControlFlowTok{return} \BuiltInTok{reduce}\NormalTok{(}\KeywordTok{lambda}\NormalTok{ x,y: x}\OperatorTok{+}\NormalTok{y, }\BuiltInTok{list}\NormalTok{(}\BuiltInTok{map}\NormalTok{(}\KeywordTok{lambda}\NormalTok{ palabras: }\BuiltInTok{len}\NormalTok{(palabras), palabras)))}
\end{Highlighting}
\end{Shaded}

\begin{Shaded}
\begin{Highlighting}[]
\NormalTok{texto }\OperatorTok{=} \StringTok{\textquotesingle{}Hello Motto\textquotesingle{}}\NormalTok{.split()}
\NormalTok{cuentaCaracteres(texto)}
\end{Highlighting}
\end{Shaded}

\begin{verbatim}
10
\end{verbatim}

\begin{Shaded}
\begin{Highlighting}[]
\CommentTok{\# Contar los caracteres de un texto en un archivo}
\NormalTok{archivo }\OperatorTok{=} \BuiltInTok{open}\NormalTok{(}\StringTok{\textquotesingle{}QueLesQuedaALosJovenes.txt\textquotesingle{}}\NormalTok{,}\StringTok{\textquotesingle{}r\textquotesingle{}}\NormalTok{)}

\NormalTok{suma }\OperatorTok{=} \DecValTok{0}
\ControlFlowTok{for}\NormalTok{ linea }\KeywordTok{in}\NormalTok{ archivo:}
\NormalTok{    palabras }\OperatorTok{=}\NormalTok{ linea.split()}
\NormalTok{    suma }\OperatorTok{+=}\NormalTok{ cuentaCaracteres(palabras)}
\BuiltInTok{print}\NormalTok{(suma)}
\NormalTok{archivo.close()}
\end{Highlighting}
\end{Shaded}

\begin{verbatim}
824
\end{verbatim}

\section{\texorpdfstring{\textbf{Ejemplo
17.}}{Ejemplo 17.}}\label{ejemplo-17.}

La siguiente función es impura porque modifica la \texttt{lista}:

\begin{Shaded}
\begin{Highlighting}[]
\KeywordTok{def}\NormalTok{ cuadradosImpuros(lista):}
    \ControlFlowTok{for}\NormalTok{ i, v }\KeywordTok{in} \BuiltInTok{enumerate}\NormalTok{(lista):}
\NormalTok{        lista[i] }\OperatorTok{=}\NormalTok{ v }\OperatorTok{**} \DecValTok{2}
    \ControlFlowTok{return}\NormalTok{ lista}

\NormalTok{numeros\_originales }\OperatorTok{=} \BuiltInTok{list}\NormalTok{(}\BuiltInTok{range}\NormalTok{(}\DecValTok{5}\NormalTok{))}
\BuiltInTok{print}\NormalTok{(numeros\_originales)}
\BuiltInTok{print}\NormalTok{(cuadradosImpuros(numeros\_originales))}
\BuiltInTok{print}\NormalTok{(numeros\_originales)}
\end{Highlighting}
\end{Shaded}

La salida del código anterior es el siguiente:

\begin{Shaded}
\begin{Highlighting}[]
\NormalTok{[}\DecValTok{0}\NormalTok{, }\DecValTok{1}\NormalTok{, }\DecValTok{2}\NormalTok{, }\DecValTok{3}\NormalTok{, }\DecValTok{4}\NormalTok{]}
\NormalTok{[}\DecValTok{0}\NormalTok{, }\DecValTok{1}\NormalTok{, }\DecValTok{4}\NormalTok{, }\DecValTok{9}\NormalTok{, }\DecValTok{16}\NormalTok{]}
\NormalTok{[}\DecValTok{0}\NormalTok{, }\DecValTok{1}\NormalTok{, }\DecValTok{4}\NormalTok{, }\DecValTok{9}\NormalTok{, }\DecValTok{16}\NormalTok{]}
\end{Highlighting}
\end{Shaded}

Escribe una versión pura de la función \texttt{cuadradosImpuros(lista)}
usando \texttt{map()} y \texttt{lambda}.

Una manera alternativa es la siguiente:

\begin{Shaded}
\begin{Highlighting}[]
\NormalTok{numeros\_originales }\OperatorTok{=} \BuiltInTok{list}\NormalTok{(}\BuiltInTok{range}\NormalTok{(}\DecValTok{5}\NormalTok{))}
\BuiltInTok{print}\NormalTok{(numeros\_originales)}
\BuiltInTok{print}\NormalTok{(}\BuiltInTok{list}\NormalTok{(}\BuiltInTok{map}\NormalTok{(}\KeywordTok{lambda}\NormalTok{ x: x }\OperatorTok{**} \DecValTok{2}\NormalTok{, numeros\_originales)))}
\BuiltInTok{print}\NormalTok{(numeros\_originales)}
\end{Highlighting}
\end{Shaded}

\begin{verbatim}
[0, 1, 2, 3, 4]
[0, 1, 4, 9, 16]
[0, 1, 2, 3, 4]
\end{verbatim}

\bookmarksetup{startatroot}

\chapter{Estructuras de datos
concisas.}\label{estructuras-de-datos-concisas.}

\textbf{Objetivo.} \ldots{}

\textbf{Funciones de Python}: \ldots{}

MACTI-Algebra\_Lineal\_01 by Luis M. de la Cruz is licensed under
Attribution-ShareAlike 4.0 International

\bookmarksetup{startatroot}

\chapter{Listas concisas}\label{listas-concisas}

En matemáticas podemos definir un conjunto como sigue:

\[S = \{x^2 : x \in (0, 1, 2, \dots, 9)\} = \{0, 1, 4, \dots, 81\}\]

En Python es posible crear este conjunto usando lo que conoce como
\emph{list comprehensions} (generación corta de listas) como sigue:

\begin{Shaded}
\begin{Highlighting}[]
\NormalTok{S }\OperatorTok{=}\NormalTok{ [x}\OperatorTok{**}\DecValTok{2} \ControlFlowTok{for}\NormalTok{ x }\KeywordTok{in} \BuiltInTok{range}\NormalTok{(}\DecValTok{10}\NormalTok{)]}
\BuiltInTok{print}\NormalTok{(S)}
\end{Highlighting}
\end{Shaded}

\begin{verbatim}
[0, 1, 4, 9, 16, 25, 36, 49, 64, 81]
\end{verbatim}

Las listas concisas son usadas para construir listas de una manera muy
concisa, natural y fácil, como lo hace un matemático. La forma precisa
de construir listas concisas es como sigue:

\begin{Shaded}
\begin{Highlighting}[]
\NormalTok{[ expresion }\ControlFlowTok{for}\NormalTok{ i }\KeywordTok{in}\NormalTok{ S }\ControlFlowTok{if}\NormalTok{ predicado ]}
\end{Highlighting}
\end{Shaded}

Donde \texttt{expresion} es una expresión que se va a aplicar a cada
elemento \texttt{i} de la secuencia \texttt{S}; opcionalmente, es
posible aplicar el \texttt{predicado} antes de aplicar la
\texttt{expresion} a cada elemento \texttt{i}.

\section{\texorpdfstring{\textbf{Ejemplo
1.}}{Ejemplo 1.}}\label{ejemplo-1.-4}

Usando listas concisas, crear el siguiente conjunto:

\[
M = \{\sqrt{x} : x \in (1,2,3,4,5,6,7,8,9,10) \;\; \text{y} \;\; x \;\; \text{es par}\} = \{ \sqrt{2}, \sqrt{4}, \sqrt{6}, \sqrt{8}, \sqrt{10}) \}
\]

\begin{Shaded}
\begin{Highlighting}[]
\ImportTok{from}\NormalTok{ math }\ImportTok{import}\NormalTok{ sqrt}

\NormalTok{M }\OperatorTok{=}\NormalTok{ [sqrt(x) }\ControlFlowTok{for}\NormalTok{ x }\KeywordTok{in} \BuiltInTok{range}\NormalTok{(}\DecValTok{2}\NormalTok{,}\DecValTok{11}\NormalTok{) }\ControlFlowTok{if}\NormalTok{ x}\OperatorTok{\%}\DecValTok{2} \OperatorTok{==} \DecValTok{0}\NormalTok{]}
\BuiltInTok{print}\NormalTok{(M)}
\end{Highlighting}
\end{Shaded}

\begin{verbatim}
[1.4142135623730951, 2.0, 2.449489742783178, 2.8284271247461903, 3.1622776601683795]
\end{verbatim}

En el ejemplo anterior se distingue lo siguiente:

\begin{enumerate}
\def\labelenumi{\arabic{enumi}.}
\item
  La secuencia de entrada: \texttt{range(2,11)}
  (\texttt{{[}2,\ 3,\ 4,\ 5,\ 6,\ 7,\ 8,\ 9,\ 10{]}}).
\item
  La etiqueta \texttt{i} que representa los miembros de la secuencia de
  entrada.
\item
  La expresión de predicado: \texttt{if\ x\ \%\ 2\ ==\ 0}.
\item
  La expresión de salida \texttt{sqrt(x)} que produce los elementos de
  la lista resultado, los cuales provienen de los miembros de la
  secuencia de entrada que satisfacen el predicado.
\end{enumerate}

\section{\texorpdfstring{\textbf{Ejemplo
2.}}{Ejemplo 2.}}\label{ejemplo-2.-2}

Obtener todos los enteros de la siguiente lista, elevarlos al cuadrado y
poner el resultado en una lista:

\begin{Shaded}
\begin{Highlighting}[]
\NormalTok{    lista }\OperatorTok{=}\NormalTok{ [}\DecValTok{1}\NormalTok{,}\StringTok{\textquotesingle{}4\textquotesingle{}}\NormalTok{,}\DecValTok{9}\NormalTok{,}\StringTok{\textquotesingle{}luiggi\textquotesingle{}}\NormalTok{,}\DecValTok{0}\NormalTok{,}\DecValTok{4}\NormalTok{,(}\StringTok{\textquotesingle{}mike\textquotesingle{}}\NormalTok{,}\StringTok{\textquotesingle{}dela+\textquotesingle{}}\NormalTok{)]}
\end{Highlighting}
\end{Shaded}

\begin{Shaded}
\begin{Highlighting}[]
\NormalTok{lista }\OperatorTok{=}\NormalTok{ [}\DecValTok{1}\NormalTok{,}\StringTok{\textquotesingle{}4\textquotesingle{}}\NormalTok{,}\DecValTok{9}\NormalTok{,}\StringTok{\textquotesingle{}luiggi\textquotesingle{}}\NormalTok{,}\DecValTok{0}\NormalTok{,}\DecValTok{4}\NormalTok{,(}\StringTok{\textquotesingle{}mike\textquotesingle{}}\NormalTok{,}\StringTok{\textquotesingle{}dela+\textquotesingle{}}\NormalTok{)]}
\end{Highlighting}
\end{Shaded}

\begin{Shaded}
\begin{Highlighting}[]
\NormalTok{resultado }\OperatorTok{=}\NormalTok{ [x}\OperatorTok{**}\DecValTok{2} \ControlFlowTok{for}\NormalTok{ x }\KeywordTok{in}\NormalTok{ lista }\ControlFlowTok{if} \BuiltInTok{isinstance}\NormalTok{(x, }\BuiltInTok{int}\NormalTok{)]}
\BuiltInTok{print}\NormalTok{( resultado )}
\end{Highlighting}
\end{Shaded}

\begin{verbatim}
[1, 81, 0, 16]
\end{verbatim}

\section{\texorpdfstring{\textbf{Ejemplo
3.}}{Ejemplo 3.}}\label{ejemplo-3.-2}

Crear la siguiente lista:
\[V = (2^0,2^1,2^2, \dots, 2^{12}) = (1, 2, 4, 8, \dots, 4096) \]

\begin{Shaded}
\begin{Highlighting}[]
\NormalTok{[}\DecValTok{2}\OperatorTok{**}\NormalTok{x }\ControlFlowTok{for}\NormalTok{ x }\KeywordTok{in} \BuiltInTok{range}\NormalTok{(}\DecValTok{13}\NormalTok{)]}
\end{Highlighting}
\end{Shaded}

\begin{verbatim}
[1, 2, 4, 8, 16, 32, 64, 128, 256, 512, 1024, 2048, 4096]
\end{verbatim}

\section{\texorpdfstring{\textbf{Ejemplo
4.}}{Ejemplo 4.}}\label{ejemplo-4.-2}

Transformar grados Celsius en Fahrenheit y vieceversa.

\textbf{Celsius \(\to\) Fahrenheit}

\begin{Shaded}
\begin{Highlighting}[]
\NormalTok{c }\OperatorTok{=}\NormalTok{ [}\DecValTok{0}\NormalTok{, }\FloatTok{22.5}\NormalTok{, }\DecValTok{40}\NormalTok{,}\DecValTok{100}\NormalTok{]}
\NormalTok{f }\OperatorTok{=}\NormalTok{ [(}\DecValTok{9}\OperatorTok{/}\DecValTok{5}\NormalTok{)}\OperatorTok{*}\NormalTok{t }\OperatorTok{+} \DecValTok{32} \ControlFlowTok{for}\NormalTok{ t }\KeywordTok{in}\NormalTok{ c]}
\BuiltInTok{print}\NormalTok{(f)}
\end{Highlighting}
\end{Shaded}

\begin{verbatim}
[32.0, 72.5, 104.0, 212.0]
\end{verbatim}

\textbf{Fahrenheit \(\to\) Celsius}

\begin{Shaded}
\begin{Highlighting}[]
\NormalTok{cn }\OperatorTok{=}\NormalTok{ [(}\DecValTok{5}\OperatorTok{/}\DecValTok{9}\NormalTok{)}\OperatorTok{*}\NormalTok{(t }\OperatorTok{{-}} \DecValTok{32}\NormalTok{) }\ControlFlowTok{for}\NormalTok{ t }\KeywordTok{in}\NormalTok{ f]}
\BuiltInTok{print}\NormalTok{(cn)}
\end{Highlighting}
\end{Shaded}

\begin{verbatim}
[0.0, 22.5, 40.0, 100.0]
\end{verbatim}

Observa que en ambos ejemplos la expresión es la fórmula de conversión
entre grados.

\bookmarksetup{startatroot}

\chapter{Anidado de listas concisas.}\label{anidado-de-listas-concisas.}

\section{\texorpdfstring{\textbf{Ejemplo
5.}}{Ejemplo 5.}}\label{ejemplo-5.-2}

Crear la siguiente lista:

\[M = \{\sqrt{x} \, |\, x \in S \text{ y } x \text{ impar }\}\]

con

\[S = \{x^2 : x \in (0 \dots 9)\} = \{0, 1, 4, \dots, 81\}\]

Este ejemplo se puede realizar como sigue:

\begin{Shaded}
\begin{Highlighting}[]
\NormalTok{S }\OperatorTok{=}\NormalTok{ [x}\OperatorTok{**}\DecValTok{2} \ControlFlowTok{for}\NormalTok{ x }\KeywordTok{in} \BuiltInTok{range}\NormalTok{(}\DecValTok{0}\NormalTok{,}\DecValTok{10}\NormalTok{)]}
\NormalTok{M }\OperatorTok{=}\NormalTok{ [sqrt(x) }\ControlFlowTok{for}\NormalTok{ x }\KeywordTok{in}\NormalTok{ S }\ControlFlowTok{if}\NormalTok{ x}\OperatorTok{\%}\DecValTok{2}\NormalTok{]}
\BuiltInTok{print}\NormalTok{(}\StringTok{\textquotesingle{}S = }\SpecialCharTok{\{\}}\StringTok{\textquotesingle{}}\NormalTok{.}\BuiltInTok{format}\NormalTok{(S))}
\BuiltInTok{print}\NormalTok{(}\StringTok{\textquotesingle{}M = }\SpecialCharTok{\{\}}\StringTok{\textquotesingle{}}\NormalTok{.}\BuiltInTok{format}\NormalTok{(M))}
\end{Highlighting}
\end{Shaded}

\begin{verbatim}
S = [0, 1, 4, 9, 16, 25, 36, 49, 64, 81]
M = [1.0, 3.0, 5.0, 7.0, 9.0]
\end{verbatim}

Sin embargo, es posible anidar las listas concisas como sigue:

\begin{Shaded}
\begin{Highlighting}[]
\NormalTok{M }\OperatorTok{=}\NormalTok{ [sqrt(x) }\ControlFlowTok{for}\NormalTok{ x }\KeywordTok{in}\NormalTok{ [x}\OperatorTok{**}\DecValTok{2} \ControlFlowTok{for}\NormalTok{ x }\KeywordTok{in} \BuiltInTok{range}\NormalTok{(}\DecValTok{0}\NormalTok{,}\DecValTok{10}\NormalTok{)] }\ControlFlowTok{if}\NormalTok{ x}\OperatorTok{\%}\DecValTok{2}\NormalTok{]}
\BuiltInTok{print}\NormalTok{(}\StringTok{\textquotesingle{}M = }\SpecialCharTok{\{\}}\StringTok{\textquotesingle{}}\NormalTok{.}\BuiltInTok{format}\NormalTok{(M))}
\end{Highlighting}
\end{Shaded}

\begin{verbatim}
M = [1.0, 3.0, 5.0, 7.0, 9.0]
\end{verbatim}

\section{\texorpdfstring{\textbf{Ejemplo
6.}}{Ejemplo 6.}}\label{ejemplo-6.-2}

Sea una matriz identidad de tamaño \(n \times n\) :

\[
\left[
\begin{matrix}
1 & 0 & 0 & \dots & 0 \\
0 & 1 & 0 & \dots & 0 \\
0 & 0 & 1 & \dots & 0 \\
\vdots&\vdots&\vdots&\ddots&\vdots \\
0 & 0 & 0 & \dots & 1 \\
\end{matrix}
\right]
\]

En Python esta matriz se puede representar por la siguiente lista:

\begin{Shaded}
\begin{Highlighting}[]
\NormalTok{[[}\DecValTok{1}\NormalTok{,}\DecValTok{0}\NormalTok{,}\DecValTok{0}\NormalTok{, ... , }\DecValTok{0}\NormalTok{],}
\NormalTok{ [}\DecValTok{0}\NormalTok{,}\DecValTok{1}\NormalTok{,}\DecValTok{0}\NormalTok{, ... , }\DecValTok{0}\NormalTok{],}
\NormalTok{ [}\DecValTok{0}\NormalTok{,}\DecValTok{0}\NormalTok{,}\DecValTok{1}\NormalTok{, ... , }\DecValTok{0}\NormalTok{],}
\NormalTok{ ................,     }
\NormalTok{ [}\DecValTok{0}\NormalTok{,}\DecValTok{0}\NormalTok{,}\DecValTok{0}\NormalTok{, ... , }\DecValTok{1}\NormalTok{]]}
\end{Highlighting}
\end{Shaded}

\begin{itemize}
\tightlist
\item
  Usando \emph{list comprehensions} anidados se puede obtener dicha
  lista:
\end{itemize}

\begin{Shaded}
\begin{Highlighting}[]
\NormalTok{n }\OperatorTok{=} \DecValTok{8}
\NormalTok{[[}\DecValTok{1} \ControlFlowTok{if}\NormalTok{ col }\OperatorTok{==}\NormalTok{ row }\ControlFlowTok{else} \DecValTok{0} \ControlFlowTok{for}\NormalTok{ col }\KeywordTok{in} \BuiltInTok{range}\NormalTok{(}\DecValTok{0}\NormalTok{,n)] }\ControlFlowTok{for}\NormalTok{ row }\KeywordTok{in} \BuiltInTok{range}\NormalTok{(}\DecValTok{0}\NormalTok{,n)]}
\end{Highlighting}
\end{Shaded}

\begin{verbatim}
[[1, 0, 0, 0, 0, 0, 0, 0],
 [0, 1, 0, 0, 0, 0, 0, 0],
 [0, 0, 1, 0, 0, 0, 0, 0],
 [0, 0, 0, 1, 0, 0, 0, 0],
 [0, 0, 0, 0, 1, 0, 0, 0],
 [0, 0, 0, 0, 0, 1, 0, 0],
 [0, 0, 0, 0, 0, 0, 1, 0],
 [0, 0, 0, 0, 0, 0, 0, 1]]
\end{verbatim}

Observa que en este caso la expresión de salida es una lista concisa:
\texttt{{[}1\ if\ col\ ==\ row\ else\ 0\ for\ col\ in\ range(0,n){]}} .

Además, la expresión de salida de esta última lista concisa es el
operador ternario: \texttt{1\ if\ col\ ==\ row\ else\ 0}

\section{\texorpdfstring{\textbf{Ejemplo
7.}}{Ejemplo 7.}}\label{ejemplo-7.-2}

Calcular números primos en el rango \texttt{{[}2,50{]}}.

En este ejercicio se usa el algoritmo conocido como criba de
Eratóstenes. Primero se encuentran todos aquellos números que tengan
algún múltiplo. En este caso solo vamos a buscar en el intervalo
\([2,50]\).

La siguiente lista concisa calcula los múltiplos de \texttt{i} (prueba
cambiando el valor de \texttt{i} a \(2,3,4,5,6,7\)) y observa el
resultado.

\begin{Shaded}
\begin{Highlighting}[]
\NormalTok{i }\OperatorTok{=} \DecValTok{4}
\NormalTok{[j }\ControlFlowTok{for}\NormalTok{ j }\KeywordTok{in} \BuiltInTok{range}\NormalTok{(i}\OperatorTok{*}\DecValTok{2}\NormalTok{, }\DecValTok{50}\NormalTok{, i)]}
\end{Highlighting}
\end{Shaded}

\begin{verbatim}
[8, 12, 16, 20, 24, 28, 32, 36, 40, 44, 48]
\end{verbatim}

Ahora, para cambiar el valor de \texttt{i} a \(2,3,4,5,6,7\) con una
lista concisa se puede hacer lo siguiente:

\begin{Shaded}
\begin{Highlighting}[]
\NormalTok{[i }\ControlFlowTok{for}\NormalTok{ i }\KeywordTok{in} \BuiltInTok{range}\NormalTok{(}\DecValTok{2}\NormalTok{,}\DecValTok{8}\NormalTok{)]}
\end{Highlighting}
\end{Shaded}

\begin{verbatim}
[2, 3, 4, 5, 6, 7]
\end{verbatim}

Usando las dos listas concisas creadas antes, generamos todos aquellos
números en el intervalo \([2,50]\) que tienen al menos un múltiplo (y
que por lo tanto no son primos)

\begin{Shaded}
\begin{Highlighting}[]
\NormalTok{noprimos }\OperatorTok{=}\NormalTok{ [j }\ControlFlowTok{for}\NormalTok{ i }\KeywordTok{in} \BuiltInTok{range}\NormalTok{(}\DecValTok{2}\NormalTok{,}\DecValTok{8}\NormalTok{) }\ControlFlowTok{for}\NormalTok{ j }\KeywordTok{in} \BuiltInTok{range}\NormalTok{(i}\OperatorTok{*}\DecValTok{2}\NormalTok{, }\DecValTok{50}\NormalTok{, i)]}
\BuiltInTok{print}\NormalTok{(}\StringTok{\textquotesingle{}NO primos: }\CharTok{\textbackslash{}n}\SpecialCharTok{\{\}}\StringTok{\textquotesingle{}}\NormalTok{.}\BuiltInTok{format}\NormalTok{(noprimos))}
\end{Highlighting}
\end{Shaded}

\begin{verbatim}
NO primos: 
[4, 6, 8, 10, 12, 14, 16, 18, 20, 22, 24, 26, 28, 30, 32, 34, 36, 38, 40, 42, 44, 46, 48, 6, 9, 12, 15, 18, 21, 24, 27, 30, 33, 36, 39, 42, 45, 48, 8, 12, 16, 20, 24, 28, 32, 36, 40, 44, 48, 10, 15, 20, 25, 30, 35, 40, 45, 12, 18, 24, 30, 36, 42, 48, 14, 21, 28, 35, 42, 49]
\end{verbatim}

Para encontrar los primos usamos una lista concisa verificando los
números que faltan en la lista de \texttt{noprimos}, esos serán los
números primos que estamos buscando:

\begin{Shaded}
\begin{Highlighting}[]
\NormalTok{primos }\OperatorTok{=}\NormalTok{ [x }\ControlFlowTok{for}\NormalTok{ x }\KeywordTok{in} \BuiltInTok{range}\NormalTok{(}\DecValTok{2}\NormalTok{,}\DecValTok{50}\NormalTok{) }\ControlFlowTok{if}\NormalTok{ x }\KeywordTok{not} \KeywordTok{in}\NormalTok{ noprimos]}
\BuiltInTok{print}\NormalTok{(}\StringTok{\textquotesingle{}Primos: }\CharTok{\textbackslash{}n}\SpecialCharTok{\{\}}\StringTok{\textquotesingle{}}\NormalTok{.}\BuiltInTok{format}\NormalTok{(primos))}
\end{Highlighting}
\end{Shaded}

\begin{verbatim}
Primos: 
[2, 3, 5, 7, 11, 13, 17, 19, 23, 29, 31, 37, 41, 43, 47]
\end{verbatim}

\textbf{Juntando todo}:

\begin{Shaded}
\begin{Highlighting}[]
\NormalTok{[x }\ControlFlowTok{for}\NormalTok{ x }\KeywordTok{in} \BuiltInTok{range}\NormalTok{(}\DecValTok{2}\NormalTok{,}\DecValTok{50}\NormalTok{) }\ControlFlowTok{if}\NormalTok{ x }\KeywordTok{not} \KeywordTok{in}\NormalTok{ [j }\ControlFlowTok{for}\NormalTok{ i }\KeywordTok{in} \BuiltInTok{range}\NormalTok{(}\DecValTok{2}\NormalTok{,}\DecValTok{8}\NormalTok{) }\ControlFlowTok{for}\NormalTok{ j }\KeywordTok{in} \BuiltInTok{range}\NormalTok{(i}\OperatorTok{*}\DecValTok{2}\NormalTok{, }\DecValTok{50}\NormalTok{, i)]]}
\end{Highlighting}
\end{Shaded}

\begin{verbatim}
[2, 3, 5, 7, 11, 13, 17, 19, 23, 29, 31, 37, 41, 43, 47]
\end{verbatim}

\section{Listas concisas con elementos no
numéricos.}\label{listas-concisas-con-elementos-no-numuxe9ricos.}

Las listas también pueden contener otro tipo de elementos, no solo
números. Por ejemplo:

\begin{Shaded}
\begin{Highlighting}[]
\NormalTok{mensaje }\OperatorTok{=} \StringTok{\textquotesingle{}La vida no es la que uno vivió, sino la que uno recuerda\textquotesingle{}}
\end{Highlighting}
\end{Shaded}

\begin{Shaded}
\begin{Highlighting}[]
\BuiltInTok{print}\NormalTok{(mensaje)}
\end{Highlighting}
\end{Shaded}

\begin{verbatim}
La vida no es la que uno vivió, sino la que uno recuerda
\end{verbatim}

\begin{Shaded}
\begin{Highlighting}[]
\NormalTok{palabras }\OperatorTok{=}\NormalTok{ mensaje.split()}
\BuiltInTok{print}\NormalTok{(palabras,end}\OperatorTok{=}\StringTok{\textquotesingle{}\textquotesingle{}}\NormalTok{)}
\end{Highlighting}
\end{Shaded}

\begin{verbatim}
['La', 'vida', 'no', 'es', 'la', 'que', 'uno', 'vivió,', 'sino', 'la', 'que', 'uno', 'recuerda']
\end{verbatim}

Vamos a crear una lista cuyos elementos contienen cada palabra de la
lista anterior en mayúsculas, en forma de título y su longitud, estos
tres elementos agregados en una tupla:

\begin{Shaded}
\begin{Highlighting}[]
\NormalTok{tabla }\OperatorTok{=}\NormalTok{ [(w.upper(), w.title(), }\BuiltInTok{len}\NormalTok{(w)) }\ControlFlowTok{for}\NormalTok{ w }\KeywordTok{in}\NormalTok{ palabras]}
\BuiltInTok{print}\NormalTok{(tabla)}
\end{Highlighting}
\end{Shaded}

\begin{verbatim}
[('LA', 'La', 2), ('VIDA', 'Vida', 4), ('NO', 'No', 2), ('ES', 'Es', 2), ('LA', 'La', 2), ('QUE', 'Que', 3), ('UNO', 'Uno', 3), ('VIVIÓ,', 'Vivió,', 6), ('SINO', 'Sino', 4), ('LA', 'La', 2), ('QUE', 'Que', 3), ('UNO', 'Uno', 3), ('RECUERDA', 'Recuerda', 8)]
\end{verbatim}

\bookmarksetup{startatroot}

\chapter{Conjuntos concisos}\label{conjuntos-concisos}

Al igual que las listas concisas, también es posible crear conjuntos
usando los mismos principios , la única diferencia es que la secuencia
que resulta es un objeto de tipo\texttt{set}.

\textbf{Definición}.

\begin{Shaded}
\begin{Highlighting}[]
\NormalTok{\{expression(variable) }\ControlFlowTok{for}\NormalTok{ variable }\KeywordTok{in}\NormalTok{ input\_set [predicate][, …]\}}
\end{Highlighting}
\end{Shaded}

\begin{enumerate}
\def\labelenumi{\arabic{enumi}.}
\item
  \texttt{expression} : Es una expresión \textbf{opcional} de salida que
  produce los miembros del nuevo conjunto a partir de los miembros del
  conjunto de entrada que satisfacen el \texttt{predicate}.
\item
  \texttt{variable} : Es una variable \textbf{requerida} que representa
  los miembros del conjunto de entrada.
\item
  \texttt{input\_set}: Representa la secuencia de entrada.
  (\textbf{requerido}).
\item
  \texttt{predicate} : Expresión \textbf{opcional} que actúa como un
  filtro sobre los miembros del conjunto de entrada.
\item
  \texttt{{[},\ …{]}{]}} : Otra \emph{comprehension} anidada
  \textbf{opcional}.
\end{enumerate}

\section{\texorpdfstring{\textbf{Ejemplo
8.}}{Ejemplo 8.}}\label{ejemplo-8.-1}

Supongamos que deseamos organizar una lista de nombres de tal manera que
no haya repeticiones, que los nombres tengan más de un caracter y que su
representación sea con la primera letra mayúscula y las demás
minúsculas. Por ejemplo, una lista aceptable sería:

\begin{Shaded}
\begin{Highlighting}[]
\NormalTok{nombres }\OperatorTok{=}\NormalTok{ [}\StringTok{\textquotesingle{}Luis\textquotesingle{}}\NormalTok{, }\StringTok{\textquotesingle{}Juan\textquotesingle{}}\NormalTok{, }\StringTok{\textquotesingle{}Angie\textquotesingle{}}\NormalTok{, }\StringTok{\textquotesingle{}Pedro\textquotesingle{}}\NormalTok{, }\StringTok{\textquotesingle{}María\textquotesingle{}}\NormalTok{, }\StringTok{\textquotesingle{}Diana\textquotesingle{}}\NormalTok{]}
\end{Highlighting}
\end{Shaded}

Leer una lista de nombres del archivo \texttt{nombres} y procesarlos
para obtener una lista similar a la descrita.

\begin{Shaded}
\begin{Highlighting}[]
\CommentTok{\# Abrimos el archivo en modo lectura}
\NormalTok{archivo }\OperatorTok{=} \BuiltInTok{open}\NormalTok{(}\StringTok{\textquotesingle{}nombres\textquotesingle{}}\NormalTok{,}\StringTok{\textquotesingle{}r\textquotesingle{}}\NormalTok{)}

\CommentTok{\# Leemos la lista de nombres y los ponemos en una lista}
\NormalTok{lista\_nombres }\OperatorTok{=}\NormalTok{ archivo.read().split()}

\CommentTok{\# Vemos la lista de nombres}
\BuiltInTok{print}\NormalTok{(lista\_nombres)}
\end{Highlighting}
\end{Shaded}

\begin{verbatim}
['A', 'LuCas', 'Sidronio', 'Michelle', 'a', 'ANGIE', 'Luis', 'lucas', 'MICHelle', 'PedrO', 'PEPE', 'Manu', 'luis', 'diana', 'sidronio', 'pepe', 'a', 'a', 'b']
\end{verbatim}

\begin{Shaded}
\begin{Highlighting}[]
\CommentTok{\# Procesamos las palabras como se requiere}
\NormalTok{nombres\_set }\OperatorTok{=}\NormalTok{ \{nombre[}\DecValTok{0}\NormalTok{].upper() }\OperatorTok{+}\NormalTok{ nombre[}\DecValTok{1}\NormalTok{:].lower() }
                     \ControlFlowTok{for}\NormalTok{ nombre }\KeywordTok{in}\NormalTok{ lista\_nombres }
                     \ControlFlowTok{if} \BuiltInTok{len}\NormalTok{(nombre) }\OperatorTok{\textgreater{}} \DecValTok{1}\NormalTok{ \}}
\BuiltInTok{print}\NormalTok{(nombres\_set)}
\end{Highlighting}
\end{Shaded}

\begin{verbatim}
{'Pepe', 'Lucas', 'Angie', 'Sidronio', 'Pedro', 'Manu', 'Luis', 'Diana', 'Michelle'}
\end{verbatim}

\begin{Shaded}
\begin{Highlighting}[]
\CommentTok{\# Transformamos el conjunto a una lista}
\NormalTok{nombres }\OperatorTok{=} \BuiltInTok{list}\NormalTok{(nombres\_set)}
\BuiltInTok{print}\NormalTok{(nombres)}
\end{Highlighting}
\end{Shaded}

\begin{verbatim}
['Pepe', 'Lucas', 'Angie', 'Sidronio', 'Pedro', 'Manu', 'Luis', 'Diana', 'Michelle']
\end{verbatim}

\section{\texorpdfstring{\textbf{Ejemplo
9.}}{Ejemplo 9.}}\label{ejemplo-9.-1}

Observa los siguientes ejemplos de conjuntos concisos y explica su
funcionamiento.

\begin{Shaded}
\begin{Highlighting}[]
\NormalTok{\{s }\ControlFlowTok{for}\NormalTok{ s }\KeywordTok{in}\NormalTok{ [}\DecValTok{1}\NormalTok{, }\DecValTok{2}\NormalTok{, }\DecValTok{1}\NormalTok{, }\DecValTok{0}\NormalTok{]\}}
\end{Highlighting}
\end{Shaded}

\begin{verbatim}
{0, 1, 2}
\end{verbatim}

\begin{Shaded}
\begin{Highlighting}[]
\NormalTok{\{s}\OperatorTok{**}\DecValTok{2} \ControlFlowTok{for}\NormalTok{ s }\KeywordTok{in}\NormalTok{ [}\DecValTok{1}\NormalTok{, }\DecValTok{2}\NormalTok{, }\DecValTok{1}\NormalTok{, }\DecValTok{0}\NormalTok{]\}}
\end{Highlighting}
\end{Shaded}

\begin{verbatim}
{0, 1, 4}
\end{verbatim}

\begin{Shaded}
\begin{Highlighting}[]
\NormalTok{\{s}\OperatorTok{**}\DecValTok{2} \ControlFlowTok{for}\NormalTok{ s }\KeywordTok{in} \BuiltInTok{range}\NormalTok{(}\DecValTok{10}\NormalTok{)\}}
\end{Highlighting}
\end{Shaded}

\begin{verbatim}
{0, 1, 4, 9, 16, 25, 36, 49, 64, 81}
\end{verbatim}

\begin{Shaded}
\begin{Highlighting}[]
\NormalTok{\{s }\ControlFlowTok{for}\NormalTok{ s }\KeywordTok{in} \BuiltInTok{range}\NormalTok{(}\DecValTok{10}\NormalTok{) }\ControlFlowTok{if}\NormalTok{ s }\OperatorTok{\%} \DecValTok{2}\NormalTok{\}}
\end{Highlighting}
\end{Shaded}

\begin{verbatim}
{1, 3, 5, 7, 9}
\end{verbatim}

\begin{Shaded}
\begin{Highlighting}[]
\NormalTok{\{(m, n) }\ControlFlowTok{for}\NormalTok{ n }\KeywordTok{in} \BuiltInTok{range}\NormalTok{(}\DecValTok{2}\NormalTok{) }\ControlFlowTok{for}\NormalTok{ m }\KeywordTok{in} \BuiltInTok{range}\NormalTok{(}\DecValTok{3}\NormalTok{, }\DecValTok{5}\NormalTok{)\}}
\end{Highlighting}
\end{Shaded}

\begin{verbatim}
{(3, 0), (3, 1), (4, 0), (4, 1)}
\end{verbatim}

\bookmarksetup{startatroot}

\chapter{Diccionarios concisos}\label{diccionarios-concisos}

\begin{itemize}
\item
  Es un método para transformar un diccionario en otro diccionario.
\item
  Durante esta transformación, los objetos dentro del diccionario
  original pueden ser incluidos o no en el nuevo diccionario dependiendo
  de una condición.
\item
  Cada objeto en el nuevo diccionario puede ser transformado como sea
  requerido.
\end{itemize}

\textbf{Definición}.

\begin{Shaded}
\begin{Highlighting}[]
\NormalTok{\{key:value }\ControlFlowTok{for}\NormalTok{ (key,value) }\KeywordTok{in}\NormalTok{ dictonary.items()\}}
\end{Highlighting}
\end{Shaded}

\section{\texorpdfstring{\textbf{Ejemplo
9.}}{Ejemplo 9.}}\label{ejemplo-9.-2}

Duplicar el valor (\emph{value}) de cada entrada (\emph{item}) de un
diccionario:

Recuerda como funcionan los diccionarios:

\begin{Shaded}
\begin{Highlighting}[]
\NormalTok{dicc }\OperatorTok{=}\NormalTok{ \{}\StringTok{\textquotesingle{}a\textquotesingle{}}\NormalTok{: }\DecValTok{1}\NormalTok{, }\StringTok{\textquotesingle{}b\textquotesingle{}}\NormalTok{: }\DecValTok{2}\NormalTok{, }\StringTok{\textquotesingle{}c\textquotesingle{}}\NormalTok{: }\DecValTok{3}\NormalTok{, }\StringTok{\textquotesingle{}d\textquotesingle{}}\NormalTok{: }\DecValTok{4}\NormalTok{\}}
\BuiltInTok{print}\NormalTok{(dicc.keys())   }\CommentTok{\# Función para obtener las claves}
\BuiltInTok{print}\NormalTok{(dicc.values()) }\CommentTok{\# Función para obtener los valores}
\BuiltInTok{print}\NormalTok{(dicc.items())  }\CommentTok{\# Función para obtener los items}
\end{Highlighting}
\end{Shaded}

\begin{verbatim}
dict_keys(['a', 'b', 'c', 'd'])
dict_values([1, 2, 3, 4])
dict_items([('a', 1), ('b', 2), ('c', 3), ('d', 4)])
\end{verbatim}

Para crear el diccionario del ejemplo hacemos lo siguiente:

\begin{Shaded}
\begin{Highlighting}[]
\CommentTok{\# Definición del diccionario}
\NormalTok{dicc }\OperatorTok{=}\NormalTok{ \{}\StringTok{\textquotesingle{}a\textquotesingle{}}\NormalTok{: }\DecValTok{1}\NormalTok{, }\StringTok{\textquotesingle{}b\textquotesingle{}}\NormalTok{: }\DecValTok{2}\NormalTok{, }\StringTok{\textquotesingle{}c\textquotesingle{}}\NormalTok{: }\DecValTok{3}\NormalTok{, }\StringTok{\textquotesingle{}d\textquotesingle{}}\NormalTok{: }\DecValTok{4}\NormalTok{, }\StringTok{\textquotesingle{}e\textquotesingle{}}\NormalTok{: }\DecValTok{5}\NormalTok{\}}

\CommentTok{\# Duplicación de los valores del diccionario}
\NormalTok{dicc\_doble }\OperatorTok{=}\NormalTok{ \{k:v}\OperatorTok{*}\DecValTok{2} \ControlFlowTok{for}\NormalTok{ (k,v) }\KeywordTok{in}\NormalTok{ dicc.items()\}}

\CommentTok{\# Mostramos el resultado}
\BuiltInTok{print}\NormalTok{(dicc)}
\BuiltInTok{print}\NormalTok{(dicc\_doble)}
\end{Highlighting}
\end{Shaded}

\begin{verbatim}
{'a': 1, 'b': 2, 'c': 3, 'd': 4, 'e': 5}
{'a': 2, 'b': 4, 'c': 6, 'd': 8, 'e': 10}
\end{verbatim}

\section{\texorpdfstring{\textbf{Ejemplo
10.}}{Ejemplo 10.}}\label{ejemplo-10.-1}

Duplicar la clave (\emph{key}) de cada entrada (\emph{item}) del
diccionario:

\begin{Shaded}
\begin{Highlighting}[]
\NormalTok{dict1\_keys }\OperatorTok{=}\NormalTok{ \{k}\OperatorTok{*}\DecValTok{2}\NormalTok{:v }\ControlFlowTok{for}\NormalTok{ (k,v) }\KeywordTok{in}\NormalTok{ dict1.items()\}}
\BuiltInTok{print}\NormalTok{(dict1\_keys)}
\end{Highlighting}
\end{Shaded}

\begin{verbatim}
{'aa': 1, 'bb': 2, 'cc': 3, 'dd': 4}
\end{verbatim}

\section{\texorpdfstring{\textbf{Ejemplo
11.}}{Ejemplo 11.}}\label{ejemplo-11.-1}

Crear un diccionario donde la clave sea un número divisible por 2 en un
rango de 0 a 10 y sus valores sean el cuadrado de la clave.

\begin{Shaded}
\begin{Highlighting}[]
\CommentTok{\# La forma tradicional}
\NormalTok{numeros }\OperatorTok{=} \BuiltInTok{range}\NormalTok{(}\DecValTok{11}\NormalTok{)}
\NormalTok{dicc }\OperatorTok{=}\NormalTok{ \{\}}

\ControlFlowTok{for}\NormalTok{ n }\KeywordTok{in}\NormalTok{ numeros:}
    \ControlFlowTok{if}\NormalTok{ n}\OperatorTok{\%}\DecValTok{2}\OperatorTok{==}\DecValTok{0}\NormalTok{:}
\NormalTok{        dicc[n] }\OperatorTok{=}\NormalTok{ n}\OperatorTok{**}\DecValTok{2}

\BuiltInTok{print}\NormalTok{(dicc)}
\end{Highlighting}
\end{Shaded}

\begin{verbatim}
{0: 0, 2: 4, 4: 16, 6: 36, 8: 64, 10: 100}
\end{verbatim}

\begin{Shaded}
\begin{Highlighting}[]
\CommentTok{\# Usando dict comprehensions}
\NormalTok{dicc\_smart }\OperatorTok{=}\NormalTok{ \{n:n}\OperatorTok{**}\DecValTok{2} \ControlFlowTok{for}\NormalTok{ n }\KeywordTok{in}\NormalTok{ numeros }\ControlFlowTok{if}\NormalTok{ n}\OperatorTok{\%}\DecValTok{2} \OperatorTok{==} \DecValTok{0}\NormalTok{\}}

\BuiltInTok{print}\NormalTok{(dicc\_smart)}
\end{Highlighting}
\end{Shaded}

\begin{verbatim}
{0: 0, 2: 4, 4: 16, 6: 36, 8: 64, 10: 100}
\end{verbatim}

\section{\texorpdfstring{\textbf{Ejemplo
12.}}{Ejemplo 12.}}\label{ejemplo-12.-1}

Intercambiar las claves y los valores en un diccionario.

\begin{Shaded}
\begin{Highlighting}[]
\NormalTok{a\_dict }\OperatorTok{=}\NormalTok{ \{}\StringTok{\textquotesingle{}a\textquotesingle{}}\NormalTok{: }\DecValTok{1}\NormalTok{, }\StringTok{\textquotesingle{}b\textquotesingle{}}\NormalTok{: }\DecValTok{2}\NormalTok{, }\StringTok{\textquotesingle{}c\textquotesingle{}}\NormalTok{: }\DecValTok{3}\NormalTok{\}}
\NormalTok{\{value:key }\ControlFlowTok{for}\NormalTok{ key, value }\KeywordTok{in}\NormalTok{ a\_dict.items()\}}
\end{Highlighting}
\end{Shaded}

\begin{verbatim}
{1: 'a', 2: 'b', 3: 'c'}
\end{verbatim}

\begin{Shaded}
\begin{Highlighting}[]
\CommentTok{\# OJO: No siempre es posible hacer lo anterior.}
\NormalTok{a\_dict }\OperatorTok{=}\NormalTok{ \{}\StringTok{\textquotesingle{}a\textquotesingle{}}\NormalTok{: [}\DecValTok{1}\NormalTok{, }\DecValTok{2}\NormalTok{, }\DecValTok{3}\NormalTok{], }\StringTok{\textquotesingle{}b\textquotesingle{}}\NormalTok{: }\DecValTok{4}\NormalTok{, }\StringTok{\textquotesingle{}c\textquotesingle{}}\NormalTok{: }\DecValTok{5}\NormalTok{\}}
\NormalTok{\{value:key }\ControlFlowTok{for}\NormalTok{ key, value }\KeywordTok{in}\NormalTok{ a\_dict.items()\}}
\end{Highlighting}
\end{Shaded}

\begin{verbatim}
TypeError: unhashable type: 'list'
\end{verbatim}

\section{\texorpdfstring{\textbf{Ejemplo
14.}}{Ejemplo 14.}}\label{ejemplo-14.-1}

Convertir Fahrenheit a Celsius y viceversa.

\begin{Shaded}
\begin{Highlighting}[]
\CommentTok{\# Usando map, lambda y diccionarios}
\NormalTok{[}\FloatTok{32.0}\NormalTok{, }\FloatTok{72.5}\NormalTok{, }\FloatTok{104.0}\NormalTok{, }\FloatTok{212.0}\NormalTok{]}
\NormalTok{fahrenheit\_dict }\OperatorTok{=}\NormalTok{ \{}\StringTok{\textquotesingle{}t1\textquotesingle{}}\NormalTok{:}\FloatTok{32.0}\NormalTok{, }\StringTok{\textquotesingle{}t2\textquotesingle{}}\NormalTok{:}\FloatTok{72.5}\NormalTok{, }\StringTok{\textquotesingle{}t3\textquotesingle{}}\NormalTok{:}\FloatTok{104.0}\NormalTok{, }\StringTok{\textquotesingle{}t4\textquotesingle{}}\NormalTok{:}\FloatTok{212.0}\NormalTok{\}}

\NormalTok{celsius }\OperatorTok{=} \BuiltInTok{list}\NormalTok{(}\BuiltInTok{map}\NormalTok{(}\KeywordTok{lambda}\NormalTok{ f: (}\DecValTok{5}\OperatorTok{/}\DecValTok{9}\NormalTok{)}\OperatorTok{*}\NormalTok{(f}\OperatorTok{{-}}\DecValTok{32}\NormalTok{), fahrenheit\_dict.values()))}

\NormalTok{celsius\_dict }\OperatorTok{=} \BuiltInTok{dict}\NormalTok{(}\BuiltInTok{zip}\NormalTok{(fahrenheit\_dict.keys(), celsius))}

\BuiltInTok{print}\NormalTok{(celsius\_dict)}
\end{Highlighting}
\end{Shaded}

\begin{verbatim}
{'t1': 0.0, 't2': 22.5, 't3': 40.0, 't4': 100.0}
\end{verbatim}

\begin{Shaded}
\begin{Highlighting}[]
\CommentTok{\# Usando dict comprehensions !}
\NormalTok{celsius\_smart }\OperatorTok{=}\NormalTok{ \{k:(}\DecValTok{5}\OperatorTok{/}\DecValTok{9}\NormalTok{)}\OperatorTok{*}\NormalTok{(v}\OperatorTok{{-}}\DecValTok{32}\NormalTok{) }\ControlFlowTok{for}\NormalTok{ (k,v) }\KeywordTok{in}\NormalTok{ fahrenheit\_dict.items()\}}
\BuiltInTok{print}\NormalTok{(celsius\_smart)}
\end{Highlighting}
\end{Shaded}

\begin{verbatim}
{'t1': 0.0, 't2': 22.5, 't3': 40.0, 't4': 100.0}
\end{verbatim}

\section{\texorpdfstring{\textbf{Ejemplo
15.}}{Ejemplo 15.}}\label{ejemplo-15.-1}

Dado un diccionario, cuyos valores son enteros, crear un nuevo
diccionario cuyos valores sean mayores que 2.

\begin{Shaded}
\begin{Highlighting}[]
\NormalTok{a\_dict }\OperatorTok{=}\NormalTok{ \{}\StringTok{\textquotesingle{}a\textquotesingle{}}\NormalTok{:}\DecValTok{1}\NormalTok{, }\StringTok{\textquotesingle{}b\textquotesingle{}}\NormalTok{:}\DecValTok{2}\NormalTok{, }\StringTok{\textquotesingle{}c\textquotesingle{}}\NormalTok{:}\DecValTok{3}\NormalTok{, }\StringTok{\textquotesingle{}d\textquotesingle{}}\NormalTok{:}\DecValTok{4}\NormalTok{, }\StringTok{\textquotesingle{}e\textquotesingle{}}\NormalTok{:}\DecValTok{5}\NormalTok{, }\StringTok{\textquotesingle{}f\textquotesingle{}}\NormalTok{:}\DecValTok{6}\NormalTok{, }\StringTok{\textquotesingle{}g\textquotesingle{}}\NormalTok{:}\DecValTok{7}\NormalTok{, }\StringTok{\textquotesingle{}h\textquotesingle{}}\NormalTok{:}\DecValTok{8}\NormalTok{\}}
\BuiltInTok{print}\NormalTok{(a\_dict)}

\NormalTok{a\_dict\_cond }\OperatorTok{=}\NormalTok{ \{ k:v }\ControlFlowTok{for}\NormalTok{ (k,v) }\KeywordTok{in}\NormalTok{ a\_dict.items() }\ControlFlowTok{if}\NormalTok{ v }\OperatorTok{\textgreater{}} \DecValTok{2}\NormalTok{ \}}
\BuiltInTok{print}\NormalTok{(a\_dict\_cond)}
\end{Highlighting}
\end{Shaded}

\begin{verbatim}
{'a': 1, 'b': 2, 'c': 3, 'd': 4, 'e': 5, 'f': 6, 'g': 7, 'h': 8}
{'c': 3, 'd': 4, 'e': 5, 'f': 6, 'g': 7, 'h': 8}
\end{verbatim}

\section{\texorpdfstring{\textbf{Ejemplo
16.}}{Ejemplo 16.}}\label{ejemplo-16.-1}

Dado un diccionario, cuyos valores son enteros, crear un nuevo
diccionario cuyos valores sean mayores que 2 y que además sean pares.

\begin{Shaded}
\begin{Highlighting}[]
\NormalTok{a\_dict\_cond2 }\OperatorTok{=}\NormalTok{ \{ k:v }\ControlFlowTok{for}\NormalTok{ (k,v) }\KeywordTok{in}\NormalTok{ a\_dict.items() }\ControlFlowTok{if}\NormalTok{ (v }\OperatorTok{\textgreater{}} \DecValTok{2}\NormalTok{) }\KeywordTok{and}\NormalTok{ (v }\OperatorTok{\%} \DecValTok{2}\NormalTok{) }\OperatorTok{==} \DecValTok{0}\NormalTok{\}}
\BuiltInTok{print}\NormalTok{(a\_dict\_cond2)}
\end{Highlighting}
\end{Shaded}

\begin{verbatim}
{'d': 4, 'f': 6, 'h': 8}
\end{verbatim}

\section{\texorpdfstring{\textbf{Ejemplo
17.}}{Ejemplo 17.}}\label{ejemplo-17.-1}

Dado un diccionario, cuyos valores son enteros, crear un nuevo
diccionario cuyos valores sean mayores que 2 y que además sean pares y
divisibles por 3.

\begin{Shaded}
\begin{Highlighting}[]
\CommentTok{\# La forma tradicional}
\NormalTok{a\_dict\_cond3\_loop }\OperatorTok{=}\NormalTok{ \{\}}

\ControlFlowTok{for}\NormalTok{ (k,v) }\KeywordTok{in}\NormalTok{ a\_dict.items():}
    \ControlFlowTok{if}\NormalTok{ (v}\OperatorTok{\textgreater{}=}\DecValTok{2} \KeywordTok{and}\NormalTok{ v}\OperatorTok{\%}\DecValTok{2} \OperatorTok{==} \DecValTok{0} \KeywordTok{and}\NormalTok{ v}\OperatorTok{\%}\DecValTok{3} \OperatorTok{==} \DecValTok{0}\NormalTok{):}
\NormalTok{        a\_dict\_cond3\_loop[k] }\OperatorTok{=}\NormalTok{ v}

\BuiltInTok{print}\NormalTok{(a\_dict\_cond3\_loop)}
\end{Highlighting}
\end{Shaded}

\begin{verbatim}
{'f': 6}
\end{verbatim}

\begin{Shaded}
\begin{Highlighting}[]
\CommentTok{\# Usando dict comprehensions}
\NormalTok{a\_dict\_cond3 }\OperatorTok{=}\NormalTok{ \{k:v }\ControlFlowTok{for}\NormalTok{ (k,v) }\KeywordTok{in}\NormalTok{ a\_dict.items() }\ControlFlowTok{if}\NormalTok{ v}\OperatorTok{\textgreater{}}\DecValTok{2} \ControlFlowTok{if}\NormalTok{ v}\OperatorTok{\%}\DecValTok{2} \OperatorTok{==} \DecValTok{0} \ControlFlowTok{if}\NormalTok{ v}\OperatorTok{\%}\DecValTok{3} \OperatorTok{==} \DecValTok{0}\NormalTok{\}}

\BuiltInTok{print}\NormalTok{(a\_dict\_cond3)}
\end{Highlighting}
\end{Shaded}

\begin{verbatim}
{'f': 6}
\end{verbatim}

\section{\texorpdfstring{\textbf{Ejemplo
18.}}{Ejemplo 18.}}\label{ejemplo-18.}

Apartir de un diccionario con valores enteros, identificar los valores
pares y los impares, y sustituir los valores por etiquetas `par' e
`impar' segun corresponda.

\begin{Shaded}
\begin{Highlighting}[]
\BuiltInTok{print}\NormalTok{(a\_dict)}
\NormalTok{a\_dict\_else }\OperatorTok{=}\NormalTok{ \{ k:(}\StringTok{\textquotesingle{}par\textquotesingle{}} \ControlFlowTok{if}\NormalTok{ v}\OperatorTok{\%}\DecValTok{2}\OperatorTok{==}\DecValTok{0} \ControlFlowTok{else} \StringTok{\textquotesingle{}impar\textquotesingle{}}\NormalTok{) }\ControlFlowTok{for}\NormalTok{ (k,v) }\KeywordTok{in}\NormalTok{ a\_dict.items()\}}
\BuiltInTok{print}\NormalTok{(a\_dict\_else)}
\end{Highlighting}
\end{Shaded}

\begin{verbatim}
{'a': 1, 'b': 2, 'c': 3, 'd': 4, 'e': 5, 'f': 6, 'g': 7, 'h': 8}
{'a': 'impar', 'b': 'par', 'c': 'impar', 'd': 'par', 'e': 'impar', 'f': 'par', 'g': 'impar', 'h': 'par'}
\end{verbatim}

\section{\texorpdfstring{\textbf{Ejemplo
19.}}{Ejemplo 19.}}\label{ejemplo-19.}

Crear un diccionaro cuyos valores sean diccionarios.

\begin{Shaded}
\begin{Highlighting}[]
\CommentTok{\# con dict comprehensions}
\NormalTok{anidado }\OperatorTok{=}\NormalTok{ \{}\StringTok{\textquotesingle{}primero\textquotesingle{}}\NormalTok{:\{}\StringTok{\textquotesingle{}a\textquotesingle{}}\NormalTok{:}\DecValTok{1}\NormalTok{\}, }\StringTok{\textquotesingle{}segundo\textquotesingle{}}\NormalTok{:\{}\StringTok{\textquotesingle{}b\textquotesingle{}}\NormalTok{:}\DecValTok{2}\NormalTok{\}, }\StringTok{\textquotesingle{}tercero\textquotesingle{}}\NormalTok{:\{}\StringTok{\textquotesingle{}c\textquotesingle{}}\NormalTok{:}\DecValTok{3}\NormalTok{\}\}}
\NormalTok{pi }\OperatorTok{=} \FloatTok{3.1415}
\NormalTok{float\_dict }\OperatorTok{=}\NormalTok{ \{e\_k:\{i\_k:i\_v}\OperatorTok{*}\NormalTok{pi }\ControlFlowTok{for}\NormalTok{ (i\_k, i\_v) }\KeywordTok{in}\NormalTok{ e\_v.items()\} }\ControlFlowTok{for}\NormalTok{ (e\_k, e\_v) }\KeywordTok{in}\NormalTok{ anidado.items()\}}

\BuiltInTok{print}\NormalTok{(float\_dict)}
\end{Highlighting}
\end{Shaded}

\begin{verbatim}
{'primero': {'a': 3.1415}, 'segundo': {'b': 6.283}, 'tercero': {'c': 9.4245}}
\end{verbatim}

\begin{Shaded}
\begin{Highlighting}[]
\CommentTok{\# La forma tradicional sería:}
\NormalTok{anidado }\OperatorTok{=}\NormalTok{ \{}\StringTok{\textquotesingle{}primero\textquotesingle{}}\NormalTok{:\{}\StringTok{\textquotesingle{}a\textquotesingle{}}\NormalTok{:}\DecValTok{1}\NormalTok{\}, }\StringTok{\textquotesingle{}segundo\textquotesingle{}}\NormalTok{:\{}\StringTok{\textquotesingle{}b\textquotesingle{}}\NormalTok{:}\DecValTok{2}\NormalTok{\}, }\StringTok{\textquotesingle{}tercero\textquotesingle{}}\NormalTok{:\{}\StringTok{\textquotesingle{}c\textquotesingle{}}\NormalTok{:}\DecValTok{3}\NormalTok{\}\}}
\NormalTok{pi }\OperatorTok{=} \FloatTok{3.1415}
\ControlFlowTok{for}\NormalTok{ (e\_k, e\_v) }\KeywordTok{in}\NormalTok{ anidado.items():}
    \ControlFlowTok{for}\NormalTok{ (i\_k, i\_v) }\KeywordTok{in}\NormalTok{ e\_v.items():}
\NormalTok{        e\_v.update(\{i\_k: i\_v }\OperatorTok{*}\NormalTok{ pi\})}
        
\NormalTok{anidado.update(\{e\_k:e\_v\})}

\BuiltInTok{print}\NormalTok{(anidado)}
\end{Highlighting}
\end{Shaded}

\begin{verbatim}
{'primero': {'a': 3.1415}, 'segundo': {'b': 6.283}, 'tercero': {'c': 9.4245}}
\end{verbatim}

\section{\texorpdfstring{\textbf{Ejemplo
20.}}{Ejemplo 20.}}\label{ejemplo-20.}

Eliminar números duplicados de una lista.

\begin{Shaded}
\begin{Highlighting}[]
\NormalTok{numeros }\OperatorTok{=}\NormalTok{ [i }\ControlFlowTok{for}\NormalTok{ i }\KeywordTok{in} \BuiltInTok{range}\NormalTok{(}\DecValTok{1}\NormalTok{,}\DecValTok{11}\NormalTok{)] }\OperatorTok{+}\NormalTok{ [i }\ControlFlowTok{for}\NormalTok{ i }\KeywordTok{in} \BuiltInTok{range}\NormalTok{(}\DecValTok{1}\NormalTok{,}\DecValTok{6}\NormalTok{)]}
\NormalTok{numeros}
\end{Highlighting}
\end{Shaded}

\begin{verbatim}
[1, 2, 3, 4, 5, 6, 7, 8, 9, 10, 1, 2, 3, 4, 5]
\end{verbatim}

\begin{Shaded}
\begin{Highlighting}[]
\CommentTok{\# Una manera es:}
\NormalTok{numeros\_unicos }\OperatorTok{=}\NormalTok{ []}
\ControlFlowTok{for}\NormalTok{ n }\KeywordTok{in}\NormalTok{ numeros:}
    \ControlFlowTok{if}\NormalTok{ n }\KeywordTok{not} \KeywordTok{in}\NormalTok{ numeros\_unicos:}
\NormalTok{        numeros\_unicos.append(n)}
\NormalTok{numeros\_unicos}
\end{Highlighting}
\end{Shaded}

\begin{verbatim}
[1, 2, 3, 4, 5, 6, 7, 8, 9, 10]
\end{verbatim}

\begin{Shaded}
\begin{Highlighting}[]
\CommentTok{\# Otra forma mas pythonica!}
\NormalTok{numeros\_unicos\_easy }\OperatorTok{=} \BuiltInTok{list}\NormalTok{(}\BuiltInTok{set}\NormalTok{(numeros))}
\NormalTok{numeros\_unicos\_easy}
\end{Highlighting}
\end{Shaded}

\begin{verbatim}
[1, 2, 3, 4, 5, 6, 7, 8, 9, 10]
\end{verbatim}

\section{\texorpdfstring{\textbf{Ejemplo
21.}}{Ejemplo 21.}}\label{ejemplo-21.}

Eliminar objetos duplicados de una lista de diccionarios.

\begin{Shaded}
\begin{Highlighting}[]
\NormalTok{datos }\OperatorTok{=}\NormalTok{ [}
\NormalTok{  \{}\StringTok{\textquotesingle{}id\textquotesingle{}}\NormalTok{: }\DecValTok{10}\NormalTok{, }\StringTok{\textquotesingle{}dato\textquotesingle{}}\NormalTok{: }\StringTok{\textquotesingle{}...\textquotesingle{}}\NormalTok{\},}
\NormalTok{  \{}\StringTok{\textquotesingle{}id\textquotesingle{}}\NormalTok{: }\DecValTok{11}\NormalTok{, }\StringTok{\textquotesingle{}dato\textquotesingle{}}\NormalTok{: }\StringTok{\textquotesingle{}...\textquotesingle{}}\NormalTok{\},}
\NormalTok{  \{}\StringTok{\textquotesingle{}id\textquotesingle{}}\NormalTok{: }\DecValTok{12}\NormalTok{, }\StringTok{\textquotesingle{}dato\textquotesingle{}}\NormalTok{: }\StringTok{\textquotesingle{}...\textquotesingle{}}\NormalTok{\},}
\NormalTok{  \{}\StringTok{\textquotesingle{}id\textquotesingle{}}\NormalTok{: }\DecValTok{10}\NormalTok{, }\StringTok{\textquotesingle{}dato\textquotesingle{}}\NormalTok{: }\StringTok{\textquotesingle{}...\textquotesingle{}}\NormalTok{\},}
\NormalTok{  \{}\StringTok{\textquotesingle{}id\textquotesingle{}}\NormalTok{: }\DecValTok{11}\NormalTok{, }\StringTok{\textquotesingle{}dato\textquotesingle{}}\NormalTok{: }\StringTok{\textquotesingle{}...\textquotesingle{}}\NormalTok{\},}
\NormalTok{]}

\BuiltInTok{print}\NormalTok{(datos)}
\end{Highlighting}
\end{Shaded}

\begin{verbatim}
[{'id': 10, 'dato': '...'}, {'id': 11, 'dato': '...'}, {'id': 12, 'dato': '...'}, {'id': 10, 'dato': '...'}, {'id': 11, 'dato': '...'}]
\end{verbatim}

\begin{Shaded}
\begin{Highlighting}[]
\CommentTok{\# La forma tradicional}
\NormalTok{objetos\_unicos }\OperatorTok{=}\NormalTok{ []}
\ControlFlowTok{for}\NormalTok{ d }\KeywordTok{in}\NormalTok{ datos:}
\NormalTok{    dato\_existe }\OperatorTok{=} \VariableTok{False}
    \ControlFlowTok{for}\NormalTok{ ou }\KeywordTok{in}\NormalTok{ objetos\_unicos:}
        \ControlFlowTok{if}\NormalTok{ ou[}\StringTok{\textquotesingle{}id\textquotesingle{}}\NormalTok{] }\OperatorTok{==}\NormalTok{ d[}\StringTok{\textquotesingle{}id\textquotesingle{}}\NormalTok{]:}
\NormalTok{          dato\_existe }\OperatorTok{=} \VariableTok{True}
          \ControlFlowTok{break}
    \ControlFlowTok{if} \KeywordTok{not}\NormalTok{ dato\_existe:}
\NormalTok{        objetos\_unicos.append(d)}
        
\BuiltInTok{print}\NormalTok{(objetos\_unicos)}
\end{Highlighting}
\end{Shaded}

\begin{verbatim}
[{'id': 10, 'dato': '...'}, {'id': 11, 'dato': '...'}, {'id': 12, 'dato': '...'}]
\end{verbatim}

\begin{Shaded}
\begin{Highlighting}[]
\CommentTok{\# Una mejor manera.}
\NormalTok{objetos\_unicos\_easy }\OperatorTok{=}\NormalTok{ \{ d[}\StringTok{\textquotesingle{}id\textquotesingle{}}\NormalTok{]:d }\ControlFlowTok{for}\NormalTok{ d }\KeywordTok{in}\NormalTok{ datos \}.values()}

\BuiltInTok{print}\NormalTok{(}\BuiltInTok{list}\NormalTok{(objetos\_unicos\_easy))}
\end{Highlighting}
\end{Shaded}

\begin{verbatim}
[{'id': 10, 'dato': '...'}, {'id': 11, 'dato': '...'}, {'id': 12, 'dato': '...'}]
\end{verbatim}

\section{\texorpdfstring{\textbf{Ejemplo
22.}}{Ejemplo 22.}}\label{ejemplo-22.}

Sea un diccionario que tiene como claves letras minúsculas y mayúsculas,
y como valores números enteros:

\begin{Shaded}
\begin{Highlighting}[]
\NormalTok{mcase }\OperatorTok{=}\NormalTok{ \{}\StringTok{\textquotesingle{}z\textquotesingle{}}\NormalTok{:}\DecValTok{23}\NormalTok{, }\StringTok{\textquotesingle{}a\textquotesingle{}}\NormalTok{:}\DecValTok{30}\NormalTok{, }\StringTok{\textquotesingle{}b\textquotesingle{}}\NormalTok{:}\DecValTok{21}\NormalTok{, }\StringTok{\textquotesingle{}A\textquotesingle{}}\NormalTok{:}\DecValTok{78}\NormalTok{, }\StringTok{\textquotesingle{}Z\textquotesingle{}}\NormalTok{:}\DecValTok{4}\NormalTok{, }\StringTok{\textquotesingle{}C\textquotesingle{}}\NormalTok{:}\DecValTok{43}\NormalTok{, }\StringTok{\textquotesingle{}B\textquotesingle{}}\NormalTok{:}\DecValTok{89}\NormalTok{\}}
\end{Highlighting}
\end{Shaded}

\begin{itemize}
\item
  Sumar los valores que corresponden a la misma letra, mayúscula y
  minúscula.
\item
  Construir un diccionario cuyas claves sean solo letras minúsculas y
  sus valores sean la suma antes calculada.
\end{itemize}

\begin{Shaded}
\begin{Highlighting}[]
\NormalTok{mcase }\OperatorTok{=}\NormalTok{ \{}\StringTok{\textquotesingle{}z\textquotesingle{}}\NormalTok{:}\DecValTok{23}\NormalTok{, }\StringTok{\textquotesingle{}a\textquotesingle{}}\NormalTok{:}\DecValTok{30}\NormalTok{, }\StringTok{\textquotesingle{}b\textquotesingle{}}\NormalTok{:}\DecValTok{21}\NormalTok{, }\StringTok{\textquotesingle{}A\textquotesingle{}}\NormalTok{:}\DecValTok{78}\NormalTok{, }\StringTok{\textquotesingle{}Z\textquotesingle{}}\NormalTok{:}\DecValTok{4}\NormalTok{, }\StringTok{\textquotesingle{}C\textquotesingle{}}\NormalTok{:}\DecValTok{43}\NormalTok{, }\StringTok{\textquotesingle{}B\textquotesingle{}}\NormalTok{:}\DecValTok{89}\NormalTok{\}}
\NormalTok{mcase\_freq }\OperatorTok{=}\NormalTok{ \{k.lower() : }
\NormalTok{              mcase.get(k.lower(), }\DecValTok{0}\NormalTok{) }\OperatorTok{+}\NormalTok{ mcase.get(k.upper(), }\DecValTok{0}\NormalTok{)}
              \ControlFlowTok{for}\NormalTok{ k }\KeywordTok{in}\NormalTok{ mcase.keys()\}}
\BuiltInTok{print}\NormalTok{(mcase\_freq)}
\end{Highlighting}
\end{Shaded}

\begin{verbatim}
{'z': 27, 'a': 108, 'b': 110, 'c': 43}
\end{verbatim}

\section{\texorpdfstring{\textbf{Ejemplo
23.}}{Ejemplo 23.}}\label{ejemplo-23.}

Es posible usar las listas y diccionarios concisos para revisar la lista
de archivos de un directorio y sus características.

\begin{Shaded}
\begin{Highlighting}[]
\ImportTok{import}\NormalTok{ os, glob}
\NormalTok{metadata }\OperatorTok{=}\NormalTok{ [(f, os.stat(f)) }\ControlFlowTok{for}\NormalTok{ f }\KeywordTok{in}\NormalTok{ glob.glob(}\StringTok{\textquotesingle{}*.ipynb\textquotesingle{}}\NormalTok{)]}
\NormalTok{metadata}
\end{Highlighting}
\end{Shaded}

\begin{verbatim}
[('Pensando_como_pythonista_1.ipynb',
  os.stat_result(st_mode=33188, st_ino=2172678637, st_dev=2097240, st_nlink=1, st_uid=1000, st_gid=100, st_size=29456, st_atime=1705689214, st_mtime=1705689214, st_ctime=1705689214)),
 ('Pythonico_es_mas_bonito_1.ipynb',
  os.stat_result(st_mode=33188, st_ino=2172678648, st_dev=2097240, st_nlink=1, st_uid=1000, st_gid=100, st_size=39608, st_atime=1705689214, st_mtime=1705689214, st_ctime=1705689214)),
 ('Pythonico_es_mas_bonito_2.ipynb',
  os.stat_result(st_mode=33188, st_ino=2172678651, st_dev=2097240, st_nlink=1, st_uid=1000, st_gid=100, st_size=29765, st_atime=1705689214, st_mtime=1705689214, st_ctime=1705689214)),
 ('T02_Expr_Decla.ipynb',
  os.stat_result(st_mode=33188, st_ino=2172811400, st_dev=2097240, st_nlink=1, st_uid=1000, st_gid=100, st_size=8386, st_atime=1709740171, st_mtime=1709683460, st_ctime=1709683460)),
 ('T03_TiposBasico_Operadores.ipynb',
  os.stat_result(st_mode=33188, st_ino=2173059438, st_dev=2097240, st_nlink=1, st_uid=1000, st_gid=100, st_size=35254, st_atime=1710044492, st_mtime=1709321664, st_ctime=1709321664)),
 ('T04_Cadenas.ipynb',
  os.stat_result(st_mode=33188, st_ino=2174792627, st_dev=2097240, st_nlink=1, st_uid=1000, st_gid=100, st_size=18314, st_atime=1709510973, st_mtime=1709323123, st_ctime=1709323123)),
 ('T00_Otros.ipynb',
  os.stat_result(st_mode=33188, st_ino=2173586078, st_dev=2097240, st_nlink=1, st_uid=1000, st_gid=100, st_size=6513, st_atime=1710044528, st_mtime=1710044528, st_ctime=1710044528)),
 ('T01_Etiquetas_y_Palabras_Reservadas.ipynb',
  os.stat_result(st_mode=33188, st_ino=2175783890, st_dev=2097240, st_nlink=1, st_uid=1000, st_gid=100, st_size=20284, st_atime=1709740171, st_mtime=1709666919, st_ctime=1709666919)),
 ('T05_Estructura_de_Datos.ipynb',
  os.stat_result(st_mode=33188, st_ino=2172811396, st_dev=2097240, st_nlink=1, st_uid=1000, st_gid=100, st_size=44066, st_atime=1709567068, st_mtime=1709566559, st_ctime=1709566559)),
 ('T06_Control_de_flujo.ipynb',
  os.stat_result(st_mode=33188, st_ino=2172811404, st_dev=2097240, st_nlink=1, st_uid=1000, st_gid=100, st_size=27294, st_atime=1709598006, st_mtime=1709598006, st_ctime=1709598006)),
 ('zT10_Comprehensions.ipynb',
  os.stat_result(st_mode=33188, st_ino=2172811425, st_dev=2097240, st_nlink=1, st_uid=1000, st_gid=100, st_size=49575, st_atime=1710099347, st_mtime=1710099347, st_ctime=1710099347)),
 ('zT11_IteradoresGeneradores.ipynb',
  os.stat_result(st_mode=33188, st_ino=2172811428, st_dev=2097240, st_nlink=1, st_uid=1000, st_gid=100, st_size=11449, st_atime=1710044551, st_mtime=1710024222, st_ctime=1710024222)),
 ('zT12_Decoradores.ipynb',
  os.stat_result(st_mode=33188, st_ino=2172812549, st_dev=2097240, st_nlink=1, st_uid=1000, st_gid=100, st_size=13349, st_atime=1709600117, st_mtime=1705689214, st_ctime=1709596220)),
 ('zT13_BibliotecaEstandar.ipynb',
  os.stat_result(st_mode=33188, st_ino=2172812551, st_dev=2097240, st_nlink=1, st_uid=1000, st_gid=100, st_size=15600, st_atime=1709679822, st_mtime=1709679822, st_ctime=1709679822)),
 ('T08_Archivos_Gestores_de_contexto.ipynb',
  os.stat_result(st_mode=33188, st_ino=2172811408, st_dev=2097240, st_nlink=1, st_uid=1000, st_gid=100, st_size=5865, st_atime=1709600072, st_mtime=1709600072, st_ctime=1709600072)),
 ('T07_Entrada_salida_estandar.ipynb',
  os.stat_result(st_mode=33188, st_ino=2173013626, st_dev=2097240, st_nlink=1, st_uid=1000, st_gid=100, st_size=12347, st_atime=1709599786, st_mtime=1709599786, st_ctime=1709599786)),
 ('T09_Funciones_y_docstring.ipynb',
  os.stat_result(st_mode=33188, st_ino=2172811410, st_dev=2097240, st_nlink=1, st_uid=1000, st_gid=100, st_size=33782, st_atime=1709774198, st_mtime=1709668427, st_ctime=1709668427)),
 ('T11_Excepciones.ipynb',
  os.stat_result(st_mode=33188, st_ino=2172811414, st_dev=2097240, st_nlink=1, st_uid=1000, st_gid=100, st_size=34820, st_atime=1709740171, st_mtime=1709680686, st_ctime=1710038069)),
 ('T10_LambdaExpressions.ipynb',
  os.stat_result(st_mode=33188, st_ino=2172811418, st_dev=2097240, st_nlink=1, st_uid=1000, st_gid=100, st_size=15325, st_atime=1710043865, st_mtime=1710043865, st_ctime=1710043865)),
 ('T12_IterablesMapFilter.ipynb',
  os.stat_result(st_mode=33188, st_ino=2172811416, st_dev=2097240, st_nlink=1, st_uid=1000, st_gid=100, st_size=32836, st_atime=1710090534, st_mtime=1710090534, st_ctime=1710090534))]
\end{verbatim}

\begin{Shaded}
\begin{Highlighting}[]
\NormalTok{metadata\_dict }\OperatorTok{=}\NormalTok{ \{f:os.stat(f) }\ControlFlowTok{for}\NormalTok{ f }\KeywordTok{in}\NormalTok{ glob.glob(}\StringTok{\textquotesingle{}*.ipynb\textquotesingle{}}\NormalTok{)\}}
\end{Highlighting}
\end{Shaded}

\begin{Shaded}
\begin{Highlighting}[]
\NormalTok{metadata\_dict.keys()}
\end{Highlighting}
\end{Shaded}

\begin{verbatim}
dict_keys(['Pensando_como_pythonista_1.ipynb', 'Pythonico_es_mas_bonito_1.ipynb', 'Pythonico_es_mas_bonito_2.ipynb', 'T02_Expr_Decla.ipynb', 'T03_TiposBasico_Operadores.ipynb', 'T04_Cadenas.ipynb', 'T00_Otros.ipynb', 'T01_Etiquetas_y_Palabras_Reservadas.ipynb', 'T05_Estructura_de_Datos.ipynb', 'T06_Control_de_flujo.ipynb', 'zT10_Comprehensions.ipynb', 'zT11_IteradoresGeneradores.ipynb', 'zT12_Decoradores.ipynb', 'zT13_BibliotecaEstandar.ipynb', 'T08_Archivos_Gestores_de_contexto.ipynb', 'T07_Entrada_salida_estandar.ipynb', 'T09_Funciones_y_docstring.ipynb', 'T11_Excepciones.ipynb', 'T10_LambdaExpressions.ipynb', 'T12_IterablesMapFilter.ipynb'])
\end{verbatim}

\begin{Shaded}
\begin{Highlighting}[]
\NormalTok{metadata\_dict[}\StringTok{\textquotesingle{}T02\_Expr\_Decla.ipynb\textquotesingle{}}\NormalTok{].st\_size}
\end{Highlighting}
\end{Shaded}

\begin{verbatim}
8386
\end{verbatim}

\bookmarksetup{startatroot}

\chapter{Iteradores y Generadores.}\label{iteradores-y-generadores.}

\textbf{Objetivo.} \ldots{}

\textbf{Funciones de Python}: \ldots{}

MACTI-Algebra\_Lineal\_01 by Luis M. de la Cruz is licensed under
Attribution-ShareAlike 4.0 International

\bookmarksetup{startatroot}

\chapter{Iteradores}\label{iteradores}

\begin{itemize}
\item
  Como vimos en la sección XXXX, en Python existen objetos que contienen
  secuencias de otros objetos (listas, tuplas, diccionarios, etc).
\item
  La mayoría de los objetos contenedores se pueden recorrer usando un
  ciclo \textbf{for \ldots{} in \ldots{}} . Este es un estilo claro y
  conveniente que impregna el universo de Python.
\end{itemize}

\textbf{Por ejemplo}:

\begin{Shaded}
\begin{Highlighting}[]
\NormalTok{mi\_cadena }\OperatorTok{=} \StringTok{"abcd"}

\BuiltInTok{print}\NormalTok{(}\StringTok{"}\CharTok{\textbackslash{}n}\StringTok{Iteración sobre una cadena: "}\NormalTok{, end}\OperatorTok{=}\StringTok{\textquotesingle{}\textquotesingle{}}\NormalTok{)}
\ControlFlowTok{for}\NormalTok{ char }\KeywordTok{in}\NormalTok{ mi\_cadena:}
    \BuiltInTok{print}\NormalTok{(char, end}\OperatorTok{=}\StringTok{\textquotesingle{} \textquotesingle{}}\NormalTok{)}
\end{Highlighting}
\end{Shaded}

\begin{verbatim}

Iteración sobre una cadena: a b c d 
\end{verbatim}

\textbf{Notas importantes}: - La instrucción \textbf{for} llama a la
función \textbf{iter()} que está definida dentro del objeto
\textbf{contenedor}. - La función \textbf{iter()} regresa como resultado
un objeto \textbf{iterador} que define el método
\textbf{\_\_next\_\_()}, con el que se puede acceder a los elementos del
objeto contenedor, uno a la vez. - Cuando no hay más elementos,
\textbf{\_\_next\_\_()} lanza una excepción de tipo
\textbf{StopIteration} que le dice al ciclo \textbf{for} que debe
terminar. - Se puede ejecutar al método \textbf{\_\_next\_\_()}, al
iterador, usando la función de biblioteca \textbf{next()}.

\textbf{Por ejemplo}:

\begin{Shaded}
\begin{Highlighting}[]
\NormalTok{iterador }\OperatorTok{=} \BuiltInTok{iter}\NormalTok{(mi\_cadena) }\CommentTok{\# Obtenemos un iterador para la cadena}
\BuiltInTok{print}\NormalTok{(}\BuiltInTok{type}\NormalTok{(iterador)) }\CommentTok{\# Obtenemos el tipo del iterador}
\BuiltInTok{print}\NormalTok{(}\BuiltInTok{next}\NormalTok{(iterador)) }\CommentTok{\# Aplicamos \_\_next\_\_()  al iterador para obtener: a}
\BuiltInTok{print}\NormalTok{(}\BuiltInTok{next}\NormalTok{(iterador)) }\CommentTok{\# Aplicamos \_\_next\_\_()  al iterador para obtener: b}
\BuiltInTok{print}\NormalTok{(}\BuiltInTok{next}\NormalTok{(iterador)) }\CommentTok{\# Aplicamos \_\_next\_\_()  al iterador para obtener: c}
\BuiltInTok{print}\NormalTok{(}\BuiltInTok{next}\NormalTok{(iterador)) }\CommentTok{\# Aplicamos \_\_next\_\_()  al iterador para obtener: d}
\end{Highlighting}
\end{Shaded}

\begin{verbatim}
<class 'str_ascii_iterator'>
a
b
c
d
\end{verbatim}

Cuando ya llegamos al final de la secuencia e intentamos aplicar
\texttt{\_\_next\_\_()} obtenemos una excepción:

\begin{Shaded}
\begin{Highlighting}[]
\BuiltInTok{next}\NormalTok{(iterador) }\CommentTok{\# Sobrepasó los elementos, se obtiene la excepción StopIteration}
\end{Highlighting}
\end{Shaded}

\begin{verbatim}
StopIteration: 
\end{verbatim}

Observa que cuando se hace el recorrido de la cadena usando el ciclo
\texttt{for} no se produce ninguna excepción debido a que maneja la
excepción para terminar el proceso adecuadamente.

Se puede crear un iterador y aplicarle la función \texttt{next()} a
cualquier secuencia, por ejemplo a una lista

\begin{Shaded}
\begin{Highlighting}[]
\CommentTok{\# Creación de una lista}
\NormalTok{cuadradosI }\OperatorTok{=}\NormalTok{ [x}\OperatorTok{*}\NormalTok{x }\ControlFlowTok{for}\NormalTok{ x }\KeywordTok{in} \BuiltInTok{range}\NormalTok{(}\DecValTok{10}\NormalTok{)]}
\BuiltInTok{print}\NormalTok{(cuadradosI)}
\end{Highlighting}
\end{Shaded}

\begin{verbatim}
[0, 1, 4, 9, 16, 25, 36, 49, 64, 81]
\end{verbatim}

\begin{Shaded}
\begin{Highlighting}[]
\CommentTok{\# Recorriendo la lista usando un iterador en una lista concisa:}
\NormalTok{iterador }\OperatorTok{=} \BuiltInTok{iter}\NormalTok{(cuadradosI)}
\NormalTok{[}\BuiltInTok{next}\NormalTok{(iterador) }\ControlFlowTok{for}\NormalTok{ x }\KeywordTok{in} \BuiltInTok{range}\NormalTok{(}\DecValTok{10}\NormalTok{)]}
\end{Highlighting}
\end{Shaded}

\begin{verbatim}
[0, 1, 4, 9, 16, 25, 36, 49, 64, 81]
\end{verbatim}

Estos objetos iterables son manejables y prácticos debido a que se
pueden usar tantas veces como se desee, pero se almacenan todos los
valores en memoria y esto no siempre es coveniente, sobre todo cuando se
tienen muchos valores.

\bookmarksetup{startatroot}

\chapter{Generadores}\label{generadores}

\begin{itemize}
\item
  Los objetos \textbf{generadores} son iteradores.
\item
  Pero solo se puede iterar sobre ellos una sola vez. Esto es porque los
  generadores no almacenan todos los valores en memoria, ellos generan
  los valores al vuelo.
\item
  Un generador se crea como sigue:
\end{itemize}

\begin{Shaded}
\begin{Highlighting}[]
\NormalTok{(expresion }\ControlFlowTok{for}\NormalTok{ x }\KeywordTok{in}\NormalTok{ secuencia)}
\end{Highlighting}
\end{Shaded}

donde \texttt{expresion} es una expresión válida de Python que genera
los elementos del generador; \texttt{x} es un elemento al que se le
aplica la \texttt{expresion} y \texttt{secuencia} es cualquier secuencia
válida en Python.

\textbf{Por ejemplo}:

\begin{Shaded}
\begin{Highlighting}[]
\CommentTok{\# Un generador simple}
\NormalTok{gen }\OperatorTok{=}\NormalTok{ (x }\ControlFlowTok{for}\NormalTok{ x }\KeywordTok{in} \BuiltInTok{range}\NormalTok{(}\DecValTok{3}\NormalTok{))}

\BuiltInTok{print}\NormalTok{(}\BuiltInTok{next}\NormalTok{(gen))}
\BuiltInTok{print}\NormalTok{(}\BuiltInTok{next}\NormalTok{(gen))}
\BuiltInTok{print}\NormalTok{(}\BuiltInTok{next}\NormalTok{(gen))}
\BuiltInTok{print}\NormalTok{(}\BuiltInTok{next}\NormalTok{(gen)) }\CommentTok{\# Produce una excepción de tipo StopIteration}
\end{Highlighting}
\end{Shaded}

\begin{verbatim}
0
1
2
\end{verbatim}

\begin{verbatim}
StopIteration: 
\end{verbatim}

\begin{Shaded}
\begin{Highlighting}[]
\CommentTok{\# Creamos el generador}
\NormalTok{cuadradosG }\OperatorTok{=}\NormalTok{ (x}\OperatorTok{*}\NormalTok{x }\ControlFlowTok{for}\NormalTok{ x }\KeywordTok{in} \BuiltInTok{range}\NormalTok{(}\DecValTok{10}\NormalTok{))}
\BuiltInTok{print}\NormalTok{(}\BuiltInTok{type}\NormalTok{(cuadradosG))}

\CommentTok{\# Recorremos el generador en un ciclo for}
\ControlFlowTok{for}\NormalTok{ i }\KeywordTok{in}\NormalTok{ cuadradosG:}
    \BuiltInTok{print}\NormalTok{(i, end}\OperatorTok{=}\StringTok{\textquotesingle{} \textquotesingle{}}\NormalTok{)}
\end{Highlighting}
\end{Shaded}

\begin{verbatim}
<class 'generator'>
0 1 4 9 16 25 36 49 64 81 
\end{verbatim}

En el ejemplo anterior tenemos: - genera el 0, es usado y lo olvida -
genera el 1, es usado y lo olvida - genera el 4, es usado y lo olvida -
etcétera.

Un generador solo se puede usar una vez, pues va calculando sus valores
uno por uno e inmediatamente los va olvidando. Si intentamos utilizar
una vez más el generador, ya no obtendremos nada:

\begin{Shaded}
\begin{Highlighting}[]
\ControlFlowTok{for}\NormalTok{ i }\KeywordTok{in}\NormalTok{ cuadradosG:    }\CommentTok{\# Este ciclo no imprimirá nada por que}
    \BuiltInTok{print}\NormalTok{(i, end}\OperatorTok{=}\StringTok{\textquotesingle{} \textquotesingle{}}\NormalTok{)   }\CommentTok{\# el generador ya se usó antes}
\end{Highlighting}
\end{Shaded}

Observa que no se produce un error porque estamos usando el generador,
que ya ha sido usado con anterioridad, dentro del ciclo \texttt{for}.

\bookmarksetup{startatroot}

\chapter{Yield}\label{yield}

\begin{itemize}
\item
  Es una palabra clave que suspende la ejecución de una función y envía
  un valor de regreso a quien la ejecuta, pero retiene la información
  suficiente para reactivar la ejecución de la función donde se quedó.
  Si la función se vuelve a ejecutar, se reanuda desde donde se detuvo
  la última vez.
\item
  Esto permite al código producir una serie de valores uno por uno, en
  vez de calcularlos y regresarlos todos.
\item
  Una función que contiene la declaración \texttt{yield} se le conoce
  como función generadora.
\end{itemize}

\textbf{Por ejemplo}:

\begin{Shaded}
\begin{Highlighting}[]
\CommentTok{\# Función generadora}
\KeywordTok{def}\NormalTok{ generadorSimple():}
    \BuiltInTok{print}\NormalTok{(}\StringTok{\textquotesingle{}yield 1 : \textquotesingle{}}\NormalTok{, end}\OperatorTok{=}\StringTok{\textquotesingle{} \textquotesingle{}}\NormalTok{)}
    \ControlFlowTok{yield} \DecValTok{1}
    \BuiltInTok{print}\NormalTok{(}\StringTok{\textquotesingle{}yield 2 : \textquotesingle{}}\NormalTok{, end}\OperatorTok{=}\StringTok{\textquotesingle{} \textquotesingle{}}\NormalTok{)}
    \ControlFlowTok{yield} \DecValTok{2}
    \BuiltInTok{print}\NormalTok{(}\StringTok{\textquotesingle{}yield 3 : \textquotesingle{}}\NormalTok{, end}\OperatorTok{=}\StringTok{\textquotesingle{} \textquotesingle{}}\NormalTok{)}
    \ControlFlowTok{yield} \DecValTok{3}

\CommentTok{\# Se construye un generador}
\NormalTok{gen }\OperatorTok{=}\NormalTok{ generadorSimple()}

\CommentTok{\# Se usa el generador}
\BuiltInTok{print}\NormalTok{(}\StringTok{\textquotesingle{}Primera ejecución de la función generadora: }\SpecialCharTok{\{\}}\StringTok{\textquotesingle{}}\NormalTok{.}\BuiltInTok{format}\NormalTok{(}\BuiltInTok{next}\NormalTok{(gen)))}
\BuiltInTok{print}\NormalTok{(}\StringTok{\textquotesingle{}Segunda ejecución de la función generadora: }\SpecialCharTok{\{\}}\StringTok{\textquotesingle{}}\NormalTok{.}\BuiltInTok{format}\NormalTok{(}\BuiltInTok{next}\NormalTok{(gen)))}
\BuiltInTok{print}\NormalTok{(}\StringTok{\textquotesingle{}Tercera ejecución de la función generadora: }\SpecialCharTok{\{\}}\StringTok{\textquotesingle{}}\NormalTok{.}\BuiltInTok{format}\NormalTok{(}\BuiltInTok{next}\NormalTok{(gen)))}
\end{Highlighting}
\end{Shaded}

\begin{verbatim}
yield 1 :  Primera ejecución de la función generadora: 1
yield 2 :  Segunda ejecución de la función generadora: 2
yield 3 :  Tercera ejecución de la función generadora: 3
\end{verbatim}

Si se intenta usar una vez más el generador obtendremos una excepción de
tipo \texttt{StopIteration}:

\begin{Shaded}
\begin{Highlighting}[]
\BuiltInTok{print}\NormalTok{(}\StringTok{\textquotesingle{}Cuarta ejecución de la función generadora: }\SpecialCharTok{\{\}}\StringTok{\textquotesingle{}}\NormalTok{.}\BuiltInTok{format}\NormalTok{(}\BuiltInTok{next}\NormalTok{(gen)))}
\end{Highlighting}
\end{Shaded}

\begin{verbatim}
StopIteration: 
\end{verbatim}

\textbf{Notas importantes}. - \textbf{yield} es usada como un
\textbf{return}, excepto que la función regresa un objeto
\textbf{generador}. - Las funciones generadoras regresan un objeto
generator. - Los objetos generadores pueden ser usados en ciclos
\textbf{for \ldots{} in \ldots{}} o \texttt{while}.

Entonces, una función generadora regresa un objeto \textbf{generador}
que es iterable, es decir, se puede usar como un \textbf{iterador}.

\begin{Shaded}
\begin{Highlighting}[]
\KeywordTok{def}\NormalTok{ construyeUnGenerador(v):}
    \ControlFlowTok{for}\NormalTok{ i }\KeywordTok{in} \BuiltInTok{range}\NormalTok{(v):       }
        \ControlFlowTok{yield}\NormalTok{ i}\OperatorTok{*}\NormalTok{i           }

\CommentTok{\# Se construye una función generadora}
\NormalTok{cuadradosY }\OperatorTok{=}\NormalTok{ construyeUnGenerador(}\DecValTok{10}\NormalTok{) }
\BuiltInTok{print}\NormalTok{(}\BuiltInTok{type}\NormalTok{(cuadradosY))}

\ControlFlowTok{for}\NormalTok{ i }\KeywordTok{in}\NormalTok{ cuadradosY:}
    \BuiltInTok{print}\NormalTok{(i)}
\end{Highlighting}
\end{Shaded}

\begin{verbatim}
<class 'generator'>
0
1
4
9
16
25
36
49
64
81
\end{verbatim}

Se recomienda usar \textbf{yield} cuando se desea iterar sobre una
secuencia, pero no se quiere almacenar toda la secuencia en memoria.

\section{\texorpdfstring{\textbf{Ejemplo
1.}}{Ejemplo 1.}}\label{ejemplo-1.-5}

Crear una función generadora que genere los cuadrados del 1 al
\(\infty\).

\begin{Shaded}
\begin{Highlighting}[]
\CommentTok{\# Función generadora que genera el cuadrado de un número}
\KeywordTok{def}\NormalTok{ cuadradoSiguiente():}
\NormalTok{    i }\OperatorTok{=} \DecValTok{1}\OperatorTok{;} 
    \ControlFlowTok{while} \VariableTok{True}\NormalTok{:}
        \ControlFlowTok{yield}\NormalTok{ i}\OperatorTok{*}\NormalTok{i                }
\NormalTok{        i }\OperatorTok{+=} \DecValTok{1}  \CommentTok{\# La siguiente ejecución se }
                \CommentTok{\# reactiva en este punto   }

\ControlFlowTok{for}\NormalTok{ numero }\KeywordTok{in}\NormalTok{ cuadradoSiguiente():}
    \ControlFlowTok{if}\NormalTok{ numero }\OperatorTok{\textgreater{}} \DecValTok{100}\NormalTok{:}
         \ControlFlowTok{break}   
    \BuiltInTok{print}\NormalTok{(numero)}
\end{Highlighting}
\end{Shaded}

\begin{verbatim}
1
4
9
16
25
36
49
64
81
100
\end{verbatim}

\section{\texorpdfstring{\textbf{Ejemplo
1.}}{Ejemplo 1.}}\label{ejemplo-1.-6}

Crear un generador de los números de Fibonacci.

\begin{Shaded}
\begin{Highlighting}[]
\CommentTok{\# Función generadora}
\KeywordTok{def}\NormalTok{ fib(limite):}
\NormalTok{    a, b }\OperatorTok{=} \DecValTok{0}\NormalTok{, }\DecValTok{1}

    \ControlFlowTok{while}\NormalTok{ a }\OperatorTok{\textless{}}\NormalTok{ limite:}
        \ControlFlowTok{yield}\NormalTok{ a }
\NormalTok{        a, b }\OperatorTok{=}\NormalTok{ b, a }\OperatorTok{+}\NormalTok{ b }\CommentTok{\# La siguiente iteración se reactiva en este punto}
\end{Highlighting}
\end{Shaded}

\begin{Shaded}
\begin{Highlighting}[]
\NormalTok{N }\OperatorTok{=} \DecValTok{100}

\CommentTok{\# Generador}
\NormalTok{x }\OperatorTok{=}\NormalTok{ fib(N)}

\ControlFlowTok{while} \VariableTok{True}\NormalTok{:}
    \ControlFlowTok{try}\NormalTok{:}
        \BuiltInTok{print}\NormalTok{(}\BuiltInTok{next}\NormalTok{(x), end}\OperatorTok{=}\StringTok{\textquotesingle{} \textquotesingle{}}\NormalTok{) }\CommentTok{\# Usamos la función next() para iterar}
    \ControlFlowTok{except} \PreprocessorTok{StopIteration}\NormalTok{:       }\CommentTok{\# Manejamos la excepción}
        \ControlFlowTok{break}
\end{Highlighting}
\end{Shaded}

\begin{verbatim}
0 1 1 2 3 5 8 13 21 34 55 89 
\end{verbatim}

\begin{Shaded}
\begin{Highlighting}[]
\CommentTok{\# Usando la función generadora directamente en un ciclo for}
\ControlFlowTok{for}\NormalTok{ i }\KeywordTok{in}\NormalTok{ fib(N): }
    \BuiltInTok{print}\NormalTok{(i, end}\OperatorTok{=}\StringTok{\textquotesingle{} \textquotesingle{}}\NormalTok{)}
\end{Highlighting}
\end{Shaded}

\begin{verbatim}
0 1 1 2 3 5 8 13 21 34 55 89 
\end{verbatim}

\bookmarksetup{startatroot}

\chapter{Decoradores.}\label{decoradores.}

\textbf{Objetivo.} \ldots{}

\textbf{Funciones de Python}: \ldots{}

MACTI-Algebra\_Lineal\_01 by Luis M. de la Cruz is licensed under
Attribution-ShareAlike 4.0 International

\bookmarksetup{startatroot}

\chapter{Definición.}\label{definiciuxf3n.}

\begin{itemize}
\tightlist
\item
  Se denomina decorador a la persona dedicada a diseñar el interior de
  oficinas, viviendas o establecimientos comerciales con criterios
  estéticos y funcionales.
\item
  En Python, un decorador es una función para modificar otra función.

  \begin{itemize}
  \tightlist
  \item
    Recibe una función.
  \item
    Regresa otra función.
  \end{itemize}
\item
  Los decoradores son herramientas bonitas y útiles de Python.
\end{itemize}

\section{\texorpdfstring{\textbf{Ejemplo
1.}}{Ejemplo 1.}}\label{ejemplo-1.-7}

La función \texttt{print\_hello()} imprime
\texttt{Hola\ mundo\ pythonico}.

\begin{Shaded}
\begin{Highlighting}[]
\KeywordTok{def}\NormalTok{ print\_hello():}
    \BuiltInTok{print}\NormalTok{(}\StringTok{\textquotesingle{}}\SpecialCharTok{\{:\^{}30\}}\StringTok{\textquotesingle{}}\NormalTok{.}\BuiltInTok{format}\NormalTok{(}\StringTok{\textquotesingle{}Hola mundo pythonico\textquotesingle{}}\NormalTok{))}
\end{Highlighting}
\end{Shaded}

Crear un decorador que agregue colores al mensaje.

\begin{Shaded}
\begin{Highlighting}[]
\KeywordTok{def}\NormalTok{ print\_hello():}
    \BuiltInTok{print}\NormalTok{(}\StringTok{\textquotesingle{}}\SpecialCharTok{\{:\^{}30\}}\StringTok{\textquotesingle{}}\NormalTok{.}\BuiltInTok{format}\NormalTok{(}\StringTok{\textquotesingle{}Hola mundo pythonico\textquotesingle{}}\NormalTok{))}
\end{Highlighting}
\end{Shaded}

\begin{Shaded}
\begin{Highlighting}[]
\CommentTok{\# Uso normal de la función}
\NormalTok{print\_hello()}
\end{Highlighting}
\end{Shaded}

\begin{verbatim}
     Hola mundo pythonico     
\end{verbatim}

\begin{Shaded}
\begin{Highlighting}[]
\ImportTok{from}\NormalTok{ colorama }\ImportTok{import}\NormalTok{ Fore, Back, Style}

\CommentTok{\# Decorador}
\KeywordTok{def}\NormalTok{ mi\_decorador1(f):}

    \CommentTok{\# La función que hace el decorado.}
    \KeywordTok{def}\NormalTok{ envoltura():}
\NormalTok{        linea }\OperatorTok{=} \StringTok{\textquotesingle{}{-}\textquotesingle{}} \OperatorTok{*} \DecValTok{30}
        \BuiltInTok{print}\NormalTok{(Fore.BLUE }\OperatorTok{+}\NormalTok{ linea }\OperatorTok{+}\NormalTok{ Style.RESET\_ALL)}
        \BuiltInTok{print}\NormalTok{(Back.GREEN }\OperatorTok{+}\NormalTok{ Fore.WHITE, end}\OperatorTok{=}\StringTok{\textquotesingle{}\textquotesingle{}}\NormalTok{)}
        
\NormalTok{        f() }\CommentTok{\# Ejecución de la función}
        
        \BuiltInTok{print}\NormalTok{(Style.RESET\_ALL, end}\OperatorTok{=}\StringTok{\textquotesingle{}\textquotesingle{}}\NormalTok{)}
        \BuiltInTok{print}\NormalTok{(Fore.BLUE }\OperatorTok{+}\NormalTok{ linea }\OperatorTok{+}\NormalTok{ Style.RESET\_ALL)}

    \CommentTok{\# Regresamos la función decorada}
    \ControlFlowTok{return}\NormalTok{ envoltura }

\CommentTok{\# Decorando la función.}
\NormalTok{print\_hello\_colored }\OperatorTok{=}\NormalTok{ mi\_decorador1(print\_hello) }\CommentTok{\# Funcion decorada}

\CommentTok{\# Ahora se ejecuta la función decorada.}
\NormalTok{print\_hello\_colored()}
\end{Highlighting}
\end{Shaded}

\begin{verbatim}
------------------------------
     Hola mundo pythonico     
------------------------------
\end{verbatim}

\section{\texorpdfstring{\textbf{Ejemplo
2.}}{Ejemplo 2.}}\label{ejemplo-2.-3}

La función \texttt{print\_message(m)} imprime el mensaje que recibe como
parámetro.

\begin{Shaded}
\begin{Highlighting}[]
\KeywordTok{def}\NormalTok{ print\_message(m):}
    \BuiltInTok{print}\NormalTok{(}\StringTok{\textquotesingle{}}\SpecialCharTok{\{:\^{}30\}}\StringTok{\textquotesingle{}}\NormalTok{.}\BuiltInTok{format}\NormalTok{(m))}
\end{Highlighting}
\end{Shaded}

Modificar el decorador creado en el ejemplo 1 para que se pueda recibir
el parámetro \texttt{m}.

\begin{Shaded}
\begin{Highlighting}[]
\CommentTok{\# Decorador}
\KeywordTok{def}\NormalTok{ mi\_decorador2(f):}

    \CommentTok{\# La función que hace el decorado.}
    \CommentTok{\# Ahora recibe un parámetro}
    \KeywordTok{def}\NormalTok{ envoltura(m):}
\NormalTok{        linea }\OperatorTok{=} \StringTok{\textquotesingle{}{-}\textquotesingle{}} \OperatorTok{*} \DecValTok{30}
        \BuiltInTok{print}\NormalTok{(Fore.BLUE }\OperatorTok{+}\NormalTok{ linea }\OperatorTok{+}\NormalTok{ Style.RESET\_ALL)}
        \BuiltInTok{print}\NormalTok{(Back.GREEN }\OperatorTok{+}\NormalTok{ Fore.WHITE, end}\OperatorTok{=}\StringTok{\textquotesingle{}\textquotesingle{}}\NormalTok{)}
        
\NormalTok{        f(m) }\CommentTok{\# Ejecución de la función}
        
        \BuiltInTok{print}\NormalTok{(Style.RESET\_ALL, end}\OperatorTok{=}\StringTok{\textquotesingle{}\textquotesingle{}}\NormalTok{)}
        \BuiltInTok{print}\NormalTok{(Fore.BLUE }\OperatorTok{+}\NormalTok{ linea }\OperatorTok{+}\NormalTok{ Style.RESET\_ALL)}

    \CommentTok{\# Regresamos la función decorada}
    \ControlFlowTok{return}\NormalTok{ envoltura }
\end{Highlighting}
\end{Shaded}

\begin{Shaded}
\begin{Highlighting}[]
\CommentTok{\# La función se puede decorar en su definición como sigue}
\AttributeTok{@mi\_decorador2}
\KeywordTok{def}\NormalTok{ print\_message(m):}
    \BuiltInTok{print}\NormalTok{(}\StringTok{\textquotesingle{}}\SpecialCharTok{\{:\^{}30\}}\StringTok{\textquotesingle{}}\NormalTok{.}\BuiltInTok{format}\NormalTok{(m))}
\end{Highlighting}
\end{Shaded}

\begin{Shaded}
\begin{Highlighting}[]
\CommentTok{\# Entonces se puede usar la función con su nombre original}
\NormalTok{print\_message(}\StringTok{\textquotesingle{}bueno, bonito y barato\textquotesingle{}}\NormalTok{)}
\end{Highlighting}
\end{Shaded}

\begin{verbatim}
------------------------------
    bueno, bonito y barato    
------------------------------
\end{verbatim}

\section{\texorpdfstring{\textbf{Ejemplo
3.}}{Ejemplo 3.}}\label{ejemplo-3.-3}

Decorar las funciones \texttt{sin()} y \texttt{cos()} de la biblioteca
\texttt{math}.

\begin{Shaded}
\begin{Highlighting}[]
\KeywordTok{def}\NormalTok{ mi\_decorador3(f):}

    \KeywordTok{def}\NormalTok{ coloreado(x):}

        \CommentTok{\# Construimos una cadena coloreada con el }
        \CommentTok{\# resultado de la evaluación de f(x)}
\NormalTok{        res }\OperatorTok{=}\NormalTok{ Fore.GREEN }\OperatorTok{+}\NormalTok{ f.}\VariableTok{\_\_name\_\_} 
\NormalTok{        res }\OperatorTok{+=} \StringTok{\textquotesingle{}(\textquotesingle{}} \OperatorTok{+}\NormalTok{ Style.BRIGHT }\OperatorTok{+} \BuiltInTok{str}\NormalTok{(x) }\OperatorTok{+}\NormalTok{ Style.RESET\_ALL }\OperatorTok{+}\NormalTok{ Fore.GREEN }\OperatorTok{+} \StringTok{\textquotesingle{}) = \textquotesingle{}} \OperatorTok{+}\NormalTok{ Style.RESET\_ALL}
\NormalTok{        res }\OperatorTok{+=} \SpecialStringTok{f\textquotesingle{}}\SpecialCharTok{\{}\NormalTok{f(x)}\SpecialCharTok{\}}\SpecialStringTok{\textquotesingle{}}

        \CommentTok{\# Imprimimos el resultado}
\NormalTok{        linea }\OperatorTok{=} \StringTok{\textquotesingle{}{-}\textquotesingle{}} \OperatorTok{*} \DecValTok{80}
        \BuiltInTok{print}\NormalTok{(Fore.BLUE }\OperatorTok{+}\NormalTok{ linea }\OperatorTok{+}\NormalTok{ Style.RESET\_ALL)}
        \BuiltInTok{print}\NormalTok{(}\StringTok{\textquotesingle{}}\SpecialCharTok{\{:\^{}80\}}\StringTok{\textquotesingle{}}\NormalTok{.}\BuiltInTok{format}\NormalTok{(res))}
        \BuiltInTok{print}\NormalTok{(Fore.BLUE }\OperatorTok{+}\NormalTok{ linea }\OperatorTok{+}\NormalTok{ Style.RESET\_ALL)}

    \ControlFlowTok{return}\NormalTok{ coloreado}

\ImportTok{from}\NormalTok{ math }\ImportTok{import}\NormalTok{ sin, cos}

\NormalTok{sin }\OperatorTok{=}\NormalTok{ mi\_decorador3(sin)}
\NormalTok{cos }\OperatorTok{=}\NormalTok{ mi\_decorador3(cos)}

\ControlFlowTok{for}\NormalTok{ f }\KeywordTok{in}\NormalTok{ [sin, cos]:}
\NormalTok{    f(}\FloatTok{3.141596}\NormalTok{)}
\end{Highlighting}
\end{Shaded}

\begin{verbatim}
--------------------------------------------------------------------------------
         sin(3.141596) = -3.3464102065883993e-06          
--------------------------------------------------------------------------------
--------------------------------------------------------------------------------
           cos(3.141596) = -0.9999999999944008            
--------------------------------------------------------------------------------
\end{verbatim}

\section{\texorpdfstring{\textbf{Ejemplo
4.}}{Ejemplo 4.}}\label{ejemplo-4.-3}

Decorar funciones con un número variable de argumentos.

\begin{Shaded}
\begin{Highlighting}[]
\ImportTok{from}\NormalTok{ random }\ImportTok{import}\NormalTok{ random, randint, choice, choices}

\KeywordTok{def}\NormalTok{ mi\_decorador4(f):}
    \KeywordTok{def}\NormalTok{ envoltura(}\OperatorTok{*}\NormalTok{args, }\OperatorTok{**}\NormalTok{kwargs):}

        \CommentTok{\# Construimos una cadena coloreada con el }
        \CommentTok{\# resultado de la evaluación de f(x)}
\NormalTok{        res }\OperatorTok{=}\NormalTok{ Fore.GREEN }\OperatorTok{+}\NormalTok{ f.}\VariableTok{\_\_name\_\_} 
\NormalTok{        res }\OperatorTok{+=} \StringTok{\textquotesingle{}(\textquotesingle{}} \OperatorTok{+}\NormalTok{ Style.BRIGHT }\OperatorTok{+} \SpecialStringTok{f\textquotesingle{}}\SpecialCharTok{\{}\NormalTok{args}\SpecialCharTok{\}}\SpecialStringTok{,}\SpecialCharTok{\{}\NormalTok{kwargs}\SpecialCharTok{\}}\SpecialStringTok{\textquotesingle{}} \OperatorTok{+}\NormalTok{ Style.RESET\_ALL }\OperatorTok{+}\NormalTok{ Fore.GREEN }\OperatorTok{+} \StringTok{\textquotesingle{}) = \textquotesingle{}} \OperatorTok{+}\NormalTok{ Style.RESET\_ALL}
\NormalTok{        res }\OperatorTok{+=} \SpecialStringTok{f\textquotesingle{}}\SpecialCharTok{\{}\NormalTok{f(}\OperatorTok{*}\NormalTok{args, }\OperatorTok{**}\NormalTok{kwargs)}\SpecialCharTok{\}}\SpecialStringTok{\textquotesingle{}}
        
        \CommentTok{\# Imprimimos el resultado}
\NormalTok{        linea }\OperatorTok{=} \StringTok{\textquotesingle{}{-}\textquotesingle{}} \OperatorTok{*} \DecValTok{80}
        \BuiltInTok{print}\NormalTok{(Fore.BLUE }\OperatorTok{+}\NormalTok{ linea }\OperatorTok{+}\NormalTok{ Style.RESET\_ALL)}
        \BuiltInTok{print}\NormalTok{(}\StringTok{\textquotesingle{}}\SpecialCharTok{\{:\^{}80\}}\StringTok{\textquotesingle{}}\NormalTok{.}\BuiltInTok{format}\NormalTok{(res))}
        \BuiltInTok{print}\NormalTok{(Fore.BLUE }\OperatorTok{+}\NormalTok{ linea }\OperatorTok{+}\NormalTok{ Style.RESET\_ALL)}
        
    \ControlFlowTok{return}\NormalTok{ envoltura}

\NormalTok{random }\OperatorTok{=}\NormalTok{ mi\_decorador4(random)}
\NormalTok{randint }\OperatorTok{=}\NormalTok{ mi\_decorador4(randint)}
\NormalTok{choice }\OperatorTok{=}\NormalTok{ mi\_decorador4(choice)}
\NormalTok{choices }\OperatorTok{=}\NormalTok{ mi\_decorador4(choices)}

\NormalTok{random()}
\NormalTok{randint(}\DecValTok{3}\NormalTok{, }\DecValTok{8}\NormalTok{)}
\NormalTok{choice([}\DecValTok{4}\NormalTok{, }\DecValTok{5}\NormalTok{, }\DecValTok{6}\NormalTok{])}

\NormalTok{p }\OperatorTok{=}\NormalTok{ [x }\ControlFlowTok{for}\NormalTok{ x }\KeywordTok{in} \BuiltInTok{range}\NormalTok{(}\DecValTok{10}\NormalTok{)]}
\NormalTok{choices(p, k}\OperatorTok{=}\DecValTok{3}\NormalTok{)}
\end{Highlighting}
\end{Shaded}

\begin{verbatim}
--------------------------------------------------------------------------------
            random((),{}) = 0.4390656899525458            
--------------------------------------------------------------------------------
--------------------------------------------------------------------------------
                  randint((3, 8),{}) = 3                  
--------------------------------------------------------------------------------
--------------------------------------------------------------------------------
               choice(([4, 5, 6],),{}) = 5                
--------------------------------------------------------------------------------
--------------------------------------------------------------------------------
choices(([0, 1, 2, 3, 4, 5, 6, 7, 8, 9],),{'k': 3}) = [5, 3, 9]
--------------------------------------------------------------------------------
\end{verbatim}

\section{\texorpdfstring{\textbf{Ejemplo
5.}}{Ejemplo 5.}}\label{ejemplo-5.-3}

Crear un decorador que calcule el tiempo de ejecución de una función.

\begin{Shaded}
\begin{Highlighting}[]
\ImportTok{import}\NormalTok{ time}

\KeywordTok{def}\NormalTok{ crono(f):}
    \CommentTok{"""}
\CommentTok{    Regresa el tiempo que toma en ejecutarse la funcion.}
\CommentTok{    """}
    \KeywordTok{def}\NormalTok{ tiempo():}
\NormalTok{        t1 }\OperatorTok{=}\NormalTok{ time.perf\_counter()}
\NormalTok{        f()}
\NormalTok{        t2 }\OperatorTok{=}\NormalTok{ time.perf\_counter()}
        \ControlFlowTok{return} \StringTok{\textquotesingle{}Elapsed time: \textquotesingle{}} \OperatorTok{+} \BuiltInTok{str}\NormalTok{((t2 }\OperatorTok{{-}}\NormalTok{ t1)) }\OperatorTok{+} \StringTok{"}\CharTok{\textbackslash{}n}\StringTok{"}
    \ControlFlowTok{return}\NormalTok{ tiempo}

\AttributeTok{@crono}
\KeywordTok{def}\NormalTok{ miFuncion():}
\NormalTok{    numeros }\OperatorTok{=}\NormalTok{ []}
    \ControlFlowTok{for}\NormalTok{ num }\KeywordTok{in}\NormalTok{ (}\BuiltInTok{range}\NormalTok{(}\DecValTok{0}\NormalTok{, }\DecValTok{10000}\NormalTok{)):}
\NormalTok{        numeros.append(num)}
    \BuiltInTok{print}\NormalTok{(}\StringTok{\textquotesingle{}}\CharTok{\textbackslash{}n}\StringTok{La suma es: \textquotesingle{}} \OperatorTok{+} \BuiltInTok{str}\NormalTok{((}\BuiltInTok{sum}\NormalTok{(numeros))))}

\BuiltInTok{print}\NormalTok{(miFuncion())}
\end{Highlighting}
\end{Shaded}

\begin{verbatim}

La suma es: 49995000
Elapsed time: 0.0012546591460704803
\end{verbatim}

\section{\texorpdfstring{\textbf{Ejemplo
6.}}{Ejemplo 6.}}\label{ejemplo-6.-3}

Detener la ejecución por un tiempo antes que una función sea ejecutada.

\begin{Shaded}
\begin{Highlighting}[]
\ImportTok{from}\NormalTok{ time }\ImportTok{import}\NormalTok{ sleep}

\KeywordTok{def}\NormalTok{ sleepDecorador(function):}

    \KeywordTok{def}\NormalTok{ duerme(}\OperatorTok{*}\NormalTok{args, }\OperatorTok{**}\NormalTok{kwargs):}
\NormalTok{        sleep(}\DecValTok{1}\NormalTok{)}
        \ControlFlowTok{return}\NormalTok{ function(}\OperatorTok{*}\NormalTok{args, }\OperatorTok{**}\NormalTok{kwargs)}
    \ControlFlowTok{return}\NormalTok{ duerme}


\AttributeTok{@sleepDecorador}
\KeywordTok{def}\NormalTok{ imprimeNumero(num):}
    \ControlFlowTok{return}\NormalTok{ num}

\ControlFlowTok{for}\NormalTok{ num }\KeywordTok{in} \BuiltInTok{range}\NormalTok{(}\DecValTok{1}\NormalTok{, }\DecValTok{6}\NormalTok{):}
    \BuiltInTok{print}\NormalTok{(imprimeNumero(num), end }\OperatorTok{=} \StringTok{\textquotesingle{} \textquotesingle{}}\NormalTok{)}

\BuiltInTok{print}\NormalTok{(}\StringTok{\textquotesingle{}}\CharTok{\textbackslash{}n}\StringTok{ {-}{-}\textgreater{} happy finish!\textquotesingle{}}\NormalTok{)}
\end{Highlighting}
\end{Shaded}

\begin{verbatim}
1 2 3 4 5 
 --> happy finish!
\end{verbatim}

\section{\texorpdfstring{\textbf{Ejemplo
7.}}{Ejemplo 7.}}\label{ejemplo-7.-3}

Crear un decorador que cheque que el argumento de una función que
calcula el factorial, sea un entero positivo.

\begin{Shaded}
\begin{Highlighting}[]
\KeywordTok{def}\NormalTok{ checaArgumento(f):}
    \KeywordTok{def}\NormalTok{ checador(x):}
        \ControlFlowTok{if} \BuiltInTok{type}\NormalTok{(x) }\OperatorTok{==} \BuiltInTok{int} \KeywordTok{and}\NormalTok{ x }\OperatorTok{\textgreater{}} \DecValTok{0}\NormalTok{:}
            \ControlFlowTok{return}\NormalTok{ f(x)}
        \ControlFlowTok{else}\NormalTok{:}
            \ControlFlowTok{raise} \PreprocessorTok{Exception}\NormalTok{(}\StringTok{"El argumento no es un entero positivo"}\NormalTok{)}
    \ControlFlowTok{return}\NormalTok{ checador}

\AttributeTok{@checaArgumento}
\KeywordTok{def}\NormalTok{ factorial(n):}
    \ControlFlowTok{if}\NormalTok{ n }\OperatorTok{==} \DecValTok{1}\NormalTok{:}
        \ControlFlowTok{return} \DecValTok{1}
    \ControlFlowTok{else}\NormalTok{:}
        \ControlFlowTok{return}\NormalTok{ n }\OperatorTok{*}\NormalTok{ factorial(n}\OperatorTok{{-}}\DecValTok{1}\NormalTok{)}
    
\ControlFlowTok{for}\NormalTok{ i }\KeywordTok{in} \BuiltInTok{range}\NormalTok{(}\DecValTok{1}\NormalTok{,}\DecValTok{10}\NormalTok{):}
    \BuiltInTok{print}\NormalTok{(i, factorial(i))}
\end{Highlighting}
\end{Shaded}

\begin{verbatim}
1 1
2 2
3 6
4 24
5 120
6 720
7 5040
8 40320
9 362880
\end{verbatim}

\begin{Shaded}
\begin{Highlighting}[]
\BuiltInTok{print}\NormalTok{(factorial(}\OperatorTok{{-}}\DecValTok{1}\NormalTok{))}
\end{Highlighting}
\end{Shaded}

\begin{verbatim}
Exception: El argumento no es un entero positivo
\end{verbatim}

\section{\texorpdfstring{\textbf{Ejemplo
8.}}{Ejemplo 8.}}\label{ejemplo-8.-2}

Contar el número de llamadas de una función.

\begin{Shaded}
\begin{Highlighting}[]
\KeywordTok{def}\NormalTok{ contadorDeLlamadas(func):}
    
    \KeywordTok{def}\NormalTok{ cuenta(}\OperatorTok{*}\NormalTok{args, }\OperatorTok{**}\NormalTok{kwargs):}
\NormalTok{        cuenta.calls }\OperatorTok{+=} \DecValTok{1}
        \ControlFlowTok{return}\NormalTok{ func(}\OperatorTok{*}\NormalTok{args, }\OperatorTok{**}\NormalTok{kwargs)}
        
    \CommentTok{\# Variable estática que lleva la cuenta}
\NormalTok{    cuenta.calls }\OperatorTok{=} \DecValTok{0}
    
    \ControlFlowTok{return}\NormalTok{ cuenta}

\AttributeTok{@contadorDeLlamadas}
\KeywordTok{def}\NormalTok{ suma(x):}
    \ControlFlowTok{return}\NormalTok{ x }\OperatorTok{+} \DecValTok{1}

\AttributeTok{@contadorDeLlamadas}
\KeywordTok{def}\NormalTok{ mulp1(x, y}\OperatorTok{=}\DecValTok{1}\NormalTok{):}
    \ControlFlowTok{return}\NormalTok{ x}\OperatorTok{*}\NormalTok{y }\OperatorTok{+} \DecValTok{1}

\BuiltInTok{print}\NormalTok{(}\StringTok{\textquotesingle{}Llamadas a suma = }\SpecialCharTok{\{\}}\StringTok{\textquotesingle{}}\NormalTok{.}\BuiltInTok{format}\NormalTok{(suma.calls))}

\ControlFlowTok{for}\NormalTok{ i }\KeywordTok{in} \BuiltInTok{range}\NormalTok{(}\DecValTok{4}\NormalTok{):}
\NormalTok{    suma(i)}
    
\NormalTok{mulp1(}\DecValTok{1}\NormalTok{, }\DecValTok{2}\NormalTok{)}
\NormalTok{mulp1(}\DecValTok{5}\NormalTok{)}
\NormalTok{mulp1(y}\OperatorTok{=}\DecValTok{2}\NormalTok{, x}\OperatorTok{=}\DecValTok{25}\NormalTok{)}

\BuiltInTok{print}\NormalTok{(}\StringTok{\textquotesingle{}Llamadas a suma = }\SpecialCharTok{\{\}}\StringTok{\textquotesingle{}}\NormalTok{.}\BuiltInTok{format}\NormalTok{(suma.calls))}
\BuiltInTok{print}\NormalTok{(}\StringTok{\textquotesingle{}Llamadas a multp1 = }\SpecialCharTok{\{\}}\StringTok{\textquotesingle{}}\NormalTok{.}\BuiltInTok{format}\NormalTok{(mulp1.calls))}
\end{Highlighting}
\end{Shaded}

\begin{verbatim}
Llamadas a suma = 0
Llamadas a suma = 4
Llamadas a multp1 = 3
\end{verbatim}

\section{\texorpdfstring{\textbf{Ejemplo
9.}}{Ejemplo 9.}}\label{ejemplo-9.-3}

Decorar una función con diferentes saludos.

\begin{Shaded}
\begin{Highlighting}[]
\KeywordTok{def}\NormalTok{ buenasTardes(func):}
    \KeywordTok{def}\NormalTok{ saludo(x):}
        \BuiltInTok{print}\NormalTok{(}\StringTok{"Hola, buenas tardes, "}\NormalTok{, end}\OperatorTok{=}\StringTok{\textquotesingle{}\textquotesingle{}}\NormalTok{)}
\NormalTok{        func(x)}
    \ControlFlowTok{return}\NormalTok{ saludo}

\KeywordTok{def}\NormalTok{ buenosDias(func):}
    \KeywordTok{def}\NormalTok{ saludo(x):}
        \BuiltInTok{print}\NormalTok{(}\StringTok{"Hola, buenos días, "}\NormalTok{, end}\OperatorTok{=}\StringTok{\textquotesingle{}\textquotesingle{}}\NormalTok{)}
\NormalTok{        func(x)}
    \ControlFlowTok{return}\NormalTok{ saludo}

\AttributeTok{@buenasTardes}
\KeywordTok{def}\NormalTok{ mensaje1(hora):}
    \BuiltInTok{print}\NormalTok{(}\StringTok{"son las "} \OperatorTok{+}\NormalTok{ hora)}

\NormalTok{mensaje1(}\StringTok{"3 pm"}\NormalTok{)}

\AttributeTok{@buenosDias}
\KeywordTok{def}\NormalTok{ mensaje2(hora):}
    \BuiltInTok{print}\NormalTok{(}\StringTok{"son las "} \OperatorTok{+}\NormalTok{ hora)}
    
\NormalTok{mensaje2(}\StringTok{"8 am"}\NormalTok{)}
\end{Highlighting}
\end{Shaded}

\begin{verbatim}
Hola, buenas tardes, son las 3 pm
Hola, buenos días, son las 8 am
\end{verbatim}

\section{\texorpdfstring{\textbf{Ejemplo
10.}}{Ejemplo 10.}}\label{ejemplo-10.-2}

El ejemplo anterior se puede realizar como sigue:

\begin{Shaded}
\begin{Highlighting}[]
\KeywordTok{def}\NormalTok{ saludo(expr):}
    \KeywordTok{def}\NormalTok{ saludoDecorador(func):}
        \KeywordTok{def}\NormalTok{ saludoGenerico(x):}
            \BuiltInTok{print}\NormalTok{(expr, end}\OperatorTok{=}\StringTok{\textquotesingle{}\textquotesingle{}}\NormalTok{)}
\NormalTok{            func(x)}
        \ControlFlowTok{return}\NormalTok{ saludoGenerico}
    \ControlFlowTok{return}\NormalTok{ saludoDecorador}

\AttributeTok{@saludo}\NormalTok{(}\StringTok{"Hola, buenas tardes, "}\NormalTok{)}
\KeywordTok{def}\NormalTok{ mensaje1(hora):}
    \BuiltInTok{print}\NormalTok{(}\StringTok{"son las "} \OperatorTok{+}\NormalTok{ hora)}

\NormalTok{mensaje1(}\StringTok{"3 pm"}\NormalTok{)}

\AttributeTok{@saludo}\NormalTok{(}\StringTok{"Hola, buenos días, "}\NormalTok{)}
\KeywordTok{def}\NormalTok{ mensaje2(hora):}
    \BuiltInTok{print}\NormalTok{(}\StringTok{"son las "} \OperatorTok{+}\NormalTok{ hora)}
    
\NormalTok{mensaje2(}\StringTok{"8 am"}\NormalTok{)}

\AttributeTok{@saludo}\NormalTok{(}\StringTok{"καλημερα "}\NormalTok{)}
\KeywordTok{def}\NormalTok{ mensaje3(hora):}
    \BuiltInTok{print}\NormalTok{(}\StringTok{" \textless{}{-}{-}{-} en griego "} \OperatorTok{+}\NormalTok{ hora)}
    
\NormalTok{mensaje3(}\StringTok{" :D "}\NormalTok{)}
\end{Highlighting}
\end{Shaded}

\begin{verbatim}
Hola, buenas tardes, son las 3 pm
Hola, buenos días, son las 8 am
καλημερα  <--- en griego  :D 
\end{verbatim}



\end{document}
